\begin{document}
%\begin{CJK*}{GBK}{song}

\def\lecturename{嵌入式技术}

\title{\insertlecture}

\author{邢超}

\institute
{
  西北工业大学航天学院
}

%\mode<presentation>{\subject{嵌入式系统}}

%  start a lecture  --------------------------
\lecture[EC]{嵌入式系统概念及应用}{}
\subtitle{介绍嵌入式系统}
\date{2014}


%\setbeamertemplate{background}{\pgfimage[width=\paperwidth,height=\paperheight]{image/flower}}
%\setbeamercovered{transparent}
%\mode<presentation>{\beamerdefaultoverlayspecification{<+->}}

\begin{frame}
  \maketitle
\end{frame}

\begin{frame}{课程内容}
\begin{center}\pgfimage[width=0.9\columnwidth]{image/content}\end{center}
\end{frame}

\begin{frame}{与其它课程关系}
\begin{center}\pgfimage[width=0.9\columnwidth]{image/relation}\end{center}
\end{frame}



\section{嵌入式技术历史}

\begin{frame}{嵌入式?}
%The Jacquard loom is a mechanical loom, invented by Joseph Marie Jacquard in 1804, that simplifies the process of manufacturing textiles with complex patterns.
\begin{center}\pgfimage[width=0.5\textwidth]{image/Jacquard.loom.full.view}\end{center}
Jacquard loom on display at Museum of Science and Industry in Manchester, England
\end{frame}

\begin{frame}{嵌入式?}
%The Jacquard loom was the first machine to use punched cards to control a sequence of operations. 
\begin{center}\pgfimage[width=0.5\textwidth]{image/Jacquard.loom.cards}\end{center}
Close-up view of the 8x26 hole punched cards---one card per pick (weft) in the fabric.
\end{frame}

\begin{frame}{嵌入式?}
\begin{center}\pgfimage[width=0.9\columnwidth]{image/FortranCardPROJ039.agr}\end{center}
Card from a Fortran program: Z(1) = Y + W(1)
\end{frame}

\begin{frame}{嵌入式?}
\begin{center}\pgfimage[width=0.9\columnwidth]{image/NOLAPunchCards1938}\end{center}
"A good operator can turn out 1,500 punch cards daily." Operators compiling hydrographic data for navigation charts on punch cards, New Orleans, 1938.
\end{frame}


\begin{frame}{嵌入式?}
\begin{center}\pgfimage[width=0.9\columnwidth]{image/printer}\end{center}
\end{frame}

\begin{frame}{嵌入式?}
\begin{center}\pgfimage[width=0.9\columnwidth]{image/postscript}\end{center}
\end{frame}


\section{嵌入式芯片与嵌入式系统}


\begin{frame}{嵌入式系统定义}
\begin{itemize}
\item 从技术的角度定义:\\
以应用为中心、以计算机技术为基础、软件硬件可裁剪、适应应用系统对功能、可靠性、成本、体积、功耗严格要求的专用计算机系统。
\item 从系统的角度定义:\\
嵌入式系统是设计完成复杂功能的硬件和软件,并使其紧密耦合在一起的计算机系统。
\end{itemize}
\end{frame}


\begin{frame}{嵌入式系统特点}
\begin{description}
\item [硬件特性:]
\begin{itemize}
\item 体积小,集成效率高
\item 面向特定应用
\item 低功耗
\end{itemize}
\item[软件特性:]
\begin{itemize}
\item 与硬件紧密相关
\item 代码效率高、高可靠性
\item 一般固化于Flash/ROM中
\end{itemize}
\end{description}
\end{frame}

\begin{frame}{嵌入式器件}
\begin{description}
\item [嵌入式微处理器:]
\begin{itemize}
\item PowerPC
\item ARM
\end{itemize}
\item[嵌入式微控制器:]
\begin{itemize}
\item 8051
\item MSP430
\end{itemize}
\item[其它]
\begin{itemize}
\item DSP(Digital Signal Processor)
\item FPGA/CPLD
\item ASIC
\item 片上系统(SoC)
\end{itemize}
\item[开发工具]
\begin{itemize}
\item HDL
\item EDA
\end{itemize}
\end{description}
\end{frame}

\begin{frame}{单片机(SCC: Single Chip Computer)/微控制器(MCU: Micro Controler Unit)}
\begin{center}\pgfimage[width=0.9\columnwidth]{image/mcu}\end{center}
\end{frame}

\begin{frame}{}
\begin{center}\pgfimage[width=0.9\columnwidth]{image/control1}\end{center}
\end{frame}
\begin{frame}{}
\begin{center}\pgfimage[width=0.9\columnwidth]{image/0119_2103}\end{center}
\end{frame}
\begin{frame}{}
\begin{center}\pgfimage[width=0.9\columnwidth]{image/0119_2104}\end{center}
\end{frame}
\begin{frame}{}
\begin{center}\pgfimage[width=0.9\columnwidth]{image/0119_2105}\end{center}
\end{frame}
\begin{frame}{}
\begin{center}\pgfimage[width=0.9\columnwidth]{image/0120_2137}\end{center}
\end{frame}
\begin{frame}{}
\begin{center}\pgfimage[width=0.9\columnwidth]{image/0120_2137_01}\end{center}
\end{frame}

\begin{frame}{单板机(SBC: Single Board Computer)}
\begin{center}\pgfimage[width=0.9\columnwidth]{image/sbc}\end{center}
\end{frame}

\begin{frame}{}
\begin{center}\pgfimage[width=0.9\columnwidth]{image/0330_2051}\end{center}
\end{frame}

\begin{frame}{OMAP3530}
\begin{center}\pgfimage[width=0.9\columnwidth]{image/omap3530}\end{center}
\end{frame}

\section{嵌入式系统应用}

\begin{frame}{嵌入式系统应用}
\begin{description}
\item [日常生活:]
\begin{itemize}
\item 机顶盒
\item 手机/PDA(Personal Digital Assistant)
\item 微波炉
\item 汽车
\item 娱乐视频
\item 校园报警器
\end{itemize}
\item[工业:]
\begin{itemize}
\item 过程监控
\item 分布式系统
\item 自动化生产线
\end{itemize}
\item[科学研究:]
\begin{itemize}
\item 国际空间站
\item 航空/航天飞行器
\item 机器人/探测器
\end{itemize}
\end{description}
\end{frame}


%生活
\begin{frame}{}
\begin{center}\pgfimage[width=0.9\columnwidth]{image/alarm}\end{center}
\end{frame}
\begin{frame}{}
\begin{center}
\pgfimage[width=0.5\columnwidth]{image/fedoracore2-1}
\pgfimage[width=0.5\columnwidth]{image/fedoracore2-2}
\end{center}
\end{frame}
\begin{frame}{}
\begin{center}\pgfimage[width=0.9\columnwidth]{image/monitor}\end{center}
\end{frame}
\begin{frame}{}
\begin{center}\pgfimage[width=0.9\columnwidth]{image/video}\end{center}
\end{frame}
\begin{frame}{}
\begin{center}\pgfimage[width=0.9\columnwidth]{image/control2}\end{center}
\end{frame}
\begin{frame}{}
\begin{center}\pgfimage[width=0.9\columnwidth]{image/control3}\end{center}
\end{frame}



\begin{frame}{Consumer Electronics}
\begin{center}\pgfimage[width=0.9\columnwidth]{image/consumer}\end{center}
\end{frame}

\begin{frame}{基于RTLinux的机器人}
\begin{center}\pgfimage[width=0.9\columnwidth]{image/robot}\end{center}
\end{frame}

%航天
\begin{frame}{Spirit/Opportunity}
\begin{center}\pgfimage[width=0.9\columnwidth]{image/spirit1}\end{center}
\end{frame}
\begin{frame}{Spirit/Opportunity}
\begin{center}\pgfimage[width=0.9\columnwidth]{image/spirit2}\end{center}
\end{frame}
\begin{frame}{Spirit/Opportunity}
\begin{center}\pgfimage[width=0.9\columnwidth]{image/spirit3}\end{center}
\end{frame}


\section{思考}
\begin{frame}{思考}
\begin{itemize}
\item 计算机应用领域发生了哪些变化?
\item 嵌入式系统有哪些特点,有哪些应用?
\item 单片机在嵌入式系统中有哪些作用?
\item 操作系统在嵌入式系统中有哪些作用?
\end{itemize}
\end{frame}


%\end{CJK*}
\end{document}
