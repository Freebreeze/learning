\begin{document}

\def\lecturename{嵌入式技术}

\title{\insertlecture}

\author{邢超}

\institute
{
  西北工业大学航天学院
}

%\mode<presentation>{\subject{嵌入式系统}}

%  start a lecture  --------------------------
\lecture[EC]{嵌入式技术}{}
\subtitle{混合语言程序设计}
\date{2015}


%\setbeamertemplate{background}{\pgfimage[width=\paperwidth,height=\paperheight]{image/flower}}
%\setbeamercovered{transparent}
%\mode<presentation>{\beamerdefaultoverlayspecification{<+->}}

\begin{frame}
  \maketitle
\end{frame}


\section{计算机程序设计语言}
\begin{frame}{程序设计发展}
\begin{itemize}
\item 纸带
\item 汇编
\item 高级语言
\begin{itemize}
\item 函数式语言(Functional language):
\begin{itemize}
\item Lisp ( LISt Processing )
\item Haskell
\item Caml ( Objective Caml )
\end{itemize}
\item 命令式语言(Imperative language):
\begin{itemize}
\item Fortran
\item C
\item Pascal
\end{itemize}
\item 脚本语言(Descript language):
\begin{itemize}
\item HTML
\item Javascript
\item Postscript
\end{itemize}
\end{itemize}
\end{itemize}
\end{frame}

\begin{frame}{程序设计语言列表}
\begin{center}\pgfimage[width=0.9\columnwidth]{image/ComputerLanguagesChart}\end{center}
\end{frame}

\begin{frame}{混合语言组合与通信}
\begin{itemize}
\item 组合方式
\begin{itemize}
\item 多个程序
\item 单个程序
\end{itemize}
\item 通信方式
\begin{itemize}
\item 文件
\item 管道
\item 网络
\item 共享库
\end{itemize}
\end{itemize}
\end{frame}



\begin{frame}{数据类型与编码}
\begin{itemize}
\item IEEE 754 float
\item Big/Little Endian
\item 数组
\end{itemize}
\end{frame}


\section{混合语言编程}
\begin{frame}{混合语言编程类型}
\begin{itemize}
\item 基础
\begin{itemize}
\item C/C++
\item Java
\end{itemize}
\item 扩展/嵌入
\begin{itemize}
\item 汇编
\item Fortran
\item MATLAB/SIMULINK
\item Scilab/Scicos
\item Lua, Python, Perl
\item Lisp, Scheme, Ocaml
\end{itemize}
\end{itemize}
\end{frame}

\begin{frame}{语言平台}
\begin{center}
\pgfimage[height=0.2\textheight]{image/java}
\pgfimage[height=0.2\textheight]{image/dotnet}
\end{center}
\begin{itemize}
\item Java Platform
\begin{itemize}
\item Java
\item JNI
\end{itemize}
\item .Net Framework
\begin{itemize}
\item C\#
\item CLR
\end{itemize}
\end{itemize}
\end{frame}



\begin{frame}{C与Fortran}
\begin{itemize}
\item C调用Fortran过程
\item Fortran调用C函数
\end{itemize}
\end{frame}

\def\lstlistingname{例}
\begin{frame}[containsverbatim]{Fortran,main}
\lstset{language=fortran}
\begin{lstlisting}
program main
  implicit none
  integer i,j,M,N;
  integer, external:: r2;
  real A(3,3),B(3,3),C(5,3);
  real, allocatable,dimension(:,:)::D;

  A=1;  B=2;  B(1:3:2,1:3:2)=3;
  B(3:2:-1,2:1:-1)=A(1:2,2:3);
  B(int(A(:,1)),1)=(/4,5,6/);
  where(B==3)
     B=5
  endwhere
  C= reshape((/ ((sin(PI/M)*cos(PI/N), 
     M=1,5), N=1,3) /),(/5,3/));
  allocate (D(size(C,1),size(C,2)));
  D= reshape((/ ((M+N*10, M=1,5), N=1,3) /),(/5,3/));
  print *, D;
  write(*,*) r2(1)
end program main
\end{lstlisting}
\end{frame}

\begin{frame}[containsverbatim]{Fortran, subroutine}
\lstset{language=fortran}
\begin{lstlisting}
recursive integer function r2(i) result(r)
integer i
integer r
write(*,*) 'recurseive...',i;
r=r2(i+1);
end function

subroutine s(a)
integer a
a=a+1;
end subroutine
\end{lstlisting}
\end{frame}

\def\lstlistingname{例}
\begin{frame}[containsverbatim]{Fortran调用C函数}
\lstset{language=fortran}
\begin{lstlisting}
PROGRAM MAIN
use iso_c_binding
INTERFACE
  subroutine fact(n)  bind(C,name="Fact")
     INTEGER(4) n(2,2)
   END subroutine
END INTERFACE
INTEGER(4) n(2,2)
write(*,*) 'before : ' , N 
call fact(N)
write(*,*) 'after : ' , N 
END
\end{lstlisting}
\end{frame}

\begin{frame}[containsverbatim]{Fortran调用C函数}
\lstset{language=C}
\begin{lstlisting}
void Fact(int a[2][2])
{
 	int i,j;
  /*
  for(i=0;i<4;i++)  *(a[0]+i)=i;
  */
  for(i=0;i<2;i++)for(j=0;j<2;j++) a[i][j]=i*2+j;
  /*
  for(i=0;i<2;i++)for(j=0;j<2;j++) *(a[0]+i*2+j)=1;
  */
}
\end{lstlisting}
\end{frame}


\begin{frame}{脚本语言编程}
\begin{itemize}
\item 扩展
\begin{itemize}
\item 速度
\item 系统调用
\end{itemize}
\item 嵌入
\begin{itemize}
\item 灵活
\item 方便
\end{itemize}
\end{itemize}
\end{frame}

\begin{frame}{C与Lua}
\begin{itemize}
\item C调用Lua函数
\item 回调函数的实现
\end{itemize}
\end{frame}



\begin{frame}[containsverbatim]{C调用Lua}
\begin{lstlisting}
int main (void){
  char buff[256];
  int width,height;
  lua_State *L = lua_open();
  luaL_openlibs(L);
  luaL_loadfile(L,"lua.txt");
  lua_getglobal(L, "width");
  lua_getglobal(L, "height");
  width = (int)lua_tonumber(L, -2);
  height = (int)lua_tonumber(L, -1);
  printf("width is %d,height is %d.\n",width,height);
  lua_getglobal(L, "f"); /* function to be called */
  lua_pushnumber(L, 1.0);
  lua_pushnumber(L, 2.0);
  lua_pcall(L, 2, 2, 0);
  printf("double: %f %f \n ", 
      lua_tonumber(L, -2), lua_tonumber(L, -1)); 
  lua_close(L);
  return 0;
}
\end{lstlisting}
\end{frame}


\begin{frame}[containsverbatim]{Lua程序lua.txt}
\lstset{language=c}
\begin{lstlisting}
width = 200
height = 300

function f(x1,x2)
return x1,x2
end
\end{lstlisting}
\end{frame}


\begin{frame}[containsverbatim]{Callback}
\lstset{language=c}
\begin{lstlisting}
static int f(lua_State *L){
  double a = lua_tonumber(L, -1);
  lua_pushnumber(L, a);
  return 0;
}
int main(void){
    lua_State *L=lua_open();
    lua_register(L,"f",f);
    double a = 1;
    char *p = "f(a)";
    lua_pushnumber(L,a);
    lua_setglobal(L,"a");
    luaL_loadstring(L,p);
    lua_pcall(L,1,1,0);
    lua_close(L);
    return 0;
}
\end{lstlisting}
\end{frame}


\begin{frame}{C与Scheme}
\begin{itemize}
\item C调用Scheme函数
\item 回调函数的实现
\end{itemize}
\end{frame}

\begin{frame}[containsverbatim]{Tinyscheme}
\lstset{language=c}
\begin{lstlisting}
#include <stdio.h>
#include <stdlib.h>
#include "dynload.h"

int main(int argc, char *argv[])
{
        scheme scmenv;
        scheme_init(&scmenv);
        scheme_set_output_port_file(&scmenv, stdout);
        scheme_load_string(&scmenv,
"(display (+ 1 2 3 4 5 6))(newline)" );
        scheme_deinit(&scmenv);
        exit(EXIT_SUCCESS);
}
\end{lstlisting}
\end{frame}

\begin{frame}[containsverbatim]{Foreign Functions}
\begin{lstlisting}
pointer square(scheme *sc, pointer args) {
 if(args!=sc->NIL) {
     if(sc->isnumber(sc->pair_car(args))) {
          double v=sc->rvalue(sc->pair_car(args));
          return sc->mk_real(sc,v*v);
     }
 }
 return sc->NIL;
}
\end{lstlisting}

Foreign functions are defined as closures: 

\begin{lstlisting}
sc->interface->scheme_define( 
     sc, 
     sc->global_env, 
     sc->interface->mk_symbol(sc,"square"), 
     sc->interface->mk_foreign_func(sc, square)); 
\end{lstlisting}
\end{frame}

\begin{frame}{Simplified Wrapper and Interface Generator (SWIG)}
\begin{itemize}
\item Building more powerful C/C++ programs. % Using SWIG, you can replace the main() function of a C program with a scripting interpreter from which you can control the application. This adds quite a lot of flexibility and makes the program "programmable." That is, the scripting interface allows users and developers to easily modify the behavior of the program without having to modify low-level C/C++ code. The benefits of this are numerous. In fact think of all of the large software packages that you use every day---nearly all of them include special a macro language, configuration language, or even a scripting engine that allows users to make customizations.
\item Rapid prototyping and debugging.%SWIG allows C/C++ programs to be placed in a scripting environment that can be used for testing and debugging. For example, you might test a library with a collection of scripts or use the scripting interpreter as an interactive debugger. Since SWIG requires no modifications to the underlying C/C++ code, it can be used even if the final product does not rely upon scripting.
\item Systems integration. %Scripting languages work fairly well for controlling and gluing loosely-coupled software components together. With SWIG, different C/C++ programs can be turned into scripting language extension modules. These modules can then be combined together to create new and interesting applications.
\item Construction of scripting language extension modules. %SWIG can be used to turn common C/C++ libraries into components for use in popular scripting languages. Of course, you will still want to make sure that no-one else has already created a module before doing this.
\end{itemize}
\end{frame}

\begin{frame}{Compared with interface definition language (IDL)}
\begin{itemize}
\item ANSI C/C++ syntax. %SWIG parses ANSI C++ that has been extended with a number of special directives. As a result, interfaces are usually built by grabbing a header file and tweaking it a little bit. This particular approach is especially useful when the underlying C/C++ program undergoes frequent modification.
\item SWIG is not a stub generator. %SWIG produces code that you simply compile and run. You don't have to fill in any stubs or write special client/server code as you do with RPC-like systems.
\item SWIG does not define a protocol nor is it a component framework. %SWIG does not define mechanisms or enforce rules regarding the way in which software components are supposed to interact with each other. Nor is it a specialized runtime library or alternative scripting language API. SWIG is merely a code generator that provides the glue necessary to hook C/C++ to other languages.
\item Designed to work with existing C/C++ code. %SWIG requires little, if any, modifications to existing code. For the most part, it encourages you to keep a clean separation between C/C++ and its scripting interface.
\item Extensibility. %SWIG provides a variety of customization options that allow you to blow your whole leg off if that's what you want to do. SWIG is not here to enforce programming morality.
\end{itemize}
\end{frame}


\begin{frame}[containsverbatim]{C file}
%So you want to get going in a hurry? To illustrate the use of SWIG, suppose you have some C functions you want added to Tcl, Perl, Python, Java and C#. Specifically, let's say you have them in a file 'example.c'
\begin{lstlisting}
 /* File : example.c */
 #include <time.h>
 double My_variable = 3.0;
 int fact(int n) {
     if (n <= 1) return 1;
     else return n*fact(n-1);
 }
 int my_mod(int x, int y) {
     return (x%y);
 }
 char *get_time()
 {
     time_t ltime;
     time(&ltime);
     return ctime(&ltime);
 }
\end{lstlisting}
\end{frame}
\begin{frame}[containsverbatim]{Interface file}
%Now, in order to add these files to your favorite language, you need to write an "interface file" which is the input to SWIG. An interface file for these C functions might look like this :
\begin{lstlisting}
 /* example.i */
 %module example
 %{
 /*Put header files here or function declarations*/
 extern double My_variable;
 extern int fact(int n);
 extern int my_mod(int x, int y);
 extern char *get_time();
 %}
 
 extern double My_variable;
 extern int fact(int n);
 extern int my_mod(int x, int y);
 extern char *get_time();
\end{lstlisting} 
\end{frame}

\begin{frame}[containsverbatim]{Building a Tcl module}
%At the UNIX prompt, type the following (shown for Linux, see the SWIG Wiki Shared Libraries page for help with other operating systems):
\begin{lstlisting}
 unix % swig -tcl example.i
 unix % gcc -fpic -c example.c example_wrap.c \
        -I/usr/local/include 
 unix % gcc -shared example.o example_wrap.o \
        -o example.so
 unix % tclsh
 % load ./example.so example
 % puts $My_variable
 3.0
 % fact 5
 120
 % my_mod 7 3
 1
 % get_time
 Sun Feb 11 23:01:07 1996
 % 
\end{lstlisting}

%The swig command produces a file example_wrap.c that should be compiled and linked with the rest of the program. In this case, we have built a dynamically loadable extension that can be loaded into the Tcl interpreter using the 'load' command.
Building a Python module

\end{frame} 

\begin{frame}[containsverbatim]{Python module}
%Turning C code into a Python module is also easy. Simply do the following (shown for Irix, see the SWIG Wiki Shared Libraries page for help with other operating systems):
\begin{lstlisting} 
 unix % swig -python example.i
 unix % gcc -c example.c example_wrap.c \
        -I/usr/local/include/python2.1
 unix % ld -shared example.o example_wrap.o \
          -o _example.so 
\end{lstlisting} 
We can now use the Python module as follows :
\begin{lstlisting} 
 >>> import example
 >>> example.fact(5)
 120
 >>> example.my_mod(7,3)
 1
 >>> example.get_time()
 'Sun Feb 11 23:01:07 1996'
 >>>
\end{lstlisting} 
\end{frame}

\begin{frame}[containsverbatim]{Building a Perl module}
%You can also build a Perl5 module as follows (shown for Solaris, see the SWIG Wiki Shared Libraries page for help with other operating systems):

\begin{lstlisting} 
 unix % swig -perl5 example.i
 unix % gcc -c example.c example_wrap.c \
        `perl -MExtUtils::Embed -e ccopts`
 unix % ld -G example.o example_wrap.o \
         -o example.so
 unix % perl
 use example;
 print $example::My_variable,"\n";
 print example::fact(5),"\n";
 print example::get_time(),"\n";
 <ctrl-d>
 3.0
 120
 Sun Feb 11 23:01:07 1996
 unix % 
\end{lstlisting}
\end{frame}

\section{思考}
\begin{frame}{思考}
\begin{itemize}
\item 混合语言程序设计有哪些方法?
\item 混合语言程序设计有哪些优缺点?
\end{itemize}
\end{frame}

\end{document}
