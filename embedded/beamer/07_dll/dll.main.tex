\begin{document}

\def\lecturename{嵌入式技术}

\title{\insertlecture}

\author{邢超}

\institute
{
  西北工业大学航天学院
}

%\mode<presentation>{\subject{嵌入式系统}}

%  start a lecture  --------------------------
\lecture[EC]{嵌入式技术}{}
\subtitle{静态库与共享库}
\date{2015}


%\setbeamertemplate{background}{\pgfimage[width=\paperwidth,height=\paperheight]{image/flower}}
%\setbeamercovered{transparent}
%\mode<presentation>{\beamerdefaultoverlayspecification{<+->}}

\begin{frame}
  \maketitle
\end{frame}


\section{简介}
\begin{frame}[containsverbatim]{程序生成过程}
\begin{itemize}
\item 编辑(源文件)
\item 编译(汇编文件,中间代码文件)
\lstset{language=c++}
\begin{lstlisting}
gcc -S hello.cpp
\end{lstlisting}
\item 汇编(目标文件,机器指令码文件)
\lstset{language=c++}
\begin{lstlisting}
gcc -c hello.cpp
\end{lstlisting}
\item 链接(可执行文件)
\lstset{language=c++}
\begin{lstlisting}
gcc hello.cpp
\end{lstlisting}
\end{itemize}
\end{frame}

\begin{frame}[containsverbatim]{链接}
\begin{itemize}
\item 静态链接
\lstset{language=c++}
\begin{lstlisting}
\end{lstlisting}
\item 动态链接
\lstset{language=c++}
\begin{lstlisting}
\end{lstlisting}
\end{itemize}
\end{frame}

\section{静态库/动态库生成}

\begin{frame}[containsverbatim]{源程序}
\lstset{language=c++}
\begin{lstlisting}
/******  say.c   *******/
#include "stdio.h"
void say()
{
	printf("Say!");
}
/****** test.c  *******/
#include "stdio.h"
void say();
main(){
	say();
}
\end{lstlisting}
\end{frame}

\begin{frame}[containsverbatim]{命令}
\lstset{language=ksh}
\begin{lstlisting}
gcc -c say.c
ar -r say.a say.o
gcc test.c say.a -o test
ldd test
gcc -fPIC -shared say.c -o say.so
gcc test.c ./say.so -o test
ldd test
\end{lstlisting}
\end{frame}


\section{动态库使用}
 
\begin{frame}[containsverbatim]{动态库位置}
\begin{itemize}
\item 
\lstset{language=c++}
\begin{lstlisting}
LD_LIBRARY_PATH
\end{lstlisting}
\item 
/etc/ld.so.cache
\begin{itemize}
\item ldconfig
\item /etc/ld.so.conf
\end{itemize}
\item 默认路径
\begin{itemize}
\item /lib
\item /usr/lib
\end{itemize}
\end{itemize}
\end{frame}

\begin{frame}[containsverbatim]{动态加载}
\lstset{language=c++,escapeinside=``}
\begin{lstlisting}
#include <stdlib.h>
#include <stdio.h>
#include <dlfcn.h>
int main(int argc,char **argv){
  void *handle;     char *error;
  double (*cosine)(double);
  handle = dlopen("/lib/libm.so.6",RTLD_LAZY);
  if(!handle){printf("%s\n",dlerror());exit(1);}
  printf("opened /lib/libm.so.6\n");
  cosine = dlsym(handle,"cos");
  if((error = dlerror())!=NULL){
    printf("%s\n",error);dlclose(handle);
    printf("closed /lib/libm.so.6\n");exit(1);}
  printf("%f\n",(*cosine)(2.0));
  dlclose(handle);
  printf("closed /lib/libm.so.6\n");
  return 0;}
/* gcc -o test test.c -ldl
/lib/libm.so.6`是动态加载库`
/usr/lib/libdl.so`是共享库` */
\end{lstlisting}
\end{frame}


%\section{静态库/动态库转换}

\section{思考}
\begin{frame}{思考}
\begin{itemize}
\item 静态库与共享库/动态库在程序设计中有哪些作用?
\item 利用GNU/Linux中静态库与共享/动态库设计一个程序。
\end{itemize}
\end{frame}


\end{document}
