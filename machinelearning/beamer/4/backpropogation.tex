\documentclass{article}
 \usepackage[BoldFont,SlantFont,CJKchecksingle]{xeCJK}
\setCJKmainfont[BoldFont=Evermore Hei]{Evermore Kai}

\usepackage{amsmath}


\begin{document}

\title{误差反传算法}
\author{邢超(xingnix@live.com)}
\maketitle


\section{ $\Delta W$ 计算}
% 利用梯度推导反传公式的过程中可以采用小扰动方法简化推导步骤

误差函数: 
\[E=(\vec{Y_E}-\vec{Y_o})^T(\vec{Y_E}-\vec{Y_o})\]



每层输出:  
\begin{align*}
\vec{X_o} &=F(W\vec X_i+\vec b)\\
&=\begin{bmatrix}
f(\vec w_1 \vec X_i+b_1)  &  f(\vec w_2 x_i+b_2) & \cdots & f(\vec w_n \vec X_i +b_n)
\end{bmatrix}^T
\end{align*}
其中 $\vec w_i$ 为 W 矩阵的第i行。

对于W的一个微小变化 $\Delta W$ ,得:
\begin{align*}
\Delta \vec X_o &=\begin{bmatrix}
f'(\vec w_1 \vec X_i+b_1)\Delta \vec w_1 \vec X_i  &  f'(\vec w_2 \vec X_i+b_2)\Delta \vec w_2 \vec X_i & \cdots & f'(\vec w_n \vec X_i +b_n)\Delta \vec w_n \vec X_i
\end{bmatrix}^T \\
&=diag[F'(W\vec{X_i}+\vec{b})]\Delta W\vec X_i \\
&=diag(F')\Delta W \vec X_i
\end{align*}


其中 $diag(F)$ 表示主对角元素为向量 F的元素的方阵。

误差函数的增量:
\begin{align*}
\Delta E &=\Delta E =-2(\vec Y_E-\vec Y_o)^T\cdot diag(F')\Delta (W \vec X_o+b)\\
& =-2(\vec Y_E-\vec Y_o)^T\cdot diag(F')\Delta W \vec X_o
\end{align*}




\section{反向传播}
根据链式法则:
\begin{align*}
\Delta \vec X_o&=diag(F')\Delta (W \vec X_i+\vec b)\\
&=diag(F')W \Delta\vec X_i\\
\Delta E &= 2(\vec{Y_o}-\vec{Y_E})^T\cdot diag(F') \cdot W \cdot \Delta \vec X_o\\
&=2(\vec{Y_o}-\vec{Y_E})^T\cdot diag(F') \cdot W \cdot diag(F')\cdot \Delta W\cdot \vec{X_i}
\end{align*}

对于多层网络权值的更新:
\begin{align*}
\Delta E &= \vec{a}^T\cdot\Delta W\cdot \vec{b} \\
\frac{\partial E}{\partial W} &= \vec{a}\cdot\vec b^T
\end{align*}
其中:
\begin{align*}
\vec{a}^T&=2(\vec Y_o-\vec Y_E)^T\cdot \prod_{n=1}^{N}[diag(F') \cdot W ]\cdot diag(F') \qquad (n=0,1,2,\ldots)\\
\vec{b}&=\vec{X_i}\qquad (\text{ inputs of the layer})
\end{align*}


\section{偏置向量}
对偏置向量的更新可考虑增广权植矩阵 $[W|b]$ ,增广输入 $\begin{bmatrix} \vec X_o  \\ 1\end{bmatrix}$ 。令 $b=\begin{bmatrix} \vec X_o \\ 1\end{bmatrix}$ 即可得到增广权值矩阵的更新公式。













\end{document}
