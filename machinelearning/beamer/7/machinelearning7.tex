% Created 2016-10-10 Mon 21:05
\documentclass{beamer}
\usepackage{fixltx2e}
\usepackage{graphicx}
\usepackage{longtable}
\usepackage{float}
\usepackage{wrapfig}
\usepackage{soul}
\usepackage{textcomp}
\usepackage{marvosym}
\usepackage{wasysym}
\usepackage{latexsym}
\usepackage{amssymb}
\usepackage{hyperref}
\tolerance=1000
\usepackage{etex}
\usepackage{amsmath}
\usepackage{pstricks}
\usepackage{pgfplots}
\usepackage{tikz}
\usepackage[europeanresistors,americaninductors]{circuitikz}
\usepackage{colortbl}
\usepackage{yfonts}
\usetikzlibrary{shapes,arrows}
\usetikzlibrary{positioning}
\usetikzlibrary{arrows,shapes}
\usetikzlibrary{intersections}
\usetikzlibrary{calc,patterns,decorations.pathmorphing,decorations.markings}
\usepackage[BoldFont,SlantFont,CJKchecksingle]{xeCJK}
\setCJKmainfont[BoldFont=Evermore Hei]{Evermore Kai}
\setCJKmonofont{Evermore Kai}
\usepackage{pst-node}
\usepackage{pst-plot}
\psset{unit=5mm}
\mode<beamer>{\usetheme{Frankfurt}}
\mode<beamer>{\usecolortheme{dove}}
\mode<article>{\hypersetup{colorlinks=true,pdfborder={0 0 0}}}
\AtBeginSection[]{\begin{frame}<beamer>\frametitle{Topic}\tableofcontents[currentsection]\end{frame}}
\setbeamercovered{transparent}
\providecommand{\alert}[1]{\textbf{#1}}

\title{计算学习理论}
\author{}
\date{}
\hypersetup{
  pdfkeywords={},
  pdfsubject={},
  pdfcreator={Emacs Org-mode version 7.9.3f}}

\begin{document}

\maketitle

\begin{frame}
\frametitle{Outline}
\setcounter{tocdepth}{3}
\tableofcontents
\end{frame}









\section{简介}
\label{sec-1}
\begin{frame}
\frametitle{简介}
\label{sec-1-1}

\begin{itemize}
\item 学习器(机器的或非机器的)应遵循什么样的规则?
\item 学习器所考虑的假设空间的大小和复杂度
\item 目标概念须近似到怎样的精度
\item 学习器输出成功的假设的可能性
\item 训练样例提供给学习器的方式
\end{itemize}

目标是为了回答以下的问题:
\begin{itemize}
\item 样本复杂度(Sample complexity)。学习器要收敛到成功假设(以较高的概率),需要多少训练样例?
\item 计算复杂度(Computational complexity)。学习器要收敛到成功假设(以较高的概率)需要多大的计算量?
\item 出错界限(Mistake bound)。在成功收敛到一个假设前,学习器对训练样例的误分类有多少次?
\end{itemize}
\end{frame}
\begin{frame}
\frametitle{概念学习任务}
\label{sec-1-2}

已知:
\begin{itemize}
\item 实例集合 $X$: 实例以其属性表示:\\
\{\em Sky, AirTemp, Humidity, Wind, Water, Forecast\}
\item 目标函数 $c$: $EnjoySport: X \rightarrow \{0,1 \}$
\item 假设空间 $H$: 布尔文字合取,如
    $$\langle ?, Cold, High, ?, ?, ? \rangle.$$
\item 训练样例集合 $D$: 目标函数的正/负样例
    $$\langle x_1, c(x_1) \rangle , \ldots \langle x_m, c(x_m) \rangle$$
\end{itemize}

求:
\begin{itemize}
\item $H$ 中的假设 $h$ , 使 $D$ 中的所有 $x$ 满足 $h(x)=c(x)$
\item $H$ 中的假设 $h$ , 使 $X$ 中的所有 $x$ 满足 $h(x)=c(x)$
\end{itemize}
\end{frame}
\section{样本复杂度}
\label{sec-2}
\begin{frame}
\frametitle{样本复杂度}
\label{sec-2-1}

需要多少样例才能学习目标概念?

\begin{itemize}
\item 学习器提出实例,询问施教者
\begin{itemize}
\item 学习器提出实例 $x$, 施教者提供 $c(x)$
\end{itemize}
\item 施教者提供训练样例
\begin{itemize}
\item 施教者提供样例序列: $\langle x, c(x) \rangle$
\end{itemize}
\item 以随机过程提出实例
\begin{itemize}
\item 随机产生实例 $x$ , 施教者提供 $c(x)$
\end{itemize}
\end{itemize}
\end{frame}
\begin{frame}
\frametitle{样本复杂度 1}
\label{sec-2-2}


学习器提出实例 $x$, 施教者提供 $c(x)$

(假定 $c$ 在学习器的假设空间 $H$ 中)

最优询问策略: play 20 questions
\begin{itemize}
\item 选取实例 $x$ , 使得 $VS$ 中一半的假设将 $x$ 分类为正,一半分类为负
\begin{itemize}
\item 可行时,需要 $\lceil \log_2 |H| \rceil$ 次查询可学习 $c$
\item 不可行时,需要更多样例
\end{itemize}
\end{itemize}
\end{frame}
\begin{frame}
\frametitle{样本复杂度 2}
\label{sec-2-3}


施教者提供样例序列: $\langle x, c(x) \rangle$

(假定 $c$ 在学习器的假设空间 $H$ 中)

最优施教策略: 依赖学习器使用的 $H$

\begin{itemize}
\item 考虑 $H=$ 最多 $n$ 个布尔文字及其否定的合取
\begin{itemize}
\item 如: $(AirTemp = Warm) \land (Wind = Strong)$ , 其中 $AirTemp, Wind,\ldots$ 各有两个可能取值。
\end{itemize}
\item 若 $H$ 中有 $n$ 个可能布尔属性 $H$, 只需 $n + 1$ 个样例
\end{itemize}
\end{frame}
\begin{frame}
\frametitle{样本复杂度 3}
\label{sec-2-4}


已知:
\begin{itemize}
\item 实例集合 $X$
\item 假设集合 $H$
\item 目标概念集合 $C$
\item 训练实例由一个 $X$ 上的固定但未知的概率分布  $\cal{D}$ 产生
\end{itemize}

针对一些目标概念 $c \in C$ , 学习器观察到训练样例(形如 $\langle x, c(x) \rangle$ )序列 $D$  , 
\begin{itemize}
\item 实例 $x$ 从分布  $\cal{D}$ 中抽取
\item 施教者提供 $c(x)$
\end{itemize}


学习器必须输出一个假设 $h$ 来估计 $c$
\begin{itemize}
\item 在后续按 $\cal{D}$ 抽取的实例上评估 $h$ 的性能
\end{itemize}


注意: 随机抽取实例,类别无噪声
\end{frame}
\section{假设的错误率}
\label{sec-3}
\begin{frame}
\frametitle{假设的错误率}
\label{sec-3-1}

\includegraphics[width=.9\linewidth]{./image/pac-err.png}
\end{frame}
\begin{frame}
\frametitle{假设的错误率}
\label{sec-3-2}

\begin{quote}
定义: 假设 $h$ 关于目标概念 $c$ 和分布 $\cal{D}$ 的真实错误率(true error)为 $h$ 误分类实例(按 $\cal{D}$ 随机抽取)的概率。
$$error_{\cal{D}}(h) \equiv \Pr_{x \in \cal{D}}[c(x) \neq h(x)]$$
\end{quote}

这里符号 $\Pr_{x \in \cal{D}}$ 代表在实例分布 $\cal{D}$ 上计算概率。
\end{frame}
\begin{frame}
\frametitle{两种错误率}
\label{sec-3-3}


\begin{itemize}
\item 针对目标概念 $c$ , 假设 $h$ 的训练错误率
\begin{itemize}
\item 在训练实例中 $h(x) \neq c(x)$ 的比例
\end{itemize}
\item 针对目标概念 $c$ , 假设 $h$ 的真实错误率
\begin{itemize}
\item 未来抽取的随机实例中 $h(x) \neq c(x)$ 的比例
\end{itemize}
\item 给定 $h$ 的训练错误率是否可确定 $h$ 的真实错误率的界限
\begin{itemize}
\item 先考虑 $h$ 的训练错误率为0的情况 (如, $h \in VS_{H,D}$ )
\end{itemize}
\end{itemize}
\end{frame}
\section{变型空间详尽化}
\label{sec-4}
\begin{frame}
\frametitle{变型空间详尽化}
\label{sec-4-1}

\includegraphics[width=.9\linewidth]{./image/pac-vs-exhausted.png}
\centerline{($r =$ training error, $error = $ true error)}
\end{frame}
\begin{frame}
\frametitle{变型空间详尽化}
\label{sec-4-2}

\begin{quote}
定义: 考虑一假设空间 $H$ ,目标概念 $c$ ,实例分布 $\cal{D}$ 以及 $c$ 的一组训练样例 $D$ 。
当 $VS_{H,D}$ 中每个假设h关于c和 $\cal{D}$ 错误率小于 $\epsilon$ 时,变型空间被称为关于 $c$ 和 $\cal{D}$ 是 $\epsilon$ - 详尽的( $\epsilon$ -exhausted)。
\end{quote}
\end{frame}
\begin{frame}
\frametitle{多少样例可使变型空间  $\epsilon$ - 详尽化}
\label{sec-4-3}


\{\bf Theorem:\} [Haussler, 1988].
\begin{quote}
变型空间的 $\epsilon$ -详尽化 ( $\epsilon$ -exhausting the version space): 若假设空间 $H$ 有限,且 $D$ 为目标概念 $c$ 的一系列 $m\geq 1$ 个独立随机抽取的样例,
那么对于任意 $0 \leq\epsilon \leq 1$ ,变型空间 $VS_{H,D}$ 不是 $\epsilon$ -详尽(关于 $c$ )的概率小于或等于:
$$|H|e^{-\epsilon m}$$
\end{quote}
\end{frame}
\begin{frame}
\frametitle{多少样例可使变型空间  $\epsilon$ - 详尽化}
\label{sec-4-4}

\begin{itemize}
\item 定理限定了任何一致学习器输出的假设 $h$ 满足 $error(h) \geq \epsilon$ 的概率。
\item 若要使此概率小于 $\delta$
  $$|H|e^{- \epsilon m} \leq \delta$$
  则
  $$m \geq \frac{1}{\epsilon}(\ln|H| + \ln(1/\delta))$$
\end{itemize}
\end{frame}
\begin{frame}
\frametitle{学习布尔文字的合取}
\label{sec-4-5}


\begin{itemize}
\item 需要多少样例足够保证至少以概率 $(1 - \delta)$ 使得
\begin{itemize}
\item 每个 $VS_{H,D}$ 中的 $h$ 满足 $error_{\cal D}(h) \leq \epsilon$
\end{itemize}
\item 使用定理:
 $$m \geq \frac{1}{\epsilon}(\ln|H| + \ln(1/\delta))$$
 假设空间 $H$ 定义为 $n$ 个布尔文字的合取,则假设空间 $H$ 的大小为 $|H| = 3^n$, 则
 $$m \geq \frac{1}{\epsilon}(\ln 3^n + \ln(1/\delta))$$
 或
 $$m \geq \frac{1}{\epsilon}(n \ln 3 + \ln(1/\delta))$$
\end{itemize}
\end{frame}
\begin{frame}
\frametitle{$EnjoySport$?}
\label{sec-4-6}



\[m \geq \frac{1}{\epsilon}(\ln|H| + \ln(1/\delta)) \]


若 $H$ 象 $EnjoySport$ 中那样, 则 $|H| = 973$, 且

$$m \geq \frac{1}{\epsilon}(\ln 973 + \ln(1/\delta))$$


若想保证以 95\% 的概率使变型空间 $VS$ 只包含 $error_{\cal D}(h) \leq .1$ 的假设, 需要 $m$ 个样例:

\[m \geq \frac{1}{.1}(\ln 973 + \ln(1/.05)) \]
\[m \geq 10 (\ln 973 + \ln 20) \]
\[m \geq 10 (6.88 + 3.00) \]
\[m \geq 98.8 \]
\end{frame}
\begin{frame}
\frametitle{PAC Learning}
\label{sec-4-7}


\begin{quote}
定义: 考虑一概念类别 $C$ 定义在长度为 $n$ 的实例集合 $X$ 上,学习器 $L$ 使用假设空间 $H$ 。
称 $C$ 是使用 $H$ 的 $L$ 可 PAC 学习的,
若对所有 $c\in C$ , $X$ 上的分布 $\cal D$ , $\epsilon$ 满足 $0<\epsilon<1/2$ ,以及 $\delta$ 满足 $0 < \delta < 1/2$ ,
学习器 $L$ 将以至少 $1-\delta$ 的概率输出一假设 $h\in H$ ,使 $error_{\cal D}(h) \leq \epsilon$ ,
所使用的时间为 $1/\epsilon$, $1/\delta$, $n$ 以及 $size(c)$ 的多项式函数。
\end{quote}
\end{frame}
\begin{frame}
\frametitle{Agnostic Learning}
\label{sec-4-8}


\begin{itemize}
\item 目前为止,假定 $c \in H$
\item 不可知学习设定:不假定 $c \in H$
\item 寻找具有最小训练错误率的 $h$
\item 此时的样本复杂度
   从 Hoeffding 边界
   $$Pr[error_{\cal D}(h) > error_D(h) + \epsilon] \leq e^{-2m\epsilon^{2}}$$
   求得
   $$m \geq \frac{1}{2 \epsilon^{2}}(\ln|H| + \ln(1/\delta))$$
\end{itemize}
\end{frame}
\begin{frame}
\frametitle{Shattering a Set of Instances}
\label{sec-4-9}


\begin{quote}
定义:将集合 $S$ 分成不相交的两个子集的划分称为集合 $S$ 的二分法划分(dichotomy)
\end{quote}

\bigskip

\begin{quote}
定义: 一实例集 $S$ 被假设空间 $H$ 拆散(shatter) ,当且仅当对 $S$ 的每个二分法划分,存在 $H$ 中的某假设与此划分一致。
\end{quote}
\end{frame}
\begin{frame}
\frametitle{Three Instances Shattered}
\label{sec-4-10}


\includegraphics[width=.9\linewidth]{./image/pac-shatter.png}
\end{frame}
\begin{frame}
\frametitle{Vapnik-Chervonenkis Dimension}
\label{sec-4-11}


\begin{quote}
定义: 定义在实例空间 $X$ 上的假设空间 $H$ 的 Vapnik-Chervonenkis 维 , 或 $VC(H)$ ,是可被 $H$ 拆散的 $X$ 的最大有限子集的大小。
如果 $X$ 的任意有限大的子集可被 $H$ 拆散,那么 $VC(H) \equiv \infty$ 。
\end{quote}
\end{frame}
\begin{frame}
\frametitle{VC Dim. of Linear Decision Surfaces}
\label{sec-4-12}


\includegraphics[width=.9\linewidth]{./image/pac-pts.png}
\end{frame}
\begin{frame}
\frametitle{Sample Complexity from VC Dimension}
\label{sec-4-13}


需要多少样例足够以至少 $(1 - \delta)$ 的概率 $\epsilon$ -详尽 $VS_{H,D}$ ?

$$m \geq \frac{1}{\epsilon}(4\log_2(2/\delta) + 8 VC(H) \log_2 (13/\epsilon))$$
\end{frame}
\section{出错界限 (Mistake Bounds)}
\label{sec-5}
\begin{frame}
\frametitle{出错界限 (Mistake Bounds)}
\label{sec-5-1}

\begin{itemize}
\item 已分析: 需要多少样本来学习?
\item 问题: 出多少次错误才能收敛?
\item 与PAC问题框架相同,考虑:
\begin{itemize}
\item 实例依分布 ${\cal D}$ 从 $X$ 中随机抽取
\item 学习器必须先预测目标值c(x),之后再由施教者给出正确的目标值。
\item 在学习器学习到目标概念前,它的预测会有多少次出错?
\end{itemize}
\end{itemize}
\end{frame}
\begin{frame}
\frametitle{Mistake Bounds: Find-S}
\label{sec-5-2}


考虑 Find-S 算法 , $H=$ 布尔文字合取

Find-S:
\begin{itemize}
\item 将 $h$ 被始化为最特殊假设 $l_{1} \land \neg l_{1} \land l_{2} \land \neg l_{2} \ldots l_{n} \land \neg l_{n}$
\item 对每个正例 $x$
\begin{itemize}
\item 从 $h$ 中移去任何不满足 $x$ 的文字
\end{itemize}
\item 输出假设 $h$
\end{itemize}

收敛到正确的 $h$ 前出错几次?
\end{frame}
\begin{frame}
\frametitle{Mistake Bounds: Halving Algorithm}
\label{sec-5-3}


考虑 Halving 算法:
\begin{itemize}
\item 使用变型空间候选消除(Candidate-Elimination)算法学习概念
\item 变型空间成员投票分类新样例
\end{itemize}

收敛到正确的 $h$ 前出错几次?
\begin{itemize}
\item $\lfloor \log_2 |H| \rfloor$
\item 0
\end{itemize}
\end{frame}
\begin{frame}
\frametitle{Optimal Mistake Bounds}
\label{sec-5-4}


对任意学习算法 $A$ 和任意目标概念 $c$ ,令 $M_{A}(C)$ 代表 $A$ 为了确切学到 $c$ , 在所有可能训练样例序列中出错的最大值。
现在对于任意非空概念类 $C$ ,令
$$M_{A}(C) \equiv \max_{c \in C} M_{A}(c)$$

定义: 令 $C$ 为任意非空概念类。$C$ 的最优出错界限(optimal mistake bound) $Opt(C)$ ,是所有可能学习算法 $A$ 中 $M_{A}(C)$ 的最小值。
$$Opt(C) \equiv \min_{A \in learning\  algorithms} M_{A}(C)$$

对任意概念类 $C$ , $C$ 的最优出错边界, Halving算法出错边界和 $C$ 的 VC 维之间关系:

$$VC(C) \leq Opt(C) \leq M_{Halving}(C) \leq log_{2}(|C|)$$
\end{frame}

\end{document}
