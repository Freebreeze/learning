% Created 2016-09-21 Wed 08:48
\documentclass{beamer}
\usepackage{fixltx2e}
\usepackage{graphicx}
\usepackage{longtable}
\usepackage{float}
\usepackage{wrapfig}
\usepackage{soul}
\usepackage{textcomp}
\usepackage{marvosym}
\usepackage{wasysym}
\usepackage{latexsym}
\usepackage{amssymb}
\usepackage{hyperref}
\tolerance=1000
\usepackage{etex}
\usepackage{amsmath}
\usepackage{pstricks}
\usepackage{pgfplots}
\usepackage{tikz}
\usepackage[europeanresistors,americaninductors]{circuitikz}
\usepackage{colortbl}
\usepackage{yfonts}
\usetikzlibrary{shapes,arrows}
\usetikzlibrary{positioning}
\usetikzlibrary{arrows,shapes}
\usetikzlibrary{intersections}
\usetikzlibrary{calc,patterns,decorations.pathmorphing,decorations.markings}
\usepackage[BoldFont,SlantFont,CJKchecksingle]{xeCJK}
\setCJKmainfont[BoldFont=Evermore Hei]{Evermore Kai}
\setCJKmonofont{Evermore Kai}
\usepackage{pst-node}
\usepackage{pst-plot}
\psset{unit=5mm}
\mode<beamer>{\usetheme{Frankfurt}}
\mode<beamer>{\usecolortheme{dove}}
\mode<article>{\hypersetup{colorlinks=true,pdfborder={0 0 0}}}
\AtBeginSection[]{\begin{frame}<beamer>\frametitle{Topic}\tableofcontents[currentsection]\end{frame}}
\setbeamercovered{transparent}
\providecommand{\alert}[1]{\textbf{#1}}

\title{评诂假设}
\author{}
\date{}
\hypersetup{
  pdfkeywords={},
  pdfsubject={},
  pdfcreator={Emacs Org-mode version 7.9.3f}}

\begin{document}

\maketitle

\begin{frame}
\frametitle{Outline}
\setcounter{tocdepth}{3}
\tableofcontents
\end{frame}










\section{动机}
\label{sec-1}
\begin{frame}
\frametitle{动机}
\label{sec-1-1}

\begin{itemize}
\item 当数据十分充足时,假设精度的估计相对容易。
\item 然而当给定的数据集非常有限时,要学习一个概念并估计其将来的精度,存在两个很关键的困难:
\begin{itemize}
\item 估计的偏差
\item 估计的方差
\end{itemize}
\end{itemize}
\end{frame}
\begin{frame}
\frametitle{估计的偏差(Bias in the estimate)。}
\label{sec-1-2}

\begin{itemize}
\item 学习到的概念在训练样例上的观察精度通常不能很好地用于估计在将来样例上的精度。
\item 因为假设是从这些样例中得出的,因此对将来样例的精度估计通常偏于乐观。
\item 尤其在学习器采用了很大的假设空间,并过度拟合训练样例时,这一情况更可能出现。
\item 要对将来的精度进行无偏估计,典型的方法是选择与训练样例和假设无关的检验样例,在这个样例集合上检验假设。
\end{itemize}
\end{frame}
\begin{frame}
\frametitle{估计的方差(Variance in the estimate)。}
\label{sec-1-3}

\begin{itemize}
\item 即使假设精度在独立的无偏检验样例上测量,得到的精度仍可能与真实精度不同,(取决于特定检验样例集合的组成)。
\item 检验样例越少,产生的方差越大。
\end{itemize}
\end{frame}
\section{估计假设精度}
\label{sec-2}
\begin{frame}
\frametitle{真实错误率}
\label{sec-2-1}

\begin{itemize}
\item 定义: 

  假设 $h$ 关于目标函数 $f$ 和分布 $D$ 的真实错误率(由 $error_D(h)$ 表示)(即:按分布 $D$ 随机抽取的实例在假设 $h$ 下被误分类的概率):
    $$error_{\cal{D}}(h) \equiv \Pr_{x \in \cal{D}}[f(x) \neq h(x)]$$

  这里,记号 $\Pr_{x \in \cal{D}}$ 表示概率在实例分布 $\cal{D}$ 上计算。
\end{itemize}
\end{frame}
\begin{frame}
\frametitle{样本错误率}
\label{sec-2-2}

\begin{itemize}
\item 定义: 
  假设 $h$ 关于目标函数 $f$ 和数据样本 $S$ 的样本错误率(标记为 $error_S(h)$ )为:
  \[error_{S}(h) \equiv \frac{1}{n}\sum_{x \in S}\delta(f(x) \neq h(x))\]

  其中 当 $f(x) \neq h(x)$ 时 $\delta(f(x) \neq h(x))$ 为 1 , 其它时为 0.
\end{itemize}
\end{frame}
\begin{frame}
\frametitle{偏差与方差}
\label{sec-2-3}


\begin{itemize}
\item 偏差 (Bias) :
\begin{itemize}
\item 当 $S$ 为训练集, $error_{S}(h)$ 为:   \#is optimistically biased
        \[bias \equiv E[ error_{S}(h)] - error_{\cal{D}}(h)\]
\item 对于无偏估计, $h$ 与 $S$ 必需独立选取。
\end{itemize}
\item 方差 (Variance)
\begin{itemize}
\item 即使对于无偏的 $S$ , $error_{S}(h)$ 仍与 $error_{\cal{D}}(h)$ 不同
\end{itemize}
\end{itemize}
\end{frame}
\begin{frame}
\frametitle{示例:}
\label{sec-2-4}


\begin{itemize}
\item 假设 $h$ 误分类 $s$ 中40个样例中的 12 个
  \[error_{S}(h) = \frac{12}{40} = 0.3\]
   $error_{\cal{D}}(h)$ ?
\item 估计量:
\begin{itemize}
\item 依概率 $\cal{D}$ 选取容量 $n$ 的样本 $s$
\item $error_{S}(h)$ 是随机变量
\item $error_{S}(h)$ 是 $error_{\cal{D}}(h)$ 的无偏估计
\end{itemize}
\end{itemize}
\end{frame}
\begin{frame}
\frametitle{离散值假设的置信区间}
\label{sec-2-5}

基于某离散值假设 $h$ 在样本 $S$ 上观察到的样本错误率,估计它的真实错误率,其中:
\begin{itemize}
\item 样本 $S$ 包含 $n$ 个样例,它们的抽取按照概率分布 $\cal D$ ,抽取过程是相互独立的,并且不依赖于 $h$
\item $n\geq 30$
\item 假设 $h$ 在这 $n$ 个样例上犯了 $r$ 个错误( $error_S(h)=r/n$ )
\end{itemize}
\end{frame}
\begin{frame}
\frametitle{已知这些条件,统计理论可给出以下断言:}
\label{sec-2-6}

\begin{itemize}
\item 没有其他信息的话, $error_{\cal{D}}(h)$ 最可能的值为 $error_S(h)$
\item 有大约 95\% 的可能性,真实错误率 $error_{\cal{D}}(h)$ 处于区间:
  $$error_{S}(h) \pm 1.96 \sqrt{\frac{error_{S}(h) (1 - error_{S}(h))}{n}}$$
\item 有大约 N\% 的可能性,真实错误率 $error_{\cal{D}}(h)$ 处于区间:
  $$error_{S}(h) \pm z_{N} \sqrt{\frac{error_{S}(h) (1 - error_{S}(h))}{n}}$$
    其中

\begin{center}
\begin{tabular}{lrrrrrrr}
 $N\%$:    &  50\%  &  68\%  &  80\%  &  90\%  &  95\%  &  98\%  &  99\%  \\
 $z_{N}$:  &  0.67  &  1.00  &  1.28  &  1.64  &  1.96  &  2.33  &  2.58  \\
\end{tabular}
\end{center}


\end{itemize}
\end{frame}
\section{采样理论基础}
\label{sec-3}
\begin{frame}
\frametitle{错误率估计和二项比例估计}
\label{sec-3-1}

\begin{itemize}
\item $error_{S}(h)$ 是一个随机变量。
\item 随机抽取容量为 $n$ 的样本 $s$ , 观测到 $r$ 个误分类样例的概率为:
  $$P(r)  = \frac{n!}{r!(n-r)!}\ error_{\cal{D}}(h)^{r}(1-error_{\cal{D}}(h))^{n-r}$$
\end{itemize}
\end{frame}
\begin{frame}
\frametitle{二项分布}
\label{sec-3-2}


若一次掷币出现正面的概率 $p = \Pr(heads)$ , 则 $n$ 次掷硬币出现 $r$ 次正面的概率 $P(r)$ :
$$P(r)  = \frac{n!}{r!(n-r)!}\ p^{r} (1- p)^{n-r}$$

\begin{itemize}
\item 期望: 
   $$E[X] \equiv  \sum_{i=0}^{n} i P(i) = np$$
\item 方差:
   $$Var(X) \equiv E[(X - E[X])^{2}] = np(1-p)$$
\item 标准差:
   $$\sigma_{X} \equiv  \sqrt{E[(X - E[X])^{2}]} = \sqrt{np(1-p)}$$
\end{itemize}
\end{frame}
\begin{frame}
\frametitle{正态分布逼近二项分布}
\label{sec-3-3}


\begin{itemize}
\item $error_{S}(h)$ 服从二项分布:
\begin{itemize}
\item 期望: $\mu_{error_{S}(h)} = error_{\cal{D}}(h)$
\item 标准差:$\sigma_{error_{S}(h)}  = \sqrt{\frac{error_{\cal{D}}(h)(1 - error_{\cal{D}}(h))}{n}}$
\end{itemize}
\item 用正态分布逼近:
\begin{itemize}
\item 期望: $\mu_{error_{S}(h)} = error_{\cal{D}}(h)$
\item 标准差:$\sigma_{error_{S}(h)} \approx \sqrt{\frac{error_{S}(h)(1 - error_{S}(h))}{n}}$
\end{itemize}
\item 得正态分布:
  $$p(x) = \frac{1}{\sqrt{2 \pi \sigma^{2}}} e^{-\frac{1}{2}(\frac{x -\mu}{\sigma})^{2}}$$
\end{itemize}
\end{frame}
\begin{frame}
\frametitle{置信区间计算}
\label{sec-3-4}

置信区间(confidence interval): 某个参数 p 的 N\% 置信区间是一个以 N\% 的概率包含p的区间。

\begin{itemize}
\item $X$ 落入区间 $(a,b)$ 的概率为:  $\int_{a}^{b} p(x) dx$
\item 其望: $E[X] = \mu$
\item 方差: $Var(X) = \sigma^{2}$
\item 标准差: $\sigma_{X} = \sigma$
\item 80\% 的概率位于 $\mu \pm 1.28 \sigma$ 之间
\item N\% 的概率位于 $\mu \pm z_{N} \sigma$ 之间

\begin{center}
\begin{tabular}{lrrrrrrr}
 $N\%$:    &  50\%  &  68\%  &  80\%  &  90\%  &  95\%  &  98\%  &  99\%  \\
 $z_{N}$:  &  0.67  &  1.00  &  1.28  &  1.64  &  1.96  &  2.33  &  2.58  \\
\end{tabular}
\end{center}


\end{itemize}
\end{frame}
\begin{frame}
\frametitle{置信区间分析}
\label{sec-3-5}

\begin{itemize}
\item $S$ 包含 $n$ 个独立抽取的样例, 且独立于 $h$
\item $n \geq 30$
\end{itemize}

则

近似 95\% 的概率, $error_{S}(h)$ 在区间

$$error_{\cal{D}}(h) \pm 1.96 \sqrt{\frac{error_{\cal{D}}(h) (1 - error_{\cal{D}}(h))}{n}}$$

同样,$error_{\cal{D}}(h)$ 在区间

\[ error_{S}(h) \pm 1.96 \sqrt{\frac{error_{\cal{D}}(h) (1 - error_{\cal{D}}(h))}{n}} \]

近似有

\[ error_{S}(h) \pm 1.96 \sqrt{\frac{error_{S}(h) (1 - error_{S}(h))}{n}} \]
\end{frame}
\begin{frame}
\frametitle{双侧和单侧边界}
\label{sec-3-6}

\begin{itemize}
\item 双侧边界:给出了估计量的上界和下界
\item 单侧边界:
\begin{itemize}
\item 只限定 $h$  的最大借误率
\end{itemize}
\end{itemize}
\end{frame}
\section{推导置信区间的一般方法}
\label{sec-4}
\begin{frame}
\frametitle{中心极限定理}
\label{sec-4-1}


考虑独立同分布的随机变量 $Y_{1} \ldots Y_{n}$ 的集合,它们服从一任意的概率分布,均值为 $\mu$ ,有限方差 $\sigma^{2}$ 。定义样本均值
\[\bar{Y}\equiv\frac{1}{n}\sum_{i=1}^{n}Y_{i} \]  
则当 $n \rightarrow \infty$ 时
$\bar{Y}$ 服从正态分布,均值为 $\mu$ 且标准差为 $\frac{\sigma^{2}}{n}$ 。
\end{frame}
\begin{frame}
\frametitle{计算置信区间}
\label{sec-4-2}


通用的过程包含以下步骤:
\begin{itemize}
\item 确定基准总体中要估计的参数 $p$ ,例如 $error_{\cal D}(h)$ 。
\item 定义一个估计量 $Y$ (如 $error_S(h)$ )它的选择应为最小方差的无偏估计量。
\item 确定估计量所服从的概率分布 ${\cal D}_Y$ ,包括其均值和方差。
\item 确定 N\% 置信区间,通过寻找阈值 $L$ 和 $U$ 以使这个按 ${\cal D}_Y$ 分布的随机变量有 $N\%$ 机会落入 $L$ 和 $U$ 之间。
\end{itemize}
\end{frame}
\section{两个假设错误率间的差异}
\label{sec-5}
\begin{frame}
\frametitle{两个假设错误率间的差异}
\label{sec-5-1}

假设 $h_{1}$ 在一拥有 $n_1$ 个独立抽取样例的样本 $S_1$ 上测试,且 $h_2$ 在 $n_2$ 个独立抽取样例的样本 $S_2$ 上测试。

\begin{itemize}
\item 估计两个假设的真实错误率间的差异
\end{itemize}
$d \equiv error_{\cal{D}}(h_{1}) - error_{\cal{D}}(h_{2})$
\begin{itemize}
\item 选取估计量
\end{itemize}
$\hat{d} \equiv error_{S_{1}}(h_{1}) - error_{S_{2}}(h_{2})$

\begin{itemize}
\item 确定估计量的概率分布
\end{itemize}
$\small \sigma_{\hat{d}} \approx \sqrt{\frac{error_{S_{1}}(h_{1})(1 - error_{S_{1}}(h_{1}))}{n_{1}} + \frac{error_{S_{2}}(h_{2})(1 - error_{S_{2}}(h_{2}))}{n_{2}}}$
\begin{itemize}
\item 寻找区间 ($L, U$) ,落入区间的概率为 N\%
\end{itemize}
$\hat{d} \pm z_{N} \sqrt{\frac{error_{S_{1}}(h_{1})(1 -
error_{S_{1}}(h_{1}))}{n_{1}} + \frac{error_{S_{2}}(h_{2})(1 -
error_{S_{2}}(h_{2}))}{n_{2}} }$
\end{frame}
\section{学习算法比较}
\label{sec-6}
\begin{frame}
\frametitle{学习算法比较}
\label{sec-6-1}


对于两个学习算法 $L_{A}$ 与 \$L$_{B}$\$\}

估计:

$E_{S \subset \cal{D}}[ error_{\cal{D}}(L_{A}(S)) - error_{\cal{D}}(L_{B}(S))]$

其中:

\begin{itemize}
\item $L(S)$ 代表给定训练样本 $S$ 时学习算法 $L$ 输出的假设
\item 下标 $S \subset \cal{D}$ 表示期望值是在基准分布 $\cal{D}$ 中抽取的样本 $S$ 上计算。
\end{itemize}

上述表达式描述的是学习算法 $L_A$ 和 $L_B$ 的差的期望值。
\end{frame}
\begin{frame}
\frametitle{计算}
\label{sec-6-2}


\begin{itemize}
\item 对于有限的样本 $D_0$ ,要估计上述的量需要将 $D_0$ 分割成训练集合 $S_0$ 和不相交的测试集合 $T_0$ 。
\begin{itemize}
\item 训练数据可以用来既训练 $L_A$ 又训练 $L_B$ ,
\item 而测试数据则用来比较两个学习到的假设的准确度。
\end{itemize}
\end{itemize}

$error_{T_{0}}(L_{A}(S_{0})) - error_{T_{0}}(L_{B}(S_{0}))$
\end{frame}
\begin{frame}
\frametitle{估计两学习算法 $L_A$ 和 $L_B$ 错误率差异的一种方法}
\label{sec-6-3}


\begin{itemize}
\item 将可用数据 $D_{0}$ 分割成 $k$ 个相同大小的不相交子集 $T_{1}, T_{2},\ldots, T_{k}$ , 其大小至少为30。
\item For $i$ from 1 to $k$, do
    使用 $T_{i}$ 作为测试集合,而剩余的数据作为训练集合Si
\begin{itemize}
\item $S_{i} \leftarrow \{D_{0} - T_{i}\}$
\item $h_{A} \leftarrow L_{A}(S_{i})$
\item $h_{B} \leftarrow L_{B}(S_{i})$
\item $\delta_{i} \leftarrow error_{T_{i}}(h_{A}) - error_{T_{i}}(h_{B})$
\end{itemize}
\item 返回值 $\bar{\delta}$ , $\bar{\delta} \equiv \frac{1}{k}\sum_{i=1}^{k} \delta_{i}$
\end{itemize}
\end{frame}
\begin{frame}
\frametitle{更合适的说法}
\label{sec-6-4}


算法对

\[ E_{S \subset D_{0}}[ error_{\cal{D}}(L_{A}(S)) - error_{\cal{D}}(L_{B}(S))] \]
进行了估计。而不是 

\[ E_{S \subset \cal{D}}[ error_{\cal{D}}(L_{A}(S)) - error_{\cal{D}}(L_{B}(S))] \]
\end{frame}
\section{成对 t 检验}
\label{sec-7}
\begin{frame}
\frametitle{成对 t 检验}
\label{sec-7-1}


比较 $h_{A}$ 与 $h_{B}$
\begin{itemize}
\item 将可用数据 $D_{0}$ 分割成 $k$ 个相同大小的不相交子集 $T_{1}, T_{2},\ldots, T_{k}$ , 其大小至少为30。
\item For $i$ from 1 to $k$, do
\begin{itemize}
\item $\delta_{i} \leftarrow error_{T_{i}}(h_{A}) - error_{T_{i}}(h_{B})$
\end{itemize}
\item 返回值 $\bar{\delta}$, $\bar{\delta} \equiv \frac{1}{k}\sum_{i=1}^{k} \delta_{i}$
\end{itemize}

$d$ 的 $N\%$ 置信区间:
\begin{itemize}
\item $\bar{\delta} \pm t_{N,k-1} \ s_{\bar{\delta}}$
\item $s_{\bar{\delta}} \equiv \sqrt{\frac{1}{k(k-1)} \sum_{i=1}^{k}(\delta_{i} -\bar{\delta})^{2}}$
\end{itemize}

\{ $\delta_{i}$ 近似正态分布\}
\end{frame}

\end{document}
