% Created 2016-09-13 Tue 10:58
\documentclass{beamer}
\usepackage{fixltx2e}
\usepackage{graphicx}
\usepackage{longtable}
\usepackage{float}
\usepackage{wrapfig}
\usepackage{soul}
\usepackage{textcomp}
\usepackage{marvosym}
\usepackage{wasysym}
\usepackage{latexsym}
\usepackage{amssymb}
\usepackage{hyperref}
\tolerance=1000
\usepackage{etex}
\usepackage{amsmath}
\usepackage{pstricks}
\usepackage{pgfplots}
\usepackage{tikz}
\usepackage[europeanresistors,americaninductors]{circuitikz}
\usepackage{colortbl}
\usepackage{yfonts}
\usetikzlibrary{shapes,arrows}
\usetikzlibrary{positioning}
\usetikzlibrary{arrows,shapes}
\usetikzlibrary{intersections}
\usetikzlibrary{calc,patterns,decorations.pathmorphing,decorations.markings}
\usepackage[BoldFont,SlantFont,CJKchecksingle]{xeCJK}
\setCJKmainfont[BoldFont=Evermore Hei]{Evermore Kai}
\setCJKmonofont{Evermore Kai}
\usepackage{pst-node}
\usepackage{pst-plot}
\psset{unit=5mm}
\usepackage{beamerarticle}
\mode<beamer>{\usetheme{Frankfurt}}
\mode<beamer>{\usecolortheme{dove}}
\mode<article>{\hypersetup{colorlinks=true,pdfborder={0 0 0}}}
\AtBeginSection[]{\begin{frame}<beamer>\frametitle{Topic}\tableofcontents[currentsection]\end{frame}}
\setbeamercovered{transparent}
\providecommand{\alert}[1]{\textbf{#1}}

\title{概念学习}
\author{}
\date{}
\hypersetup{
  pdfkeywords={},
  pdfsubject={},
  pdfcreator={Emacs Org-mode version 7.9.3f}}

\begin{document}

\maketitle

\begin{frame}
\frametitle{Outline}
\setcounter{tocdepth}{3}
\tableofcontents
\end{frame}











\section{概念学习}
\label{sec-1}
\begin{frame}
\frametitle{什么是概念学习}
\label{sec-1-1}
\begin{block}{定义}
\label{sec-1-1-1}

概念学习是指从有关某个布尔函数的输入输出训练样例中,推断出该布尔函数。
\end{block}
\end{frame}
\begin{frame}
\frametitle{示例}
\label{sec-1-2}


\begin{center}
\begin{tabular}{lllllll}
 Sky    &  AirTemp  &  Humid   &  Wind    &  Water  &  Forecst  &  EnjoySpt  \\
\hline
 Sunny  &  Warm     &  Normal  &  Strong  &  Warm   &  Same     &  Yes       \\
 Sunny  &  Warm     &  High    &  Strong  &  Warm   &  Same     &  Yes       \\
 Rainy  &  Cold     &  High    &  Strong  &  Warm   &  Change   &  No        \\
 Sunny  &  Warm     &  High    &  Strong  &  Cool   &  Change   &  Yes       \\
\end{tabular}
\end{center}
\end{frame}
\begin{frame}
\frametitle{假设}
\label{sec-1-3}


\begin{itemize}
\item 考虑各属性约束的合取式 $h$
\item 每个属性可取值为:
\begin{itemize}
\item 由“?”表示任意值
\item 明确指定的属性值(如 Water=Warm)
\item 由“ $\emptyset$ ”表示不接受任何值(如 Water=$\emptyset$ )
\end{itemize}
\item 例如:

\begin{center}
\begin{tabular}{llllll}
 Sky              &  AirTemp  &  Humid  &  Wind      &  Water  &  Forecst         \\
\hline
 $\langle Sunny$  &  $?$      &  $?$    &  $Strong$  &  $?$    &  $Same \rangle$  \\
\end{tabular}
\end{center}


\end{itemize}
\end{frame}
\begin{frame}
\frametitle{任务}
\label{sec-1-4}


\begin{itemize}
\item 已知:
\begin{itemize}
\item 实例集X:可能的日子,每个日子由下面的属性描述:
\begin{itemize}
\item Sky(可取值为Sunny,Cloudy和Rainy)
\item AirTemp(可取值为Warm和Cold)
\item Humidity(可取值为Normal和High)
\item Wind(可取值为Strong和Weak)
\item Water(可取值为Warm和Cool)
\item Forecast(可取值为Same和Change)
\end{itemize}
\item 假设集H:每个假设描述为6个属性Sky,AirTemp,Humidity,Wind,Water和Forecast的值约束的合取。约束可以为“?”(表示接受任意值),“ $\emptyset$ ”(表示拒绝所有值),或一特定值。
\item 目标概念c: EnjoySport:  $X\rightarrow \{0, 1\}$
\item 训练样例集D:目标函数的正例和反例  $\langle x_1, c(x_1) \rangle , \ldots \langle x_m, c(x_m) \rangle$
\end{itemize}
\item 求解:
\begin{itemize}
\item $H$ 中的一假设 $h$ ,使对于 $X$ 中任意 $x$ , $h(x)=c(x)$ 。
\end{itemize}
\end{itemize}
\end{frame}
\begin{frame}
\frametitle{归纳学习假设}
\label{sec-1-5}


任一假设如果在足够大的训练样例集中很好地逼近目标函数,它也能在未见实例中很好地逼近目标函数。
\end{frame}
\begin{frame}
\frametitle{more-general-than-or-equal-to}
\label{sec-1-6}


\center
\includegraphics[width=\textwidth]{./image/vs-gen-spec.png}
\end{frame}
\begin{frame}
\frametitle{more-general-than-or-equal-to}
\label{sec-1-7}


定义: 令 $h_j$ 和 $h_k$ 为在 $X$ 上定义的布尔函数。定义一个 more-general-than-or-equal-to 关系,记做 $\geq_g$ 。 称 $h_j \geq_g h_k$ 当且仅当

$$(\forall x\in X)[(h_k(x)=1)\rightarrow (h_j(x)=1)]$$
\end{frame}
\section{FIND-S}
\label{sec-2}
\begin{frame}
\frametitle{FIND-S 算法}
\label{sec-2-1}


\begin{itemize}
\item 将 $h$ 初始化为 $H$ 中最特殊假设
\item 对每个正例 $x$
\begin{itemize}
\item 对h的每个属性约束 $a_i$
\begin{itemize}
\item 如果 $x$ 满足 $a_i$
        那么 不做任何事
\item 否则将 $h$ 中 $a_i$ 替换为满足 $x$ 的紧邻的更一般约束
\end{itemize}
\end{itemize}
\item 输出假设h
\end{itemize}
\end{frame}
\begin{frame}
\frametitle{FIND-S 的假设空间搜索}
\label{sec-2-2}

\center
\includegraphics[width=0.8\textwidth]{./image/finds-gen-spec.png}
\end{frame}
\section{变型空间和候选消除算法}
\label{sec-3}
\begin{frame}
\frametitle{表示}
\label{sec-3-1}


\begin{itemize}
\item 一致(Consistent): 一个假设h与训练样例集合D一致(consistent),当且仅当对 $\langle x, c(x) \rangle$ in $D$ 都有 h(x)=c(x)。
\end{itemize}

\[Consistent(h,D) \equiv (\forall \langle x, c(x) \rangle \in D)\  h(x)=c(x) \]

\begin{itemize}
\item 变型空间(version space): 关于假设空间 $H$ 和训练样例集 $D$ 的变型空间(version space),标记为 $VS_{H,D}$ ,是 $H$ 中与训练样例 $D$ 一致的所有假设构成的子集。
\end{itemize}

\[VS_{H,D} \equiv \{h \in H|Consistent(h,D)\} \]
\end{frame}
\begin{frame}
\frametitle{列表后消除算法(List-Then-Eliminate)}
\label{sec-3-2}


\begin{itemize}
\item 变型空间 $VersionSpace\leftarrow$ 包含H中所有假设的列表
\item 对每个训练样例 $\langle x, c(x)\rangle$
\begin{itemize}
\item 从变型空间中移除所有 $h(x)\neq c(x)$ 的假设h
\end{itemize}
\item 输出 $VersionSpace$ 中的假设列表
\end{itemize}
\end{frame}
\begin{frame}
\frametitle{变型空间的更简洁表示}
\label{sec-3-3}


\begin{itemize}
\item 定义: 关于假设空间H和训练数据D的一般边界(General boundary)G,是在H中与D相一致的极大一般(maximally general)成员的集合。
    $S\equiv\{ g\in H | Consistent(g, D)\land(\neg\exists g'\in H)[(g' >_g g)\land Consistent(g', D)]\}$
\item 定义: 关于假设空间H和训练数据D的特殊边界(Specific boundary)S,是在H中与D相一致的极大特殊(maximally specific)成员的集合。
    $S\equiv\{ s\in H | Consistent(s, D)\land(\neg\exists s'\in H)[(s >_g s')\land Consistent(s', D)]\}$
\end{itemize}
\end{frame}
\begin{frame}
\frametitle{例}
\label{sec-3-4}

\center
\includegraphics[width=0.8\textwidth]{./image/figure-vs3.png}
\end{frame}
\begin{frame}
\frametitle{变型空间表示定理}
\label{sec-3-5}


令 $X$ 为一任意的实例集合, $H$ 与为 $X$ 上定义的布尔假设的集合。令 $c: X\rightarrow\{0, 1\}$ 为 $X$ 上定义的任一目标概念,并令 $D$ 为任一训练样例的集合 $\{\langle x, c(x)\rangle\}$ 。对所有的 $X,H,c,D$ 以及良好定义的 $S$ 和 $G$ :
 $$VS_{H,D} = \{ h\in H | (\exists s\in S) (\exists g\in G) (g\geq_g h \geq_g s)\}$$
\end{frame}
\begin{frame}
\frametitle{候选消除算法(Candidate Elimination Algorithm)}
\label{sec-3-6}

\begin{itemize}
\item 将G集合初始化为H中极大一般假设
\item 将S集合初始化为H中极大特殊假设
\item 对每个训练样例d,进行以下操作:
\item 如果d是一正例
\begin{itemize}
\item 从G中移去所有与d不一致的假设
\item 对S中每个与d不一致的假设s
\begin{itemize}
\item 从S中移去s
\item 把s的所有的极小泛化式h加入到S中,其中h满足:

                 h与d一致,而且G的某个成员比h更一般
\item 从S中移去所有这样的假设:它比S中另一假设更一般
\end{itemize}
\end{itemize}
\item 如果d是一个反例
\begin{itemize}
\item 从S中移去所有与d不一致的假设
\item 对G中每个与d不一致的假设g
\begin{itemize}
\item 从G中移去g
\item 把g的所有的极小特殊化式h加入到G中,其中h满足:

                h与d一致,而且S的某个成员比h更特殊
\end{itemize}
\item 从G中移去所有这样的假设:它比G中另一假设更特殊
\end{itemize}
\end{itemize}
\end{frame}
\begin{frame}
\frametitle{算法示例}
\label{sec-3-7}

\center
\includegraphics[width=0.6\textwidth]{./image/vsexamp-initialize.png}
\end{frame}
\begin{frame}
\frametitle{算法示例}
\label{sec-3-8}

\center
\includegraphics[width=0.6\textwidth]{./image/vsexamp0.png}
\end{frame}
\begin{frame}
\frametitle{算法示例}
\label{sec-3-9}

\center
\includegraphics[width=0.6\textwidth]{./image/vsexamp1.png}
\end{frame}
\begin{frame}
\frametitle{算法示例}
\label{sec-3-10}

\center
\includegraphics[width=0.7\textwidth]{./image/vsexamp2.png}
\end{frame}
\begin{frame}
\frametitle{算法示例}
\label{sec-3-11}

\center
\includegraphics[width=0.7\textwidth]{./image/vsexamp3.png}
\end{frame}
\begin{frame}
\frametitle{算法示例}
\label{sec-3-12}

\center
\includegraphics[width=0.7\textwidth]{./image/figure-vs3.png}
\end{frame}
\section{归纳偏置}
\label{sec-4}
\begin{frame}
\frametitle{一个有偏的假设空间}
\label{sec-4-1}

EnjoySportw例子中,假设空间限制为只包含属性值的合取。不能够表示最简单的析取形式的目标概念,如“Sky=Sunny or Sky=Cloudy”。
给定以下三个训练样例,它们来自于该析取式假设,我们的算法将得到一个空的变型空间。

\samll

\begin{center}
\begin{tabular}{lllllll}
 Sky     &  AirTemp  &  Humidity  &  Wind    &  Water  &  Forecast  &  EnjoySport  \\
\hline
 Sunny   &  Warm     &  Normal    &  Strong  &  Cool   &  Change    &  Yes         \\
 Cloudy  &  Warm     &  Normal    &  Strong  &  Cool   &  Change    &  Yes         \\
 Rainy   &  Warm     &  Normal    &  Strong  &  Cool   &  Change    &  No          \\
\end{tabular}
\end{center}
\end{frame}
\begin{frame}
\frametitle{无法生成变型空间的原因}
\label{sec-4-2}


与头两个样例一致,并且能在给定假设空间H中表示的最特殊的假设是:

S2: <?, Warm, Nornal, Strong, Cool, Change>

将第三个样例错误地划为正例。
\end{frame}
\begin{frame}
\frametitle{无偏的学习器}
\label{sec-4-3}

为EnjoySport学习任务定义一个新的假设空间H',允许使用前面的假设的任意析取、合取和否定式。例如目标概念“Sky=Sunny  or Sky=Cloudy”可被描述为:

<Sunny, ?, ?, ?, ?, ?> ∨ <Cloudy, ?, ?, ?, ?, ?>

排除了表达能力的问题,但概念学习算法将完全无法从训练样例中泛化!
\end{frame}
\begin{frame}
\frametitle{无法泛化的原因}
\label{sec-4-4}

如下,假定我们提供了3个正例(x1,x2,x3)以及两个反例(x4,x5)给学习器。这时,变型空间的S边界包含的假设正好是三个正例的析取:

$$S: \{ (x_1 \lor x_2 \lor x_3) \}$$

因为这是能覆盖3个正例的最特殊假设。相似地,G边界将由那些刚好能排除掉反例的那些假设组成。
$G: \{\neg (x_4\lor x_5)\}$
\end{frame}
\begin{frame}
\frametitle{归纳编置}
\label{sec-4-5}

定义:考虑对于实例集合X的概念学习算法  $L$  , 令
\begin{itemize}
\item $c$  为 $X$ 上定义的任一概念,
\item $D_c = \{\langle x, c(x) \rangle \}$ 为 $c$ 的任意训练样例集合。
\item $L(x_i,D_c)$ 表示经过数据 $D_c$ 的训练后 $L$ 赋予实例 $x_i$ 的分类。
\end{itemize}

$L$ 的归纳偏置是最小断言集合 $B$ ,它使任意目标概念 $c$ 和相应的训练样例 $D_c$ 满足

\[
(\forall x_i \in X) [(B \land D_c \land x_i) \vdash L(x_i,D_c)]
\]

 $A \vdash B$ 表示 $A$  逻辑蕴涵 $B$
\end{frame}
\begin{frame}
\frametitle{归纳系统与等价的演绎系统}
\label{sec-4-6}

\center
\includegraphics[width=0.7\textwidth]{./image/figure-vs-bias-new.png}
\end{frame}
\begin{frame}
\frametitle{归纳偏置不同的学习器}
\label{sec-4-7}

\begin{itemize}
\item 机械学习器(Rote-Learner)
\begin{itemize}
\item 简单地将每个观察到的训练样例存储下来。
\item 后续的实例的分类通过在内存中匹配进行。
\begin{itemize}
\item 如果实例在内存中找到了,存储的分类结果被输出。
\item 否则系统拒绝进行分类。
\end{itemize}
\end{itemize}
\item 候选消除算法
\begin{itemize}
\item 新的实例只在变型空间所有成员都进行同样分类时才输出分类结果
\item 否则系统拒绝分类。
\end{itemize}
\item Find-S
\begin{itemize}
\item 算法寻找与训练样例一致的最特殊的假设
\item 用这一假设来分类后续实例。
\end{itemize}
\end{itemize}
\end{frame}

\end{document}
