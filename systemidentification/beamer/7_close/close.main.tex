\begin{document}
%\begin{CJK*}{GBK}{song}
\newcommand{\vect}[1]{\boldsymbol{#1}}

\def\lecturename{系统辨识}

\title{\insertlecture}

\author{邢超}

\institute
{
  西北工业大学航天学院
}

%\mode<presentation>{\subject{嵌入式系统}}

%  start a lecture  --------------------------
\lecture[]{闭环系统辨识}{}
\subtitle{}
\date{2012}


%\setbeamertemplate{background}{\pgfimage[width=\paperwidth,height=\paperheight]{image/flower}}
%\setbeamercovered{transparent}
%\mode<presentation>{\beamerdefaultoverlayspecification{<+->}}

\begin{frame}
  \maketitle
\end{frame}


\section{可辨识性概念}
\begin{frame}{可辨识性概念}
闭环系统可辨识性与控制器的结构、阶次和反馈通道的噪声有关。
例:
\begin{eqnarray*}
y_k &=& -ay_{k-1}+b u_{k-1}+\varepsilon_k \\
\end{eqnarray*}
反馈1:
\begin{eqnarray*}
u_k &=&  y_k \\
y_k &=& -ay_{k-1}+b y_{k-1}+\varepsilon_k \\
y_k &=& (-a+b) y_{k-1}+\varepsilon_k \\
\end{eqnarray*}
反馈2:
\begin{eqnarray*}
u_k &=&  y_{k-1} \\
y_k &=& -ay_{k-1}+b y_{k-2}+\varepsilon_k \\
Y &=& \Phi \theta +\varepsilon
\end{eqnarray*}
\end{frame}

\section{SISO闭环系统辨识}
\begin{frame}{闭环系统}
\begin{eqnarray*}
y_k &=& -\sum_{i=1}^{n_a} a_i y_{k-i} + \sum_{i=q}^{n_b} b_i u_{k-i} + \varepsilon_k \\
u_k &=& -\sum_{i=1}^{n_c} c_i u_{k-i} + \sum_{i=p}^{n_d} d_i y_{k-i} + s_k  \\
Y &=& \Phi\theta \\
\Phi &=& \begin{bmatrix}
-y_{n} & \cdots & -y_{n+1-n_a} & u_{n+1-q} &\cdots & u_{n+1-n_b} \\
-y_{n+1} & \cdots & -y_{n+2-n_a} & u_{n+2-q} &\cdots & u_{n+2-n_b} \\
\vdots &        &\vdots         &  \vdots  &       &\vdots \\
-y_{n+N-1} & \cdots & -y_{n+N-n_a} & u_{n+N-q} &\cdots & u_{n+N-n_b} 
\end{bmatrix} \\
\theta &=& \begin{bmatrix}a_1 & \cdots & a_{n_a} & b_q &\cdots & b_{n_b} \end{bmatrix}^T  \\
Y &=& \begin{bmatrix} Y_{n+1} & \cdots  & Y_{n+N} \end{bmatrix}^T
\end{eqnarray*}
\end{frame}

\begin{frame}{直接辨识}
\begin{itemize}
\item 当$s(k)$为白噪声时:
\begin{itemize}
\item 前向通道最小二乘估计为惟一性估计。 
\item 如果$\varepsilon(k)=0$或者是与$s(k)$无关的白噪声:则前向通道最小二乘估计为一致性与惟一性估计。
\end{itemize}
\item 当$s(k)=0$时,如果$n_c>n_b-q$或$n_d>n_a-q$:
\begin{itemize}
\item 则最小二乘估计为惟一性估计。
\item $\varepsilon(k)=0$或为白噪声时,若满足 $q>0$ 或 $p>0$ ,前向通道最小二乘估计为一致估计。
\end{itemize}
\item 若$\varepsilon(k)$为有色噪声,最小二乘估计不是一致性估计,可用改进的最小二乘估计方法得到一致性估计。
\end{itemize}
\end{frame}

% \section{思考}
% \begin{frame}{思考}
% \begin{itemize}
% \item 极大似然法辨识思想
% \end{itemize}
% \end{frame}


%\end{CJK*}
\end{document}
