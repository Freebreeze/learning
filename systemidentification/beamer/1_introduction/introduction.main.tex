\begin{document}
%\begin{CJK*}{GBK}{song}

\def\lecturename{系统辨识}

\title{\insertlecture}

\author{邢超}

\institute
{
  西北工业大学航天学院
}

%\mode<presentation>{\subject{嵌入式系统}}

%  start a lecture  --------------------------
\lecture[SI]{系统辨识简介}{}
\subtitle{系统辨识基本概念}
\date{2012}


%\setbeamertemplate{background}{\pgfimage[width=\paperwidth,height=\paperheight]{image/flower}}
%\setbeamercovered{transparent}
%\mode<presentation>{\beamerdefaultoverlayspecification{<+->}}

\begin{frame}
  \maketitle
\end{frame}

%\begin{frame}{课程内容}
%\begin{center}\pgfimage[width=0.9\columnwidth]{image/content}\end{center}
%\end{frame}

%\begin{frame}{与其它课程关系}
%\begin{center}\pgfimage[width=0.9\columnwidth]{image/relation}\end{center}
%\end{frame}



\section{主要内容}

\begin{frame}{课程学习内容}
\begin{itemize}
\item 系统辨识目的
\item 辨识的方法
\item 辨识的具体步骤
\end{itemize}
如何学习:
\begin{itemize}
\item 学习该课程是做什么的?
\item 主要解决什么问题?
\item 有哪些方法?
\item 每种方法的优缺点、适用范围?
\end{itemize}
\end{frame}

\section{系统辨识的基本概念}
\begin{frame}{系统辨识的地位和目的}
\begin{description}
\item[控制理论]经典控制理论、现代控制理论、智能控制理论
\item[经典控制]应用时域法、根轨迹法、频域法,设计被控对象的控制器。
\item[现代控制]线性系统理论、最优控制理论和最优估计理论等。
\item[智能控制]神经网络、专家系统及人工智能。
\end{description}
\end{frame}

\begin{frame}{线性系统理论}
现代控制的基础,主要解决系统的模型描述和基础知识。即线性系统一般可描述为:
\begin{eqnarray}
\dot x = Ax+Bu \\
y=Cx+Du
\end{eqnarray}
\begin{description}
\item[最优控制] 解决在某一性能指标约束下,如何解算最优输入u(t);
\item[最优估计] 主要解决状态变量X的估计和预测。
\end{description}
\end{frame}


\begin{frame}{系统辨识目的}
\begin{itemize}
\item 上述问题解决的先决条件:
\begin{itemize}
\item 模型中的A、B、C、D已知。
\item 亦即系统的结构和参数已知,
\item 也就是要知道系统的传递函数、或是脉冲传递函数、或是差分方程、或是系统的频率特性。
\end{itemize}
\item 那么,如何获取系统的结构和参数?
\item 系统辨识目的:如何获取系统的模型及其参数?
\end{itemize}
\end{frame}

\section{系统的模型描述}
\begin{frame}{系统的模型定义与特点}
\begin{description}
\item[模型定义]
    系统的本质的部分信息简缩成的一种有用的描述形式。
\item[模型特点]
\begin{itemize}
\item 同一系统有多个模型描述; 
\item 同一模型可以反映不同的实际系统;
\item 模型的精确度与复杂度。
\end{itemize}
\end{description}
\end{frame}

\begin{frame}{模型表示形式}
\begin{itemize}
\item 直觉模型
\item 物理模型
\item 图表模型
\item 数学模型。
\end{itemize}
其中,图表模型为非参数模型,数学模型为参数模型。
\end{frame}





\begin{frame}{数学模型分类}
\begin{description}
\item[时域]
\begin{itemize}
\item 微分方程
\item 差分方程
\item 状态方程
\end{itemize}
\item[复域]
\begin{itemize}
\item 传递函数
\item 脉冲传递函数
\end{itemize}
\item[频域]
\begin{itemize}
\item 频率特性
\item 描述函数
\end{itemize}
\end{description}
\end{frame}



\begin{frame}{系统辨识中的模型}
   系统辨识获取系统的非参数模型和参数模型。
\begin{description}
\item[非参数模型]
\begin{itemize}
\item 频率特性曲线
\item 脉冲响应曲线
\end{itemize}
\item[参数模型]
\begin{itemize}
\item 差分方程
\item 传递函数
\item 脉冲传递函数
\end{itemize}
\item[模型转换]
\begin{itemize}
\item 参数模型间可以相互变换;
\item 非参数模型可以变换为参数模型。
\end{itemize}
\end{description}
\end{frame}

\section{数学模型的建立方法和原则}
\begin{frame}{模型建立方法}
\begin{itemize}
\item    理论分析方法:本科阶段已学
\item    实验测试法:利用系统输入/输出数据,建立系统的数学模型。系统辨识采用该方法。
\end{itemize}
\end{frame}

\begin{frame}{建模原则}
\begin{itemize}
\item 模型的使用目的明确; 
\item 物理概念清楚;
\item 辨识具有无偏性和一致性;
\item 符合节省原理。需辨识参数数目要少。
\end{itemize}
\end{frame}

\section{系统辨识流程与分类}


\begin{frame}{系统辨识定义}
\begin{itemize}
\item    定义:在系统输入和输出数据基础上,从一组给定的模型类中,确定一个与所测系统等价的模型。
\item    系统辨识三要素:数据、模型类与准则。
\begin{itemize}
\item    数据:记录的输入/输出数据,往往含有噪声;
\item    模型类:选定模型;
\item    准则:亦即代价函数,通常为误差准则。
\end{itemize}
\end{itemize}
\end{frame}

\begin{frame}{系统辨识一般流程}
    系统辨识分为模型结构辨识和模型参数辨识。
其一般流程为:
\begin{itemize}
\item    明确所辨识系统模型的使用目的; 
\item    预选待辨识系统的数学模型种类;
\item    进行辨识的实验设计,记录I/O数据;
\item    数据预处理,野点剔除;
\item    模型结构辨识,辨识系统阶次n;
\item    选择参数估计方法,辨识系统其它参数;
\item    模型验证。
\end{itemize}
本课程重点:参数估计方法
\end{frame}

\begin{frame}{系统辨识分类}
\begin{itemize}
\item    线性系统辨识和非线性系统辨识; 
\item    集中参数辨识和分布参数辨识;
\item    系统结构参数辨识和系统参数辨识;
\item    经典辨识和近代辨识;
\item    开环系统辨识和闭环系统辨识;
\item    离线辨识和在线辨识。
\end{itemize}
\end{frame}

\begin{frame}{离线辨识}
\begin{itemize}
\item 过程:系统模型及阶次n选定后,记录下系统全部的I/O数据,然后再用参数估计方法,辨识系统的模型参数。
\item 特点:需存储数据量大,计算量大,辨识精度较高。事后数据处理方法,不能用于实时控制系统。
\end{itemize}
\end{frame}

\begin{frame}{在线辨识}
\begin{itemize}
\item 过程:系统模型及阶次n选定后,先获取一小部分数据,估计系统模型参数,再获取新的I/O数据,采用递推修正算法获得新的参数估计值,重复上述过程,直至系统运行停止。
\item 特点:数据量小,计算量小,辨识精度稍低。是一种在线数据处理方法,用于实时控制系统。
\end{itemize}
\end{frame}

\section{系统辨识误差准则}
\begin{frame}{系统辨识误差准则}
误差准则通常被表示为误差的泛函
\begin{eqnarray}
J(\theta)=\Sigma_{k=1}^Nf(\epsilon(k))
\end{eqnarray} 
$\epsilon(k)$为模型与实际系统的误差,可以是输出误差或输入误差,也可以是广义误差。一般函数$f$取为误差平方:
\begin{eqnarray}
f(\epsilon(k))=\epsilon^2(k)
\end{eqnarray}
\begin{itemize}
\item 输入误差$\epsilon(k)=u(k)-u_m(k)=u(k)-S^{-1}[y_m(k)]$
\item 输出误差$\epsilon(k)=y(k)-y_m(k)$
\end{itemize}
本课程均采用输出误差。
\end{frame}


\section{思考}
\begin{frame}{思考}
\begin{itemize}
\item 系统辨识与其它课程的关系?
\item 如何学习系统辨识?
\end{itemize}
\end{frame}


%\end{CJK*}
\end{document}
