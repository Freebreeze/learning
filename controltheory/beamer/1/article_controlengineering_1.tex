% Created 2013-10-20 Sun 14:59
\documentclass{article}
\usepackage[utf8]{inputenc}
\usepackage[T1]{fontenc}
\usepackage{fixltx2e}
\usepackage{graphicx}
\usepackage{longtable}
\usepackage{float}
\usepackage{wrapfig}
\usepackage{soul}
\usepackage{textcomp}
\usepackage{marvosym}
\usepackage{wasysym}
\usepackage{latexsym}
\usepackage{amssymb}
\usepackage{hyperref}
\tolerance=1000
\usepackage{amsmath}
\usepackage[usenames]{color}
\usepackage{pstricks}
\usepackage{pgfplots}
\usepackage{tikz}
\usepackage[europeanresistors,americaninductors]{circuitikz}
\usepackage{colortbl}
\usepackage{yfonts}
\usetikzlibrary{shapes,arrows}
\usetikzlibrary{positioning}
\usetikzlibrary{arrows,shapes}
\usetikzlibrary{intersections}
\usetikzlibrary{calc,patterns,decorations.pathmorphing,decorations.markings}
\usepackage[BoldFont,SlantFont,CJKchecksingle]{xeCJK}
\setCJKmainfont[BoldFont=Evermore Hei]{Evermore Kai}
\setCJKmonofont{Evermore Kai}
\xeCJKsetup{CJKglue=\hspace{0pt plus .08 \baselineskip }}
\usepackage{pst-node}
\usepackage{pst-plot}
\psset{unit=5mm}
\usepackage{beamerarticle}
\mode<beamer>{\usetheme{Frankfurt}}
\mode<beamer>{\usecolortheme{dove}}
\mode<article>{\hypersetup{colorlinks=true,pdfborder={0 0 0}}}
\AtBeginSection[]{\begin{frame}<beamer>\frametitle{Topic}\tableofcontents[currentsection]\end{frame}}
\setbeamercovered{transparent}
\providecommand{\alert}[1]{\textbf{#1}}

\title{自动控制的基本概念}
\author{}
\date{}
\hypersetup{
  pdfkeywords={},
  pdfsubject={},
  pdfcreator={Emacs Org-mode version 7.9.3f}}

\begin{document}

\maketitle

\begin{frame}
\frametitle{Outline}
\setcounter{tocdepth}{3}
\tableofcontents
\end{frame}












\section{反馈与控制}
\label{sec-1}
\subsection{自动控制理论}
\label{sec-1-1}
\begin{frame}
\frametitle{什么是自动控制}
\label{sec-1-1-1}

 无人工直接参与的情况下,利用控制装置(控制器)使被控对象按照给定的规律变化。
\end{frame}
\begin{frame}
\frametitle{自动控制理论}
\label{sec-1-1-2}

\begin{itemize}
\item <2->经典控制理论
\item <3->现代控制理论
\end{itemize}
\end{frame}
\begin{frame}
\frametitle{控制方式}
\label{sec-1-1-3}

\begin{itemize}
\item 开环控制
\item 闭环控制
\item 复合控制
\end{itemize}
\end{frame}
\subsection{开环控制}
\label{sec-1-2}
\begin{frame}
\frametitle{开环控制}
\label{sec-1-2-1}

\begin{itemize}
\item <2->定义:开环控制是指控制器与被控对象之间只有顺向作用而没有反向联系,称为开环控制。
\item <3->系统的输出量对系统的输入量无影响
\item <4->开环系统对控制偏差无修正能力。
\begin{itemize}
\item <5->按给定量控制
\item <6->按扰动量控制
\end{itemize}
\end{itemize}
\end{frame}
\begin{frame}
\frametitle{按给定量控制}
\label{sec-1-2-2}



\begin{tikzpicture}[node distance=2em,auto,>=latex', thick]
%\path[use as bounding box] (-1,0) rectangle (10,-2); 
\path[->] node[] (r) {$U_g$}; 
%\path[->] node[ circle,inner sep=2pt,minimum size=1pt,draw,label=below left:$ $,right =of r] (p1) { }; 
%\path[->](r) edge node {} (p1) ; 
\path[blue] node[draw, right =of r] (n) {信号变换与驱动电路}; 
\path[->] (r) edge node[midway] {} (n) ; 
\path[red] node[draw, inner sep=5pt,right =of n] (g) {电机}; 
\path[->] (n) edge node [midway]{$ $} (g); 
\path[->] node[ right =of g] (o) {$n$}; 
\path[->] (g) edge node {} (o); 
%\path[blue] node[draw, below =of g] (h) {传感器};
%\path[->,draw] (g.east)+(1em,0) |- (h.east) ; 
%\path[->,draw] (h.west) -| (p1) ; 
\end{tikzpicture} 

\begin{itemize}
\item 输入量: 电压 $U_g$
\item 输出量: 电机转速 $n$
\item $n=kU_g$
\end{itemize}
\end{frame}
\begin{frame}
\frametitle{按扰动量控制}
\label{sec-1-2-3}

   对扰动进行补偿,使扰动的影响减小


\begin{tikzpicture}[node distance=2em,auto,>=latex', thick]
%\path[use as bounding box] (-1,0) rectangle (10,-2); 
\path[->] node[] (r) {$U_0$}; 
\path[->] node[ circle,inner sep=2pt,minimum size=1pt,draw,label=below left:$ $,right =of r] (p1) { }; 
\path[->](r) edge node {} (p1) ; 
\path[blue] node[draw, right =of p1] (n) {驱动电路}; 
\path[->] (p1) edge node[midway] {$U_c$} (n) ; 
\path[red] node[draw, inner sep=5pt,right =of n] (g) {电机}; 
\path[->] (n) edge node [midway]{$ $} (g); 
\path[->] node[ right =of g] (o) {$n$}; 
\path[->] (g) edge node {} (o); 
\path[blue] node[draw, below =of g] (l) {负载扰动};
\path[dashed,draw] (g.south) edge (l) ; 
\path[blue] node[draw, below =of n] (h) {扰动测量};
\path[->,draw] (l) edge  node[midway] {$i$} (h) ; 
\path[->,draw] (h.west) -| node[midway]{$U_b$} (p1) ; 
\end{tikzpicture} 

\begin{itemize}
\item $U_c=U_0+U_b$
\item 负载增加导致 $n\downarrow , i\uparrow$
\item $i\uparrow\rightarrow U_b\uparrow\rightarrow U_c\uparrow\rightarrow n\uparrow$
\end{itemize}
\end{frame}
\begin{frame}
\frametitle{开环控制特点}
\label{sec-1-2-4}

\begin{enumerate}
\item <2->优点:原理简单,结构简单,反应速度快,灵敏度高
\item <3->缺点:
\begin{itemize}
\item 对控制偏差无修正能力
\item 控制精度取决于各控制元器件的精度
\end{itemize}
\item <4->适应场合:对控制精度要求不高的系统
\item <5-> 结构图:   输入 $\rightarrow$ 控制器 $\rightarrow$ 被控对象 $\rightarrow$ 输出  (顺向作用)
\end{enumerate}
\end{frame}
\subsection{闭环控制}
\label{sec-1-3}
\begin{frame}
\frametitle{闭环控制}
\label{sec-1-3-1}


\begin{tikzpicture}[node distance=2em,auto,>=latex', thick]
%\path[use as bounding box] (-1,0) rectangle (10,-2); 
\path[->] node[] (r) {期望}; 
\path[->] node[ circle,inner sep=2pt,minimum size=1pt,draw,label=below left:$ $,right =of r] (p1) { }; 
\path[->](r) edge node {} (p1) ; 
\path[blue] node[draw, right =of p1] (n) {控制器}; 
\path[->] (p1) edge node[midway] {偏差} (n) ; 
\path[red] node[draw, inner sep=5pt,right =of n] (g) {被控对象}; 
\path[->] (n) edge node [midway]{$ $} (g); 
\path[->] node[ right =of g] (o) {输出}; 
\path[->] (g) edge node {} (o); 
\path[blue] node[draw, below =of g] (h) {传感器};
\path[->,draw] (g.east)+(1em,0) |- (h.east) ; 
\path[->,draw] (h.west) -| (p1) ; 
\end{tikzpicture} 

\begin{itemize}
\item <2-> 定义: 闭环控制是指在输出量处,通过 \textbf{反馈} 回路使得输出量对输入量施加影响
\item <3-> 控制目的:通过在输入端引入输出量,使得输入处的偏差 $\rightarrow0$
\item <4->闭环控制按偏差进行调节。
\end{itemize}
\end{frame}
\begin{frame}
\frametitle{反馈}
\label{sec-1-3-2}

\begin{itemize}
\item 反馈: 指将系统的输出返回到输入端并以某种方式改变输入,进而影响系统功能的过程。
\item 正反馈: 输出变化时,反馈对输出造成的影响与输出变化趋势相同
\item 负反馈:输出变化时,反馈对输出造成的影响与输出变化趋势相反
\end{itemize}
\end{frame}
\begin{frame}
\frametitle{示例:人手工竖杆}
\label{sec-1-3-3}



\begin{tikzpicture}[node distance=2em,auto,>=latex', thick]
%\path[use as bounding box] (-1,0) rectangle (10,-2); 
\path[->] node[] (r) {0}; 
\path[->] node[ circle,inner sep=2pt,minimum size=1pt,draw,label=below left:$ $,right =of r] (p1) { }; 
\path[->](r) edge node {} (p1) ; 
\path[blue] node[draw, right =of p1] (n) {脑}; 
\path[->] (p1) edge node[midway] {偏差} (n) ; 
\path[blue] node[draw, right =of n] (d) {手}; 
\path[->] (n) edge node[midway] {} (d) ; 
\path[red] node[draw, inner sep=5pt,right =of d] (g) {杆}; 
\path[->] (d) edge node [midway]{$ $} (g); 
\path[->] node[ right =of g] (o) {$\theta$}; 
\path[->] (g) edge node {} (o); 
\path[blue] node[draw, below =of d] (h) {眼};
\path[->,draw] (g.east)+(1em,0) |- (h.east) ; 
\path[->,draw] (h.west) -| (p1) ; 
\end{tikzpicture} 


\begin{itemize}
\item 反馈通道:眼
\item 执行机构:手
\item 被控制量:杆与竖直方向夹角  $\theta\rightarrow 0$
\end{itemize}
\end{frame}
\begin{frame}
\frametitle{示例:倒立摆系统}
\label{sec-1-3-4}



\begin{tikzpicture}[node distance=2em,auto,>=latex', thick]
%\path[use as bounding box] (-1,0) rectangle (10,-2); 
\path[blue] node[draw, right =of n] (d) {电机}; 
\path[red] node[draw, inner sep=5pt,right =of d] (g) {小车}; 
\path[red,draw] (g.south)+(-0.7em,-0.25em) circle (0.25em) (g.south)+(0.7em,-0.25em) circle (0.25em);
\path[red,draw] (g.north)--+(60:3em);
\path[draw,dashed] (g.north)--+(90:3em);
\path[draw,dashed] (g.north)++(90:2.5em) arc (90:60:2.5em);
\path  (g.north)+(75:3em) node {$\theta$};
\path[] (d) edge node [midway]{$ $} (g); 
\path[blue] node[draw, right =of g] (h) {传感器};
\path[] (g) edge node {$\theta,r$}(h) ; 
\path[red,draw] (d.south)|-($(g.south)+(0,-0.51em)$)-| (h.south);

\path[blue] node[draw, below =of g] (n) {控制器}; 
\path[<-,draw] (d.west)--+(-1em,0) |- (n.west) ; 
\path[->,draw] (h.east)--+(1em,0) |- (n.east) ; 
\end{tikzpicture} 


\begin{itemize}
\item 执行机构:电机
\item 反馈通道:角度传感器、位置传感器
\item 被控制量: $\theta\rightarrow 0, r\rightarrow 0$
\end{itemize}
\end{frame}
\begin{frame}
\frametitle{示例:直流电机速度反馈控制系统}
\label{sec-1-3-5}



\begin{tikzpicture}[node distance=2em,auto,>=latex', thick]
%\path[use as bounding box] (-1,0) rectangle (10,-2); 
\path[->] node[] (r) {$U_g$}; 
\path[->] node[ circle,inner sep=2pt,minimum size=1pt,draw,label=below left:$ $,right =of r] (p1) { }; 
\path[->](r) edge node {} (p1) ; 
\path[blue] node[draw, right =of p1] (n) {放大器}; 
\path[->] (p1) edge node[midway] {$U_d$} (n) ; 
\path[red] node[draw, inner sep=5pt,right =of n] (d) {驱动电路}; 
\path[->] (n) edge node [midway]{$ $} (d); 
\path[red] node[draw, inner sep=5pt,right =of d] (g) {电机}; 
\path[->] (d) edge node [midway]{$ $} (g); 
\path[red] node[draw, inner sep=5pt,below =of g] (s) {测速电机};
\path[red] (g) edge node {$n$} (s); 
\path[->,draw] (s.west) -| node[near start] {$U_f$} node[very near end] {$-$} (p1) ; 
\end{tikzpicture} 

\begin{eqnarray}
  n  &=& K U_d \\
  U_d &=& U_g-U_f \\
  U_f &=& K' n
\end{eqnarray}

负载增大后: $n\downarrow\rightarrow U_f\downarrow\rightarrow U_d\uparrow\rightarrow n\uparrow$
\end{frame}
\begin{frame}
\frametitle{闭环控制的特点}
\label{sec-1-3-6}

\begin{enumerate}
\item <2->按偏差进行调节
\item <3->控制精度较高,取决于反馈通道元器件的精度,而反馈通道所包围的电路中的元器件的元件精度可降低
\item <4->抗干扰能力强
\end{enumerate}
\end{frame}
\begin{frame}
\frametitle{复合控制}
\label{sec-1-3-7}

扰动补偿+闭环控制

例:直流电机速度复合控制

\begin{tikzpicture}[node distance=2em,auto,>=latex', thick]
\tikzstyle{every node}=[font=\small]
%\path[use as bounding box] (-1,0) rectangle (10,-2); 
\path[->] node[] (r) {$U_g$}; 
\path[->] node[ circle,inner sep=2pt,minimum size=1pt,draw,label=below left:$ $,right =of r] (p1) { }; 
\path[->](r) edge node {} (p1) ; 
\path[blue] node[draw, right =of p1] (n) {放大器}; 
\path[->] (p1) edge node[midway] {$U_d$} (n) ; 
\path[->] node[ circle,inner sep=2pt,minimum size=1pt,draw,label=below left:$ $,right =of n] (p2) { }; 
\path[->](n) edge node {} (p2) ; 
\path[red] node[draw, inner sep=5pt,right =of p2] (d) {驱动电路}; 
\path[->] (p2) edge node [midway]{$ $} (d); 
\path[red] node[draw, inner sep=5pt,right =of d] (g) {电机}; 
\path[->] (d) edge node [midway]{$ $} (g); 
\path[red] node[draw, inner sep=5pt,below =of g] (s) {测速电机};
\path[red] (g) edge node {$n$} (s); 
\path[->,draw] (s.west) -| node[near start] {$U_f$} node[very near end] {$-$} (p1) ; 

\path[blue] node[draw, above =of g] (l) {负载扰动};
\path[dashed,draw] (g.north) edge (l) ; 
\path[blue] node[draw, left =of l] (h) {扰动测量};
\path[->,draw] (l) edge  node[midway] {$i$} (h) ; 
\path[->,draw] (h.west) -| node[near end]{$U_b$} (p2) ; 
\end{tikzpicture} 
\end{frame}
\section{信号与系统}
\label{sec-2}
\subsection{概念与分类}
\label{sec-2-1}
\begin{frame}
\frametitle{基本概念}
\label{sec-2-1-1}

\begin{itemize}
\item 信号: 随时间和空间变化的某种物理量.
\begin{itemize}
\item 信号通常是时间变量 $t$ 的函数
\item 信号的特性可从两方面来描述
\begin{itemize}
\item 时域特性
\item 频域特性
\end{itemize}
\end{itemize}
\item 系统: 能够对信号完成某种变换或运算功能的集合体称为系统
\end{itemize}
\end{frame}
\begin{frame}
\frametitle{闭环系统组成}
\label{sec-2-1-2}


\begin{tikzpicture}[node distance=2em,auto,>=latex', thick ]
\tikzstyle{every node}=[font=\small]
%\path[use as bounding box] (-1,0) rectangle (10,-2); 
\path[->] node[text width =1em] (r) {给定}; 
\path[->] node[ circle,inner sep=2pt,minimum size=1pt,draw,label=below left:$ $,right =of r] (p1) { }; 
\path[->](r) edge node {} (p1) ; 
\path[blue] node[text width=1em,draw, right =of p1] (n) {串联校正}; 
\path[->] (p1) edge node[midway] {} (n) ; 
\path[->] node[ circle,inner sep=2pt,minimum size=1pt,draw,label=below left:$ $,right =of n] (p2) { }; 
\path[->](n) edge node {} (p2) ; 
\path[red] node[draw, inner sep=5pt,right =of p2] (a) {放大}; 
\path[->] (p2) edge node[midway] {} (a) ; 
\path[red] node[draw, inner sep=5pt,right =of a] (e) {执行}; 
\path[->] (a) edge node[midway] {} (e) ; 
\path[red] node[draw, text width=1em,inner sep=5pt,right =of e] (g) {被控对象}; 
\path[->] (e) edge node [midway]{$ $} (g); 
\path[->] node[ right =of g,text width=1em] (o) {输出}; 
\path[->] (g) edge node {} (o); 

\path[blue] node[draw, below =of a] (l) {局部反馈};
\path[->,draw] (e.east)+(1em,0) |- (l.east) ; 
\path[->,draw] (l.west) -| node[very near end] {$-$}(p2) ; 

\path[blue] node[draw, below =of l] (h) {主反馈};
\path[->,draw] (g.east)+(1em,0) |- (h.east) ; 
\path[->,draw] (h.west) -| node[very near end] {$-$}(p1) ; 
\end{tikzpicture} 
\end{frame}
\begin{frame}
\frametitle{闭环系统中的信号}
\label{sec-2-1-3}

\begin{itemize}
\item <2->输入信号:给定信号及干扰信号
\item <2->输出信号:被控量的物理量
\item <3->反馈信号:反馈元部件的输出
\item <4->偏差信号:给定信号与主反馈信号之差
\item <4->误差信号:输出量的希望值与实际值之差
\item <5->干扰信号:系统受到的内外干扰
\end{itemize}
\end{frame}
\begin{frame}
\frametitle{典型信号}
\label{sec-2-1-4}

\begin{itemize}
\item <2->阶跃信号(函数)  $r(t)=\begin{cases} A & t\geq 0 \\ 0 & t < 0 \end{cases}$
\item <3->脉冲信号(函数)  $r(t)=\begin{cases}\frac{A}{\epsilon}  & 0\leq t\leq \epsilon\\ 0 & others\end{cases}$
\item <4->正弦信号(函数)  $r(t)=A\sin(\omega t), t>0$
\item <5->斜坡信号(函数)  $r(t)=Vt  ,     t>0$
\item <6->加速度信号(函数)$r(t)=\frac{1}{2}at^2,  t>0$
\end{itemize}
\end{frame}
\begin{frame}
\frametitle{按给定量的运动规律分类}
\label{sec-2-1-5}

\begin{enumerate}
\item <2->镇定系统:输入 $r(t)$ 不变
\item <3->程序控制系统:输入 $r(t)$ 按规律变化
\item <4->随动系统:输入 $r(t)$ 随机变化
\end{enumerate}
\end{frame}
\begin{frame}
\frametitle{按系统性能分类}
\label{sec-2-1-6}

\begin{enumerate}
\item <2->线性系统和非线性系统
\begin{itemize}
\item 线性系统: 系统的输入和输出因果关系可以用线性微分方程描述
\item 非线性系统: $r(t)$ 和 $c(t)$ 关系只能用非线性方程描述
\end{itemize}
\item <3->定常系统与时变系统
\begin{itemize}
\item 定常系统:微分方程中各项系数为常数 $a_0c''(t)+a_1c'(t)=r(t)$
\item 时变系统:各项系数中有随时间变化的量 $a_0(t)c''(t)+a_1(t)c'(t)=r(t)$
\end{itemize}
\item <4->连续系统与离散系统
\begin{itemize}
\item 连续系统:系统中信号是时间t的连续函数的模拟量
\item 离散系统:系统中存在脉冲量或数字信号
\end{itemize}
\item <5->确定性和不确定性系统
    确定性系统:系统中微分方程参数变化是精确可知的
    不确定性系统:参数变化只是部分可知或近似可知
\end{enumerate}
\end{frame}
\subsection{控制系统基本要求}
\label{sec-2-2}
\begin{frame}
\frametitle{基本要求:稳定性、稳态性能、瞬态性能}
\label{sec-2-2-1}

\begin{enumerate}
\item <2->稳定性(稳): 正常工作的先决条件
\item <3->稳态性能(准): 指标:稳态误差
\item <4->瞬态性能(快):
\begin{enumerate}
\item <5->峰值时间:$t_p$
\item <6->调节时间:$t_s$
\item <7->超调量:$\sigma \% = \frac{c(t_p)-c(\infty)}{c(\infty)}$
\end{enumerate}
\end{enumerate}
\end{frame}
\begin{frame}
\frametitle{示例:响应曲线}
\label{sec-2-2-2}
\begin{itemize}

\item 初始值:0,期望值:1\\
\label{sec-2-2-2-1}%
\begin{tikzpicture}[scale=2]
\coordinate (o) at (0,0);
\coordinate (ox) at (3,0);
\coordinate (oy) at (0,1.5);
\draw[->] (o) -- (ox);
\draw[->] (o) -- (oy);
\draw (o) node[below] {$o$};
\draw [red,thick,smooth] plot coordinates {(0,0) (1,1) (1.5,1.2) (2,1.05) (2.5,0.95) (3,1)};
\draw[thick,blue,dashed] (0,1) -- (3,1);
\draw[thick,violet,dashed] (0,0.95) -- (3,0.95);
\draw[thick,violet,dashed] (0,1.05) -- (3,1.05);
\draw[thick,red,dashed] (1.5,1.2) -- (1.5,0);\draw (1.5,0) node[below] {$t_p$};
\draw[thick,red,dashed] (2,1.05) -- (2,0);\draw (2,0) node[below] {$t_s$};
\draw (o) node[left] {$0$}
;\draw (0,1) node[left] {$1$};
\end{tikzpicture}


\item 指标:\\
\label{sec-2-2-2-2}%
\begin{itemize}
\item <2->超调量 $\sigma\%$
\item <3->调节时间 $t_s$
\item <4->上升时间 $t_r$
\begin{itemize}
\item $100\%$ 的 $t_r$
\item $90\%$ 的 $t_r$
\item $70\%$ 的 $t_r$
\end{itemize}
\item <5->峰值时间 $t_p$
\item <6->振荡次数
\end{itemize}

\end{itemize} % ends low level
\end{frame}
\begin{frame}
\frametitle{指标}
\label{sec-2-2-3}

\begin{itemize}
\item 超调量:  $(c(t_p)-c(\infty))/c(\infty)$
\item 调节时间: 若有 $t_s$ ,当 $t\geq t_s$ 时有 $|c(t)-c(\infty)|\leq 0.05c(\infty)$ (或 $0.03c(\infty)$ )成立,则 $t_s$ 为该系统调节时间。
\item 上升时间 $t_r$ ,定义
\begin{itemize}
\item $100\%$ 的 $t_r,c(t)$ 首次达到 $c(\infty)$ 的时间
\item $90\%$ 的 $t_r,c(t)$ 首次达到 $90\%c(\infty)$ 的时间
\item $70\%$ 的 $t_r,c(t)$ 首次达到 $70\%c(\infty)$ 的时间
\end{itemize}
\item 峰值时间 $t_p$ : $c(t_p)=Max(c(t))$
\item 振荡次数:在 $t<=t_s$ 期间, $c(t)$ 围绕 $c(\infty)$ 上下振荡的次数
\end{itemize}
\end{frame}
\begin{frame}
\frametitle{课程内容:}
\label{sec-2-2-4}

\begin{enumerate}
\item <2->一般概念
\item <3->数学模型
\item <4->分析方法
\begin{enumerate}
\item 时域分析法
\item 根轨迹法
\item 频域分析法
\end{enumerate}
\end{enumerate}
\end{frame}

\end{document}
