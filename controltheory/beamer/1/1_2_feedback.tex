% Created 2014-10-09 星期四 10:31
\documentclass{beamer}
\usepackage[utf8]{inputenc}
\usepackage[T1]{fontenc}
\usepackage{fixltx2e}
\usepackage{graphicx}
\usepackage{longtable}
\usepackage{float}
\usepackage{wrapfig}
\usepackage{soul}
\usepackage{textcomp}
\usepackage{marvosym}
\usepackage{wasysym}
\usepackage{latexsym}
\usepackage{amssymb}
\usepackage{hyperref}
\tolerance=1000
\usepackage{etex}
\usepackage{amsmath}
\usepackage{pstricks}
\usepackage{pgfplots}
\usepackage{tikz}
\usepackage[europeanresistors,americaninductors]{circuitikz}
\usepackage{colortbl}
\usepackage{yfonts}
\usetikzlibrary{shapes,arrows}
\usetikzlibrary{positioning}
\usetikzlibrary{arrows,shapes}
\usetikzlibrary{intersections}
\usetikzlibrary{calc,patterns,decorations.pathmorphing,decorations.markings}
\usepackage[BoldFont,SlantFont,CJKchecksingle]{xeCJK}
\setCJKmainfont[BoldFont=Evermore Hei]{Evermore Kai}
\setCJKmonofont{Evermore Kai}
\usepackage{pst-node}
\usepackage{pst-plot}
\psset{unit=5mm}
\mode<beamer>{\usetheme{Frankfurt}}
\mode<beamer>{\usecolortheme{dove}}
\mode<article>{\hypersetup{colorlinks=true,pdfborder={0 0 0}}}
\AtBeginSection[]{\begin{frame}<beamer>\frametitle{Topic}\tableofcontents[currentsection]\end{frame}}
\setbeamercovered{transparent}
\providecommand{\alert}[1]{\textbf{#1}}

\title{自动控制的基本概念}
\author{}
\date{}
\hypersetup{
  pdfkeywords={},
  pdfsubject={},
  pdfcreator={Emacs Org-mode version 7.9.3f}}

\begin{document}

\maketitle

\begin{frame}
\frametitle{Outline}
\setcounter{tocdepth}{3}
\tableofcontents
\end{frame}
















\section{开环控制}
\label{sec-1}
\begin{frame}
\frametitle{开环控制}
\label{sec-1-1}

\begin{itemize}
\item <2->定义:开环控制是指控制器与被控对象之间只有顺向作用而没有反向联系,称为开环控制。
\item <3->系统的输出量对系统的输入量无影响
\item <4->开环系统对控制偏差无修正能力。
\begin{itemize}
\item <5->按给定量控制
\item <6->按扰动量控制
\end{itemize}
\end{itemize}
\end{frame}
\begin{frame}
\frametitle{按给定量控制}
\label{sec-1-2}



\begin{tikzpicture}[node distance=2em,auto,>=latex', thick]
%\path[use as bounding box] (-1,0) rectangle (10,-2); 
\path[->] node[] (r) {$U_g$}; 
%\path[->] node[ circle,inner sep=2pt,minimum size=1pt,draw,label=below left:$ $,right =of r] (p1) { }; 
%\path[->](r) edge node {} (p1) ; 
\path[blue] node[draw, right =of r] (n) {信号变换与驱动电路}; 
\path[->] (r) edge node[midway] {} (n) ; 
\path[red] node[draw, inner sep=5pt,right =of n] (g) {电机}; 
\path[->] (n) edge node [midway]{$ $} (g); 
\path[->] node[ right =of g] (o) {$n$}; 
\path[->] (g) edge node {} (o); 
%\path[blue] node[draw, below =of g] (h) {传感器};
%\path[->,draw] (g.east)+(1em,0) |- (h.east) ; 
%\path[->,draw] (h.west) -| (p1) ; 
\end{tikzpicture} 

\begin{itemize}
\item 输入量: 电压 $U_g$
\item 输出量: 电机转速 $n$
\item $n=kU_g$
\end{itemize}
\end{frame}
\begin{frame}
\frametitle{按扰动量控制}
\label{sec-1-3}

   对扰动进行补偿,使扰动的影响减小


\begin{tikzpicture}[node distance=2em,auto,>=latex', thick]
%\path[use as bounding box] (-1,0) rectangle (10,-2); 
\path[->] node[] (r) {$U_0$}; 
\path[->] node[ circle,inner sep=2pt,minimum size=1pt,draw,label=below left:$ $,right =of r] (p1) { }; 
\path[->](r) edge node {} (p1) ; 
\path[blue] node[draw, right =of p1] (n) {驱动电路}; 
\path[->] (p1) edge node[midway] {$U_c$} (n) ; 
\path[red] node[draw, inner sep=5pt,right =of n] (g) {电机}; 
\path[->] (n) edge node [midway]{$ $} (g); 
\path[->] node[ right =of g] (o) {$n$}; 
\path[->] (g) edge node {} (o); 
\path[blue] node[draw, below =of g] (l) {负载扰动};
\path[dashed,draw] (g.south) edge (l) ; 
\path[blue] node[draw, below =of n] (h) {扰动测量};
\path[->,draw] (l) edge  node[midway] {$i$} (h) ; 
\path[->,draw] (h.west) -| node[midway]{$U_b$} (p1) ; 
\end{tikzpicture} 

\begin{itemize}
\item $U_c=U_0+U_b$
\item 负载增加导致 $n\downarrow , i\uparrow$
\item $i\uparrow\rightarrow U_b\uparrow\rightarrow U_c\uparrow\rightarrow n\uparrow$
\end{itemize}
\end{frame}
\begin{frame}
\frametitle{开环控制特点}
\label{sec-1-4}

\begin{enumerate}
\item <2->优点:原理简单,结构简单,反应速度快,灵敏度高
\item <3->缺点:
\begin{itemize}
\item 对控制偏差无修正能力
\item 控制精度取决于各控制元器件的精度
\end{itemize}
\item <4->适应场合:对控制精度要求不高的系统
\item <5-> 结构图:   输入 $\rightarrow$ 控制器 $\rightarrow$ 被控对象 $\rightarrow$ 输出  (顺向作用)
\end{enumerate}
\end{frame}
\section{闭环控制}
\label{sec-2}
\begin{frame}
\frametitle{闭环控制}
\label{sec-2-1}


\begin{tikzpicture}[node distance=2em,auto,>=latex', thick]
%\path[use as bounding box] (-1,0) rectangle (10,-2); 
\path[->] node[] (r) {期望}; 
\path[->] node[ circle,inner sep=2pt,minimum size=1pt,draw,label=below left:$ $,right =of r] (p1) { }; 
\path[->](r) edge node {} (p1) ; 
\path[blue] node[draw, right =of p1] (n) {控制器}; 
\path[->] (p1) edge node[midway] {偏差} (n) ; 
\path[red] node[draw, inner sep=5pt,right =of n] (g) {被控对象}; 
\path[->] (n) edge node [midway]{$ $} (g); 
\path[->] node[ right =of g] (o) {输出}; 
\path[->] (g) edge node {} (o); 
\path[blue] node[draw, below =of g] (h) {传感器};
\path[->,draw] (g.east)+(1em,0) |- (h.east) ; 
\path[->,draw] (h.west) -| (p1) ; 
\end{tikzpicture} 

\begin{itemize}
\item <2-> 定义: 闭环控制是指在输出量处,通过 \textbf{反馈} 回路使得输出量对输入量施加影响
\item <3-> 控制目的:通过在输入端引入输出量,使得输入处的偏差 $\rightarrow0$
\item <4->闭环控制按偏差进行调节。
\end{itemize}
\end{frame}
\begin{frame}
\frametitle{反馈}
\label{sec-2-2}

\begin{itemize}
\item 反馈: 指将系统的输出返回到输入端并以某种方式改变输入,进而影响系统功能的过程。
\item 正反馈: 输出变化时,反馈对输出造成的影响与输出变化趋势相同
\item 负反馈:输出变化时,反馈对输出造成的影响与输出变化趋势相反
\end{itemize}
\end{frame}
\begin{frame}
\frametitle{示例:人手工竖杆}
\label{sec-2-3}
\begin{columns}
\begin{column}{0.3\textwidth}
\begin{block}{示意图}
\label{sec-2-3-1}




%LaTeX with PSTricks extensions
%%Creator: 0.48.2
%%Please note this file requires PSTricks extensions
\psset{xunit=.5pt,yunit=.5pt,runit=.5pt}
\begin{pspicture}(83.9397583,146.67623901)
{
\newrgbcolor{curcolor}{0 0 0}
\pscustom[linewidth=1,linecolor=curcolor]
{
\newpath
\moveto(77.59144242,73.33811721)
\curveto(77.59144242,66.87824605)(71.75959493,61.64148578)(64.56564648,61.64148578)
\curveto(57.37169803,61.64148578)(51.53985055,66.87824605)(51.53985055,73.33811721)
\curveto(51.53985055,79.79798837)(57.37169803,85.03474864)(64.56564648,85.03474864)
\curveto(71.75959493,85.03474864)(77.59144242,79.79798837)(77.59144242,73.33811721)
\closepath
}
}
{
\newrgbcolor{curcolor}{0 0 0}
\pscustom[linewidth=1,linecolor=curcolor]
{
\newpath
\moveto(63.7681501,61.64148901)
\lineto(64.2998101,25.48826201)
\lineto(39.3115501,0.49999901)
}
}
{
\newrgbcolor{curcolor}{0 0 0}
\pscustom[linewidth=1,linecolor=curcolor]
{
\newpath
\moveto(64.2998101,24.95659701)
\lineto(80.7814301,4.75332301)
}
}
{
\newrgbcolor{curcolor}{0 0 0}
\pscustom[linewidth=1,linecolor=curcolor]
{
\newpath
\moveto(83.4397601,36.65322901)
\lineto(64.2998101,49.41319201)
\lineto(36.1215601,36.12156401)
\lineto(0.5000001,146.17624001)
}
}
\end{pspicture}

\end{block}
\end{column}
\begin{column}{0.7\textwidth}
\begin{block}{分析}
\label{sec-2-3-2}

\begin{tikzpicture}[node distance=2em,auto,>=latex', thick]
%\path[use as bounding box] (-1,0) rectangle (10,-2); 
\path[->] node[] (r) {0}; 
\path[->] node[ circle,inner sep=2pt,minimum size=1pt,draw,label=below left:$ $,right =of r] (p1) { }; 
\path[->](r) edge node {} (p1) ; 
\path[blue] node[draw, right =of p1] (n) {脑}; 
\path[->] (p1) edge node[midway] {偏差} (n) ; 
\path[blue] node[draw, right =of n] (d) {手}; 
\path[->] (n) edge node[midway] {} (d) ; 
\path[red] node[draw, inner sep=5pt,right =of d] (g) {杆}; 
\path[->] (d) edge node [midway]{$ $} (g); 
\path[->] node[ right =of g] (o) {$\theta$}; 
\path[->] (g) edge node {} (o); 
\path[blue] node[draw, below =of d] (h) {眼};
\path[->,draw] (g.east)+(1em,0) |- (h.east) ; 
\path[->,draw] (h.west) -| (p1) ; 
\end{tikzpicture} 


\begin{itemize}
\item 反馈通道:眼
\item 执行机构:手
\item 被控制量:杆与竖直方向夹角  $\theta\rightarrow 0$
\end{itemize}
\end{block}
\end{column}
\end{columns}
\end{frame}
\begin{frame}
\frametitle{示例:倒立摆系统}
\label{sec-2-4}



\begin{tikzpicture}[node distance=2em,auto,>=latex', thick]
%\path[use as bounding box] (-1,0) rectangle (10,-2); 
\path[blue] node[draw, right =of n] (d) {电机}; 
\path[red] node[draw, inner sep=5pt,right =of d] (g) {小车}; 
\path[red,draw] (g.south)+(-0.7em,-0.25em) circle (0.25em) (g.south)+(0.7em,-0.25em) circle (0.25em);
\path[red,draw] (g.north)--+(60:3em);
\path[draw,dashed] (g.north)--+(90:3em);
\path[draw,dashed] (g.north)++(90:2.5em) arc (90:60:2.5em);
\path  (g.north)+(75:3em) node {$\theta$};
\path[] (d) edge node [midway]{$ $} (g); 
\path[blue] node[draw, right =of g] (h) {传感器};
\path[] (g) edge node {$\theta,r$}(h) ; 
\path[red,draw] (d.south)|-($(g.south)+(0,-0.51em)$)-| (h.south);

\path[blue] node[draw, below =of g] (n) {控制器}; 
\path[<-,draw] (d.west)--+(-1em,0) |- (n.west) ; 
\path[->,draw] (h.east)--+(1em,0) |- (n.east) ; 
\end{tikzpicture} 


\begin{itemize}
\item 执行机构:电机
\item 反馈通道:角度传感器、位置传感器
\item 被控制量: $\theta\rightarrow 0, r\rightarrow 0$
\end{itemize}
\end{frame}
\begin{frame}
\frametitle{示例:直流电机速度反馈控制系统}
\label{sec-2-5}



\begin{tikzpicture}[node distance=2em,auto,>=latex', thick]
%\path[use as bounding box] (-1,0) rectangle (10,-2); 
\path[->] node[] (r) {$U_g$}; 
\path[->] node[ circle,inner sep=2pt,minimum size=1pt,draw,label=below left:$ $,right =of r] (p1) { }; 
\path[->](r) edge node {} (p1) ; 
\path[blue] node[draw, right =of p1] (n) {放大器}; 
\path[->] (p1) edge node[midway] {$U_d$} (n) ; 
\path[red] node[draw, inner sep=5pt,right =of n] (d) {驱动电路}; 
\path[->] (n) edge node [midway]{$ $} (d); 
\path[red] node[draw, inner sep=5pt,right =of d] (g) {电机}; 
\path[->] (d) edge node [midway]{$ $} (g); 
\path[red] node[draw, inner sep=5pt,below =of g] (s) {测速电机};
\path[red] (g) edge node {$n$} (s); 
\path[->,draw] (s.west) -| node[near start] {$U_f$} node[very near end] {$-$} (p1) ; 
\end{tikzpicture} 

\begin{eqnarray}
  n  &=& K U_d \\
  U_d &=& U_g-U_f \\
  U_f &=& K' n
\end{eqnarray}

负载增大后: $n\downarrow\rightarrow U_f\downarrow\rightarrow U_d\uparrow\rightarrow n\uparrow$
\end{frame}
\begin{frame}
\frametitle{负反馈放大器}
\label{sec-2-6}

\begin{tikzpicture}[node distance=2em,auto,>=latex', thick]
%\path[use as bounding box] (-1,0) rectangle (10,-2); 
\path[->] node[] (r) {r}; 
\path[->] node[ circle,inner sep=2pt,minimum size=1pt,draw,label=below left:$ $,right =of r] (p1) { }; 
\path[->](r) edge node {} (p1) ; 
\path[blue] node[draw, right =of p1] (g) {$K$}; 
\path[->] (p1) edge node[midway] {E} (g) ; 
\path[->] node[ circle,inner sep=2pt,minimum size=1pt,draw,label=below left:$ $,right =of g] (p2) { }; 
\path[->] (g) edge node [midway]{$ $} (p2); 
\path[->] node[ right =of p2] (o) {$c$}; 
\path[->] (p2) edge node {} (o); 

\path[blue] node[draw, below =of g] (h) {$A$};
\path[->,draw] (g.east)+(1em,0) |- (h.east) ; 
\path[->,draw] (h.west) -| (p1) ; 

\path[blue] node[ above =of p2] (f) {$f$};
\path[->] (f) edge node [midway]{$ $} (p2); 
\end{tikzpicture} 

\begin{eqnarray}
c &=& Ke+f \\
e &=& r-Ac \\
c &=& \frac{Kc}{1+KA}+\frac{f}{1+KA}
\end{eqnarray}
\end{frame}
\begin{frame}
\frametitle{闭环控制的特点}
\label{sec-2-7}

\begin{enumerate}
\item <2->按偏差进行调节
\item <3->控制精度较高,取决于反馈通道元器件的精度,而反馈通道所包围的电路中的元器件的元件精度可降低
\item <4->抗干扰能力强
\end{enumerate}
\end{frame}
\begin{frame}
\frametitle{复合控制}
\label{sec-2-8}

扰动补偿+闭环控制

例:直流电机速度复合控制

\begin{tikzpicture}[node distance=2em,auto,>=latex', thick]
\tikzstyle{every node}=[font=\small]
%\path[use as bounding box] (-1,0) rectangle (10,-2); 
\path[->] node[] (r) {$U_g$}; 
\path[->] node[ circle,inner sep=2pt,minimum size=1pt,draw,label=below left:$ $,right =of r] (p1) { }; 
\path[->](r) edge node {} (p1) ; 
\path[blue] node[draw, right =of p1] (n) {放大器}; 
\path[->] (p1) edge node[midway] {$U_d$} (n) ; 
\path[->] node[ circle,inner sep=2pt,minimum size=1pt,draw,label=below left:$ $,right =of n] (p2) { }; 
\path[->](n) edge node {} (p2) ; 
\path[red] node[draw, inner sep=5pt,right =of p2] (d) {驱动电路}; 
\path[->] (p2) edge node [midway]{$ $} (d); 
\path[red] node[draw, inner sep=5pt,right =of d] (g) {电机}; 
\path[->] (d) edge node [midway]{$ $} (g); 
\path[red] node[draw, inner sep=5pt,below =of g] (s) {测速电机};
\path[red] (g) edge node {$n$} (s); 
\path[->,draw] (s.west) -| node[near start] {$U_f$} node[very near end] {$-$} (p1) ; 

\path[blue] node[draw, above =of g] (l) {负载扰动};
\path[dashed,draw] (g.north) edge (l) ; 
\path[blue] node[draw, left =of l] (h) {扰动测量};
\path[->,draw] (l) edge  node[midway] {$i$} (h) ; 
\path[->,draw] (h.west) -| node[near end]{$U_b$} (p2) ; 
\end{tikzpicture} 
\end{frame}

\end{document}
