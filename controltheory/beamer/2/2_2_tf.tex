% Created 2015-09-16 Wed 13:11
\documentclass{beamer}
\usepackage{fixltx2e}
\usepackage{graphicx}
\usepackage{longtable}
\usepackage{float}
\usepackage{wrapfig}
\usepackage{soul}
\usepackage{textcomp}
\usepackage{marvosym}
\usepackage{wasysym}
\usepackage{latexsym}
\usepackage{amssymb}
\usepackage{hyperref}
\tolerance=1000
\usepackage{etex}
\usepackage{amsmath}
\usepackage{pstricks}
\usepackage{pgfplots}
\usepackage{tikz}
\usepackage[europeanresistors,americaninductors]{circuitikz}
\usepackage{colortbl}
\usepackage{yfonts}
\usetikzlibrary{shapes,arrows}
\usetikzlibrary{positioning}
\usetikzlibrary{arrows,shapes}
\usetikzlibrary{intersections}
\usetikzlibrary{calc,patterns,decorations.pathmorphing,decorations.markings}
\usepackage[BoldFont,SlantFont,CJKchecksingle]{xeCJK}
\setCJKmainfont[BoldFont=Evermore Hei]{Evermore Kai}
\setCJKmonofont{Evermore Kai}
\usepackage{pst-node}
\usepackage{pst-plot}
\psset{unit=5mm}
\mode<beamer>{\usetheme{Frankfurt}}
\mode<beamer>{\usecolortheme{dove}}
\mode<article>{\hypersetup{colorlinks=true,pdfborder={0 0 0}}}
\mode<beamer>{\AtBeginSection[]{\begin{frame}<beamer>\frametitle{Topic}\tableofcontents[currentsection]\end{frame}}}
\setbeamercovered{transparent}
\subtitle{控制系统的复域数学模型}
\mode<article>{\renewcommand{\labelitemii}{$\cdot$}}
\providecommand{\alert}[1]{\textbf{#1}}

\title{自动控制系统的数学模型}
\author{}
\date{}
\hypersetup{
  pdfkeywords={},
  pdfsubject={},
  pdfcreator={Emacs Org-mode version 7.9.3f}}

\begin{document}

\maketitle

\begin{frame}
\frametitle{Outline}
\setcounter{tocdepth}{3}
\tableofcontents
\end{frame}













\section{Laplace变换}
\label{sec-1}
\begin{frame}
\frametitle{概念}
\label{sec-1-1}

\begin{itemize}
\item <2->定义
      \begin{eqnarray*}
      {\cal L}[F(t)] &=& F(s) \\
       &=& \int_0^{+\infty}f(t)e^{-st}dt
      \end{eqnarray*}
\item <3->作用:将微积分运算变成代数运算
\end{itemize}
\end{frame}
\begin{frame}
\frametitle{常用函数的Laplace变换}
\label{sec-1-2}

\begin{itemize}
\item <2->单位脉冲函数 $f(t)=\delta(t) \rightarrow   F(s)=1$
\item <3->阶跃函数 $f(t)=A,(t\geq 0) \rightarrow   F(s)=\frac{A}{s}$
\item <4->斜坡函数(速度)  $f(t)=vt,(t\geq0) \rightarrow F(s)=\frac{v}{s^2}$
\item <5->加速度函数  $f(t)=\frac{1}{2}at^2,(t\geq 0) \rightarrow  F(s)=\frac{a}{s^3}$
\item <6->指数函数 $f(t)=e^{at} \rightarrow  F(s)=\frac{1}{s-a}$
\item <7->正弦函数 $f(t)=A\sin(\omega t)\rightarrow F(s)=\frac{A\omega}{s^2+\omega^2}$
\end{itemize}
\end{frame}
\begin{frame}
\frametitle{Laplace变换的性质}
\label{sec-1-3}

\begin{itemize}
\item <2->线性: $f(t)=f_1(t)+f_2(t)\rightarrow  F(s)=F_1(s)+F_2(s)$
\item <3->衰减: $g(t)=f(t)e^{-at} \rightarrow G(s)=F(s+a)$
\item <4->延迟: $g(t)=f(t-a) \rightarrow  G(s)=F(s)e^{-as}$ , 一个信号经过 $e^{-as}$ 后,相当于对这个信号作延迟运算
\item <5->积分: $g(t)=\int_0^t f(\tau) d\tau \rightarrow  G(s)=\frac{F(s)}{s}$ , $\frac{1}{s}$ 相当于积分器
\item <6->微分: $g(t)=f(t)'\rightarrow  G(s)=sF(s)-f(0)$ , 零初始条件下, $G(s)=sF(s)$ , $s$ 相当于微分器
\item <7->初值定理: 若 $f(t)$ 在 $t=0$ 处无脉冲分量则 $f(0)=\lim_{t->0}f(t)=\lim_{s->\infty}sF(s)$
\item <8->终值定理: 若 $F(s)$ 极点全部在左半平面,则 $f(\infty)=lim_{t->\infty}f(t)=lim_{s->0}sF(s)$
\end{itemize}
\end{frame}
\begin{frame}
\frametitle{传递函数概念}
\label{sec-1-4}

\begin{itemize}
\item <2->概念:数学模型,从现阶段来讲,描述的是输入与输出的数学运算关系。 传递函数可以表示出系统的结构,可以用来研究系统结构,参数变化对系统性能的影响。
\item <3->定义:线性定常系统的传递函数是指在零初始条件下,系统输出量的拉氏变换与输入量的拉氏变换之比,记作:$G(s)=\frac{C(s)}{R(s)}$ 。
\end{itemize}
\end{frame}
\section{传递函数的零点与极点}
\label{sec-2}
\begin{frame}
\frametitle{传递函数的形式}
\label{sec-2-1}

\begin{itemize}
\item <2->传递函数有三种表达形式
\begin{itemize}
\item <3->分子分母多项式
\item <4->零极点形式
\item <5->典型环节形式
\end{itemize}
\end{itemize}
\end{frame}
\begin{frame}
\frametitle{分子分母多项式}
\label{sec-2-2}

\begin{itemize}
\item <2->在零初始条件下,对微分方程两端进行拉氏变换,有
     \begin{eqnarray*}
     \only<3->{ a_n c^{(n)}(t)+...+a_0c(t) &=& b_m r^{(m)}(t)+...+b_0r } \\
     \only<4->{ a_n s^n C(s)+...+a_0C(s) &=& b_m s^m R(s)+...+b_0 R(s) }\\
     \only<5->{ (a_n s^n+...+a_0)C(s) &=& (b_m s^m+...+b_0)R(s) }\\
     \only<6->{ G(s) &=& \frac{C(s)}{R(s)} }\\
     \only<7->{ &=& \frac{b_m s^m+b_{m-1}s^{m-1}+...+b_0}{a_n s^n+a_{n-1}s^{n-1}+...+a_0} }
     \end{eqnarray*}
\item <8->传递函数只与系统的结构和参数有关
\end{itemize}
\end{frame}
\begin{frame}
\frametitle{零极点形式}
\label{sec-2-3}

 $$G(s)=\frac{k_g\prod_{i=1}^m(s-z_i)}{\prod_{j=1}^n(s-p_j)}$$
\begin{itemize}
\item $k_g$ 根轨迹增益
\item $z_i$ 零点
\item $p_j$ 极点
\end{itemize}
\end{frame}
\begin{frame}
\frametitle{典型环节形式}
\label{sec-2-4}

  $$G(s)=\frac{K\prod_{i=1}^m(\tau_i s+1)}{s^{\nu}\prod_{j=1}^n(\tau_j s+1)}$$

\begin{itemize}
\item $K$ :系统增益
\end{itemize}
   
\end{frame}
\begin{frame}
\frametitle{传递函数的零极点对系统输出的影响}
\label{sec-2-5}

\begin{itemize}
\item <2-> $N(s)C(s)=M(s)R(s)$
\begin{itemize}
\item $M(s)=(b_m s^m+b_{m-1}s^{m-1}+...+b_0)$
\item $N(s)=(a_n s^n+a_{n-1}s^{n-1}+...+a_0)$
\item $G(s)=\frac{C(s)}{R(s)}=\frac{M(s)}{N(s)}$
\end{itemize}
\item <3-> 运动的模态
\begin{itemize}
\item 脉冲响应 $c(t)=c_1e^{\lambda_1 t}+...+c_ne^{\lambda_n t}$
\item 系统运动模态( $e^{\lambda_1 t},\cdots,e^{\lambda_n t}$ )由极点决定
\item 各模态所占比例与零点有关
\end{itemize}
\end{itemize}
\end{frame}
\section{典型环节传递函数}
\label{sec-3}
\begin{frame}
\frametitle{比例环节}
\label{sec-3-1}

\begin{eqnarray*}
c(t) &=& kr(t) \\
C(s) &=& kR(s) \\
G(s) &=& \frac{C(s)}{R(s)} \\
   &=& k
\end{eqnarray*}
\end{frame}
\begin{frame}[t]
\frametitle{积分环节}
\label{sec-3-2}
\begin{columns}
\begin{column}{0.65\textwidth}
\begin{block}<1->{传递函数}
\label{sec-3-2-1}


\begin{itemize}
\item $c(t) = \int r(t)dt$
\item $C(s) = \frac{R(s)}{s}$
\item $G(s) = \frac{1}{s}$
\end{itemize}

\begin{circuitikz}[x=0.7cm]
\draw
            (5,.5) node [op amp] (opamp) {}
           (0,1) node [ left ] {$U_{r}$} 
            to [R, l=$R$,o-*,i_=$I$] (opamp.-)
            (opamp.+) to ($( opamp.+)+(0,-0.5)$) node [ground] {}
           (opamp.out) to [short] +(0,1.5) to   [C, l_=$C$, i_=$I_c$] ($(opamp.-)+(0,1)$) to [short] (opamp.-) 
           (opamp.out) to [short , *-o] (7,.5) node [ right ] {$U_{c}$};
\end{circuitikz}
\end{block}
\end{column}
\begin{column}{0.35\textwidth}
\begin{block}<2->{推导}
\label{sec-3-2-2}


\begin{enumerate}
\item <2-> $U_r    = I R$
\item <3-> $U_r(s) = I(s)R$
\item <2-> $C\frac{dU_c}{dt}    = I_c=-I$
\item <3-> $U_c(s) = -\frac{I(s)}{Cs}$
\item <4-> $U_c(s) = -\frac{U_r(s)}{RCs}$
\item <5-> $\frac{U_c(s)}{U_r(s)} = -\frac{1}{RCs}$
\end{enumerate}
\end{block}
\end{column}
\end{columns}
\end{frame}
\begin{frame}[t]
\frametitle{微分环节}
\label{sec-3-3}
\begin{columns}
\begin{column}{0.5\textwidth}
\begin{block}<1->{传递函数}
\label{sec-3-3-1}

\begin{itemize}
\item $c(t)=r'(t)$
\item $C(s)=sR(s)$
\item $G(s)=s$
\end{itemize}

\begin{circuitikz}[american voltages,x=0.7cm]
%       o---c --+-------o
%               |
%      U_r      R      U_c
%               |
%       o-------+-------o
\draw
  % rotor circuit
  (0,0) to  [short, o-o] (5,0)
  to [open, v^>=$U_c$,-o](5,3)
  to [short] (3,3)
  to [R, l=$R$, i_={$I$}] (3,0)

  (0,0) to [open, v>=$U_r$,-o] (0,3)
  to [C,l=$C$] (3,3);
\end{circuitikz}
\end{block}
\end{column}
\begin{column}{0.5\textwidth}
\begin{block}<2->{推导}
\label{sec-3-3-2}

\begin{eqnarray*}
  U_r &= &\frac{1}{C}\int I dt +U_c \\
  U_r(s) &=& \frac{I(s)}{Cs}+U_c(s) \\
  IR &=& U_c \\
  I(s)R&=&U_c(s) \\
  U_r(s) &=& \frac{U_c(s)}{RCs}+U_c(s)\\
  \frac{U_c(s)}{U_r(s)} &=&\frac{RCs}{1+RCs} \\
  &\approx & RCs , \qquad (RC\ll 1)
\end{eqnarray*}
\mode<article>{实际物理系统 $n\geq m$ . 其中: $n$ :传递函数分母阶次, $m$ 分子阶次}
\end{block}
\end{column}
\end{columns}
\end{frame}
\begin{frame}
\frametitle{一阶惯性环节}
\label{sec-3-4}
\begin{columns}
\begin{column}{0.5\textwidth}
\begin{block}<1->{传递函数}
\label{sec-3-4-1}

\begin{eqnarray*}
G(s) &=& \frac{1}{Ts+1}
\end{eqnarray*}
其中 $T=RC$ 为时间常数

\begin{circuitikz}[american voltages,x=0.7cm]
%       o---R --+-------o
%               |
%      U_r      C      U_c
%               |
%       o-------+-------o
\draw
  % rotor circuit
  (0,0) to  [short, o-o] (5,0)
  to [open, v^>=$U_c$,-o](5,3)
  to [short] (3,3)
  to [C, l_=$C$, i_={$I$}] (3,0)

  (0,0) to [open, v>=$U_r$,-o] (0,3)
  to [R,l=$R$] (3,3);
\end{circuitikz}
\end{block}
\end{column}
\begin{column}{0.5\textwidth}
\begin{block}{推导}
\label{sec-3-4-2}


\begin{eqnarray*}
 U_r &= &IR dt +U_c \\
 U_r(s) &=& I(s)R+U_c(s) \\
 U_c &=& \frac{1}{C}\int I dt \\
 I(s)&=&CsU_c \\
 U_r(s) &=& U_c(s)RCs+U_c(s)\\
 \frac{U_c(s)}{U_r(s)} &=&\frac{1}{1+RCs} 
\end{eqnarray*}
\end{block}
\end{column}
\end{columns}
\end{frame}
\begin{frame}
\frametitle{一阶微分环节}
\label{sec-3-5}

  $$G(s)=1+\tau s$$
\end{frame}
\begin{frame}
\frametitle{二阶振荡环节}
\label{sec-3-6}
\begin{columns}
\begin{column}{0.45\textwidth}
\begin{block}<1->{LC振荡电路}
\label{sec-3-6-1}


\begin{circuitikz}[american voltages,x=0.7cm]
%       o-R --L-+-------o
%               |
%      U_r      C      U_c
%               |
%       o-------+-------o
\draw
  % rotor circuit
  (0,0) to  [short, o-o] (6,0)
  to [open, v^>=$U_c$,-o](6,3)
  to [short] (4,3)
  to [C, l_=$C$, i_={$I$}] (4,0)

  (0,0) to [open, v>=$U_r$,-o] (0,3)
  to [R,l=$R$] (2,3)
  to [L,l=$L$] (4,3);
\end{circuitikz}
\begin{eqnarray*}
U_r &= &IR + U_L+ U_c \\
U_c &=& \frac{1}{C}\int I dt \\
U_L &=& L\frac{dI}{dt} 
\end{eqnarray*}
\end{block}
\end{column}
\begin{column}{0.55\textwidth}
\begin{block}<2->{推导}
\label{sec-3-6-2}


\begin{eqnarray*}
 U_r(s) &=& I(s)R+U_L(s)+U_c(s) \\
 U_c(s) &=& \frac{I(s)}{Cs}\\
 I(s)&=&CsU_c \\
 U_L(s) &=& LsI(s) \\
        &=& LCs^2U_c(s) \\
 U_r(s) &=& (Rcs+LCs^2+1)U_c(s)\\
 \frac{U_c(s)}{U_r(s)} &=&\frac{1}{LCs^2+RCs+1}
\end{eqnarray*}
\end{block}
\end{column}
\end{columns}
\end{frame}
\begin{frame}
\frametitle{二阶振荡环节标准形式}
\label{sec-3-7}

\begin{itemize}
\item <2-> 标准形式:
       \begin{eqnarray*}
        G(s) &=& \frac{\omega^2}{s^2+2\xi\omega_n s+\omega_n^2}\\
             &=& \frac{1}{T^2s^2+2\xi Ts+1}
       \end{eqnarray*}
        其中 $T\omega_n=1$
\item <3->术语:
\begin{itemize}
\item $\omega_n$ : 无阻尼振荡频率或自然频率
\item $\xi$ : 阻尼比或阻尼系数
\item $T$ : 时间常数
\end{itemize}
\end{itemize}

\mode<article>{例:姿态角、角速度、加速度计等其数学模型均为二阶振荡环节}
\end{frame}
\begin{frame}
\frametitle{二阶微分环节}
\label{sec-3-8}

   $$G(s)=\tau^2s^2+2\xi\tau s + 1$$
\end{frame}
\begin{frame}
\frametitle{延迟环节}
\label{sec-3-9}

\begin{eqnarray*}
c(t) &=& r(t-\tau) \\
C(s) &=& R(s)e^{-\tau s} \\
G(s) &=&e^{-\tau s}
\end{eqnarray*}
\end{frame}

\end{document}
