% !TeX root = rules.beamer.tex
\DeclareMathOperator*{\argmin}{arg\,min}
\usetikzlibrary{shapes,arrows,positioning,decorations}
\usetikzlibrary{decorations.pathmorphing} 
\usetikzlibrary{decorations.shapes} 
\usetikzlibrary{decorations.fractals} 
\usetikzlibrary{spy} % L A T EX and plain T E X
\begin{document}
\newcommand{\vect}[1]{\boldsymbol{#1}}

\def\lecturename{自动控制原理}

\title{\insertlecture}

\author{邢超}

\institute
{
  西北工业大学航天学院
}

%\mode<presentation>{\subject{嵌入式系统}}

%  start a lecture  --------------------------
\lecture[]{线性系统的根轨迹法}{}
\subtitle{根轨迹绘制基本法则}
\date{2014}


%\setbeamertemplate{background}{\pgfimage[width=\paperwidth,height=\paperheight]{image/flower}}
%\setbeamercovered{transparent}
%\mode<presentation>{\beamerdefaultoverlayspecification{<+->}}

\begin{frame}
  \maketitle
\end{frame}


\section{根轨迹分支数、对称性和连续性}
\begin{frame}{根轨迹分支数、对称性和连续性}
\begin{itemize}
\item 根轨迹分支数等于开环零点数与开环极点数中的较大者,即$\max(m,n)$\only<article>{(多项式的阶数为$\max(m,n)$)}
\item 根轨迹关于实轴对称\only<article>{(实系数多项式的根为实数或共轭复数)}
\item 根轨迹是连续的\only<article>{(参考隐函数存在定理)}
\end{itemize}
\end{frame}

\section{根轨迹的起点和终点}
\begin{frame}{开环零极点数相等时}
\begin{eqnarray}
K^{\ast}\frac{\prod\limits_{j=1}^m(s-z_j)}{\prod\limits_{i=1}^n(s-p_i)} &=& -1  \\
{\prod\limits_{i=1}^n(s-p_i)}+K^{*}{\prod\limits_{j=1}^m(s-z_j)} &=& 0 \\
\frac{1}{K^{*}}{\prod\limits_{i=1}^n(s-p_i)}+{\prod\limits_{j=1}^m(s-z_j)} &=& 0
\end{eqnarray}
\begin{itemize}
\item 起点:开环极点
\item 终点:开环零点
\end{itemize}
\only<article>{此时根轨迹只存在于有限区域内。}
\end{frame}

\begin{frame}{开环极点数大于零点数}
\begin{eqnarray}
K^* &=& \frac{\prod\limits_{i=1}^n|s-p_i|}{\prod\limits_{j=1}^m|s-z_j|} \\
K^* &=& \frac{|s|^{n-m}\prod\limits_{i=1}^n|1-\frac{p_i}{s}|}{\prod\limits_{j=1}^m|1-\frac{z_j}{s}|} 
\end{eqnarray}
\only<article>{此时根轨迹随$K^*$的增大而趋于无穷远,且远处的根轨迹趋近于$\frac{K^*}{(s-c)^{n-m}}$。}
\end{frame}


\section{根轨迹的渐近线}
\only<article>{当$n>m$时,根轨迹随$K^*$的增大趋于无穷远,且远处的根轨迹趋近于$\frac{K^*}{(s-c)^{n-m}}$。此时,$\frac{K^*}{(s-c)^{n-m}}$的根轨迹可看作渐近线。}
% \begin{frame}{根轨迹的渐近线(方向)}
% \only<article>{将$\frac{K^*}{(s-c)^{n-m}}$简化为$\frac{K^*}{s^{n-m}}$,求渐近线方向}
% \begin{eqnarray}
% K^{\ast}\frac{\prod\limits_{j=1}^m(s-z_j)}{\prod\limits_{i=1}^n(s-p_i)} &=& -1  \\
% K^{\ast}\frac{\prod\limits_{j=1}^m(1-\frac{z_j}{s})}{s^{n-m}\prod\limits_{i=1}^n(1-\frac{p_i}{s})} &=& -1  \\
% K^{\ast}\frac{1}{s^{n-m}} &\approx& -1  
% \end{eqnarray}
% \end{frame}

\begin{frame}{根轨迹的渐近线}
\begin{eqnarray}
K^{\ast}\frac{1}{(s-c)^{n-m}} &\approx& -1  \\
K^{\ast}\frac{\prod\limits_{j=1}^m(1-\frac{z_j}{s})}{s^{n-m}\prod\limits_{i=1}^n(1-\frac{p_i}{s})} &\approx& \frac{K^*}{s^{n-m}(1-\frac{c}{s})^{n-m}} \\
(1-\frac{c}{s})^{n-m}\prod\limits_{j=1}^m(1-\frac{z_j}{s}) &\approx& \prod\limits_{i=1}^n(1-\frac{p_i}{s})  
\end{eqnarray}
\end{frame}

\begin{frame}{根轨迹的渐近线}
\begin{eqnarray}
(1-\frac{c}{s})^{n-m} &=&\only<2,5->{ 1-\frac{(n-m)c}{s}+\cdots    }\\
\prod\limits_{j=1}^m(1-\frac{z_j}{s}) &=&\only<3,5-> { 1-\frac{\sum\limits_{j=1}^m z_j }{s}+\cdots }\\
\prod\limits_{i=1}^n(1-\frac{p_i}{s}) &=& \only<4-> { 1-\frac{\sum\limits_{i=1}^n p_i}{s}+\cdots }
\end{eqnarray}
\end{frame}

\begin{frame}{根轨迹的渐近线}
\begin{eqnarray}
\only<article>{(1-\frac{(n-m)c}{s}+\cdots)(1-\frac{\sum_{j=1}^{m}z_j}{s}+\cdots) &=& 1-\frac{\sum_{i=1}^n p_i}{s}+\cdots \\ }
1-\frac{(n-m)c}{s}-\frac{\sum_{j=1}^{m}z_j}{s}+\cdots &\approx& 1-\frac{\sum_{i=1}^n }{s}+\cdots \nonumber \\ 
-(n-m)c-\sum_{j=1}^{m}z_j &=& \sum_{i=1}^n p_i \\
c &=& \frac{\sum\limits_{i=1}^n p_i -\sum\limits_{j=1}^{m}z_j}{n-m}
\end{eqnarray}
\end{frame}

\section{根轨迹在实轴上的分布}
\begin{frame}{根轨迹在实轴上的分布$s \in R$(开环零极点全为实数)}
\begin{eqnarray}
K^{\ast}\frac{\prod\limits_{j=1}^m(s-z_j)}{\prod\limits_{i=1}^n(s-p_i)} &=& -1 \\
K^{\ast}\frac{\prod\limits_{j=1}^m\rho_j}{\prod\limits_{i=1}^n A_i} &=& -1 \\
K^{\ast}\frac{\prod\limits_{j=1}^m |\rho_j| }{\prod\limits_{i=1}^n |A_i|} \frac{{j=1}(-1)^q}{(-1)^r}&=& -1 
\end{eqnarray}
实轴上某区域,若其右边开环实数零、极点个数之和为奇数,则该区域必是根轨迹。
\end{frame}

\begin{frame}{根轨迹在实轴上的分布$s\in R$(开环零极点存在共轭复数)}
\begin{eqnarray}
K^{\ast}G_q(s)(s-z_q)(s-\bar z_q) &=& -1 \\
K^{\ast}G_i(s)(s-a-jb_q)(s-a+jb_q) &=& -1 \\
K^{\ast}G_i(s)((s-a)^2+b_q^2) &=& -1 
\end{eqnarray}
实轴上某区域,若其右边开环实数零、极点个数之和为奇数,则该区域必是根轨迹。
\end{frame}

\section{根轨迹的起始角与终止角}

\begin{frame}{根轨迹的起始角与终止角}
\begin{center}
\begin{tikzpicture}
\draw[blue] (-1,2) 
\foreach \x/\y in { -1/2,-0.9987409/1.9962392,-0.9974632/1.9924566,-0.9961665/1.9886519,-0.9948504/1.9848249,-0.9935143/1.9809754,-0.9921578/1.9771032,-0.9907804/1.9732078,-0.9893816/1.9692893,-0.9879609/1.9653471,-0.9865176/1.9613812,-0.9707099/1.9203577,-0.9518765/1.8766821,-0.9290213/1.8301611,-0.9007159/1.7807913,-0.8649497/1.7290805,-0.8191523/1.6767276,-0.7610164/1.6276691,-0.6909830/1.5882514,-0.6145415/1.5638845,-0.539338/1.5546278,-0.4701030/1.5564299,-0.4080364/1.5650218,-0.3526261/1.5774586,-0.3029050/1.5919735,-0.2579303/1.6075439,-0.2169078/1.6235776,-0.1791978/1.6397288,-0.1442911/1.6557936,-0.1117808/1.6716525,-0.0813397/1.6872366,-0.0527020/1.702508,-0.0256491/1.7174482,0.0243972/1.746317,0.0476707/1.7602527,0.0699300/1.7738666,0.0912690/1.7871691,0.1117689/1.8001713,0.1315006/1.812885,0.1505260/1.8253217,0.1688999/1.8374928,0.1866707/1.8494093,0.2038815/1.861082,0.2205707/1.8725209,0.2367730/1.8837359,0.2525195/1.8947361,0.2678384/1.9055303,0.2827553/1.9161268,0.2972934/1.9265334,0.3114740/1.9367575,0.3253166/1.9468062,0.3388391/1.9566862,0.3520578/1.9664036,0.3649879/1.9759645,0.3776435/1.9853745,0.3900376/1.9946389,0.4021822/2.0037628,0.4140885/2.012751,0.425767/2.0216081}
{
-- (\x,\y)
}
;
\draw [help lines] (-3,-2) grid (1,3);
\draw[->] (-5,0)--(2,0) node[right] {$ $} ;
\draw[->] (0,-2)--(0,3.2) node[above] {$j$} ;
\foreach \x/\y in {0/0,-3/0,-1/2,-1/-2}
{
\draw  (\x,\y) node[red,thick] {$\times$};
}
\only<beamer>{
	\foreach \ii/\sx/\sy/\sxx/\syy in 
%{1/0.2972934/1.9265334/-0.0015997/1.9434599, 2/0.2205707/1.8725209/-0.0054099/1.8961228, 3/0.1315006/1.812885/-0.0133992/1.8368469, 4/0.0243972/1.746317/-0.0293212/1.7596199, 5/-0.1117808/1.6716525/-0.0620373/1.6532639, 6/-0.3029050/1.5919735/-0.1369698/1.4948477, 7/-0.6145415/1.5638845/-0.33775/1.250717, 8/-0.9007159/1.7807913/-0.5874244/1.0890766, 9/-0.9879609/1.9653471/-0.6718206/1.0553846, 10/-0.9948504/1.9848249/-0.6786501/1.0530395}
	{1/0.1315006/1.812885/-0.0133992/1.8368469,  2/-0.3029050/1.5919735/-0.1369698/1.4948477, 3/-0.6145415/1.5638845/-0.33775/1.250717, 4/-0.9007159/1.7807913/-0.5874244/1.0890766, 5/-0.9879609/1.9653471/-0.6718206/1.0553846 }
	{
		\foreach \x/\y in {0/0,-3/0,-1/-2}
		{
			\draw<\ii>[->] (\x,\y)--(\sx,\sy);
			%\filldraw[fill=green!20,draw=red] (\x,\y) -- +(0.5,0) arc(0:\pgfmathatantwo{\sx-\x}{\sy-\y}:5mm);
			%\pgfmathparse{atan2(\sx-\x,\sy-\y)}\pgfmathresult
				%\filldraw[fill=green!20,draw=red] (\x,\y) -- +(0.5,0) arc(0:\pgfmathatan2{\sx-\x}{\sy-\y}:5mm);
			\begin{scope}
			\clip<\ii> (\x,\y)--+(10,0)--(\sx,\sy)--cycle;
			%\clip[draw] (\x,\y)--+(1,0)--+(\sx-\x,\sy-\y)--cycle;
			\filldraw<\ii>[fill=green!20,draw=red] (\x,\y) -- +(0.5,0) arc(0:360:5mm);
			\end{scope}
		}
		\foreach \x/\y in {-1/2}{
			\begin{scope}
			\clip<\ii> (\x,\y)--+(10,0)--(\sxx,\syy)--cycle;
			\draw<\ii>[thick,red] (\x,\y) -- +(0.5,0) arc(0:360:5mm);
			\end{scope}
			\draw<\ii>[->,thick,red] (\x,\y) --(\sxx,\syy);
			\draw<\ii>[->] (\x,\y)--(\sx,\sy);
		}
	}
}
\only<article>{
  \foreach \x/\y in {0/0,-3/0,-1/-2}{
    \draw[->] (\x,\y)--(-1,2);
      \begin{scope}
        \clip (\x,\y)--+(10,0)--(-0.9879609,1.9653471)--cycle;
        \filldraw[fill=green!20,draw=red] (\x,\y) -- +(0.5,0) arc(0:360:5mm);
      \end{scope} }
  \draw[->,thick,red] (-1,2) --(-0.6718206,1.0553846);}
\end{tikzpicture}
\end{center}
\end{frame}

\bgroup
\setbeamertemplate{sidebar right}{}

\begin{frame}{起始角}
\begin{eqnarray*}
\sum_{l=1}^{m}\angle(s-z_{l})-\sum_{i=1}^{n}\angle(s-p_{i}) & = & (2k+1)\pi\\
s & = & p_{q}+\delta re^{j\theta}\\
\sum_{l=1}^{m}\angle(p_{q}+\delta re^{j\theta}-z_{l})-\sum_{i=1}^{n}\angle(p_{q}+\delta re^{j\theta}-p_{i}) & = & (2k+1)\pi\\
\lim_{\delta r\rightarrow0}\sum_{l=1}^{m}\angle(p_{q}+\delta re^{j\theta}-z_{l})-\sum_{i=1}^{n}\angle(p_{q}+\delta re^{j\theta}-p_{i}) & = & (2k+1)\pi\\
\sum_{l=1}^{m}\angle(p_{q}-z_{l})-\sum_{p_{q}=p_{i}}\theta-\sum_{p_{q}\not=p_{i}}\angle(p_{q}-p_{i}) & = & (2k+1)\pi
\end{eqnarray*}
\end{frame}

\begin{frame}{终止角}
\begin{eqnarray*}
\sum_{l=1}^{m}\angle(s-z_{l})-\sum_{i=1}^{n}\angle(s-p_{i}) & = & (2k+1)\pi\\
s & = & z_{q}+\delta re^{j\theta}\\
\sum_{l=1}^{m}\angle(z_{q}+\delta re^{j\theta}-z_{l})-\sum_{i=1}^{n}\angle(z_{q}+\delta re^{j\theta}-p_{i}) & = & (2k+1)\pi\\
\lim_{\delta r\rightarrow0}\sum_{l=1}^{m}\angle(z_{q}+\delta re^{j\theta}-z_{l})-\sum_{i=1}^{n}\angle(z_{q}+\delta re^{j\theta}-p_{i}) & = & (2k+1)\pi\\
\sum_{z_{q}\not=z_{l}}\angle(z_{q}-z_{l})+\sum_{z_{q}=z_{l}}\theta-\sum_{i=1}^{n}\angle(z_{q}-p_{i}) & = & (2k+1)\pi
\end{eqnarray*}
\end{frame}

\egroup

\section{根轨迹的分离点与分离角}

\begin{frame}{分离点与分离角}
\begin{itemize}
 \item<2->分离点: 两条或两条以上根轨迹分支在  $S$  平面上相交后又分开的点称为根轨迹的分离点。
 \item<3->分离角: 根轨迹进入分离点的切线方向与离开分离点的切线方向之间的夹角
\end{itemize}
\end{frame}


\begin{frame}{分离点计算}
\begin{align*}
G(s) &=  \frac{K^* M(s)}{N(s)}\\
D(s) &= N(s)+K^* M(s) \\
N(s) &= - K^* M(s) & D(s) &= 0 \\
N'(s) &= -K^* M'(s) & D'(s) &= 0 \\
\frac{N'(s)}{N(s)} &=\frac{M'(s)}{M(s)} \\
\frac{\frac{d}{ds}\prod_{i=1}^{n}(s-p_i)}{\prod_{i=1}^{n}(s-p_i)} &=\frac{\frac{d}{ds}\prod_{j=1}^{m}(s-z_j)}{\prod_{j=1}^{m}(s-z_j)} \\
\sum_{i=1}^n\frac{1}{s-p_i} &= \sum_{j=1}^m\frac{1}{s-z_j}
\end{align*}
\end{frame}

\begin{frame}{分离角计算:}
设 $K^*=K_0$时有重极点。可将原特征方程变换
\begin{align*}
D(s)&=N(s)+K^*M(S)\\
D(s)&=N(s)+(K'+K_0)M(S)\\
D(s)&=N(s)+K_0M(S)+K'M(s)\\
D'(s)&=N'(s)+K'M(s)
\end{align*}
\mode<article>{
新系统 $\frac{M(s)}{N'(s)}$ 的开环极点(N'(S)=0的根)与 $K^*=K_0$ 时原系统闭环特征方程的根相同。可得:}
\begin{itemize}
\item 原系统的分离点等于新系统的起始点
\item 分离角等于 $180^\circ$ 根轨迹与0度根轨迹起始角之差
\begin{itemize}
\item $180^\circ$ 根轨迹起始角: $\theta_1$
\item 0度根轨迹起始角: $\theta_2$
\item 其它零(极)点到分离点的辐角之和(差): $\theta_2$
\end{itemize}
\begin{align*}
L\theta_1  &= \theta_0+(2k_1+1)\pi\\
L\theta_2 &= \theta_0+2k_2\pi\\
L\Delta\theta &=(2(k_1-k_2)+1)\pi\\
\Delta\theta &=\frac{(2k+1)\pi}{L}
\end{align*}
\end{itemize}

\end{frame}

\section{根轨迹与虚轴的交点}
\begin{frame}{根轨迹与虚轴的交点}
\begin{eqnarray*}
1+G(s)H(s) & = & 0\\
s & = & j\omega\\
1+G(j\omega)H(j\omega) & = & 0
\end{eqnarray*}

\end{frame}

\section{根之和($n-1>m$)}
\begin{frame}{根之和($n-1>m$)}
\begin{eqnarray*}
\prod_{i=1}^{n}(s-p_{i})+K^{*}\prod_{j=1}^{m}(s-z_{j}) & = & 0\\
\prod_{i=1}^{n}(s-p_{i}) & = & s^{n}+\sum_{i=1}^{n}(-p_{i})s^{n-1}+\cdots\\
\prod_{j=1}^{m}(s-z_{j}) & = & s^{m}+\sum_{j=1}^{m}(-z_{j})s^{m-1}+\cdots\\
\prod_{i=1}^{n}(s-p_{i})+K^{*}\prod_{j=1}^{m}(s-z_{j}) & = & s^{n}+\sum_{i=1}^{n}(-p_{i})s^{n-1}+\cdots
\end{eqnarray*}

\end{frame}

\begin{frame}{根之和($n-1>m$)}
\begin{eqnarray*}
\prod_{i=1}^{n}(s-p_{i})+K^{*}\prod_{j=1}^{m}(s-z_{j}) & = & s^{n}+\sum_{i=1}^{n}(-p_{i})s^{n-1}+\cdots\\
\prod_{i=1}^{n}(s-p_{i})+K^{*}\prod_{j=1}^{m}(s-z_{j}) & = & \prod_{i=1}^{n}(s-s_{i})\\
\prod_{i=1}^{n}(s-s_{i}) & = & s^{n}+\sum_{i=1}^{n}(-s_{i})s^{n-1}+\cdots\\
\sum_{i=1}^{n}(-p_{i})s^{n-1} & = & \sum_{i=1}^{n}(-s_{i})s^{n-1}\\
\sum_{i=1}^{n}(-p_{i}) & = & \sum_{i=1}^{n}(-s_{i})\\
\sum_{i=1}^{n}p_{i} & = & \sum_{i=1}^{n}s_{i}
\end{eqnarray*}

\end{frame}
\end{document}
	

%%% Local Variables:
%%% TeX-master: "rules.beamer"
%%% End:
