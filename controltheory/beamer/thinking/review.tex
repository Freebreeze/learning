% Created 2014-11-05 星期三 11:17
\documentclass{beamer}
\usepackage{fixltx2e}
\usepackage{graphicx}
\usepackage{longtable}
\usepackage{float}
\usepackage{wrapfig}
\usepackage{soul}
\usepackage{textcomp}
\usepackage{marvosym}
\usepackage{wasysym}
\usepackage{latexsym}
\usepackage{amssymb}
\usepackage{hyperref}
\tolerance=1000
\usepackage{amsmath}
\usepackage[usenames]{color}
\usepackage{pstricks}
\usepackage{pgfplots}
\usepackage{tikz}
\usepackage[europeanresistors,americaninductors]{circuitikz}
\usepackage{colortbl}
\usepackage{yfonts}
\usetikzlibrary{shapes,arrows}
\usetikzlibrary{positioning}
\usetikzlibrary{arrows,shapes}
\usetikzlibrary{intersections}
\usetikzlibrary{calc,patterns,decorations.pathmorphing,decorations.markings}
\usepackage[BoldFont,SlantFont,CJKchecksingle]{xeCJK}
\setCJKmainfont[BoldFont=Evermore Hei]{Evermore Kai}
\setCJKmonofont{Evermore Kai}
\xeCJKsetup{CJKglue=\hspace{0pt plus .08 \baselineskip }}
\usepackage{pst-node}
\usepackage{pst-plot}
\psset{unit=5mm}
\usepackage{beamerarticle}
\mode<beamer>{\usetheme{Frankfurt}}
\mode<beamer>{\usecolortheme{dove}}
\mode<article>{\hypersetup{colorlinks=true,pdfborder={0 0 0}}}
\AtBeginSection[]{\begin{frame}<beamer>\frametitle{Topic}\tableofcontents[currentsection]\end{frame}}
\setbeamercovered{transparent}
\providecommand{\alert}[1]{\textbf{#1}}

\title{自控原理体会}
\author{}
\date{}
\hypersetup{
  pdfkeywords={},
  pdfsubject={},
  pdfcreator={Emacs Org-mode version 7.9.3f}}

\begin{document}

\maketitle

\begin{frame}
\frametitle{Outline}
\setcounter{tocdepth}{3}
\tableofcontents
\end{frame}





\section{自动控制的一般概念}
\label{sec-1}
\subsection{开环与闭环}
\label{sec-1-1}
\begin{frame}
\frametitle{指南车}
\label{sec-1-1-1}

\begin{pspicture}
\Rnode{f}{\color{purple}干扰}
{\hskip 1em }\Rnode{a}{\psframebox[framesep=0.1em]{指针}}
{\hskip 1em}\Rnode{c}{方向}
\ncline{->}{f}{a}
\ncline{->}{a}{c}

{\hskip 1em}

\Rnode{r}{\psframebox[framesep=0.1em]{$-1$}}
{\hskip 1em }\Rnode{p}{\color{purple}$\circ$}
\nput[labelsep=1em]{90}{p}{\color{purple} \Rnode{f}{干扰}}
{\hskip 1em }\Rnode{a}{\psframebox[framesep=0.1em]{指针}}
{\hskip 1em}\Rnode{c}{方向}
\ncline{->}{r}{p}
\ncline{->}{p}{a}
\ncline{->}{a}{c}
\ncline{->}{f}{p}
\ncangles[angleA=-90,angleB=180,armA=0.3em,armB=0.7em]{->}{f}{r}

{\hskip 1em}

\Rnode{p}{\color{purple}$\circ$}
\nput[labelsep=1em]{90}{p}{\color{purple} \Rnode{f}{干扰}}
{\hskip 1em }\Rnode{a}{\psframebox[framesep=0.1em]{指针}}
{\hskip 1em}\Rnode{c}{方向}
\ncline{->}{p}{a}
\ncline{->}{a}{c}
\ncline{->}{f}{p}
\ncangles[angleA=-90,angleB=180,armA=0.3em,armB=1em]{->}{f}{p}
\nput[labelsep=0]{-150}{p}{\color{purple}$-$}
\end{pspicture}
\end{frame}
\begin{frame}
\frametitle{水位控制}
\label{sec-1-1-2}

\begin{pspicture}
\Rnode{r}{进水}
{\hskip 1em }\Rnode{p}{\color{purple}$\circ$}
\nput[labelsep=1em]{90}{p}{\color{purple} \Rnode{f}{干扰}}
{\hskip 1em }\Rnode{a}{\psframebox[framesep=0.1em]{容器}}
{\hskip 1em}\Rnode{c}{水位}
\ncline{->}{r}{p}
\ncline{->}{p}{a}
\ncline{->}{a}{c}
\ncline{->}{f}{p}

{\hskip 1em}

\Rnode{r}{进水}
{\hskip 1em }\Rnode{p1}{\color{purple}$\circ$}
{\hskip 1em }\Rnode{p2}{\color{purple}$\circ$}
\nput[labelsep=1em]{90}{p2}{\color{purple} \Rnode{f}{干扰}}
{\hskip 1em }\Rnode{a}{\psframebox[framesep=0.1em]{容器}}
\nput[labelsep=0.5em]{-90}{a}{\color{purple} \Rnode{h}{\psframebox[framesep=0.1em]{$-1$}}}
{\hskip 1em}\Rnode{c}{水位}
\ncline{->}{r}{p1}
\ncline{->}{p1}{p2}
\ncline{->}{p2}{a}
\ncline{->}{a}{c}
\ncline{->}{f}{p2}
\ncbar[angleA=0,angleB=0,armA=0.3em,armB=0.3em]{->}{a}{h}
\ncangle[angleA=180,angleB=-90,armA=0.5em]{->}{h}{p1}

{\hskip 1em}

\Rnode{r}{进水}
{\hskip 1em }\Rnode{p1}{\color{purple}$\circ$}
{\hskip 1em }\Rnode{p2}{\color{purple}$\circ$}
\nput[labelsep=1em]{90}{p2}{\color{purple} \Rnode{f}{干扰}}
{\hskip 1em }\Rnode{a}{\psframebox[framesep=0.1em]{容器}}
{\hskip 1em}\Rnode{c}{水位}
\ncline{->}{r}{p1}
\ncline{->}{p1}{p2}
\ncline{->}{p2}{a}
\ncline{->}{a}{c}
\ncline{->}{f}{p2}
\ncangles[angleA=0,angleB=-90,armA=0.5em,armB=0.7em]{->}{a}{p1}
\nput[labelsep=0]{-130}{p1}{\color{purple}$-$}
\end{pspicture}
\end{frame}
\subsection{正反馈与负反馈}
\label{sec-1-2}
\begin{frame}
\frametitle{力-速度模型}
\label{sec-1-2-1}

\begin{pspicture}
\Rnode{r}{$r$}{\hskip 1em }\Rnode{p}{\color{purple}$\circ$}{\hskip 1em}
%\Rnode{s}{\psline(0,0.3em)(1em,1em)\psarc{<-}(0,0.3em){0.7em}{-10}{80}{\hskip 1em}}{\hskip 1em}
\Rnode{a}{\psframebox[framesep=0.1em]{$\int$}}{\hskip 1em}\Rnode{c}{$x$}
\ncline{->}{r}{p}
\ncline{-}{p}{a}
\ncline{->}{a}{c}
\ncangles[angleA=0,angleB=-60,armA=0.5em,armB=0.7em]{->}{a}{p}%\naput[labelsep=0,npos=3.6]{\red $-$}
\nput[labelsep=0]{-100}{p}{\color{purple} $-$}

{\hskip 1em}

\Rnode{r}{$r$}{\hskip 1em }\Rnode{p}{\color{purple}$\circ$}{\hskip 1em}
\Rnode{a}{\psframebox[framesep=0.1em]{$\int$}}{\hskip 2em}\Rnode{c}{$x$}
\nput[labelsep=0.1em]{-90}{a}{\color{purple} \Rnode{h}{\psframebox[framesep=0.1em]{$-1$}}}
\ncline{->}{r}{p}
\ncline{-}{p}{a}
\ncline{->}{a}{c}
%\ncangles[angleA=0,angleB=-60,armA=0.5em,armB=0.7em]{->}{a}{p}
\ncangle[angleA=180,angleB=-90]{->}{h}{p}
\ncangle[angleA=0,angleB=0,armA=0.5em,armB=0.5em]{->}{a}{h}
\end{pspicture}

\begin{eqnarray*}
x & =& \int_0^t v dt \\
v &=& r-x
\end{eqnarray*}
\end{frame}
\begin{frame}
\frametitle{$AB<0$}
\label{sec-1-2-2}

\begin{pspicture}
\Rnode{r}{$r$}{\hskip 1em }\Rnode{p}{\color{purple}$\circ$}{\hskip 1em}
\Rnode{a}{\psframebox[framesep=0.1em]{$A$}}{\hskip 2em}\Rnode{c}{$x$}
\nput[labelsep=0.1em]{-90}{a}{\color{purple} \Rnode{h}{\psframebox[framesep=0.1em]{$B$}}}
\ncline{->}{r}{p}
\ncline{-}{p}{a}
\ncline{->}{a}{c}
%\ncangles[angleA=0,angleB=-60,armA=0.5em,armB=0.7em]{->}{a}{p}
\ncangle[angleA=180,angleB=-90,armB=0]{->}{h}{p}
\ncangle[angleA=0,angleB=0,armA=0.5em,armB=0.5em]{->}{a}{h}

\newline

\Rnode{r}{$r$}{\hskip 1em }\Rnode{p}{\color{purple}$\circ$}{\hskip 1em}
\Rnode{a}{\psframebox[framesep=0.1em]{$A$}}{\hskip 2em}\Rnode{c}{$x$}
\nput[labelsep=0.1em]{-90}{a}{\color{purple} \Rnode{h}{\psframebox[framesep=0.1em]{$-1$}}}
\ncline{->}{r}{p}
\ncline{-}{p}{a}
\ncline{->}{a}{c}
%\ncangles[angleA=0,angleB=-60,armA=0.5em,armB=0.7em]{->}{a}{p}
\ncangle[angleA=180,angleB=-90,armB=0]{->}{h}{p}
\ncangle[angleA=0,angleB=0,armA=0.5em,armB=0.5em]{->}{a}{h}

\\

\newline

\Rnode{r}{$r$}{\hskip 1em }\Rnode{p}{\color{purple}$\circ$}{\hskip 1em}
\Rnode{a}{\psframebox[framesep=0.1em]{$A$}}{\hskip 2em}\Rnode{c}{$x$}
%\nput[labelsep=0.1em]{-90}{a}{\color{purple} \Rnode{h}{\psframebox[framesep=0.1em]{$B$}}}
\ncline{->}{r}{p}
\ncline{-}{p}{a}
\ncline{->}{a}{c}
\ncangles[angleA=0,angleB=-60,armA=0.5em,armB=0.7em]{->}{a}{p}
%\ncangle[angleA=180,angleB=-70,armB=0]{->}{h}{p}
%\ncangle[angleA=0,angleB=0,armA=0.5em,armB=0.5em]{->}{a}{h}
\nput[labelsep=0]{-100}{p}{\color{purple} $-$}
\end{pspicture}

减小元器件A非线性的影响:
\begin{eqnarray*}
c & =& A(x+Bc)\\
  &=& \frac{Ax}{1-AB}\\
 &\approx& \frac{x}{-B}\\
\frac{dc}{dA} &=& \frac{x(1-AB)-Ax(-B)}{(1-AB)^2}\\
              &=& \frac{x}{(1-AB)^2}
\end{eqnarray*}
\end{frame}
\section{二阶系统时域分析}
\label{sec-2}
\subsection{利用单个复根分析二阶欠阻尼系统}
\label{sec-2-1}
\begin{frame}
\frametitle{典型二阶系统传递函数:}
\label{sec-2-1-1}

\begin{eqnarray*}
\phi(s) &=& \frac{\omega_n^2}{s^2+2\xi\omega_n s+\omega_n^2}\\
&= & \frac{a^2+b^2}{s^2+2a s+a^2+b^2}\\
&=&  \frac{a^2+b^2}{(s+a+bi)(s+a-bi)}\\
\omega_n &=& a^2+b^2 \\
\xi &=& \cos\beta\\
a &=&\omega_n\cos\beta \\
b &=&\omega_n\sin\beta
\end{eqnarray*}
\end{frame}
\begin{frame}
\frametitle{分解成两个复一阶系统并联}
\label{sec-2-1-2}

\begin{eqnarray*}
\phi(s) &=&\frac{a^2+b^2}{(s+a+bi)(-a-bi+a-bi)}+\frac{a^2+b^2}{(-a+bi+a+bi)(s+a-bi)}\\
&=& \frac{a^2+b^2}{-2bi(s+a+bi)}+\frac{a^2+b^2}{2bi(s+a-bi)}
\end{eqnarray*}
\end{frame}
\begin{frame}
\frametitle{输入为单位阶跃信号,输出为:}
\label{sec-2-1-3}

\begin{eqnarray*}
C(s) &=& \frac{a^2+b^2}{s(-2bi)(s+a+bi)}+\frac{a^2+b^2}{s(2bi)(s+a-bi)}\\
     &=& \frac{a^2+b^2}{s(-2bi)(0+a+bi)}+\frac{a^2+b^2}{(-a-bi)(-2bi)(s+a+bi)}
\begin{itemize}
\item \frac{a^2+b^2}{s(2bi)(0+a-bi)}+\frac{a^2+b^2}{(-a+bi)(2bi)(s+a-bi)}\\
\end{itemize}
     &=& \frac{a-bi}{s(-2bi)}+\frac{a-bi}{(2bi)(s+a+bi)}
\begin{itemize}
\item \frac{a+bi}{s(2bi)}+\frac{-a-bi}{(2bi)(s+a-bi)}\\
\end{itemize}
     &=& \frac{b+ai}{s(2b)}+\frac{-b-ai}{(2b)(s+a+bi)}
\begin{itemize}
\item \frac{b-ai}{s(2b)}+\frac{-b+ai}{(2b)(s+a-bi)}\\
\end{itemize}
     &=& \frac{1}{s}+\frac{-b-ai}{(2b)(s+a+bi)}+\frac{-b+ai}{(2b)(s+a-bi)}\\
     &=& \frac{1}{s}+\frac{\sqrt{a^2+b^2}}{2b}e^{(-\frac{\pi}{2}-\beta)i}\cdot\frac{1}{s+a+bi} 
\begin{itemize}
\item \frac{\sqrt{a^2+b^2}}{2b}e^{(\frac{\pi}{2}+\beta)i}\cdot\frac{1}{(s+a-bi)}\\
\end{itemize}
     &=& \frac{1}{s}+\frac{\omega_n}{2b}e^{(-\frac{\pi}{2}-\beta)i}\cdot\frac{1}{s+a+bi} 
\begin{itemize}
\item \frac{\omega_n}{2b}e^{(\frac{\pi}{2}+\beta)i}\cdot\frac{1}{s+a-bi}\\
\end{itemize}
c(t) &=& 1+\frac{\omega_n}{2b}e^{(-\frac{\pi}{2}-\beta)i}e^{(-a-bi)t} 
\begin{itemize}
\item \frac{\omega_n}{2b}e^{(\frac{\pi}{2}+\beta)i}e^{(-a+bi)t}\\
\end{itemize}
     &=& 1+\frac{\omega_n}{2b}e^{-at} e^{(-bt-\frac{\pi}{2}-\beta)i}
\begin{itemize}
\item \frac{\omega_n}{2b}e^{-at}e^{(bt+\frac{\pi}{2}+\beta)i}\\
\end{itemize}
     &=& 1+\frac{\omega_n}{b}e^{-at} \cos(bt+\frac{\pi}{2}+\beta)\\
     &=& 1-\frac{\omega_n}{b}e^{-at} \sin(bt+\beta)\\
\end{eqnarray*}
\end{frame}
\begin{frame}
\frametitle{两个复一阶系统的输出分别为:}
\label{sec-2-1-4}

\begin{eqnarray*}
C_1(s) &=& \frac{a^2+b^2}{s(-2bi)(s+a+bi)}\\
     &=& \frac{a^2+b^2}{s(-2bi)(0+a+bi)}+\frac{a^2+b^2}{(-a-bi)(-2bi)(s+a+bi)}\\
     &=& \frac{a-bi}{s(-2bi)}+\frac{a-bi}{(2bi)(s+a+bi)}\\
     &=& \frac{b+ai}{2bs}+\frac{-b-ai}{2b(s+a+bi)}\\
     &=& \frac{b+ai}{2bs}+\frac{\sqrt{a^2+b^2}}{2b}e^{(-\frac{\pi}{2}-\beta)i}\cdot\frac{1}{s+a+bi} \\
     &=& \frac{b+ai}{2bs}+\frac{\omega_n}{2b}e^{(-\frac{\pi}{2}-\beta)i}\cdot\frac{1}{s+a+bi} \\
c_1(t) &=& \frac{1}{2}+\frac{ai}{2b}+\frac{\omega_n}{2b}e^{(-\frac{\pi}{2}-\beta)i}e^{(-a-bi)t} \\
     &=& \frac{1}{2}+\frac{ai}{2b}+\frac{\omega_n}{2b}e^{-at} e^{(-bt-\frac{\pi}{2}-\beta)i}\\
\Re[c_1(t)] &=& \frac{1}{2}+\frac{\omega_n}{2b}e^{-at}\cos(bt+\frac{\pi}{2}+\beta)\\
            &=& \frac{1}{2}-\frac{\omega_n}{2b}e^{-at}\sin(bt+\beta)\\
            &=& \frac{c(t)}{2}\\
\Im[c_1(t)] &=& \frac{ai}{2b}+\frac{\omega_n}{2b}e^{-at} \sin(-bt-\frac{\pi}{2}-\beta)\\
            &=& \frac{ai}{2b}-\frac{\omega_n}{2b}e^{-at} \cos(bt+\beta)\\
\end{eqnarray*}

与

\begin{eqnarray*}
C_2(s) &=& \frac{a^2+b^2}{s(2bi)(s+a-bi)}\\
     &=& \frac{a^2+b^2}{s(2bi)(0+a-bi)}+\frac{a^2+b^2}{(-a+bi)(2bi)(s+a-bi)}\\
     &=& \frac{a+bi}{s(2bi)}+\frac{-a-bi}{(2bi)(s+a-bi)}\\
     &=& \frac{b-ai}{2bs)}+\frac{-b+ai}{2b(s+a-bi)}\\
     &=& \frac{b-ai}{2bs)}+\frac{\sqrt{a^2+b^2}}{2b}e^{(\frac{\pi}{2}+\beta)i}\cdot\frac{1}{(s+a-bi)}\\
     &=& \frac{b-ai}{2bs)}+\frac{\omega_n}{2b}e^{(\frac{\pi}{2}+\beta)i}\cdot\frac{1}{s+a-bi}\\
c_2(t) &=& \frac{1}{2}+\frac{-ai}{2b}+\frac{\omega_n}{2b}e^{(\frac{\pi}{2}+\beta)i}e^{(-a+bi)t}\\
     &=& \frac{1}{2}+\frac{-ai}{2b}+\frac{\omega_n}{2b}e^{-at}e^{(bt+\frac{\pi}{2}+\beta)i}\\
\Re[c_2(t)] &=& \frac{1}{2}+\frac{\omega_n}{2b}e^{-at}\cos(bt+\frac{\pi}{2}+\beta)\\
     &=& 1-\frac{\omega_n}{2b}e^{-at} \sin(bt+\beta)\\
     &=& \Re[c_1](t) \\
\Im[c_2(t)] &=& \frac{-ai}{2b}+\frac{\omega_n}{2b}e^{-at}\sin(bt+\frac{\pi}{2}+\beta)\\
            &=& \frac{-ai}{2b}+\frac{\omega_n}{2b}e^{-at}\cos(bt+\beta)\\
            &=& -\Im[c_2(t)] \\
c_2(t) &=& c_1^*(t)
\end{eqnarray*}

因此,在单位阶跃输入的情况下,复一阶系统输出信号的实部与典型二阶系统的输出相等.

正如简谐振动是圆周运动的投影.正弦函数也可以看作复数的投影.欠阻尼二阶系统中的振荡特性,可以看作是复平面曲线运动的投影.
\end{frame}
\begin{frame}[fragile]
\frametitle{结果演示}
\label{sec-2-1-5}


\begin{verbatim}

t=[0:0.01:10];
a=1;
b=1;
omega=sqrt(a^2+b^2);
be=asin(b/omega);
y1=omega/2/b*exp(-a*t).*exp((-b*t-be-pi/2)*i);
y2=omega/2/b*exp(-a*t).*exp((b*t+be+pi/2)*i);
figure( 1, "visible", "on" );
subplot(2,2,1);plot3(t,1+y1)
subplot(2,2,2);plot(t,real(1+y1));
subplot(2,2,3);plot(t,imag(1+y1));
subplot(2,2,4);plot(1+y1);
print -dpng chart.png -S500,500;
ans="chart.png";
\end{verbatim}
\end{frame}
\section{传递函数模型}
\label{sec-3}
\subsection{能控能观性}
\label{sec-3-1}
\begin{frame}
\frametitle{单输入多状态}
\label{sec-3-1-1}

\begin{eqnarray*}
X_1(s) & =&\frac{1}{s+1}U(s)+\frac{1}{s+1}x_1(0) \\
X_2(s) &=& \frac{2}{s+1}U(s)+\frac{1}{s+1}x_2(0)
\end{eqnarray*}
极点相同,两个状态变量无法同时满足任意初始条件下达到给定值.例如,当初始值为0时,始终有: $x_2(t)=x_2(t)$
\begin{eqnarray*}
X_1(s) & =&\frac{1}{s+1}U(s)+\frac{1}{s+1}x_1(0) \\
X_2(s) &=& \frac{1}{s+2}U(s)+\frac{1}{s+2}x_2(0)
\end{eqnarray*}
极点不同,任意初始条件下,两个状态变量可同时达到给定值.如:设 $u(t)=\begin{cases}a & 0<t<1 \\ b & 1<t<2 \end{cases}$ ,代入式中可求得a,b.
\end{frame}
\begin{frame}
\frametitle{多状态单输}
\label{sec-3-1-2}

\begin{eqnarray*}
Y(s) & =&\frac{2}{s+1}X_1(0)+\frac{1}{s+1}x_2(0) 
\end{eqnarray*}
极点相同,根据输出,无法计算出两个状态的各自的值.例如,以最小二乘法进行计算,无法得到两个状态的初始值.
\begin{eqnarray*}
Y(s) & =&\frac{2}{s+2}X_1(0)+\frac{1}{s+1}x_2(0) 
\end{eqnarray*}
极点不同,根据输出,可以计算两个状态变量各自的值,例如,可以使用最小二乘法通过输出序列计算出两个状态的初始值.
\end{frame}
\begin{frame}
\frametitle{零极点相消}
\label{sec-3-1-3}

存在零极点相消时,导致传递函数中对应的状态变量消失,以致出现不能控或不能观的状态变量.
\begin{eqnarray*}
G(s)&= &\frac{1}{s+1} 
\end{eqnarray*}
能控不能观:
\begin{eqnarray*}
G(s) &=& \frac{1}{s+2}\cdot\frac{s+2}{s+1}\\
X_1(s) &=& \frac{1}{s+2}U(s)\\
X_2(s) &=& \frac{1}{s+1}X_1(s)\\
Y(s) &= & X_1(s)+X_2(s) \\
     &=& \frac{1}{s+1}U(s)
\end{eqnarray*}

\begin{pspicture}
\Rnode{u}{\color{purple}$u$}
{\hskip 1em }\Rnode{a}{\psframebox[framesep=0.1em]{$\frac{1}{s+2}$}}
{\hskip 1em }\Rnode{b}{\psframebox[framesep=0.1em]{$\frac{s+2}{s+1}$}}
{\hskip 1em }\Rnode{y}{\color{purple}$y$}
\ncline{->}{u}{a}
\ncline{->}{a}{b}
\ncline{->}{b}{y}
\end{pspicture}

能观不能控:
\begin{eqnarray*}
G(s) &=& \frac{s+2}{s+1}\cdot\frac{1}{s+2}\\
Y(s) &= &X_2(s)\\
X_2(s) &=& \frac{1}{s+2}(X_1(S)+U(s))\\
       &=& \frac{1}{s+1}U(s)\\
X_1(s) &=& \frac{1}{s+1}U(s)
\end{eqnarray*}

\begin{pspicture}
\Rnode{u}{\color{purple}$u$}
{\hskip 1em }\Rnode{a}{\psframebox[framesep=0.1em]{$\frac{s+2}{s+1}$}}
{\hskip 1em }\Rnode{b}{\psframebox[framesep=0.1em]{$\frac{1}{s+2}$}}
{\hskip 1em }\Rnode{y}{\color{purple}$y$}
\ncline{->}{u}{a}
\ncline{->}{a}{b}
\ncline{->}{b}{y}
\end{pspicture}
\end{frame}
\subsection{极点配置与状态观测器}
\label{sec-3-2}
\begin{frame}
\frametitle{利用状态反馈任意配置极点}
\label{sec-3-2-1}

\begin{eqnarray*}
Y(s) & =& \frac{M(s)}{N(s)}U(s)\\
     &=& M(s)X(s) \\
X(s) &=& \frac{U(s)}{N(s)}\\
N(s) &=& s^n+\sum_{i=0}^{n-1} a_i s^i \\
N(s)X(s) &=& s^n X(s)+\sum_{i=0}^{n-1} a_i s^i X(s)  \\
         &=& U(s) \\
U(s) &=& R(s)+\sum_{i=0}^{n-1}k_i s^i X(s) \\
N'(s)X(s) &=& R(s) \\
N'(s) &=& s^n +\sum_{i=0}^{n-1}(a_i+k_i)s^i  \\
Y(s) &=& \frac{M(s)}{N'(s)}R(s)
\end{eqnarray*}
其中 $s^i X(s)$ 为状态变量, $k_i$ 为状态反馈系数, 可任意配置传递函数极点
\end{frame}
\begin{frame}
\frametitle{利用矩阵推导状态观测器}
\label{sec-3-2-2}

将状态变量初值看作扰动

\begin{eqnarray*}
sX(s) &= & AX(s)+BU(s)+F \qquad , F=X_0\\
Y(s) &=& CX(s)\\
s\hat{X}(s) &=& A\hat{X}(s)+BU(s)+KC(X(s)-\hat{X}(s))\\
\hat{Y}(s) &=& C\hat{X}(s)\\
E(s) &=& X(s)-\hat{X}(s) \\
sE(s) &=& AE(s)+F-KCE(s)\\
sE(s) &=& (A-KC)E(s)+F\\
E(s) &=& (sI-A+KC)^{-1}F
\end{eqnarray*}


对于不稳定的被控对象,利用观测到的状态构建状态反馈即可实现稳定性.

已知:
\begin{eqnarray*}
(sI-A)X(s) &= & BU(s)+F \qquad , F=X_0\\
(sI-A+KC)E(s) &=& F
\end{eqnarray*}

引入观测状态作为反馈:
\begin{eqnarray*}
(sI-A)X(s) & =& BK_f\hat{X}(s)+F \\
(sI-A)X(s) & =& BK_f(X(s)-E(s))+F \\
(sI-A-BK_f)X(s) & =& BK_fE(s)+F 
\end{eqnarray*}

可得新的等效系统模型:
\begin{eqnarray*}
(sI-A-BK_f)X(s) & =& BK_fE(s)+F \\
(sI-A+KC)E(s) &=& F
\end{eqnarray*}

合理选择 $K,K_f$ 即可实现状态观测与状态反馈.
\end{frame}
\begin{frame}
\frametitle{利用传递函数推导}
\label{sec-3-2-3}

\begin{pspicture}
\Rnode{u}{\color{purple}$u$}
{\hskip 3em }\Rnode{p1}{\color{purple}$\circ$}
{\hskip 1em }\Rnode{a}{\psframebox[framesep=0.1em]{$\frac{1}{s}$}}
{\hskip 2em }\Rnode{p2}{\color{purple}$\circ$}
{\hskip 1em }\Rnode{m1}{\psframebox[framesep=0.1em]{$\frac{M(s)}{N(s)}$}}
{\hskip 1em }\Rnode{p6}{\color{purple}$\circ$}
\nput[labelsep=0]{130}{p6}{\color{purple}$-$}
\nput[labelsep=2em]{90}{p1}{\color{purple} \Rnode{c}{$c$}}
\nput[labelsep=3em]{90}{a}{\color{purple} \Rnode{f}{\psframebox[framesep=0.1em]{$a$}}}
\nput[labelsep=1em]{-90}{a}{\color{purple} \Rnode{b}{\psframebox[framesep=0.1em]{$p$}}}
\ncline{->}{u}{p1}
\ncline{->}{p1}{a}
\ncline{->}{a}{p2}
\ncline{->}{p2}{m1}
\ncline{->}{m1}{p6}
\ncline{->}{c}{p1}
\ncangle[angleA=0,angleB=180,armB=3em]{->}{u}{f}
\ncangle[angleA=0,angleB=90]{->}{f}{p2}
\ncangle[angleA=0,angleB=0,armA=1em]{->}{a}{b}
\ncangle[angleA=180,angleB=-90]{->}{b}{p1}

\nput[labelsep=5em]{-90}{a}{
{\hskip 3em }\Rnode{p3}{\color{purple}$\circ$}
{\hskip 1em }\Rnode{p4}{\color{purple}$\circ$}
{\hskip 1em }\Rnode{g}{\psframebox[framesep=0.1em]{$\frac{1}{s}$}}
{\hskip 2em }\Rnode{p5}{\color{purple}$\circ$}
{\hskip 1em }\Rnode{m2}{\psframebox[framesep=0.1em]{$\frac{M(s)}{N(s)}$}}
\nput[labelsep=1em]{90}{g}{\color{purple} \Rnode{fg}{\psframebox[framesep=0.1em]{$a$}}}
\nput[labelsep=1em]{-90}{g}{\color{purple} \Rnode{bg}{\psframebox[framesep=0.1em]{$p$}}}
\nput[labelsep=1em]{-90}{bg}{\color{purple} \Rnode{k}{\psframebox[framesep=0.1em]{$k$}}}
}
\ncline{->}{p3}{p4}
\ncline{->}{p4}{g}
\ncline{->}{g}{p5}
\ncline{->}{p5}{m2}
\ncangle[angleA=0,angleB=180,armB=3em]{->}{p3}{fg}
\ncangle[angleA=0,angleB=90]{->}{fg}{p5}
\ncangle[angleA=0,angleB=0,armA=1em]{->}{g}{bg}
\ncangle[angleA=180,angleB=-90]{->}{bg}{p4}
\ncangle[angleA=0,angleB=180]{->}{u}{p3}
\ncangle[angleA=0,angleB=-90]{->}{m2}{p6}
\ncbar[angleA=0,angleB=0,armA=1em]{->}{p6}{k}
\ncangle[angleA=180,angleB=-90]{->}{k}{p3}
\end{pspicture}


\begin{pspicture}
\Rnode{u}{\color{purple}$u$}
{\hskip 3em }\Rnode{p1}{\color{purple}$\circ$}
{\hskip 1em }\Rnode{a}{\psframebox[framesep=0.1em]{$\frac{1}{s}$}}
{\hskip 2em }\Rnode{p2}{\color{purple}$\circ$}
{\hskip 2em }\Rnode{p6}{\color{purple}$\circ$}
\nput[labelsep=0]{130}{p6}{\color{purple}$-$}
\nput[labelsep=2em]{90}{p1}{\color{purple} \Rnode{c}{$c$}}
\nput[labelsep=3em]{90}{a}{\color{purple} \Rnode{f}{\psframebox[framesep=0.1em]{$a$}}}
\nput[labelsep=1em]{-90}{a}{\color{purple} \Rnode{b}{\psframebox[framesep=0.1em]{$p$}}}
\ncline{->}{u}{p1}
\ncline{->}{p1}{a}
\ncline{->}{a}{p2}
\ncline{->}{p2}{p6}
\ncline{->}{c}{p1}
\ncangle[angleA=0,angleB=180,armB=3em]{->}{u}{f}
\ncangle[angleA=0,angleB=90]{->}{f}{p2}
\ncangle[angleA=0,angleB=0,armA=1em]{->}{a}{b}
\ncangle[angleA=180,angleB=-90]{->}{b}{p1}

\nput[labelsep=5em]{-90}{p1}{
{\hskip 3em }\Rnode{p3}{\color{purple}$\circ$}
{\hskip 1em }\Rnode{p4}{\color{purple}$\circ$}
{\hskip 1em }\Rnode{g}{\psframebox[framesep=0.1em]{$\frac{1}{s}$}}
{\hskip 2em }\Rnode{p5}{\color{purple}$\circ$}
\nput[labelsep=1em]{90}{g}{\color{purple} \Rnode{fg}{\psframebox[framesep=0.1em]{$a$}}}
\nput[labelsep=1em]{-90}{g}{\color{purple} \Rnode{bg}{\psframebox[framesep=0.1em]{$p$}}}
\nput[labelsep=1em]{-90}{bg}{\color{purple} \Rnode{k}{\psframebox[framesep=0.1em]{$\frac{kM(s)}{N(s)}$}}}
}
\ncline{->}{p3}{p4}
\ncline{->}{p4}{g}
\ncline{->}{g}{p5}
\ncangle[angleA=0,angleB=180,armB=3em]{->}{p3}{fg}
\ncangle[angleA=0,angleB=90]{->}{fg}{p5}
\ncangle[angleA=0,angleB=0,armA=1em]{->}{g}{bg}
\ncangle[angleA=180,angleB=-90]{->}{bg}{p4}
\ncangle[angleA=0,angleB=180]{->}{u}{p3}
\ncangle[angleA=0,angleB=-90]{->}{p5}{p6}
\ncbar[angleA=0,angleB=0,armA=1em]{->}{p6}{k}
\ncangle[angleA=180,angleB=-90]{->}{k}{p3}
\end{pspicture}

\begin{pspicture}
\Rnode{u}{\color{purple}$u$}
{\hskip 3em }\Rnode{p1}{\color{purple}$\circ$}
{\hskip 1em }\Rnode{a}{\psframebox[framesep=0.1em]{$\frac{1}{s}$}}
{\hskip 2em }\Rnode{p2}{\color{purple}$\circ$}
{\hskip 2em }\Rnode{p6}{\color{purple}$\circ$}
\nput[labelsep=0]{130}{p2}{\color{purple}$-$}
\nput[labelsep=0.5em]{110}{p6}{\color{purple}$-$}
\nput[labelsep=2em]{90}{p1}{\color{purple} \Rnode{c}{$c$}}
\nput[labelsep=3em]{90}{a}{\color{purple} \Rnode{f}{\psframebox[framesep=0.1em]{$a$}}}
\nput[labelsep=1em]{-90}{a}{\color{purple} \Rnode{b}{\psframebox[framesep=0.1em]{$p$}}}
\ncline{->}{u}{p1}
\ncline{->}{p1}{a}
\ncline{->}{a}{p2}
\ncline{->}{p2}{p6}
\naput{\color{purple}$e$}
\ncline{->}{c}{p1}
\ncangle[angleA=0,angleB=180,armB=3em]{->}{u}{f}
\ncangle[angleA=0,angleB=90]{->}{f}{p6}
\ncangle[angleA=0,angleB=0,armA=1em]{->}{a}{b}
\ncangle[angleA=180,angleB=-90]{->}{b}{p1}

\nput[labelsep=5em]{-90}{p1}{
{\hskip 3em }\Rnode{p3}{\color{purple}$\circ$}
{\hskip 1em }\Rnode{p4}{\color{purple}$\circ$}
{\hskip 1em }\Rnode{g}{\psframebox[framesep=0.1em]{$\frac{1}{s}$}}
% {\hskip 2em }\Rnode{p5}{\color{purple}$\circ$}
\nput[labelsep=1em]{-90}{g}{\color{purple} \Rnode{bg}{\psframebox[framesep=0.1em]{$p$}}}
\nput[labelsep=1em]{-90}{bg}{\color{purple} \Rnode{fg}{\psframebox[framesep=0.1em]{$a$}}}
\nput[labelsep=1em]{-90}{fg}{\color{purple} \Rnode{k}{\psframebox[framesep=0.1em]{$\frac{kM(s)}{N(s)}$}}}
}
\ncline{->}{p3}{p4}
\ncline{->}{p4}{g}
\ncangle[angleA=0,angleB=180,armB=3em]{->}{p3}{fg}
\ncangle[angleA=0,angleB=-90]{->}{fg}{p6}
\ncangle[angleA=0,angleB=0,armA=1em]{->}{g}{bg}
\ncangle[angleA=180,angleB=-90]{->}{bg}{p4}
\ncangle[angleA=0,angleB=180]{->}{u}{p3}
\ncangle[angleA=0,angleB=-90]{->}{g}{p2}
\ncbar[angleA=0,angleB=0,armA=1em]{->}{p6}{k}
\ncangle[angleA=180,angleB=-90]{->}{k}{p3}
\end{pspicture}

\begin{eqnarray*}
\frac{e}{u} &= & 0\\
\frac{e}{c} &=& \frac{1}{s+p}\frac{1}{1+\frac{1}{s+p}\frac{KM}{N+KMa} \\
            &=& \frac{1}{s+p+\frac{KM}{N+KMa} \\
            &=& \frac{N+KMa}{(s+p)(N+KMa)+KM}
\end{eqnarray*}

只需合理选取k,使系统稳定,则可以准确获取状态.(系统极点与反馈K单独作用于被控对象时相同)

系统不稳定时,可以根据观测到的状态构建状态反馈.

\begin{pspicture}
\Rnode{u}{\color{purple}$u$}
{\hskip 3em }\Rnode{p0}{\color{purple}$\circ$}
{\hskip 1em }\Rnode{p1}{\color{purple}$\circ$}
{\hskip 1em }\Rnode{a}{\psframebox[framesep=0.1em]{$\frac{1}{s}$}}
{\hskip 2em }\Rnode{p2}{\color{purple}$\circ$}
{\hskip 2em }\Rnode{p6}{\color{purple}$\circ$}
\nput[labelsep=0]{130}{p2}{\color{purple}$-$}
\nput[labelsep=0.5em]{110}{p6}{\color{purple}$-$}
\nput[labelsep=2em]{90}{p1}{\color{purple} \Rnode{c}{$c$}}
\nput[labelsep=3em]{90}{a}{\color{purple} \Rnode{f}{\psframebox[framesep=0.1em]{$a$}}}
\nput[labelsep=1em]{-90}{a}{\color{purple} \Rnode{b}{\psframebox[framesep=0.1em]{$p$}}}
\ncline{->}{u}{p0}
\ncline{->}{p0}{p1}
\ncline{->}{p1}{a}
\ncline{->}{a}{p2}
\ncline{->}{p2}{p6}
\naput{\color{purple}$e$}
\ncline{->}{c}{p1}
\ncangle[angleA=0,angleB=180,armB=5em]{->}{u}{f}
\ncangle[angleA=0,angleB=90]{->}{f}{p6}
\ncangle[angleA=0,angleB=0,armA=1em]{->}{a}{b}
\ncangle[angleA=180,angleB=-90]{->}{b}{p1}

\nput[labelsep=5em]{-90}{p1}{
{\hskip 3em }\Rnode{p3}{\color{purple}$\circ$}
{\hskip 1em }\Rnode{p4}{\color{purple}$\circ$}
{\hskip 1em }\Rnode{g}{\psframebox[framesep=0.1em]{$\frac{1}{s}$}}
}
% {\hskip 2em }\Rnode{p5}{\color{purple}$\circ$}
\nput[labelsep=1em]{-90}{g}{\color{purple} \Rnode{bg}{\psframebox[framesep=0.1em]{$p$}}}
\nput[labelsep=1em]{-90}{bg}{\color{purple} \Rnode{fg}{\psframebox[framesep=0.1em]{$a$}}}
\nput[labelsep=1em]{-90}{fg}{\color{purple} \Rnode{k}{\psframebox[framesep=0.1em]{$\frac{kM(s)}{N(s)}$}}}
\nput[labelsep=1em]{90}{g}{\color{purple} \Rnode{kf}{\psframebox[framesep=0.1em]{$k_f$}}}
\ncline{->}{p3}{p4}
\ncline{->}{p4}{g}
\ncangle[angleA=0,angleB=180,armB=3em]{->}{p3}{fg}
\ncangle[angleA=0,angleB=-90]{->}{fg}{p6}
\ncangle[angleA=0,angleB=0,armA=1em]{->}{g}{bg}
\ncangle[angleA=180,angleB=-90]{->}{bg}{p4}
\ncangle[angleA=0,angleB=180,armA=1em,armB=3em]{->}{u}{p3}
\ncangle[angleA=0,angleB=-90]{->}{g}{p2}
\ncbar[angleA=0,angleB=0,armA=1em]{->}{p6}{k}
\ncangle[angleA=180,angleB=-90]{->}{k}{p3}
\ncbar[angleA=0]{->}{g}{kf}
\ncangle[angleA=180,angleB=90]{->}{kf}{p3}
\ncangle[angleA=180,angleB=-90]{->}{kf}{p0}
\end{pspicture}

可以看到, 有无反馈 $k_f$ 对误差信号 $e$ 无任何影响.即 $e$ 仍按原来的规律趋近于0. 因此,原系统等效于:

\begin{pspicture}
\Rnode{u}{\color{purple}$u$}
{\hskip 3em }\Rnode{p0}{\color{purple}$\circ$}
{\hskip 1em }\Rnode{p1}{\color{purple}$\circ$}
{\hskip 1em }\Rnode{a}{\psframebox[framesep=0.1em]{$\frac{1}{s}$}}
{\hskip 2em }\Rnode{p2}{\color{purple}$\circ$}
{\hskip 2em }\Rnode{m}{\color{purple}\psframebox[framesep=0.1em]{$\frac{M(s)}{N(s)}$}}
{\hskip 2em }\Rnode{y}{\color{purple}$y$}
\nput[labelsep=2em]{90}{p1}{\color{purple} \Rnode{c}{$c$}}
\nput[labelsep=2em]{90}{p0}{\color{purple} \Rnode{e}{$e$}}
\nput[labelsep=3em]{90}{a}{\color{purple} \Rnode{f}{\psframebox[framesep=0.1em]{$a$}}}
\nput[labelsep=1em]{-90}{a}{\color{purple} \Rnode{b}{\psframebox[framesep=0.1em]{$p$}}}
\nput[labelsep=1em]{-90}{b}{\color{purple} \Rnode{k}{\psframebox[framesep=0.1em]{$k_f$}}}
\ncline{->}{u}{p0}
\ncline{->}{p0}{p1}
\ncline{->}{p1}{a}
\ncline{->}{a}{p2}
\ncline{->}{p2}{m}
\ncline{->}{m}{y}
\ncline{->}{c}{p1}
\ncline{->}{e}{p0}
\ncangle[angleA=0,angleB=180,armB=5em]{->}{u}{f}
\ncangle[angleA=0,angleB=90]{->}{f}{p2}
\ncangle[angleA=0,angleB=0,armA=1em]{->}{a}{b}
\ncangle[angleA=180,angleB=-90]{->}{b}{p1}
\ncangle[angleA=0,angleB=0,armA=1em]{->}{a}{k}
\ncangle[angleA=180,angleB=-90]{->}{k}{p0}
\end{pspicture}

此系统与未加反馈 $k_f$ 的原系统一起组成新的等效系统.其中信号 $e$ 从原系统中得到,按原系统中的规律趋近于零.因此,若选取的状态反馈 $k_f$ 能使被控对象稳定,则从观测到的状态引出的反馈同样能使系统稳定.
\end{frame}
\section{泰勒级数计算}
\label{sec-4}
\subsection{有理分式展开为泰勒级数}
\label{sec-4-1}

自动控制原理教材提到动态误差系数计算以及Z反变换计算,本质上都是计算泰勒级数的系数.

有理分式 $G(s)=\frac{M(s)}{N(s)}$ , 其中 $M(s)$ 与 \$N(s)\$均为 $s$ 的多项式, 
当极点均不为零时, 可在 $s=0$ 处展开为泰勒级数. 

\begin{eqnarray}
\frac{M(s)}{N(s)} &=&c_0+c_1s+c_2s^2+\cdots+c_n s^n+\cdots  \\
\end{eqnarray}

利用部分分式表示形式, 可以得到各项系数 $c_n$ . 当 $N(s)$ 无重根时:

\begin{eqnarray}
\frac{M(s)}{N(s)} &=& \sum_{i=1}^I \frac{a_i}{s-p_i} \\
\frac{a_i}{s-p_i} &=& \frac{a_i}{s-p_i} \\
                  &=& \frac{-a_i/p_i}{1-s/p_i} \\
                  &=& \frac{-a_i}{p_i}+\cdots+\frac{-a_i}{p_i}(\frac{s}{p_i})^n+\cdots \\
\sum_{i=1}^I \frac{a_i}{s-p_i} &=& \sum_{i=1}^I \frac{-a_i}{p_i}+\cdots+\sum_{i=1}^I \frac{-a_i}{p_i}(\frac{s}{p_i})^n+\cdots 
\end{eqnarray}

当 $N(s)$ 有重根时,需要针对重根作进一步的推导,设 $\lambda$ 为 $J$ 重根,则:

\begin{eqnarray}
\frac{M(s)}{N(s)} &=& \sum_{i=1}^I \frac{a_i}{s-p_i}+\sum_{j=1}^{J}\frac{b_j}{(s-\lambda)^j} \\
\frac{b_j}{(s-\lambda)^j} &=& \frac{(-1)^{j-1}}{(j-1)!}\cdot\frac{d^{j-1}}{ds^{j-1}}(\frac{b_j}{s-\lambda}) \\
\frac{b_j}{s-\lambda} &=& \frac{-b_j/\lambda}{1-s/\lambda} \\
                  &=& \frac{-b_j}{\lambda}+\cdots+\frac{-b_j}{\lambda}(\frac{s}{\lambda})^n+\cdots \\
\frac{b_j}{(s-\lambda)^j} &=& \cdots+\frac{(-1)^{j-1}}{(j-1)!}\frac{-b_j}{\lambda^{n+1}}\frac{d^js^n}{ds^j}+\cdots \\
\end{eqnarray}
\subsection{利用留数计算}
\label{sec-4-2}

\begin{eqnarray*}
s &=&\frac{1}{t} \\
\frac{M(s)}{N(s)}&=&\frac{M(1/t)}{N(1/t)}\\
\frac{M(1/t)}{N(1/t)} &=& c_0+c_1t^{-1}+\cdots+c_nt^{-n}+\cdots \\
\end{eqnarray*}
计算 $\frac{M(1/t)}{N(1/t)}t^{n-1}$ 的留数可得 $c_n$
\section{正弦输入时的稳态误差消除}
\label{sec-5}
\subsection{使用前馈}
\label{sec-5-1}

\begin{align*}
R(s) &=\frac{1}{s^2+1}\\
G(s) &=\frac{1}{s+1}\\
G_r(s) &= \frac{2s}{s+1} \\
\Phi_e(s) &=\frac{1-G G_r}{1+G}\\
&= \frac{s^2+1}{s+2}\\
\Phi_e(s)R(s) &=\frac{1}{s+2}\\
\lim_{t\to\infty}e_{ss}(t) &=\lim_{s\to 0}\Phi_e(s)R(s)\\
 &=0
\end{align*}
\subsection{内模控制}
\label{sec-5-2}

\begin{align*}
R(s) &=\frac{1}{s^2+1}\\
G(s) &=\frac{1}{s+1}\\
G_c(s) &= \frac{1+2s}{s^2+1} \\
\Phi_e(s) &=\frac{1}{1+G_c G}\\
&= \frac{(s^2+1)(s+1)}{(1+s)(s^2+1)+2s+1}\\
\Phi_e(s)R(s) &=\frac{s+1}{(1+s)(s^2+1)+2s+1}\\
\lim_{t\to\infty}e_{ss}(t) &=\lim_{s\to 0}\Phi_e(s)R(s)\\
 &=0
\end{align*}

\end{document}
