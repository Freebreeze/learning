% Created 2013-10-24 Thu 13:36
\documentclass[table]{beamer}
\usepackage[utf8]{inputenc}
\usepackage[T1]{fontenc}
\usepackage{fixltx2e}
\usepackage{graphicx}
\usepackage{longtable}
\usepackage{float}
\usepackage{wrapfig}
\usepackage{soul}
\usepackage{textcomp}
\usepackage{marvosym}
\usepackage{wasysym}
\usepackage{latexsym}
\usepackage{amssymb}
\usepackage{hyperref}
\tolerance=1000
\usepackage{amsmath}
\usepackage[usenames]{color}
\usepackage{pstricks}
\usepackage{pgfplots}
\usepackage{tikz}
\usepackage[europeanresistors,americaninductors]{circuitikz}
\usepackage{colortbl}
\usepackage{yfonts}
\usetikzlibrary{shapes,arrows}
\usetikzlibrary{positioning}
\usetikzlibrary{arrows,shapes}
\usetikzlibrary{intersections}
\usetikzlibrary{calc,patterns,decorations.pathmorphing,decorations.markings}
\usepackage[BoldFont,SlantFont,CJKchecksingle]{xeCJK}
\setCJKmainfont[BoldFont=Evermore Hei]{Evermore Kai}
\setCJKmonofont{Evermore Kai}
\xeCJKsetup{CJKglue=\hspace{0pt plus .08 \baselineskip }}
\usepackage{pst-node}
\usepackage{pst-plot}
\psset{unit=5mm}
\usepackage{beamerarticle}
\mode<beamer>{\usetheme{Frankfurt}}
\mode<beamer>{\usecolortheme{dove}}
\mode<article>{\hypersetup{colorlinks=true,pdfborder={0 0 0}}}
\mode<beamer>{\AtBeginSection[]{\begin{frame}<beamer>\frametitle{Topic}\tableofcontents[currentsection]\end{frame}}}
\setbeamercovered{transparent}
\subtitle{基本概念}
\providecommand{\alert}[1]{\textbf{#1}}

\title{线性系统的根轨迹法}
\author{}
\date{}
\hypersetup{
  pdfkeywords={},
  pdfsubject={},
  pdfcreator={Emacs Org-mode version 7.9.3f}}

\begin{document}

\maketitle

\begin{frame}
\frametitle{Outline}
\setcounter{tocdepth}{3}
\tableofcontents
\end{frame}















\section{零极点与根轨迹}
\label{sec-1}
\begin{frame}
\frametitle{根轨迹法}
\label{sec-1-1}

\begin{itemize}
\item <2->系统性能由闭环极点决定
\item <3->目的:分析系统参数变化对系统性能影响
\item <4->根轨迹法:根据系统开环零极点作出系统闭环极点随参数变动的轨迹
\end{itemize}
\end{frame}
\begin{frame}
\frametitle{分析 $K$ 变化对系统性能的影响}
\label{sec-1-2}


\begin{tikzpicture}[node distance=2em,auto,>=latex', thick]
%\path[use as bounding box] (-1,0) rectangle (10,-2); 
\path[->] node[] (r) { $ r(t) $ }; 
\path[->] node[ circle,inner sep=2pt,minimum size=1pt,draw,label=below left: $   $ ,right =of r] (p1) { }; 
\path[->](r) edge node {} (p1) ; 
%\path[red] node[draw, right =of p1] (n) { $ N $ }; 
%\path[->] (p1) edge node[midway] { $ x(t) $ } (n) ; 
\path[red] node[draw, inner sep=5pt,right =of p1] (g) { $ \frac{K}{s(s+1)} $ }; 
\path[->] (p1) edge node [midway]{ $   $ } (g); 
\path[->] node[ right =of g] (o) { $ c(t) $ }; 
\path[->] (g) edge node {} (o); 
\path[->, draw] (g.east)+(1em,0) -- +(1em,-3em) -| node[very near end] { $ - $ } (p1); 
\end{tikzpicture} 

\begin{eqnarray*}
\Phi(s) & = &\frac{K}{s^2+s+K} \\
D(s) &=& s^2+s+K \\
D(s) &=& 0 \\
s_{1,2} &=& \frac{-1\pm\sqrt{1-4K}}{2}
\end{eqnarray*}

\mode<beamer>{\rowcolors[]{1}{green!10}{blue!10}}
\begin{tabular}{l!{\vrule}c<{\onslide<2->}c<{\onslide<3->}c<{\onslide<4->}c<{\onslide<5->}c<{\onslide<6->}c<{\onslide<7->}}
K   & 0  &  0.25 & 1          & 4         & $\cdots$ & $\infty$ \\
$s_1$ & 0  &  -0.5 & -0.5+0.87j & -0.5+1.9j & $\cdots$ & -0.5+ $\infty$ j \\
$s_1$ & -1 &  -0.5 & -0.5-0.87j & -0.5-1.9j & $\cdots$ & -0.5- $\infty$ j 
\end{tabular}
\end{frame}
\section{根轨迹的基本条件}
\label{sec-2}
\begin{frame}
\frametitle{根轨迹条件:系统模型}
\label{sec-2-1}


\begin{tikzpicture}[node distance=2em,auto,>=latex', thick]
%\path[use as bounding box] (-1,0) rectangle (10,-2); 
\path[->] node[] (r) { $ r(t) $ }; 
\path[->] node[ circle,inner sep=2pt,minimum size=1pt,draw,label=below left: $   $ ,right =of r] (p1) { }; 
\path[->](r) edge node {} (p1) ; 
%\path[red] node[draw, right =of p1] (n) { $ N $ }; 
%\path[->] (p1) edge node[midway] { $ x(t) $ } (n) ; 
\path[red] node[draw, inner sep=5pt,right =of p1] (g) { $ G(s) $ }; 
\path[->] (p1) edge node [midway]{ $   $ } (g); 
\path[->] node[ right =of g] (o) { $ c(t) $ }; 
\path[->] (g) edge node {} (o); 
\path[red] node[draw, inner sep=5pt,below =of g] (h) { $ H(s) $ }; 
\path[->,draw] (g.east)+(1em,0) |- (h.east); 
\path[->, draw] (h) -| node[very near end] { $ - $ } (p1); 
%\path[->, draw] (g.east)+(1em,0) -- +(1em,-3em) -| node[very near end] { $ - $ } (p1); 
\end{tikzpicture} 

\begin{itemize}
\item <2-> 开环传递函数(零极点形式):  
          \[G(s)=\frac{K_g\prod_{i=1}^m(s-z_i)}{\prod_{j=1}^n(s-p_j)}\]
\item <3-> 变动的参数:根轨迹增益  $K_g$
\item <4-> $K_g$ 从  $0\rightarrow\infty$  时,闭环极点的运动轨迹
\item <5-> 对于负反馈系统,称为  $180^\circ$  根轨迹.
\end{itemize}
\end{frame}
\begin{frame}
\frametitle{根轨迹条件:幅值条件与相角条件}
\label{sec-2-2}


\begin{eqnarray*}
\Phi(s) &=& \frac{G(s)}{1+G(s)H(s)} \\
D(s) &= &1+G(s)H(s) 
     = 0 \\
G(s)H(s) &=& -1 \\
\frac{K_g\prod_{i=1}^m(s+z_i)}{\prod_{j=1}^n(s+p_j)} &=& -1 
\end{eqnarray*}

\begin{itemize}
\item <2->幅值条件:  $\frac{K_g\prod_{i=1}^m|s+z_i|}{\prod_{j=1}^n|s+p_j|} = 1$
\item <3->相角条件:  $\sum_{i=1}^m\angle (s+z_i)-\sum_{j=1}^n\angle (s+p_j) = (2k+1)\pi$
\item <4->命题
\begin{itemize}
\item <4->满足相角条件的点一定是根轨迹上的点
\item <5->根轨迹上的点何时成为系统闭环极点由  $K_g$  决定
\end{itemize}
\item <6->结论: 绘制根轨迹只依据相角条件即可
\end{itemize}
\end{frame}

\end{document}
