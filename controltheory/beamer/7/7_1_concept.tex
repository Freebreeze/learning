% Created 2014-11-28 星期五 12:40
\documentclass[table]{beamer}
\usepackage{fixltx2e}
\usepackage{graphicx}
\usepackage{longtable}
\usepackage{float}
\usepackage{wrapfig}
\usepackage{soul}
\usepackage{textcomp}
\usepackage{marvosym}
\usepackage{wasysym}
\usepackage{latexsym}
\usepackage{amssymb}
\usepackage{hyperref}
\tolerance=1000
\usepackage{etex}
\usepackage{amsmath}
\usepackage{pstricks}
\usepackage{pgfplots}
\pgfplotsset{compat=1.8}
\usepackage{tikz}
\usepackage[europeanresistors,americaninductors]{circuitikz}
\usepackage{colortbl}
\usepackage{yfonts}
\usetikzlibrary{shapes,arrows}
\usetikzlibrary{positioning}
\usetikzlibrary{arrows,shapes}
\usetikzlibrary{intersections}
\usetikzlibrary{calc,patterns,decorations.pathmorphing,decorations.markings}
\usepackage[BoldFont,SlantFont,CJKchecksingle]{xeCJK}
\setCJKmainfont[BoldFont=Evermore Hei]{Evermore Kai}
\setCJKmonofont{Evermore Kai}
\usepackage{pst-node}
\usepackage{pst-plot}
\psset{unit=5mm}
\mode<beamer>{\usetheme{Frankfurt}}
\mode<beamer>{\usecolortheme{dove}}
\mode<article>{\hypersetup{colorlinks=true,pdfborder={0 0 0}}}
\mode<beamer>{\AtBeginSection[]{\begin{frame}<beamer>\frametitle{Topic}\tableofcontents[currentsection]\end{frame}}}
\setbeamercovered{transparent}
\subtitle{离散系统基本概念}
\providecommand{\alert}[1]{\textbf{#1}}

\title{线性离散系统分析}
\author{}
\date{}
\hypersetup{
  pdfkeywords={},
  pdfsubject={},
  pdfcreator={Emacs Org-mode version 7.9.3f}}

\begin{document}

\maketitle

\begin{frame}
\frametitle{Outline}
\setcounter{tocdepth}{3}
\tableofcontents
\end{frame}












\section{特点}
\label{sec-1}
\begin{frame}
\frametitle{离散系统与离散信号}
\label{sec-1-1}

\mode<article>{离散信号只在离散的时刻有值,通常也把只在离散时刻有非零值的脉冲序列称为离散信号。}
\begin{itemize}
\item <2->离散信号:脉冲或数字信号。
\item <3->离散系统:控制系统中有一处或几处信号是脉冲或数字信号
\end{itemize}
\end{frame}
\section{采样控制系统}
\label{sec-2}
\begin{frame}
\frametitle{采样控制系统}
\label{sec-2-1}

\mode<article>{通常被控对象是连续系统,采用离散系统作为控制器时需要将连续信号采样,得到离散信号供控制器使用,还需要将控制器的输出复现为连续信号输入到被控对象。}
\begin{itemize}
\item <2->采样: 连续信号转变为离散脉冲序列的过程
\begin{itemize}
\item <3->周期采样:离散信号的获取是周期性的
\item <4->非周期采样:离散信号的获取是非周期的
\end{itemize}
\item <5->复现:把脉冲序列转变为连续信号的过程
\end{itemize}
\begin{columns}
\begin{column}{0.35\textwidth}
\begin{block}<2->{连续信号}
\label{sec-2-1-1}

\begin{tikzpicture}[scale=0.5]
\begin{axis}[grid=both]
\addplot+[smooth,mark=none] plot coordinates
    {(0,2) (0.1,1) (0.3,0.5) (0.35,4) (0.5,3)
     (0.6,2) (0.7,1.5) (1,1.5)};
\end{axis}
\end{tikzpicture}
\end{block}
\end{column}
\begin{column}{0.35\textwidth}
\begin{block}<2->{采样:}
\label{sec-2-1-2}

\begin{tikzpicture}[scale=0.5]
\begin{axis}[grid=both]
\addplot+[ycomb] plot coordinates
    {(0,2) (0.1,1) (0.3,0.5) (0.35,4) (0.5,3)
     (0.6,2) (0.7,1.5) (1,1.5)};
\end{axis}
\end{tikzpicture}
\end{block}
\end{column}
\begin{column}{0.35\textwidth}
\begin{block}<5->{复现:}
\label{sec-2-1-3}

\begin{tikzpicture}[scale=0.5]
\begin{axis}[grid=both]
\addplot+[const plot] plot coordinates
    {(0,2) (0.1,1) (0.3,0.5) (0.35,4) (0.5,3)
     (0.6,2) (0.7,1.5) (1,1.5)};
\end{axis}
\end{tikzpicture}
\end{block}
\end{column}
\end{columns}
\end{frame}
\begin{frame}
\frametitle{采样器与保持器}
\label{sec-2-2}

\begin{itemize}
\item 典型采样控制系统中既有连续的模拟信号,又有离散的脉冲信号,因此需要:
\begin{itemize}
\item 采样器: 模拟信号转换为脉冲信号
\item 保持器: 脉冲信号转换为模拟信号
\end{itemize}
\end{itemize}

\begin{tikzpicture}[node distance=2.2em,auto,>=latex', thick]
%\path[use as bounding box] (-1,0) rectangle (10,-2); 
\path[->] node[] (r) {$r(t)$}; 
\path[->] node[ circle,inner sep=2pt,minimum size=1pt,draw,label=below left:$   $ ,right =of r] (p1) {}; 
\path[->](r) edge node {} (p1) ; 
\path[->] node[minimum size=2em,right =of p1] (s) {}; 
\draw[blue] (s.west)--(s.north east);\draw[blue,->] (s.north west) arc (70:0:1.7em);\draw (s.south) node {$T$};\draw[blue] (s.north) node[above] {$S$};
\path[](p1) edge node[midway] {$e(t)$} (s) ; 
\path[->] node[draw, inner sep=5pt,right =of s] (k) {$K$}; 
\path[->] (s) edge node[midway] {$e^*(t)$} (k); 
\path[blue,->] node[draw, inner sep=5pt,right =of k] (gh) {$G_h(s)$}; 
\path[->] (k) edge node {} (gh); 
\path[red,->] node[draw, inner sep=5pt,right =of gh] (gp) {$G_p(s)$}; 
\path[->] (gh) edge node[midway] {$e_h(t)$} (gp); 
\path[->] node[ right =of gp] (o) {$c(t)$}; 
\path[->] (gp) edge node {} (o); 
\path[->] node[draw, inner sep=5pt,below =of gh] (h) {$H(s)$}; 
\path[->, draw] (o.west)+(-1em,0) |-   (h.east); 
\path[->, draw] (h.west) -| node[very near end] {$-$} (p1); 
%\path[->, draw] (g.east)+(1em,0) -- +(1em,-3em) -| node[very near end] {$-$} (p1); 
\end{tikzpicture} 

\begin{itemize}
\item $e^*(t)$  :采样信号
\item $G_h(s)$  :保持器
\item $e_h(t)$  :复现信号
\item $S$  :理想采样开关
\item $T$  :采样周期
\end{itemize}
\end{frame}
\section{数字控制系统}
\label{sec-3}
\begin{frame}
\frametitle{数字控制系统}
\label{sec-3-1}

\begin{itemize}
\item 以数字计算机作为控制器控制连续对象
\item 系统中既有连续信号,又有数字信号,实现两种信号之间的转换装置为A/D,D/A.
\end{itemize}
\end{frame}
\begin{frame}
\frametitle{模数转换器(A/D)}
\label{sec-3-2}

\begin{itemize}
\item 将连续信号转换为数字信号.
\item <2->工作过程:
\begin{itemize}
\item <2->采样过程:  $e(t)\rightarrow e^*(t)$
\item <3->量化过程:  $e^*(t)\rightarrow \bar{e}^*(t)$
\end{itemize}
\end{itemize}
\begin{columns}
\begin{column}{0.35\textwidth}
\begin{block}<2->{连续信号}
\label{sec-3-2-1}

\begin{tikzpicture}[scale=0.5]
\begin{axis}[grid=both]
\addplot+[smooth,mark=none] plot coordinates
    {(0,2) (0.1,1) (0.3,0.5) (0.35,4) (0.5,3)
     (0.6,2) (0.7,1.5) (1,1.5)};
\end{axis}
\end{tikzpicture}
\end{block}
\end{column}
\begin{column}{0.35\textwidth}
\begin{block}<2->{采样:}
\label{sec-3-2-2}

\begin{tikzpicture}[scale=0.5]
\begin{axis}[grid=both]
\addplot+[ycomb] plot coordinates
    {(0,2) (0.1,1) (0.3,0.5) (0.35,4) (0.5,3)
     (0.6,2) (0.7,1.5) (1,1.5)};
\end{axis}
\end{tikzpicture}
\end{block}
\end{column}
\begin{column}{0.35\textwidth}
\begin{block}<3->{量化:}
\label{sec-3-2-3}

\begin{tikzpicture}[scale=0.5]
\begin{axis}[grid=both]
\addplot+[ycomb] plot coordinates
    {(0,2) (0.1,1) (0.3,1) (0.35,4) (0.5,3)
     (0.6,2) (0.7,2) (1,2)};
\end{axis}
\end{tikzpicture}
\end{block}
\end{column}
\end{columns}
\end{frame}
\begin{frame}
\frametitle{数模转换器(D/A)}
\label{sec-3-3}

\begin{itemize}
\item 将离散的数字信号转换为连续模拟信号
\item <2->工作过程:
\begin{itemize}
\item 解码过程: 将离散数字信号转换为离散模拟信号
\item 复现过程: 将离散的模拟信号转换为连续的模拟信号
\end{itemize}
\end{itemize}
\begin{columns}
\begin{column}{0.5\textwidth}
\begin{block}<2->{数字信号}
\label{sec-3-3-1}

\begin{tikzpicture}[scale=0.7]
\begin{axis}[grid=both]
\addplot+[ycomb] plot coordinates
    {(0,2) (0.1,1) (0.3,1) (0.35,4) (0.5,3)
     (0.6,2) (0.7,2) (1,2)};
\end{axis}
\end{tikzpicture}
\end{block}
\end{column}
\begin{column}{0.5\textwidth}
\begin{block}<2->{复现:}
\label{sec-3-3-2}

\begin{tikzpicture}[scale=0.7]
\begin{axis}[grid=both]
\addplot+[const plot] plot coordinates
    {(0,2) (0.1,1) (0.3,1) (0.35,4) (0.5,3)
     (0.6,2) (0.7,2) (1,2)};
\end{axis}
\end{tikzpicture}
\end{block}
\end{column}
\end{columns}
\end{frame}
\begin{frame}
\frametitle{量化方法}
\label{sec-3-4}

\begin{itemize}
\item <2->只舍不入: 只取量化单位  $q$  的整数部分
       \begin{eqnarray*}
        E(e) &=& \frac{q}{2} \\
        \sigma^2 &=& \frac{q^2}{3}
       \end{eqnarray*}
\item <3->有舍有入: 类似四舍五入
       \begin{eqnarray*}
        E(e) &=& 0 \\
        \sigma^2 &=& \frac{q^2}{12}
       \end{eqnarray*}
\end{itemize}
\end{frame}
\begin{frame}
\frametitle{减小量化误差方法}
\label{sec-3-5}

\begin{itemize}
\item 减小  $q$  , 即增大字长  $i$  :   
         \[q=\frac{x_{max}-x_{min}}{2^i}\]
\end{itemize}
\end{frame}
\section{离散系统研究方法}
\label{sec-4}
\begin{frame}
\frametitle{离散系统研究方法}
\label{sec-4-1}

\begin{itemize}
\item 连续系统: Laplacian 变换
\item <2->离散系统: Z变换
\item <3->离散系统学习要点
\begin{itemize}
\item <4->离散数学模型,离散系统与连续系统对比
\item <5->离散系统的稳定性,稳态性能与动态性能分析
\end{itemize}
\end{itemize}
\end{frame}

\end{document}
