% Created 2013-11-02 Sat 16:37
\documentclass[table]{article}
\usepackage[utf8]{inputenc}
\usepackage[T1]{fontenc}
\usepackage{fixltx2e}
\usepackage{graphicx}
\usepackage{longtable}
\usepackage{float}
\usepackage{wrapfig}
\usepackage{soul}
\usepackage{textcomp}
\usepackage{marvosym}
\usepackage{wasysym}
\usepackage{latexsym}
\usepackage{amssymb}
\usepackage{hyperref}
\tolerance=1000
\usepackage{amsmath}
\usepackage[usenames]{color}
\usepackage{pstricks}
\usepackage{pgfplots}
\pgfplotsset{compat=1.8}
\usepackage{tikz}
\usepackage[europeanresistors,americaninductors]{circuitikz}
\usepackage{colortbl}
\usepackage{yfonts}
\usetikzlibrary{shapes,arrows}
\usetikzlibrary{positioning}
\usetikzlibrary{arrows,shapes}
\usetikzlibrary{intersections}
\usetikzlibrary{calc,patterns,decorations.pathmorphing,decorations.markings}
\usepackage[BoldFont,SlantFont,CJKchecksingle]{xeCJK}
\xeCJKsetup{CJKglue=\hspace{0pt plus .08 \baselineskip }}
\setCJKmainfont[BoldFont=Evermore Hei]{Evermore Kai}
\setCJKmonofont{Evermore Kai}
\usepackage{pst-node}
\usepackage{pst-plot}
\psset{unit=5mm}
\usepackage{beamerarticle}
\mode<beamer>{\usetheme{Frankfurt}}
\mode<beamer>{\usecolortheme{dove}}
\mode<article>{\hypersetup{colorlinks=true,pdfborder={0 0 0}}}
\mode<beamer>{\AtBeginSection[]{\begin{frame}<beamer>\frametitle{Topic}\tableofcontents[currentsection]\end{frame}}}
\setbeamercovered{transparent}
\subtitle{离散系统介绍}
\providecommand{\alert}[1]{\textbf{#1}}

\title{线性离散系统分析}
\author{}
\date{}
\hypersetup{
  pdfkeywords={},
  pdfsubject={},
  pdfcreator={Emacs Org-mode version 7.9.3f}}

\begin{document}

\maketitle

\begin{frame}
\frametitle{Outline}
\setcounter{tocdepth}{3}
\tableofcontents
\end{frame}













\section{离散系统基本概念}
\label{sec-1}
\subsection{特点}
\label{sec-1-1}
\begin{frame}
\frametitle{离散系统与离散信号}
\label{sec-1-1-1}

\begin{itemize}
\item <2->离散系统:控制系统中有一处或几处信号是一串脉冲或数码
\item <3->离散信号:脉冲或数码(数字信号)
\end{itemize}
\end{frame}
\subsection{采样控制系统}
\label{sec-1-2}
\begin{frame}
\frametitle{采样控制系统}
\label{sec-1-2-1}

\begin{itemize}
\item <2->采样: 连续信号转变为离散脉冲序列的过程
\begin{itemize}
\item <3->周期采样:离散信号的获取是周期性的
\item <4->非周期采样:离散信号的获取是非周期的
\end{itemize}
\item <5->复现:把脉冲序列转变为连续信号的过程
\end{itemize}
\begin{itemize}

\item 连续信号
\label{sec-1-2-1-1}%
\begin{tikzpicture}[scale=0.5]
\begin{axis}[grid=both]
\addplot+[smooth,mark=none] plot coordinates
    {(0,2) (0.1,1) (0.3,0.5) (0.35,4) (0.5,3)
     (0.6,2) (0.7,1.5) (1,1.5)};
\end{axis}
\end{tikzpicture}

\item 采样:
\label{sec-1-2-1-2}%
\begin{tikzpicture}[scale=0.5]
\begin{axis}[grid=both]
\addplot+[ycomb] plot coordinates
    {(0,2) (0.1,1) (0.3,0.5) (0.35,4) (0.5,3)
     (0.6,2) (0.7,1.5) (1,1.5)};
\end{axis}
\end{tikzpicture}

\item 复现:
\label{sec-1-2-1-3}%
\begin{tikzpicture}[scale=0.5]
\begin{axis}[grid=both]
\addplot+[const plot] plot coordinates
    {(0,2) (0.1,1) (0.3,0.5) (0.35,4) (0.5,3)
     (0.6,2) (0.7,1.5) (1,1.5)};
\end{axis}
\end{tikzpicture}
\end{itemize} % ends low level
\end{frame}
\begin{frame}
\frametitle{采样器与保持器}
\label{sec-1-2-2}

\begin{itemize}
\item 典型采样控制系统中既有连续的模拟信号,又有离散的脉冲信号,因此需要:
\begin{itemize}
\item 采样器: 模拟信号转换为脉冲信号
\item 保持器: 脉冲信号转换为模拟信号
\end{itemize}
\end{itemize}

\begin{tikzpicture}[node distance=2.2em,auto,>=latex', thick]
%\path[use as bounding box] (-1,0) rectangle (10,-2); 
\path[->] node[] (r) {$r(t)$}; 
\path[->] node[ circle,inner sep=2pt,minimum size=1pt,draw,label=below left:$   $ ,right =of r] (p1) {}; 
\path[->](r) edge node {} (p1) ; 
\path[->] node[minimum size=2em,right =of p1] (s) {}; 
\draw (s.west)--(s.north east);\draw[->] (s.north west) arc (70:0:1.7em);\draw (s.south) node {$T$};\draw (s.north) node[above] {$S$};
\path[](p1) edge node[midway] {$e(t)$} (s) ; 
\path[red,->] node[draw, inner sep=5pt,right =of s] (gh) {$G_h(s)$}; 
\path[->] (s) edge node[midway] {$e^*(t)$} (gh); 
\path[red,->] node[draw, inner sep=5pt,right =of gh] (gp) {$G_p(s)$}; 
\path[->] (gh) edge node[midway] {$e_h(t)$} (gp); 
\path[->] node[ right =of gp] (o) {$c(t)$}; 
\path[->] (gp) edge node {} (o); 
\path[blue,->] node[draw, inner sep=5pt,below =of gh] (h) {$H(s)$}; 
\path[->, draw] (o.west)+(-1em,0) |-   (h.east); 
\path[->, draw] (h.west) -| node[very near end] {$-$} (p1); 
%\path[->, draw] (g.east)+(1em,0) -- +(1em,-3em) -| node[very near end] {$-$} (p1); 
\end{tikzpicture} 

\begin{itemize}
\item $e^*(t)$  :采样信号
\item $G_h(s)$  :保持器
\item $e_h(t)$  :复现信号
\item $S$  :理想采样开关
\item $T$  :采样周期
\end{itemize}
\end{frame}
\subsection{数字控制系统}
\label{sec-1-3}
\begin{frame}
\frametitle{数字控制系统}
\label{sec-1-3-1}

\begin{itemize}
\item 以数字计算机作为控制器控制连续对象
\item 系统中既有连续信号,又有数字信号,实现两种信号之间的转换装置为A/D,D/A.
\end{itemize}
\end{frame}
\begin{frame}
\frametitle{模数转换器(A/D)}
\label{sec-1-3-2}

\begin{itemize}
\item 将连续信号转换为数字信号.
\item <2->工作过程:
\begin{itemize}
\item <2->采样过程:  $e(t)\rightarrow e^*(t)$
\item <3->量化过程:  $e^*(t)\rightarrow \bar{e}^*(t)$
\end{itemize}
\end{itemize}
\begin{itemize}

\item 连续信号
\label{sec-1-3-2-1}%
\begin{tikzpicture}[scale=0.5]
\begin{axis}[grid=both]
\addplot+[smooth,mark=none] plot coordinates
    {(0,2) (0.1,1) (0.3,0.5) (0.35,4) (0.5,3)
     (0.6,2) (0.7,1.5) (1,1.5)};
\end{axis}
\end{tikzpicture}

\item 采样:
\label{sec-1-3-2-2}%
\begin{tikzpicture}[scale=0.5]
\begin{axis}[grid=both]
\addplot+[ycomb] plot coordinates
    {(0,2) (0.1,1) (0.3,0.5) (0.35,4) (0.5,3)
     (0.6,2) (0.7,1.5) (1,1.5)};
\end{axis}
\end{tikzpicture}

\item 量化:
\label{sec-1-3-2-3}%
\begin{tikzpicture}[scale=0.5]
\begin{axis}[grid=both]
\addplot+[ycomb] plot coordinates
    {(0,2) (0.1,1) (0.3,1) (0.35,4) (0.5,3)
     (0.6,2) (0.7,2) (1,2)};
\end{axis}
\end{tikzpicture}

\end{itemize} % ends low level
\end{frame}
\begin{frame}
\frametitle{数模转换器(D/A)}
\label{sec-1-3-3}

\begin{itemize}
\item 将离散的数字信号转换为连续模拟信号
\item <2->工作过程:
\begin{itemize}
\item 解码过程: 将离散数字信号转换为离散模拟信号
\item 复现过程: 将离散的模拟信号转换为连续的模拟信号
\end{itemize}
\end{itemize}
\begin{itemize}

\item 数字信号
\label{sec-1-3-3-1}%
\begin{tikzpicture}[scale=0.7]
\begin{axis}[grid=both]
\addplot+[ycomb] plot coordinates
    {(0,2) (0.1,1) (0.3,1) (0.35,4) (0.5,3)
     (0.6,2) (0.7,2) (1,2)};
\end{axis}
\end{tikzpicture}

\item 复现:
\label{sec-1-3-3-2}%
\begin{tikzpicture}[scale=0.7]
\begin{axis}[grid=both]
\addplot+[const plot] plot coordinates
    {(0,2) (0.1,1) (0.3,1) (0.35,4) (0.5,3)
     (0.6,2) (0.7,2) (1,2)};
\end{axis}
\end{tikzpicture}

\end{itemize} % ends low level
\end{frame}
\begin{frame}
\frametitle{量化方法}
\label{sec-1-3-4}

\begin{itemize}
\item <2->只舍不入: 只取量化单位  $q$  的整数部分
       	\begin{eqnarray*}
	 E(e) &=& \frac{q}{2} \\
	 \sigma^2 &=& \frac{q^2}{3}
       	\end{eqnarray*}
\item <3->有舍有入: 类似四舍五入
       	\begin{eqnarray*}
	 E(e) &=& 0 \\
	 \sigma^2 &=& \frac{q^2}{12}
       	\end{eqnarray*}
\end{itemize}
\end{frame}
\begin{frame}
\frametitle{减小量化误差方法}
\label{sec-1-3-5}

\begin{itemize}
\item 减小  $q$  , 即增大字长  $i$  :   
         \[q=\frac{x_{max}-x_{min}}{2^i}\]
\end{itemize}
\end{frame}
\subsection{离散系统研究方法}
\label{sec-1-4}
\begin{frame}
\frametitle{离散系统研究方法}
\label{sec-1-4-1}

\begin{itemize}
\item 连续系统: Laplacian 变换
\item <2->离散系统: Z变换
\item <3->离散系统学习要点
\begin{itemize}
\item <4->离散数学模型
\item <5->离散系统的稳定性,稳态性能与动态性能分析
\end{itemize}
\end{itemize}
\end{frame}
\section{信号的采样与保持}
\label{sec-2}






\mode<article>{香农(Shannon)定理, 零阶保持器, 一阶保持器}
\subsection{信号的采样}
\label{sec-2-1}
\begin{frame}
\frametitle{采样信号}
\label{sec-2-1-1}

\begin{itemize}
\item <2->若采样开关为理想采样开关,则有:  
       	\[e^*(t)=e(t)\delta_T(t)\]
      其中  $\delta_T(t)$  为理想单位脉冲序列:  
	       \[\delta_T(t)=\sum_{n=0}^{\infty}\delta(t-nT)\]
\item <3->得:
      \begin{eqnarray*}
      e^{*}(t) & = & e(t)\sum_{n=0}^{\infty}\delta(t-nT) \\
       	&=&  \sum_{n=0}^{\infty}e(t)\delta(t-nT) \\
       	&=&  \sum_{n=0}^{\infty}e(nT)\delta(t-nT) 
      \end{eqnarray*}
\end{itemize}
\end{frame}
\begin{frame}
\frametitle{采样信号的Laplacian变换}
\label{sec-2-1-2}

      \begin{eqnarray*}
      {\cal L}(e^{*}(t)) & =  & {\cal L}( \sum_{n=0}^{\infty}e(nT)\delta(t-nT) )\\
      &=&  \sum_{n=0}^{\infty}e(nT){\cal L}(\delta(t-nT) )\\
      &=&  \sum_{n=0}^{\infty}e(nT)e^{-nTs}
      \end{eqnarray*}
\begin{itemize}
\item <2->离散信号的Laplacian变换为  $s$  的超越函数,不易分析. 可利用Z变换分析.
\end{itemize}
\end{frame}
\begin{frame}
\frametitle{采样信号的频谱分析}
\label{sec-2-1-3}

\begin{itemize}
\item 目的:分析  $e^*(t)$  与  $e(t)$  的关系,得到香农定理.
\item <2->将  $\delta_T(t)$  以 Fourier 级数表示,得: 
       \begin{eqnarray*}
       \delta_T(t) & = &\sum_{n=-\infty}^{\infty}C_n e^{jn\omega_s t} \\
       C_n &=&\frac{1}{T}\int_{-\frac{T}{2}}^{\frac{T}{2}}\delta_T(t)e^{-jn\omega_s t}dt \\
         &=&\frac{1}{T}\int_{-\frac{T}{2}}^{\frac{T}{2}}\delta(t)dt \\ 
         &=& \frac{1}{T} \\
       \omega_s &=& \frac{1}{T} 
     \end{eqnarray*}
\end{itemize}
\end{frame}
\begin{frame}
\frametitle{采样信号的频谱分析(续)}
\label{sec-2-1-4}

\begin{itemize}
\item 采样
       \begin{eqnarray*}
       \delta_T(t) &=& \frac{1}{T}\sum_{n=-\infty}^{\infty}e^{jn\omega_s t} \\
       e^*(t) &=& \frac{1}{T}\sum_{n=-\infty}^{\infty}e(t)e^{jn\omega_s t} \\
       E^*(s) &=& \frac{1}{T}\sum_{n=-\infty}^{\infty}E(s+jn\omega_s ) \\
       E^*(j\omega) &=& \frac{1}{T}\sum_{n=-\infty}^{\infty}E(j(\omega+n\omega_s)) 
       \end{eqnarray*}
\item <2->$e^*(t)$  的频谱为以  $\omega_s$  为周期的无穷多个频谱之和.
\item <3->设  $e(t)$ 带宽有限,最高角频率为  $\omega_h$ , 则当  $\omega_s>2\omega_h$  时,  $e^*(t)$  频谱的各部分不会相互重叠.
\end{itemize}
\end{frame}
\begin{frame}
\frametitle{香农定理}
\label{sec-2-1-5}

\begin{itemize}
\item <2->若采样器的输入信号  $e(t)$ 只有有限带宽,且其最高频率分量为  $\omega_h$  ,
\item <3->当采样周期满足  
	 \[T\leq\frac{2\pi}{2\omega_h}\]  
     则信号  $e(t)$  可以完全从  $e^*(t)$  中恢复出来.
\end{itemize}
\end{frame}
\begin{frame}
\frametitle{工程中  $T$  的选取}
\label{sec-2-1-6}

\begin{itemize}
\item <2->$T$  减小,失真度小,增加计算量
\item <3->$T$  增大,动态性能差,稳定性难保证
\item <4->经验方法:
\begin{itemize}
\item <4->在随动系统中,若校正后系统截止频率为  $\omega_c$ ,则采样频率为  $\omega_s=10\omega_c$  , 即  $T=\frac{\pi}{5\omega_c}$
\item <5->按  $t_r,t_s$  选取,   $T=\frac{T_r}{10},T=\frac{t_s}{40}$
\end{itemize}
\end{itemize}
\end{frame}
\subsection{采样函数Laplacian变换性质}
\label{sec-2-2}
\begin{frame}
\frametitle{采样函数Laplacian变换性质:$G^*(s)=G^*(s+jk\omega_s)$}
\label{sec-2-2-1}

\begin{itemize}
\item <2-> 证明:
      \begin{eqnarray*}
      G^*(s) &=& \frac{1}{T}\sum_{n=-\infty}^{\infty}G(s+jn\omega_s) \\
      G^*(s+jk\omega_s) &=& \frac{1}{T}\sum_{n=-\infty}^{\infty}G(s+j(n+k)\omega_s) \\
       &=& \frac{1}{T}\sum_{n=-\infty}^{\infty}G(s+jn\omega_s)\\
       &=& G^*(s)
      \end{eqnarray*}
\end{itemize}
\end{frame}
\begin{frame}
\frametitle{采样函数Laplacian变换性质:$[G(s)E^*(s)]^*=G^*(s)E^*(s)$}
\label{sec-2-2-2}

\begin{itemize}
\item <2-> 证明
      \begin{eqnarray*}
      [G(s)E^*(s)]^* &= & \frac{1}{T}\sum_{n=-\infty}^{\infty}[G(s+jn\omega_s)E^*(s+jn\omega_s)] \\
       &=& \frac{1}{T}\sum_{n=-\infty}^{\infty}[G(s+jn\omega_s)E^*(s)] \\
       &=& (\frac{1}{T}\sum_{n=-\infty}^{\infty}G(s+jn\omega_s))E^*(s) \\
       &=& G^*(s)E^*(s)
      \end{eqnarray*}
\end{itemize}
\end{frame}
\subsection{信号的保持}
\label{sec-2-3}
\begin{frame}
\frametitle{信号的保持}
\label{sec-2-3-1}

\begin{itemize}
\item <2->将数字信号及脉冲信号转换成连续的模拟信号,采用保持器.主要解决  $nT$  与  $(n+1)T$  之间的插值问题.
\item <3->保持器是具有外推功能的元件,外推公式为:  
	  \[e(nT+\Delta t)=a_0+a_1 \Delta t+a_2(\Delta t)^2+\cdots+a_m(\Delta t)^m\]
    式中  $a_0,\cdots,a_m$  由过去各采样时刻  $(m+1)$ 个离散的信号  $e^*((n-i)T),(i=0,\cdots,m)$  惟一确定.
\item <4-> $m=0$ 时称为零阶保持器,
\item <5-> $m=1$ 时称为一阶保持器.
\end{itemize}
\end{frame}
\begin{frame}
\frametitle{零阶保持器}
\label{sec-2-3-2}

\begin{itemize}
\item <2-> $e(nT+\Delta t)=a_0$  , 当  $\Delta t=0$  时,有  $e(nT)=a_0$ , 即按常值外推,  $e(t)=e(nT),t\in [ nT,(n+1)T)$
\item <3-> 设零阶保持器输入为  $r^*(t)=\delta(t)$  ,则输出为  $e(t)=1,t\in [ nT,(n+1)T)$  因此
      \begin{eqnarray*}
      {\cal L} (r^*) &=& 1 \\
      {\cal L} (e) &=& \frac{1}{s}-\frac{e^{-Ts}}{s} \\
      G_h(s) &=& \frac{E(s)}{R^*(s)}
	     = \frac{1-e^{-Ts}}{s} \\
      G_h(j\omega) &= &\frac{1-e^{-jT\omega}}{j\omega} 
       = \frac{e^{-j\omega T/2}(e^{j\omega T/2}-e^{-j\omega T/2})}{j\omega}\\
       &=& \frac{2\sin\frac{\omega T}{2}}{\omega}e^{-j\omega T/2} \\
       &=& \frac{2\sin\frac{\pi\omega}{\omega_s}}{\omega}e^{-j\pi\omega /\omega_s} 
      \end{eqnarray*}
\end{itemize}
\end{frame}
\begin{frame}
\frametitle{零阶保持器频率特性}
\label{sec-2-3-3}
\begin{itemize}

\item Bode图
\label{sec-2-3-3-1}%
\begin{tikzpicture}[scale=0.5]
%g=1-e^{-2\pi s}/s
\begin{axis}[
%axis x line=middle,axis y line= left, 
xticklabel=$\pgfmathprintnumber{\tick}\omega_s$ ,
yticklabel=$\pgfmathprintnumber{\tick}T$ ,
ylabel=$|G_h(j\omega)|$ ,xlabel=$\omega$ ,
every axis plot post/.append style={mark=none},
grid=both,
ymin=0,ymax=1.1,xmin=0.1,xmax=3]
\addplot[violet,thick] shell {octave -q --eval "w=[0.1:0.1:3]';m=abs(1/2/pi*(1-exp(-2*pi*j*w))./(j*w));disp([w,m]);" };
%\legend{$|G_h(j\omega)|$ , $\angle G_h(j\omega)$};
\end{axis}
\end{tikzpicture}

\begin{tikzpicture}[scale=0.5]
%g=1-e^{-2\pi s}/s
\begin{axis}[
%axis x line=middle,axis y line= left, 
xticklabel=$\pgfmathprintnumber{\tick}\omega_s$ ,
yticklabel=$\pgfmathprintnumber{\tick}\pi$ ,
ylabel=$\angle G_h(j\omega)$ ,xlabel=$\omega$ ,
every axis plot post/.append style={mark=none},
grid=both,
ymin=-3.5,ymax=0,xmin=0,xmax=3]
\addplot plot coordinates  {(0,0) (3,-3) };
\addplot plot coordinates  {(0,-1) (1,-1) (1,0)};
%\legend{$|G_h(j\omega)|$ , $\angle G_h(j\omega)$};
\end{axis}
\end{tikzpicture}


\item 零阶保持器特性
\label{sec-2-3-3-2}%
\begin{itemize}
\item 低通
\item 相角迟后
\item 时间延迟
\end{itemize}
\end{itemize} % ends low level
\end{frame}
\begin{frame}
\frametitle{一阶保持器}
\label{sec-2-3-4}

\begin{eqnarray*}
 e(nT+\Delta t) &=& a_0+a_1 \Delta t, \qquad (0\leq \Delta t < T) \\
 a_0& = & e(nT) \\
 a_1&=& \frac{e(nT)-e((n-1)T)}{T} \\
G_h(s) &=& T(1+s)\left(\frac{1-e^{-Ts}}{Ts}\right)^2 \\
G_h(j\omega) &=& \sqrt{1+(\omega T)^2}\left(\frac{2\sin\frac{\omega T}{2}}{\omega }\right)^2e^{-j(\omega T-\arctan\omega T)}
\end{eqnarray*}

\begin{itemize}
\item <2-> 其相角迟后比零阶保持器大得多,大大降低了系统相位裕度  $\gamma$  ,
\item <3-> 因此一般只用零阶保持器.
\end{itemize}
\end{frame}
\section{Z变换}
\label{sec-3}
\subsection{Z变换}
\label{sec-3-1}
\begin{frame}
\frametitle{Z变换定义}
\label{sec-3-1-1}

\begin{itemize}
\item 采样信号  $e^*(t)$  的Laplacian变换  
       \[E^*(s)=\sum_{n=0}^{\infty}e(nT)e^{-nsT}\]
\item <2->令  $Z=e^{sT}$ ,则  
       \[e^{-nsT}=Z^{-n}\]
\item <3->得:  
       \[E(Z)=\sum_{n=0}^{\infty}e(nT)Z^{-n}\]
\item <3->记作  
       \[E(Z)={\cal Z}[e^*(t)]={\cal Z}[e(t)]\]
\end{itemize}
\end{frame}
\begin{frame}
\frametitle{Z变换方法}
\label{sec-3-1-2}

\begin{itemize}
\item 级数求合法
\begin{itemize}
\item 按照Z变换的定义求解
\end{itemize}
\item 部分分式法:
\begin{itemize}
\item 先求出  $e(t)$  的Laplacian变换  $E(s)$  ,将其展开成部分分式之和,使每部分对应的Z变换是已知的.
\end{itemize}
\end{itemize}
\end{frame}
\begin{frame}
\frametitle{级数求合法示例: 单位阶跃信号 $1(t)$}
\label{sec-3-1-3}

\begin{itemize}
\item <2-> 解: 
       \begin{eqnarray*}
       e(nT)&=&1 , \\
       E(z) &=  &\sum_{n=0}^{\infty}e(nT)z^{-n} \\
       	&=& \sum_{n=0}^{\infty}z^{-n} \\
       &=& \frac{1}{1-z^{-1}} \\
       &=& \frac{1}{z-1}
       \end{eqnarray*}
       其中  $\qquad |z^{-1}|<1$
\end{itemize}
\end{frame}
\begin{frame}
\frametitle{级数求合法示例: $\delta_T(t)=\sum_{n=0}^{\infty}\delta(t-nT)$}
\label{sec-3-1-4}

\begin{itemize}
\item <2->解:
      \begin{eqnarray*}
      e^*(t) & = & \sum_{n=0}^{\infty}e(nT)\delta(t-nT) \\
       &=& \sum_{n=0}^{\infty}\delta(t-nT) \\
      e(nT) &=& 1\\
      E(z) &=& \sum_{n=0}^{\infty}z^{-n}\\
       &=& \frac{1}{1-z^{-1}} \\
      &=& \frac{z}{z-1}
      \end{eqnarray*}
      其中 $\qquad |z^{-1}|<1$
\item <3->$1(t)$  与  $\delta_T(t)$  对应的Z变换相同.
\end{itemize}
\end{frame}
\begin{frame}
\frametitle{部分分式法示例:  $E(s)=\frac{a}{s(s+a)}$}
\label{sec-3-1-5}

\begin{itemize}
\item <2->解:
      \begin{eqnarray*}
      E(s) & = & \frac{1}{s}-\frac{1}{s+a}\\
      e(t) &=& 1-e^{-at} \\
      E(z) &=& \frac{1}{1-z^{-1}} -\frac{1}{1-z^{-1}e^{-aT}}
      \end{eqnarray*}
\item <3->Z变换表:
      \[\begin{matrix}
      \delta(t) & 1 & 1 \\
      1(t) & \frac{1}{s} & \frac{1}{1-z^{-1}} \\
      t & \frac{1}{s^2} & \frac{Tz^{-1}}{(1-z^{-1})^2} \\
      e^{-at} & \frac{1}{s+a} &\frac{1}{1-e^{-aT}z^{-1}}
      \end{matrix}\]
\end{itemize}
\end{frame}
\begin{frame}
\frametitle{Z变换性质}
\label{sec-3-1-6}

\begin{itemize}
\item <2->线性定理:    ${\cal Z}[\alpha e_1(t)+\beta e_2(t)]=\alpha E_1(z)+\beta E_2(z)$
\item <3->实数位移定理:  ${\cal Z}[e(t+kT)] = z^k[E(z)-\sum_{n=0}^{k-1}e(nT)z^{-n}]$
\item <4->复数位移定理:  ${\cal Z}[e^{\pm at}e(t)] = E(ze^{\mp aT})$
\item <5->终值定理:  $\lim_{n\rightarrow\infty}e(nT)=\lim_{z\rightarrow 1}(1-z^{-1})E(z)$
\item <6->卷积定理: 若  $g(nT)=x(nT)*y(nT)$  则  $G(z)=X(z)Y(z)$  . ($x(nT)*y(nT)=\sum_{k=0}^{\infty}x(kT)y((n-k)T)$)
\end{itemize}
\end{frame}
\subsection{Z反变换}
\label{sec-3-2}
\begin{frame}
\frametitle{Z反变换}
\label{sec-3-2-1}

  \[e(nT)={\cal Z}^{-1}[E(z)]\]
\begin{itemize}
\item 幂级数展开法
\item 部分分式法
\begin{itemize}
\item 展开成部分分式后查表
\end{itemize}
\item 反演积分法
\end{itemize}
\end{frame}
\begin{frame}
\frametitle{幂级数展开法}
\label{sec-3-2-2}

\begin{eqnarray*}
E(z) & = &\frac{b_0+b_1 z^{-1}+\cdots+b_m z^{-m}}{1+a_1 z^{-1}+\cdots+a_n z^{-n}} \\
 &=& c_0+c_1 z^{-1}+\cdots +c_n z^{-n} \\
 &=& \sum_{n=0}^{\infty}c_n z^{-n} \\
e^{*}(t) &=& \sum_{n=0}^{\infty}c_n\delta(t-nT) \\
e(nT) &=& c_n 
\end{eqnarray*}
\end{frame}
\begin{frame}
\frametitle{反演积分法}
\label{sec-3-2-3}

\begin{eqnarray*}
E(z) & = & \sum_{n=0}^{\infty}e(nT)z^{-n} \\
  &=& e(0)+e(T)z^{-1}+\cdots+e(nT)z^{-n}+\cdots \\
E(z)z^{n-1} &=& e(0)z^{n-1}+e(T)z^{n-2}+\cdots+e(nT)z^{-1}+\cdots \\
e(nT)&=& Res(E(z)z^{n-1})
\end{eqnarray*}
\end{frame}
\begin{frame}
\frametitle{反演积分法示例:  $E(z)=\frac{z^2}{(z-1)(z-0.5)}$  求  $e(nT)$}
\label{sec-3-2-4}

\begin{itemize}
\item <2->解:
      \begin{eqnarray*}
      E(z)z^{n-1} & = &\frac{z^{n+1}}{(z-1)(z-0.5)} \\
      Res_1 &=& \lim_{z\rightarrow 1}\frac{(z-1)z^{n+1}}{(z-1)(z-0.5)} \\
	 &=& 2 \\
      Res_2 &=& \lim_{z\rightarrow 0.5}\frac{(z-0.5)z^{n+1}}{(z-1)(z-0.5)} \\
	 &=& -0.5^n \\
      e(nT) &=& Res_1+Res_2 \\
       &=& 2-0.5^n
      \end{eqnarray*}
\end{itemize}
\end{frame}
\section{离散系统数学模型}
\label{sec-4}
\subsection{差分方程}
\label{sec-4-1}
\begin{frame}
\frametitle{差分方程模型}
\label{sec-4-1-1}

\begin{itemize}
\item <2->n阶后向差分方程
      \begin{eqnarray*}
       & &c(k)+a_1 c(k-1)+\cdots+a_n c(k-n) \\
       &=& b_0 r(k) +b_1 r(k-1) + \cdots + b_m r(k-m)
      \end{eqnarray*}
      即  $k$  时刻的输出  $c(k)$  与k时刻前  $n$  个时刻输出及前  $m$  个输入,当前时刻输入有关.
\item <3->n阶前向差分方程
      \begin{eqnarray*}
    & &  c(k+n)+a_1 c(k+n-1)+\cdots+a_n c(k) \\
    &=& b_0 r(k+m)+b_1 r(k+m-1)+\cdots+ b_m r(k)
      \end{eqnarray*}
\end{itemize}
\end{frame}
\begin{frame}
\frametitle{差分方程解法: 迭代法}
\label{sec-4-1-2}

\begin{itemize}
\item 利用差分方程的递推关系,逐步计算  $c(k)$  的值
\item <2->例:  $c(k)=r(k)+5 c(k-1) -6 c(k-2)$  输入  $r(k)=1$ , 初始条件:  $c(0)=0,c(1)=1$
\item <3->解:
      \begin{eqnarray*}
      c(2) & = & 6\\
      c(3) & =& 25 \\
      c(4) &=& 90
      \end{eqnarray*}
\end{itemize}
\end{frame}
\begin{frame}
\frametitle{z变换法}
\label{sec-4-1-3}

\begin{itemize}
\item 将差分方程与输入进行Z变换,得到输出的Z变换,再进行Z反变换.
\item <2->例: 差分方程  $c(t+2T)+3c(t+T)+2c(t)=0$  初始条件  $c(0)=0,c(1)=1$
\item <3->解:
      \begin{align*}
       &  z^2(c(z)-c(0)-c(1)z^{-1})+3z(c(z)-c(0))+2c(z)  =  0 \\
       &  (z^2+3z+2)c(z) = z \\
       &  c(z) = \frac{z}{z^2+3z+2} 
       	=  \frac{z}{z+1}-\frac{z}{z+2}
       	=  \frac{1}{1+z^{-1}}-\frac{1}{1+2z^{-1}}\\
       & c(k) = (-1)^k-(-2)^k
      \end{align*}
      其中  $k=0,1,2,\cdots$
\end{itemize}
\end{frame}
\subsection{脉冲传递函数}
\label{sec-4-2}
\begin{frame}
\frametitle{脉冲传递函数定义}
\label{sec-4-2-1}

\begin{itemize}
\item 连续系统:传递函数 (s域)
\item 离散系统:脉冲传递函数 (z域)
\item <2->定义:输出  $c^*(t)$   的Z变换与输入  $r^*(t)$  的Z变换之比(零初始条件下)叫做系统的脉冲传递函数.记为 
	 \[G(z)=\frac{C(z)}{R(z)}\]
\end{itemize}
\end{frame}
\begin{frame}
\frametitle{脉冲传递函数意义}
\label{sec-4-2-2}

\begin{itemize}
\item 加权序列: 输入  $r^*(t)=\delta(t)$  的输出序列称为加权序列,记为  $k^*(t)$
\item <2->脉冲传递函数: 
      \begin{eqnarray*}
      G(z) &=& \frac{{\cal Z}[k^*(t)]}{{\cal Z}[r^*(t)]} \\
      &=& {\cal Z}[k^*(t)]\\
      &=& k(z)
      \end{eqnarray*}
\item <3-> 脉冲传递函数为加权序列  $k^*(t)$  的Z变换
\end{itemize}
\end{frame}
\begin{frame}
\frametitle{两种模型之间的变换关系:}
\label{sec-4-2-3}

      \begin{eqnarray*}
      c(nT)+\sum_{k=1}^n a_k c((n-k)T) &=& \sum_{k=0}^m b_k r((n-k)T) \\
      G(z) &=& \frac{\sum_{k=0}^{m}b_k z^{-k}}{1+\sum_{k=1}^n a_k z^{-k}}
      \end{eqnarray*}
\begin{itemize}
\item <2-> 差分方程在零初始条件下进行Z变换,得脉冲传递函数.
\end{itemize}
\end{frame}
\begin{frame}
\frametitle{脉冲传递函数计算}
\label{sec-4-2-4}

\begin{itemize}
\item 差分方程Z变换:  $G(z)=\frac{C(z)}{R(z)}$
\item 从传递函数  $G(s)$  求解(部分分式法)
\item <2->例:  $c(nT)=r[(n-k)T]$
\item <3->解:
       \begin{eqnarray*}
       C(z) &=& z^{-k}R(z) \\
       G(z) &=& \frac{C(z)}{R(z)} \\
         &=& z^{-k}
       \end{eqnarray*}
\end{itemize}
\end{frame}
\subsection{开环系统的脉冲传递函数}
\label{sec-4-3}
\begin{frame}
\frametitle{开环系统脉冲传递函数}
\label{sec-4-3-1}

\mode<article>{按定义求,即:  $G(z)=\frac{{\cal Z} [c^*(t)]}{{\cal Z}[r^*(t)]}$ }
\begin{itemize}

\item 结构图
\label{sec-4-3-1-1}%
\begin{tikzpicture}[node distance=2.2em,auto,>=latex', thick] 
%\path[use as bounding box] (-1,0) rectangle (10,-2); 
\path[->] node[] (r) {$r(t)$}; 
%\path[->] node[ circle,inner sep=2pt,minimum size=1pt,draw,label=below left:$   $ ,right =of r] (p1) {}; 
%\path[->](r) edge node {} (p1) ; 
\path[->] node[minimum size=2em,right =of r] (s1) {}; 
\draw (s1.west)--(s1.north east);\draw[->] (s1.north west) arc (70:0:1.7em);\draw (s1.south) node {$T$};%\draw (s1.north) node[above] {$S$};
\path[](r) edge node[midway] {$   $} (s1) ; 
\path[red,->] node[draw, inner sep=5pt,right =of s1] (g1) {$G_1(s)$}; 
\path[->] (s1) edge node[midway] {$r^*(t)$} (g1); 
\path[->] node[minimum size=2em,right =of g1] (s2) {}; 
\draw (s2.west)--(s2.north east);\draw[->] (s2.north west) arc (70:0:1.7em);\draw (s2.south) node {$T$};%\draw (s2.north) node[above] {$S$};
\path[](g1) edge node[midway] {$d(t)$} (s2) ; 
\path[red,->] node[draw, inner sep=5pt,right =of s2] (g2) {$G_2(s)$}; 
\path[->] (s2) edge node[midway] {$d^*(t)$} (g2); 
\path[->] node[ right =of g2] (o) {$c(t)$}; 
\path[->] (g2) edge node {} (o); 
\path[->] node[minimum size=2em,above =of o] (sc) {}; 
\draw[dashed] (sc.west)--(sc.north east);\draw[dashed,->] (sc.north west) arc (70:0:1.7em);\draw[dashed] (sc.south) node {$T$};%\draw (sc.north) node[above] {$S$};
\path[dashed,draw](o.west)+(-1em,0) |- (sc.west) ; 
\path node[ right =of sc] (c) {$c^*(t)$}; 
\path[dashed,->] (sc) edge node {} (c); 
%\path[->, draw] (g.east)+(1em,0) -- +(1em,-3em) -| node[very near end] {$-$} (p1); 
\end{tikzpicture} 


\item 推导
\label{sec-4-3-1-2}%
\begin{eqnarray*}
D(z) &=& R(z)G_1(z) \\
C(z) & = & D(z)G_2(z) 
       = G_1(z)G_2(z)R(z) \\
G(z) &=& G_1(z)G_2(z)
\end{eqnarray*}
\end{itemize} % ends low level
\end{frame}
\begin{frame}
\frametitle{开环系统脉冲传递函数(续)}
\label{sec-4-3-2}
\begin{itemize}

\item 结构图
\label{sec-4-3-2-1}%
\begin{tikzpicture}[node distance=2.2em,auto,>=latex', thick] 
%\path[use as bounding box] (-1,0) rectangle (10,-2); 
\path[->] node[] (r) {$r(t)$}; 
%\path[->] node[ circle,inner sep=2pt,minimum size=1pt,draw,label=below left:$   $ ,right =of r] (p1) {}; 
%\path[->](r) edge node {} (p1) ; 
\path[->] node[minimum size=2em,right =of r] (s1) {}; 
\draw (s1.west)--(s1.north east);\draw[->] (s1.north west) arc (70:0:1.7em);\draw (s1.south) node {$T$};%\draw (s1.north) node[above] {$S$};
\path[](r) edge node[midway] {$   $} (s1) ; 
\path[red,->] node[draw, inner sep=5pt,right =of s1] (g1) {$G_1(s)$}; 
\path[->] (s1) edge node[midway] {$r^*(t)$} (g1); 
\path[red] node[draw, inner sep=5pt,right =of g1] (g2) {$G_2(s)$}; 
\path[->] (g1) edge node[midway] {$   $} (g2); 
\path[->] node[ right =of g2] (o) {$c(t)$}; 
\path[->] (g2) edge node {} (o); 
\path[->] node[minimum size=2em,above =of o] (sc) {}; 
\draw[dashed] (sc.west)--(sc.north east);\draw[dashed,->] (sc.north west) arc (70:0:1.7em);\draw[dashed] (sc.south) node {$T$};%\draw (sc.north) node[above] {$S$};
\path[dashed,draw](o.west)+(-1em,0) |- (sc.west) ; 
\path node[ right =of sc] (c) {$c^*(t)$}; 
\path[dashed,->] (sc) edge node {} (c); 
%\path[->, draw] (g.east)+(1em,0) -- +(1em,-3em) -| node[very near end] {$-$} (p1); 
\end{tikzpicture} 

\item 推导
\label{sec-4-3-2-2}%
\begin{eqnarray*}
C^*(s) & = & [R^*(s)G_1(s)G_2(s)]^* 
       = R^*(s)[G_1(s)G_2(s)]^* \\
C(z) &=& R(z) G_1G_2(z) \\
G(z) &=& G_1G_2(z)
\end{eqnarray*}
\end{itemize} % ends low level
\end{frame}
\begin{frame}
\frametitle{开环系统脉冲传递函数(续):有零阶保持器时:}
\label{sec-4-3-3}
\begin{itemize}

\item 结构图
\label{sec-4-3-3-1}%
\begin{tikzpicture}[node distance=2.2em,auto,>=latex', thick] 
%\path[use as bounding box] (-1,0) rectangle (10,-2); 
\path[->] node[] (r) {$r(t)$}; 
%\path[->] node[ circle,inner sep=2pt,minimum size=1pt,draw,label=below left:$   $ ,right =of r] (p1) {}; 
%\path[->](r) edge node {} (p1) ; 
\path[->] node[minimum size=2em,right =of r] (s1) {}; 
\draw (s1.west)--(s1.north east);\draw[->] (s1.north west) arc (70:0:1.7em);\draw (s1.south) node {$T$};%\draw (s1.north) node[above] {$S$};
\path[](r) edge node[midway] {$   $} (s1) ; 
\path[red,->] node[draw, inner sep=5pt,right =of s1] (g1) {$G_h(s)$}; 
\path[->] (s1) edge node[midway] {$r^*(t)$} (g1); 
\path[red] node[draw, inner sep=5pt,right =of g1] (g2) {$G_p(s)$}; 
\path[->] (g1) edge node[midway] {$   $} (g2); 
\path[->] node[ right =of g2] (o) {$c(t)$}; 
\path[->] (g2) edge node {} (o); 
\path[->] node[minimum size=2em,above =of o] (sc) {}; 
\draw[dashed] (sc.west)--(sc.north east);\draw[dashed,->] (sc.north west) arc (70:0:1.7em);\draw[dashed] (sc.south) node {$T$};%\draw (sc.north) node[above] {$S$};
\path[dashed,draw](o.west)+(-1em,0) |- (sc.west) ; 
\path node[ right =of sc] (c) {$c^*(t)$}; 
\path[dashed,->] (sc) edge node {} (c); 
%\path[->, draw] (g.east)+(1em,0) -- +(1em,-3em) -| node[very near end] {$-$} (p1); 
\end{tikzpicture} 

\item 推导
\label{sec-4-3-3-2}%
\begin{itemize}
\item <2->   $C^*(s)  =  [R^*(s)\cdot \frac{1-e^{-Ts}}{s}\cdot G_p(s)]^*$
\item <3->   $C^*(s)  = R^*(s)[(1-e^{-Ts})\cdot\frac{G_p(s)}{s}]^*$
\item <4->   $C^*(s)  = R^*(s)[\frac{G_p(s)}{s}-e^{-Ts}\cdot\frac{G_p(s)}{s}]^*$
\item <5->   $C(z) = R(z){\cal Z}[\frac{G_p(z)}{s}]-z^{-1}{\cal Z}[\frac{G_p(z)}{s}]$
\item <6->   $G(z) = (1-z^{-1}){\cal Z}[\frac{G_p(z)}{s}]$
\end{itemize}
\end{itemize} % ends low level
\end{frame}
\begin{frame}
\frametitle{开环系统脉冲传递函数示例:  $G_p(s)=\frac{a}{s(s+a)}$}
\label{sec-4-3-4}

\begin{itemize}
\item <2->解:
      \begin{eqnarray*}
      G(z) & = &(1-z^{-1}){\cal Z}[\frac{a}{s^2(s+a)}] \\
       &=& (1-z^{-1}){\cal Z}[\frac{1}{s^2}-\frac{1}{a}(\frac{1}{s}-\frac{1}{s+a})] \\
       &=& (1-z^{-1})\left[\frac{Tz^{-1}}{(1-z^{-1})^2}-\frac{1}{a}(\frac{1}{1-z^{-1}}-\frac{1}{1-z^{-1}e^{-aT}} )\right]
      \end{eqnarray*}
\end{itemize}
\end{frame}
\subsection{闭环系统的脉冲传递函数}
\label{sec-4-4}
\begin{frame}
\frametitle{闭环系统的脉冲传递函数}
\label{sec-4-4-1}

\mode<article>{按定义求:}

\begin{tikzpicture}[node distance=2.2em,auto,>=latex', thick] 
%\path[use as bounding box] (-1,0) rectangle (10,-2); 
\path[->] node[] (r) {$r(t)$}; 
\path[->] node[ circle,inner sep=2pt,minimum size=1pt,draw,label=below left:$   $ ,right =of r] (p1) {}; 
\path[->](r) edge node {} (p1) ; 
\path[->] node[minimum size=2em,right =of p1] (s1) {}; 
\draw (s1.west)--(s1.north east);\draw[->] (s1.north west) arc (70:0:1.7em);\draw (s1.south) node {$T$};%\draw (s1.north) node[above] {$S$};
\path[](p1) edge node[midway] {$e(t)$} (s1) ; 
%\path[red,->] node[draw, inner sep=5pt,right =of s1] (g1) {$G_h(s)$}; 
%\path[->] (s1) edge node[midway] {$r^*(t)$} (g1); 
\path[red] node[draw, inner sep=5pt,right =of s1] (g2) {$G(s)$}; 
\path[->] (s1) edge node[midway] {$e^*(t)$} (g2); 
\path[->] node[ right =of g2] (o) {$c(t)$}; 
\path[->] (g2) edge node {} (o); 
\path[->] node[minimum size=2em,above =of o] (sc) {}; 
\draw[dashed] (sc.west)--(sc.north east);\draw[dashed,->] (sc.north west) arc (70:0:1.7em);\draw[dashed] (sc.south) node {$T$};%\draw (sc.north) node[above] {$S$};
\path[dashed,draw](o.west)+(-1em,0) |- (sc.west) ; 
\path node[ right =of sc] (c) {$c^*(t)$}; 
\path[dashed,->] (sc) edge node {} (c); 
\path[red] node[draw, inner sep=5pt,below =of g2] (h) {$H(s)$}; 
\path[->,draw] (g2.east)+(1em,0) |- (h.east);
\path[->,draw] (h.west) -| node [very near end] {$-$} (p1);
%\path[->, draw] (g.east)+(1em,0) -- +(1em,-3em) -| node[very near end] {$-$} (p1); 
\path[->] node[minimum size=2em,above =of p1] (sr) {}; 
\draw[dashed] (sr.west)--(sr.north east);\draw[dashed,->] (sr.north west) arc (70:0:1.7em);\draw[dashed] (sr.south) node {$T$};%\draw (sr.north) node[above] {$S$};
\path[dashed,draw](r.east)+(1em,0) |- (sr.west) ; 
\path node[ right =of sr] (i) {$r^*(t)$}; 
\path[dashed,->] (sr) edge node {} (i); 
\end{tikzpicture} 
\begin{itemize}

\item 脉冲传递函数
\label{sec-4-4-1-1}%
\begin{eqnarray*}
\Phi(z) & = & \frac{C(z)}{R(z)}
 = \frac{{\cal Z}[c^*(t)]}{{\cal Z}[r^*(t)]} \\
\Phi_e(z) & = & \frac{E(z)}{R(z)}
    = \frac{{\cal Z}[e^*(t)]}{{\cal Z}[r^*(t)]} 
\end{eqnarray*}

\item 解:
\label{sec-4-4-1-2}%
\begin{eqnarray*}
C(s) &=& G(s)E^*(s) \\
E(s) &=& R(s)-H(s)C(s) \\
     &=& R(s)-H(s)G(s)E^*(st) \\
\end{eqnarray*}
\end{itemize} % ends low level
\end{frame}
\begin{frame}
\frametitle{闭环系统的脉冲传递函数(续)}
\label{sec-4-4-2}

\begin{eqnarray*}
C(s) &=& G(s)E^*(s) \\
E(s) &=& R(s)-H(s)C(s) \\
 &=& R(s)-H(s)G(s)E^*(st) \\
E^*(s) &=& R^*(s)-HG^*(s)E^*(s)\\
  &=& \frac{R^*(s)}{1+HG^*(s)} \\
\Phi_e(z) &=& \frac{1}{1+HG(z)} \\
C^*(s) &=& G^*(s)E^*(s)\\
 &=& \frac{G^*(s)R^*(s)}{1+HG^*(s)} \\
\Phi(z) &=& \frac{G(z)}{1+HG(z)} 
\end{eqnarray*}
\end{frame}
\begin{frame}
\frametitle{闭环系统的脉冲传递函数示例:}
\label{sec-4-4-3}


\begin{tikzpicture}[node distance=2.2em,auto,>=latex', thick] 
%\path[use as bounding box] (-1,0) rectangle (10,-2); 
\path[->] node[] (r) {$r(t)$}; 
\path[->] node[ circle,inner sep=2pt,minimum size=1pt,draw,label=below left:$   $ ,right =of r] (p1) {}; 
\path[->](r) edge node {} (p1) ; 
%\path[->] node[minimum size=2em,right =of p1] (s1) {}; 
%\draw (s1.west)--(s1.north east);\draw[->] (s1.north west) arc (70:0:1.7em);\draw (s1.south) node {$T$};%\draw (s1.north) node[above] {$S$};
%\path[](p1) edge node[midway] {$e(t)$} (s1) ; 
%\path[red,->] node[draw, inner sep=5pt,right =of s1] (g1) {$G_h(s)$}; 
%\path[->] (s1) edge node[midway] {$r^*(t)$} (g1); 
\path[red] node[draw, inner sep=5pt,right =of p1] (g2) {$G(s)$}; 
\path[->] (p1) edge node[midway] {$e(t)$} (g2); 
\path[->] node[minimum size=2em,right =of g2] (sc) {}; 
\draw (sc.west)--(sc.north east);\draw[->] (sc.north west) arc (70:0:1.7em);\draw (sc.south) node {$T$};%\draw (sc.north) node[above] {$S$};
\path (g2) edge node[midway] {$c(t)$} (sc); 
%\path[draw](o.west)+(-1em,0) |- (sc.west) ; 
%\path node[ right =of sc] (c) {$c^*(t)$}; 
%\path[dashed,->] (sc) edge node {} (c); 
\path[->] node[ right =of sc] (o) {$c^*(t)$}; 
\path[->] (sc) edge node {} (o); 

\path[red] node[draw, inner sep=5pt,below =of g2] (h) {$H(s)$}; 
\path[->,draw] (o.west)+(-1em,0) |- (h.east);
\path[->,draw] (h.west) -| node [very near end] {$-$} (p1);
%\path[->, draw] (g.east)+(1em,0) -- +(1em,-3em) -| node[very near end] {$-$} (p1); 
\path[->] node[minimum size=2em,above =of p1] (sr) {}; 
\draw[dashed] (sr.west)--(sr.north east);\draw[dashed,->] (sr.north west) arc (70:0:1.7em);\draw[dashed] (sr.south) node {$T$};%\draw (sr.north) node[above] {$S$};
\path[dashed,draw](r.east)+(1em,0) |- (sr.west) ; 
\path node[ right =of sr] (i) {$r^*(t)$}; 
\path[dashed,->] (sr) edge node {} (i); 
\end{tikzpicture} 

\begin{itemize}
\item <2->解:
      \begin{eqnarray*}
      E(s) &=& R(s)-H(s)C^*(s)\\
      C(s) & = & G(s)E(s) 
          = G(s)R(s)-G(s)H(s)C^*(s)\\
      C^*(s) &=& GR^*(s)-GH^*(s)C^*(s)
             = \frac{GR^*(s)}{1+GH^*(s)C^*(s)}
      \end{eqnarray*}
\item <3->没有闭环脉冲传递函数
\end{itemize}
\end{frame}
\subsection{Z变换局限性与修正Z变换}
\label{sec-4-5}
\begin{frame}
\frametitle{Z变换局限性}
\label{sec-4-5-1}

\begin{itemize}
\item <2->采样间隔 $\tau$  要远小于系统最小时间常数
\item <3->c(nT)不能反映采样间隔中的信息
\item <4->G(s)要满足:  $n\geq m+2$  ,否则  $c^*(t)$  与  $c(t)$  差别较大.
\end{itemize}
\end{frame}
\begin{frame}
\frametitle{修正Z变换}
\label{sec-4-5-2}

\begin{itemize}
\item 目的:求取采样间隔中的输出值
\item 原理:
\begin{itemize}
\item <2->将周期为  $T$  的原输入采样信号序列 $r^*(t)$ 再次以周期  $\frac{T}{n}$  采样,即得:  $R'(z)=R(z^n)$
\item <3->计算在采样周期  $\frac{T}{n}$  下的响应,即得到原采样间隔中的值.
\end{itemize}
\item 方法:
\begin{itemize}
\item <4->原输入信号Z变换为  $R(z)$ , 将 $z$  替换为:  $z^n$  .
\item <5->以  $\frac{T}{n}$ 重新计算系统脉冲传递函数.
\end{itemize}
\end{itemize}
\begin{itemize}

\item $R(z)$
\label{sec-4-5-2-1}%
\begin{tikzpicture}[scale=0.5]
\begin{axis}[xticklabel=$\pgfmathprintnumber{\tick}T$]
\addplot+[ycomb] plot coordinates
    {(0,2) (1,1) (2,0.5) (3,4) (4,3) (5,2) (6,1.5) (7,1.5)};
\end{axis}
\end{tikzpicture}

\item $R(z^2)$,($T'=\frac{T}{2}$)
\label{sec-4-5-2-2}%
\begin{tikzpicture}[scale=0.5]
\begin{axis}[ymin=0,xticklabel=$\pgfmathprintnumber{\tick}T$]
\addplot+[ycomb] plot coordinates
    {(0,2) (0.5, 0) (1,1) (1.5,0) (2,0.5)(2.5,0) (3,4)(3.5,0) (4,3)(4.5,0) (5,2)(5.5,0) (6,1.5)(6.5,0) (7,1.5)};
\end{axis}
\end{tikzpicture}
\end{itemize} % ends low level
\end{frame}
\begin{frame}
\frametitle{修正Z变换示例:}
\label{sec-4-5-3}

\[G(z)=\frac{z}{z-e^{-T}}\]
   $T=1$ ,  $r(t)=1(t)$ , 要求每采样周期中间插入两点.

\begin{itemize}
\item 解:
      \begin{eqnarray*}
      G(z) &= & \frac{z}{z-e^{-1/3}} \\
      r(z) &=& \frac{1}{1-z^{-1}} \\
      r'(z) &=& r(z^3) \\
      &=& \frac{1}{1-z^{-3}} \\
      c'(z) &=& \frac{1}{1-z^{-1}e^{-1/3}}\cdot\frac{1}{1-z^{-3}}
      \end{eqnarray*}
\end{itemize}
\end{frame}
\section{离散系统稳定性}
\label{sec-5}
\subsection{稳定性}
\label{sec-5-1}
\begin{frame}
\frametitle{S域到Z域的映射}
\label{sec-5-1-1}
\begin{itemize}

\item $S\leftrightarrow Z$
\label{sec-5-1-1-1}%
\begin{eqnarray*}
z & = & e^{sT}\\
s &=& \sigma+j\omega \\
z &=& e^{\sigma T}e^{j\omega T} \\
|z| &=& e^{\sigma T} \\
\angle z &=& \omega T
\end{eqnarray*}
\begin{itemize}

\item 当  $\sigma=0$  时,\\
\label{sec-5-1-1-1-1}%
对应到  $z$  平面的单位圆,此时,  $\omega$  从  $-\infty\rightarrow\infty$ 时,  $z$  平面上的点己绕单位圆运动了无数圈,称  $[-\frac{\omega_s}{2},\frac{\omega_s}{2}]$  为主要带.

\end{itemize} % ends low level

\item 主要映射关系:
\label{sec-5-1-1-2}%
\begin{itemize}
\item 等  $\sigma$  线: 单位圆:   $|z|=e^{\sigma T}$
\item 等  $\omega$  线: 过原点射线:  $\angle z=\omega T$
\item 等  $\xi$  线(S平面过原点射线): 对数螺线
\end{itemize}

\end{itemize} % ends low level
\end{frame}
\begin{frame}
\frametitle{离散系统的稳定性}
\label{sec-5-1-2}

\begin{itemize}
\item 稳定性定义:离散系统在有界输入序列下,输出序列也有界.
\item <2->连续系统中:闭环系统的特征根实部 $\sigma$  小于0.
\item <3->离散系统中:  $|z|<1$ ,($|z|=e^{\sigma}$)
\begin{itemize}
\item 差分方程:特征根的模均小于1
\item 闭环脉冲传递函数:离散系统闭环特征根在Z平面的单位圆内($|z_i|<1$)
\end{itemize}
\end{itemize}
\end{frame}
\subsection{稳定性判据}
\label{sec-5-2}

\mode<article>{解特征方程,根据 $|z_i|<1$ 判断}
\begin{frame}
\frametitle{W域的劳斯判据}
\label{sec-5-2-1}

\begin{itemize}
\item Z域变换到W域:
       \begin{eqnarray*}
       	z & = & x+jy\\
       	w &= & u+jv \\
       	z & = &\frac{w+1}{w-1} \\
       	w &= & \frac{z+1}{z-1} \\
       \end{eqnarray*}
\item <2->等价关系:
       \begin{eqnarray*}
       u+jv &=& \frac{x^2+y^2-1-2yj}{(x-1)^2+y^2} \\
       |z|<1 &\Leftrightarrow& u<0 
       \end{eqnarray*}
\item <3->可在W域中使用Routh判据.
\end{itemize}
\end{frame}
\begin{frame}
\frametitle{W域的劳斯判据示例:}
\label{sec-5-2-2}


\begin{tikzpicture}[node distance=2.2em,auto,>=latex', thick]
%\path[use as bounding box] (-1,0) rectangle (10,-2); 
\path[->] node[] (r) {$r(t)$}; 
\path[->] node[ circle,inner sep=2pt,minimum size=1pt,draw,label=below left:$   $ ,right =of r] (p1) {}; 
\path[->](r) edge node {} (p1) ; 
\path[->] node[minimum size=2em,right =of p1] (s1) {}; 
\draw (s1.west)--(s1.north east);\draw[->] (s1.north west) arc (70:0:1.7em);\draw (s1.south) node {$T$};%\draw (s1.north) node[above] {$S$};
\path[](p1) edge node[midway] {$e(t)$} (s1) ; 
%\path[red,->] node[draw, inner sep=5pt,right =of s1] (g1) {$G_h(s)$}; 
%\path[->] (s1) edge node[midway] {$r^*(t)$} (g1); 
\path[red] node[draw, inner sep=5pt,right =of s1] (g2) {$\frac{K}{s(1+0.1s)}$}; 
\path[->] (s1) edge node[midway] {$e^*(t)$} (g2); 
\path[->] node[ right =of g2] (o) {$c(t)$}; 
\path[->] (g2) edge node {} (o); 
\path[->] node[minimum size=2em,above =of o] (sc) {}; 
\draw[dashed] (sc.west)--(sc.north east);\draw[dashed,->] (sc.north west) arc (70:0:1.7em);\draw[dashed] (sc.south) node {$T$};%\draw (sc.north) node[above] {$S$};
\path[dashed,draw](o.west)+(-1em,0) |- (sc.west) ; 
\path node[ right =of sc] (c) {$c^*(t)$}; 
\path[dashed,->] (sc) edge node {} (c); 
\path[red] node[ inner sep=5pt,below =of g2] (h) {$   $}; 
\path[draw] (g2.east)+(1em,0) |- (h.west);
\path[->,draw] (h.west) -| node [very near end] {$-$} (p1);
%\path[->, draw] (g.east)+(1em,0) -- +(1em,-3em) -| node[very near end] {$-$} (p1); 
\path[->] node[minimum size=2em,above =of p1] (sr) {}; 
\draw[dashed] (sr.west)--(sr.north east);\draw[dashed,->] (sr.north west) arc (70:0:1.7em);\draw[dashed] (sr.south) node {$T$};%\draw (sr.north) node[above] {$S$};
\path[dashed,draw](r.east)+(1em,0) |- (sr.west) ; 
\path node[ right =of sr] (i) {$r^*(t)$}; 
\path[dashed,->] (sr) edge node {} (i); 
\end{tikzpicture} 

分有无采样开关两种情况讨论为使系统稳定, $K$ 需要满足的条件.
\begin{itemize}

\item 解:\\
\label{sec-5-2-2-1}%
无采样开关时:
   \[D(s)=0.1s^2+s+k\]
   得:  $k>0$ 
\end{itemize} % ends low level
\end{frame}
\begin{frame}
\frametitle{W域的劳斯判据示例(续):有采样开关时:}
\label{sec-5-2-3}

\begin{eqnarray*}
G(z) &=  &{\cal Z}[\frac{K}{s(1+0.1s)}] 
  = \frac{0.632kz}{z^2-1.368z+0.368} \\
\Phi(z) &=& \frac{G(z)}{1+G(z)} 
\end{eqnarray*}
\begin{eqnarray*}
D(z) &=& z^2+(0.632k-1.368)z+0.368\\
z &=& \frac{w+1}{w-1} \\
D(w) &=& 0.632Kw^2+1.264w+(2.736-0.632k)
\end{eqnarray*}
\end{frame}
\begin{frame}
\frametitle{W域的劳斯判据示例(续):有采样开关时:}
\label{sec-5-2-4}

\begin{itemize}
\item Routh表:
      \[\begin{matrix}
      w^2 & 0.632k & 2.7360-0.632k \\
      w^1 & 1.264  & 0 \\
      w^0 & 2.736-0.632k
      \end{matrix}\]
\item 得:
      \begin{eqnarray*}
      0.632k &>  & 0\\
      2.736-0.632k& >& 0
      \end{eqnarray*}
\item 得:  
      \[0<k<4.33\]
\item <2->采样开关对稳定性有很大影响.
\end{itemize}
\end{frame}
\subsection{离散系统稳定性影响因素}
\label{sec-5-3}
\begin{frame}
\frametitle{离散系统稳定性影响因素}
\label{sec-5-3-1}

\begin{itemize}
\item <2->系统开环增益
\begin{itemize}
\item $k\uparrow$  则离散系统稳定性下降
\item $k\downarrow$  则离散系统稳定性提高
\end{itemize}
\item <3->采样周期
\begin{itemize}
\item $T\uparrow$  则离散系统稳定性下降
\item $T\downarrow$  则离散系统稳定性提高
\end{itemize}
\end{itemize}
\end{frame}
\section{离散系统稳态误差}
\label{sec-6}
\subsection{离散系统稳态误差}
\label{sec-6-1}
\begin{frame}
\frametitle{离散系统稳态误差}
\label{sec-6-1-1}

\begin{itemize}
\item 连续系统稳定误差:
\begin{itemize}
\item Laplacian 变换的终值定理
\item 静态误差系数
\item 动态误差系数
\end{itemize}
\item <2->离散系统稳态误差
\begin{itemize}
\item Z变换终值定理
      \begin{eqnarray*}
      \lim_{t\rightarrow\infty}e^*(t) & = &\lim_{z\rightarrow 1}(z-1)E(z)\\
       &=& \lim_{z\rightarrow 1}(z-1)\Phi_e(z)R(z)
      \end{eqnarray*}
\end{itemize}
\end{itemize}
\end{frame}
\begin{frame}
\frametitle{离散系统稳态误差示例:}
\label{sec-6-1-2}

\begin{tikzpicture}[node distance=2.2em,auto,>=latex', thick]
%\path[use as bounding box] (-1,0) rectangle (10,-2); 
\path[->] node[] (r) {$r(t)$}; 
\path[->] node[ circle,inner sep=2pt,minimum size=1pt,draw,label=below left:$   $ ,right =of r] (p1) {}; 
\path[->](r) edge node {} (p1) ; 
\path[->] node[minimum size=2em,right =of p1] (s1) {}; 
\draw (s1.west)--(s1.north east);\draw[->] (s1.north west) arc (70:0:1.7em);\draw (s1.south) node {$T$};%\draw (s1.north) node[above] {$S$};
\path[](p1) edge node[midway] {$e(t)$} (s1) ; 
%\path[red,->] node[draw, inner sep=5pt,right =of s1] (g1) {$G_h(s)$}; 
%\path[->] (s1) edge node[midway] {$r^*(t)$} (g1); 
\path[red] node[draw, inner sep=5pt,right =of s1] (g2) {$\frac{1}{s(1+0.1s)}$}; 
\path[->] (s1) edge node[midway] {$e^*(t)$} (g2); 
\path[->] node[ right =of g2] (o) {$c(t)$}; 
\path[->] (g2) edge node {} (o); 
\path[->] node[minimum size=2em,above =of o] (sc) {}; 
\draw[dashed] (sc.west)--(sc.north east);\draw[dashed,->] (sc.north west) arc (70:0:1.7em);\draw[dashed] (sc.south) node {$T$};%\draw (sc.north) node[above] {$S$};
\path[dashed,draw](o.west)+(-1em,0) |- (sc.west) ; 
\path node[ right =of sc] (c) {$c^*(t)$}; 
\path[dashed,->] (sc) edge node {} (c); 
\path[red] node[ inner sep=5pt,below =of g2] (h) {$   $}; 
\path[draw] (g2.east)+(1em,0) |- (h.west);
\path[->,draw] (h.west) -| node [very near end] {$-$} (p1);
%\path[->, draw] (g.east)+(1em,0) -- +(1em,-3em) -| node[very near end] {$-$} (p1); 
\path[->] node[minimum size=2em,above =of p1] (sr) {}; 
\draw[dashed] (sr.west)--(sr.north east);\draw[dashed,->] (sr.north west) arc (70:0:1.7em);\draw[dashed] (sr.south) node {$T$};%\draw (sr.north) node[above] {$S$};
\path[dashed,draw](r.east)+(1em,0) |- (sr.west) ; 
\path node[ right =of sr] (i) {$r^*(t)$}; 
\path[dashed,->] (sr) edge node {} (i); 
\end{tikzpicture} 
其中  $T=0.1,r_1(t)=1(t),r_2(t)=t$  求离散系统相应的稳态误差
\begin{itemize}

\item 解:\\
\label{sec-6-1-2-1}%
\begin{eqnarray*}
      G(z) &=& \frac{z(1-0.368)}{(z-1)(z-0.368)} \\
      \Phi_e(z) &= &\frac{1}{1+G(z)} 
       = \frac{(z-1)(z-0.368)}{z^2-0.736z+0.368}
      \end{eqnarray*}

\end{itemize} % ends low level
\end{frame}
\begin{frame}
\frametitle{离散系统稳态误差示例(续)}
\label{sec-6-1-3}
\begin{itemize}

\item $r_1(t) =  1(t)$ 时\\
\label{sec-6-1-3-1}%
\begin{eqnarray*}
      R_1(z) &=& \frac{1}{1-z^{-1}} \\
      \lim_{z\rightarrow 1}(1-z^{-1})\Phi_e(z)R(z) &=& 0
      \end{eqnarray*}

\item $r_2(t) = t(t)$ 时\\
\label{sec-6-1-3-2}%
\begin{eqnarray*}
      R_1(z) &=& \frac{Tz^{-1}}{(1-z^{-1})^2} \\
      \lim_{z\rightarrow 1}(1-z^{-1})\Phi_e(z)R(z) &=& \lim_{z\rightarrow 1}\frac{T(z-0.368)}{z^2-0.736z+0.368}\\
       &=& T \\
       &=& 0.1
      \end{eqnarray*}
\end{itemize} % ends low level
\end{frame}
\subsection{离散系统型别与静态误差系数}
\label{sec-6-2}
\begin{frame}
\frametitle{离散系统型别}
\label{sec-6-2-1}

\begin{itemize}
\item 连续系统型别:  
       \[G_o(s)=\frac{M(s)}{s^{\nu}N(s)}\]  
      若  $\nu=0,1,2$  则分别称为0型,I型,II型系统.
\item <2->离散系统型别:  
       \[G_o(z)=\frac{M(z)}{(z-1)^{\nu}N(z)}\]  
      若  $\nu=0,1,2$  则分别称为0型,I型,II型系统.
       ($G_o(z)$  为单位负反馈开环脉冲传递函数)
\end{itemize}
\end{frame}
\begin{frame}
\frametitle{静态误差系数:0型系统:}
\label{sec-6-2-2}
\begin{itemize}

\item 连续系统
\label{sec-6-2-2-1}%
\begin{eqnarray*}
K_p &=& \lim_{s\rightarrow 0}G_o(s)  \\
r(t)&=& 1 \\
e_{ss} &=& \frac{1}{1+K_p} 
\end{eqnarray*}

\item 离散系统
\label{sec-6-2-2-2}%
\begin{eqnarray*}
K_p &=& \lim_{z\rightarrow 1}(1+G_o(z))  \\
r(t)&=& 1(t) \\
e_{ss} &=& \frac{1}{K_p} 
\end{eqnarray*}
\end{itemize} % ends low level
\end{frame}
\begin{frame}
\frametitle{静态误差系数:I型系统:}
\label{sec-6-2-3}
\begin{itemize}

\item 连续系统
\label{sec-6-2-3-1}%
\begin{eqnarray*}
K_p &=& \lim_{s\rightarrow 0}sG_o(s)  \\
r(t)&=& t \\
e_{ss} &=& \frac{1}{K_v} 
\end{eqnarray*}

\item 离散系统
\label{sec-6-2-3-2}%
\begin{eqnarray*}
K_p &=& \lim_{z\rightarrow 1} (z-1)G_o(z)  \\
r(t)&=& t \\
e_{ss} &=& \frac{T}{K_v} 
\end{eqnarray*}
\end{itemize} % ends low level
\end{frame}
\begin{frame}
\frametitle{静态误差系数:II型系统:}
\label{sec-6-2-4}
\begin{itemize}

\item 连续系统
\label{sec-6-2-4-1}%
\begin{eqnarray*}
K_p &=& \lim_{s\rightarrow 0}s^2G_o(s)  \\
r(t)&=& \frac{t^2}{2} \\
e_{ss} &=& \frac{1}{K_a} 
\end{eqnarray*}

\item 离散系统
\label{sec-6-2-4-2}%
\begin{eqnarray*}
K_p &=& \lim_{z\rightarrow 0}(z-1)^2G_o(s)  \\
r(t)&=& \frac{t^2}{2} \\
e_{ss} &=& \frac{T^2}{K_a} 
\end{eqnarray*}

\end{itemize} % ends low level
\end{frame}
\section{离散系统动态性能分析}
\label{sec-7}







\mode<article>{连续系统:时域分析,根轨迹法,频域法,离散系统也有类似方法,这里只讨论时域响应}
\subsection{离散系统时间响应}
\label{sec-7-1}
\begin{frame}
\frametitle{离散系统时间响应计算}
\label{sec-7-1-1}

\begin{itemize}
\item 求  $\Phi(z)$  计算  $C(z)=\Phi(z)R(z)$ ,Z反变换求出  $C^*(t)$
\item 不存在  $\Phi(z)$  时,直接计算 $C(z)$  , Z反变换求出  $C^*(t)$
\end{itemize}
\end{frame}
\begin{frame}
\frametitle{离散系统时间响应计算示例:}
\label{sec-7-1-2}
\begin{itemize}

\item 结构图
\label{sec-7-1-2-1}%
\begin{tikzpicture}[node distance=2.2em,auto,>=latex', thick]
%\path[use as bounding box] (-1,0) rectangle (10,-2); 
\path[->] node[] (r) {$r(t)$}; 
\path[->] node[ circle,inner sep=2pt,minimum size=1pt,draw,label=below left:$   $ ,right =of r] (p1) {}; 
\path[->](r) edge node {} (p1) ; 
\path[->] node[minimum size=2em,right =of p1] (s1) {}; 
\draw (s1.west)--(s1.north east);\draw[->] (s1.north west) arc (70:0:1.7em);\draw (s1.south) node {$T$};%\draw (s1.north) node[above] {$S$};
\path[](p1) edge node[midway] {$e(t)$} (s1) ; 
\path[red,->] node[draw, inner sep=5pt,right =of s1] (g1) {$G_h(s)$}; 
\path[->] (s1) edge node[midway] {$e^*(t)$} (g1); 
\path[red] node[draw, inner sep=5pt,right =of g1] (g2) {$\frac{K}{s(1+s)}$}; 
\path[->] (g1) edge node[midway] {$   $} (g2); 
\path[->] node[ right =of g2] (o) {$c(t)$}; 
\path[->] (g2) edge node {} (o); 
\path[->] node[minimum size=2em,above =of o] (sc) {}; 
\draw[dashed] (sc.west)--(sc.north east);\draw[dashed,->] (sc.north west) arc (70:0:1.7em);\draw[dashed] (sc.south) node {$T$};%\draw (sc.north) node[above] {$S$};
\path[dashed,draw](o.west)+(-1em,0) |- (sc.west) ; 
\path node[ right =of sc] (c) {$c^*(t)$}; 
\path[dashed,->] (sc) edge node {} (c); 
\path[red] node[ inner sep=5pt,below =of g2] (h) {$   $}; 
\path[draw] (g2.east)+(1em,0) |- (h.west);
\path[->,draw] (h.west) -| node [very near end] {$-$} (p1);
%\path[->, draw] (g.east)+(1em,0) -- +(1em,-3em) -| node[very near end] {$-$} (p1); 
\path[->] node[minimum size=2em,above =of p1] (sr) {}; 
\draw[dashed] (sr.west)--(sr.north east);\draw[dashed,->] (sr.north west) arc (70:0:1.7em);\draw[dashed] (sr.south) node {$T$};%\draw (sr.north) node[above] {$S$};
\path[dashed,draw](r.east)+(1em,0) |- (sr.west) ; 
\path node[ right =of sr] (i) {$r^*(t)$}; 
\path[dashed,->] (sr) edge node {} (i); 
\end{tikzpicture} 

其中  $r(t)=1(t),T=1s$  求系统动态性能指标.


\item 解:\\
\label{sec-7-1-2-2}%
\begin{align*}
\only<2-3>{& G_o(z) = (1-z^{-1}){\cal Z}[\frac{1}{s^2+1}]  = \frac{0.368z+0.264}{(z-1)(z-0.368)} \\}
\only<3-4>{& \Phi(z)  = \frac{G_o(z)}{1+G_o(z)}  = \frac{0.368z+0.264}{z^2-z+0.632} \\}
\only<4-5>{& C(z) = \Phi(z)R(z)= \frac{0.368z^{-1}+0.264z^{-2}}{1-2z^{-1}+1.632z^{-2}-0.632z^{-3}} \\}
\only<5-6>{& C(z) = 0.368z^{-1}+z^{-2}+1.4z^{-3}+1.4z^{-4}1.147z^{-5}+0.895z^{-6}+0.802z^{-7}+0.868z^{-8}+\cdots \\}
\only<6-7>{& t_r=2s,t_p=4s,t_s=12s,\sigma\%=40\%}
\end{align*}
\end{itemize} % ends low level
\end{frame}
\subsection{采样器,保持器对系统动态性能的影响}
\label{sec-7-2}
\begin{frame}
\frametitle{采样器,保持器对系统动态性能的影响}
\label{sec-7-2-1}

\begin{itemize}
\item 定性说明:
\begin{itemize}
\item 采样器: 使系统稳定性下降,使  $\sigma\%\uparrow,t_r\downarrow,t_s\downarrow$
\item <2->保持器: 使系统稳定性下降,使  $\sigma\%\uparrow,t_r\uparrow,t_s\uparrow$
\end{itemize}
\item <3->对大迟延系统,无上述定性结论
\end{itemize}
\end{frame}
\subsection{闭环极点与动态响应的关系}
\label{sec-7-3}
\begin{frame}
\frametitle{闭环极点与动态响应的关系}
\label{sec-7-3-1}

\begin{eqnarray*}
z & = & e^{sT}\\
 &=& e^{\sigma T}e^{j\omega T}
\end{eqnarray*}

\begin{itemize}
\item 若闭环极点  $|z|>1$  , 则有  $\sigma>0$  , 系统不稳定.
\item 若闭环极点  $|z|=1$  , 则有  $\sigma=0$  , 等幅振荡.
\item 若闭环极点  $|z|<1$  , 则有  $\sigma<0$  , 系统稳定.
\begin{itemize}
\item <2->闭环极点为正实数: 单调收敛
\item <2->闭环极点为负实数: 振荡收敛
\item <2->闭环极点为具有正实部的复数: 低频振荡收敛
\item <2->闭环极点为具有负实部的复数: 高频振荡收敛
\item <2->若  $|z|\rightarrow 0$  ,  $\sigma\rightarrow -\infty$ , 收敛极快
\item <3->系统期望的闭环极点在Z平面单位圆的右半圆内
\end{itemize}
\end{itemize}
\end{frame}

\end{document}
