% Created 2015-10-14 Wed 17:47
\documentclass{beamer}
\usepackage{fixltx2e}
\usepackage{graphicx}
\usepackage{longtable}
\usepackage{float}
\usepackage{wrapfig}
\usepackage{soul}
\usepackage{textcomp}
\usepackage{marvosym}
\usepackage{wasysym}
\usepackage{latexsym}
\usepackage{amssymb}
\usepackage{hyperref}
\tolerance=1000
\usepackage{etex}
\usepackage{amsmath}
\usepackage{pstricks}
\usepackage{pgfplots}
\usepackage{tikz}
\usepackage[europeanresistors,americaninductors]{circuitikz}
\usepackage{colortbl}
\usepackage{yfonts}
\usetikzlibrary{shapes,arrows}
\usetikzlibrary{positioning}
\usetikzlibrary{arrows,shapes}
\usetikzlibrary{intersections}
\usetikzlibrary{calc,patterns,decorations.pathmorphing,decorations.markings}
\usepackage[BoldFont,SlantFont,CJKchecksingle]{xeCJK}
\setCJKmainfont[BoldFont=Evermore Hei]{Evermore Kai}
\setCJKmonofont{Evermore Kai}
\usepackage{pst-node}
\usepackage{pst-plot}
\psset{unit=5mm}
\mode<beamer>{\usetheme{Frankfurt}}
\mode<beamer>{\usecolortheme{dove}}
\mode<article>{\hypersetup{colorlinks=true,pdfborder={0 0 0}}}
\mode<beamer>{\AtBeginSection[]{\begin{frame}<beamer>\frametitle{Topic}\tableofcontents[currentsection]\end{frame}}}
\setbeamercovered{transparent}
\subtitle{线性定常系统的稳定性}
\providecommand{\alert}[1]{\textbf{#1}}

\title{线性系统时域分析法}
\author{}
\date{}
\hypersetup{
  pdfkeywords={},
  pdfsubject={},
  pdfcreator={Emacs Org-mode version 7.9.3f}}

\begin{document}

\maketitle

\begin{frame}
\frametitle{Outline}
\setcounter{tocdepth}{3}
\tableofcontents
\end{frame}













\section{稳定性的概念}
\label{sec-1}
\begin{frame}
\frametitle{稳定性}
\label{sec-1-1}


\psset{unit=1em}
\begin{pspicture}(0,0)(30em,5em)
%\begin{pspicture}[showgrid](0,0)(30em,10em)
%
%       o    \   o   /  \     /
%       ^     \  ^  /    \ o /
%      / \     \/ \/      \_/   
%
\psset{unit=1em}
%\multips(1.5,2.5)(4,0){3}{\pscircle(0,0){0.5}}
\psset{linecolor=red}
%\psline(0,0)(50,6)
\pscircle(4.5,4){1}
\pscurve(0,0)(4.5,3)(9,0)
\psset{linecolor=violet}
\pscircle(14.5,2){1}
\pscurve(10,0)(12,2)(14.5,1)(17.5,2)(19,0)
\psset{linecolor=blue}
\pscircle(24.5,1){1}
\pscurve(20,2)(24.5,0)(29,2)
%\psplot{0}{6}{x sqrt}
%\rput(3,3){中文}
%\pscustom{\dim{1}
%\code{
%dup dup  scale currentlinewidth exch div setlinewidth 
%0 0 moveto 3 6 lineto stroke}}
\end{pspicture}

定义:系统处于平衡状态时,若有干扰使系统偏离平衡状态,当扰动消除后,系统仍能回到原来的平衡状态,则称该系统是稳定的,反之称为不稳定.
\end{frame}
\begin{frame}
\frametitle{稳定的充要条件}
\label{sec-1-2}


\begin{itemize}
\item 系统的传递函数
     \begin{eqnarray*}
     \Phi(s) & = & \frac{C(s)}{R(s)} = \frac{k_{g}\prod_{i=1}^{m}(s-z_{i})}{\prod_{j=1}^{n}(s-p_{j})} 
     \end{eqnarray*}
\item 设系统输入为单位脉冲信号: $r(t)=\delta(t),R(s)=1$
     \begin{eqnarray*}
     C(s) & = & \Phi(s)R(s) =  \frac{k_{g}\prod_{i=1}^{m}(s-z_{i})}{\prod_{j=1}^{n}(s-p_{j})} \\
          & = &  k_{g}\sum_{j=1}^{n}\frac{k_{j}}{s-p_{j}} \\
     c(t) & = & k_{g}\sum_{j=1}^{n}k_{j}e^{p_{j}t}
     \end{eqnarray*}
\item 当 $\Re(p_{j})<0$ 时,有 $\lim_{t\rightarrow\infty}c(t) = 0$
\item 稳定性充要条件:系统的闭环极点均具有负实部
\end{itemize}
\end{frame}
\section{古尔维茨判据及劳斯判据}
\label{sec-2}
\begin{frame}
\frametitle{特征多项式}
\label{sec-2-1}

\begin{eqnarray*}
G(s) & = & \frac{M(s)}{N(s)}\\
M(s) &=&  b_{o}s^{m}+\cdots+b_{n} \\
N(s) & =& a_{0}s^{n}+\cdots+a_{n}
\end{eqnarray*}
其中:
\begin{itemize}
\item $N(s)$ 称为特征多项式
\item $N(s)=0$ 称为特征方程(只与 $a_{0},\cdots,a_{n}$ 有关)
\end{itemize}
\end{frame}
\begin{frame}
\frametitle{古尔维茨判据(代数判据):}
\label{sec-2-2}

\begin{itemize}
\item <2-> 稳定的必要条件:特征多项式中各项系数大于零.
\item <3-> 稳定的充要条件:特征多项式中各项系数所构成的主行列式及顺序主子式全部大于零.
\item <3-> 主行列式:
     \begin{equation*}
     \Delta_{n}=\left|
     \begin{matrix}
     a_{1}  & a_{3} & a_{5} & \cdots & 0  \\
     a_{0}  & a_{2} & a_{4} & \cdots & 0  \\
     0      & a_{1} & a_{3} & \cdots & 0  \\
     \vdots & a_{0} & a_{2} & \cdots & 0  \\
     \vdots &\vdots &\vdots &        & 0   \\
       0    &  0    & \cdots & a_{n-2}& a_{n}
     \end{matrix}
     \right|
     \end{equation*}
\item <4-> 顺序主子式有n-1个: $\Delta_{1},\cdots,\Delta_{n-1}$
\item <5-> 李纳德-戚帕特判据:若所有奇次古尔维茨行列式为正,则偶次古尔维茨行列式必为正,反之亦然
\end{itemize}
\end{frame}
\begin{frame}
\frametitle{劳斯-古尔维茨判据,简称劳斯判据}
\label{sec-2-3}

\begin{itemize}
\item <2->构造劳斯表判断系统是否稳定
     \begin{equation*}
     \begin{matrix}
     s^{n}   &  a_{0}  &  a_{2}  & a_{4}  & \cdots  \\
     s^{n-1} &  a_{1}  & a_{3}   & a_{5}  & \cdots  \\
     s^{n-2} &  c_{1}=\frac{a_1 a_2 - a_0 a_3}{a_1} &  c_{2}=\frac{a_1 a_4 - a_0 a_5}{a_1} & \cdots \\
     s^{n-3} &  d_{1}=\frac{c_1 a_3 - a_1 c_2}{c_1} &  d_{2}=\frac{c_1 a_5 - a_1 c_3}{c_1} & \cdots \\
     \vdots  &   \vdots                             &                                      & \cdots  \\
     s^{0}   &   a_{n}                              &                                      &
     \end{matrix}
     \end{equation*}
\item <3->劳斯判据:
\begin{itemize}
\item <4->系统稳定的充要条件:劳斯表中第一列元素均大于零
\item <5->若第一列元素有小于零的,则系统不稳定,且正实部根的个数等于第一列元素变号的次数.
\end{itemize}
\end{itemize}
\end{frame}
\begin{frame}
\frametitle{劳斯表与多项式除法}
\label{sec-2-4}

\begin{equation*}
     \begin{matrix}
     s^{n}   &  a_{0}  &  a_{2}  & a_{4}  & \cdots  \\
     s^{n-1} &  a_{1}  & a_{3}   & a_{5}  & \cdots  \\
     s^{n-2} &  c_{1}=a_2 -\frac{ a_0 a_3}{a_1} &  c_{2}=a_4 - \frac{a_0 a_5}{a_1} & \cdots \\
     s^{n-3} &  d_{1}=a_3 -\frac{ a_1 c_2}{c_1} &  d_{2}=a_5 -\frac{ a_1 c_3}{c_1} & \cdots \\
     \vdots  &   \vdots                             &                                      & \cdots  \\
     s^{0}   &   a_{n}                              &                                      &
     \end{matrix}
\end{equation*}
即:
\begin{align*}
    a_0 s^n+a_2 s^{n-2}+\cdots &= s \frac{a_0}{a_1}(a_1 s^{n-1}+a_3 s^{n-3}+\cdots)+c_1 s^{n-2}+c_2 s^{n-4}+\cdots\\
    c_1 s^{n-2}+c_2 s^{n-4}+\cdots &=s \frac{c_1}{d_1}(d_1 s^{n-3}+d_2 s^{n-5}+\cdots)+\cdots
\end{align*}
\end{frame}
\section{劳斯判据示例}
\label{sec-3}
\begin{frame}
\frametitle{Routh 判据示例1: $3s^{4}+10s^{3}+5s^{2}+s+2=0$}
\label{sec-3-1}


解:
\begin{block}<2->{劳斯表:}
\label{sec-3-1-1}

\begin{equation*}
\begin{matrix}
s^{4} & 3   &  5  &  2 \\
s^{3} & 10  &  1        \\
s^{2} & 4.7 &  2       \\
s^{1} & -\frac{15.3}{4.7}  \\
s^{0} & 2   
\end{matrix}
\end{equation*}
\end{block}
\begin{block}<3->{结论}
\label{sec-3-1-2}

系统不稳定,有2个不稳定根.
\end{block}
\end{frame}
\begin{frame}
\frametitle{Routh 判据示例2:}
\label{sec-3-2}

\begin{psmatrix}[rowsep=0.4,colsep=0.5]
%              .---------------.
%              |     s+1       |
% R-->o--> K ->| ------------- |--+--> C
%   _ ^        |  (s(s+2)(s+3) |  |
%     |        '---------------'  |
%     '---------------------------'
%
%
% 1                        2                        3                           4              5    6
$r$ &  \pscirclebox[framesep=-0.2em]{$\times$}& \psframebox{$K$} &  \psframebox{$\frac{s+1}{s(s+2)(s+3)}$}&  & $y$ \\
%link
\ncline{->}{1,1}{1,2}
\ncline{->}{1,2}{1,3}
\ncline{->}{1,3}{1,4}
\ncline{->}{1,4}{1,6}
\ncangles[angleA=0,angleB=-90,armA=0.5em,armB=1em]{->}{1,4}{1,2}
\naput[npos=3.6]{$-$}
\end{psmatrix}

求使系统稳定的 $K$ 的范围

\mode<article>{解:}
\begin{columns}
\begin{column}{0.55\textwidth}
\begin{block}<2->{传递函数}
\label{sec-3-2-1}


\begin{eqnarray*}
G(s)     & = & \frac{K(s+1)}{s(s+2)(s+3)}\\
\Phi(s)  & = & \frac{K(s+1)}{s^{3}+5s^{2}+(6+K)s+K}
\end{eqnarray*}
\end{block}
\end{column}
\begin{column}{0.45\textwidth}
\begin{block}<3->{劳斯表:}
\label{sec-3-2-2}


\[
\begin{matrix}
s^{3}  &    1   &  6+K  \\
s^{2}  &    5   &  K  \\
s^{1}  &   \frac{30+4K}{5}  & 0 \\
s^{0}  &  K
\end{matrix}
\]

稳定条件:
\begin{eqnarray*}
30+4K & > & 0 \\
K & >  & 0
\end{eqnarray*}

得: $K>0$
\end{block}
\end{column}
\end{columns}
\end{frame}
\section{劳斯判据应用中的特殊情况}
\label{sec-4}
\begin{frame}
\frametitle{Routh表特殊情况:第一列元素有零元,系统不稳定:}
\label{sec-4-1}

\begin{itemize}
\item 先用 $\epsilon$ 代替,再令 $\epsilon\rightarrow 0$
\item 系统升阶: $(s+a)D(s),a>0$
\end{itemize}
\begin{block}<2->{例:}
\label{sec-4-1-1}

系统特征方程: $D(s)=s^{4}+2s^{3}+s^{2}+2s+1=0$ 判断系统的稳定性,求出正实部根的个数.
\end{block}
\begin{columns}
\begin{column}{0.3\textwidth}
\begin{block}<3->{Roth表:}
\label{sec-4-1-2}


\[
\begin{matrix}
s^{4}   &  1  &   1  & 1 \\
s^{3}   &  2  &  2  \\
s^{2}   &  0(\epsilon)  &  1   \\
s^{1}   &  \frac{2\epsilon-2}{\epsilon} & 0 \\
s^{0}   &  1   \\
\end{matrix}
\]
\end{block}
\end{column}
\begin{column}{0.7\textwidth}
\begin{block}<4->{$(s+1)D(s)=s^{5}+3s^{4}+3s^{3}+3s^{2}+3s+1$}
\label{sec-4-1-3}


\[
\begin{matrix}
s^{5} & 1 & 3 & 3 \\
s^{4} & 3 & 3 & 1 \\
s^{3} & 2 & \frac{8}{3} \\
s^{2} & -1 & 1 \\
s^{1} & \frac{14}{3} & 0 \\
s^{0} & 1
\end{matrix}
\]
\end{block}
\end{column}
\end{columns}
\end{frame}
\begin{frame}
\frametitle{特殊情况:全行为零时:}
\label{sec-4-2}

\begin{itemize}
\item 全零行的数值由上一行求导代替
\item 辅助方程:全零行的上一行可构成辅助方程,辅助方程的解全部为系统特征根
\end{itemize}
\begin{block}<2->{例:}
\label{sec-4-2-1}

$D(s)=s^{4}+5s^{3}+5s^{2}-5s-6$ 求系统不稳定根的个数
\end{block}
\begin{columns}
\begin{column}{0.35\textwidth}
\begin{block}<3->{Routh表}
\label{sec-4-2-2}

\[
\begin{matrix}
s^{4} &  1 &  5 & -6 \\
s^{3} &  5 & -5 \\
s^{2} &  6 & -6 \\
s^{1} &  0(12)  & 0 \\
s^{0} &  -6
\end{matrix}
\]
\end{block}
\end{column}
\begin{column}{0.65\textwidth}
\begin{block}<4->{辅助方程}
\label{sec-4-2-3}

\begin{itemize}
\item 对 $6s^{2}-6$ 求导得 $12s$
\item 辅助方程 $6s^2-6=0$ ,得 $s_{1,2}=\pm 1$ 系统不稳定根为1
\end{itemize}
\end{block}
\end{column}
\end{columns}
\end{frame}
\section{劳斯判据求解系统参数}
\label{sec-5}
\begin{frame}
\frametitle{例: $D(s)=s^{4}+2s^{3}+ks^{2}+s+2$ 求 $K$ 值稳定范围}
\label{sec-5-1}


解:
\begin{columns}
\begin{column}{0.35\textwidth}
\begin{block}<2->{Routh表}
\label{sec-5-1-1}

\[
\begin{matrix}
s^{4} & 1 & K & 2\\
s^{3} & 2 & 1 \\
s^{2} & \frac{2K-1}{2} & 2\\
s^{1} & 1-\frac{8}{2K-1} & 0 \\
s^{0} & 2 \\
\end{matrix}
\]
\begin{eqnarray*}
2K-1 & > & 0 \\
%   K & > 1 \\
1-\frac{8}{2K-1} &>& 0  
%   K &> \frac{9}{2} 
\end{eqnarray*}
得: $K>\frac{9}{2}$
\end{block}
\end{column}
\begin{column}{0.7\textwidth}
\begin{block}<3->{当 $K=\frac{9}{2}$ 时:}
\label{sec-5-1-2}


Routh表:

$$
\begin{matrix}
s^{4} & 1 & 4.5 & 2\\
s^{3} & 2 & 1 \\
s^{2} & 4 & 2\\
s^{1} & 0(8) & 0 \\
s^{0} & 2 \\
\end{matrix}
$$
其中辅助方程为 $4s^{2}+2=0$ ,可解得 $s=\pm\frac{\sqrt{2}}{2}j$ 
\end{block}
\end{column}
\end{columns}
\end{frame}
\begin{frame}
\frametitle{例: $D(s)=Ts^{3}+s^{2}+K=0$}
\label{sec-5-2}
\begin{columns}
\begin{column}{0.5\textwidth}
\begin{block}<2->{Routh表}
\label{sec-5-2-1}

\[
\begin{matrix}
s^{3} & T  & 0\\
s^{2} & 1  & K \\
s^{1} & -TK \\
s^{0} & K 
\end{matrix}
\]
\end{block}
\end{column}
\begin{column}{0.5\textwidth}
\begin{block}<3->{无解}
\label{sec-5-2-2}

\begin{eqnarray*}
T & > & 0 \\
K & > & 0 \\
TK& < & 0 
\end{eqnarray*}
\end{block}
\end{column}
\end{columns}
\end{frame}
\section{相对稳定性}
\label{sec-6}
\begin{frame}
\frametitle{例:$D(s)=s^{3}+5.5s^{2}+8.5s+3$}
\label{sec-6-1}

\begin{itemize}
\item 判断系统是否具有相对稳定性:$\sigma=1$
\end{itemize}

解:
\begin{columns}
\begin{column}{0.3\textwidth}
\begin{block}<2->{Routh表}
\label{sec-6-1-1}

\[
\begin{matrix}
s^{3} & 1 &  8.5 \\
s^{2} & 5.5  & 3 \\
s^{1} & 8.5-\frac{3}{5.5} & 0 \\
s^{0} & 3
\end{matrix}
\]
\end{block}
\end{column}
\begin{column}{0.7\textwidth}
\begin{block}<3->{将 $s=z-\sigma$ 代入 $D(s)$ 中}
\label{sec-6-1-2}

得 $D(z)=z^{3}+2.5z^{2}+0.5z-1$ ,不稳定.
\begin{block}<4->{Routh表:}
\label{sec-6-1-2-1}



\[
\begin{matrix}
s^{3} & 1 &  0.5 \\
s^{2} & 2.5  & -1 \\
s^{1} & 0.5+\frac{1}{2.5} & 0 \\
s^{0} & -1
\end{matrix}
\]
\end{block}
\begin{block}<5->{结论}
\label{sec-6-1-2-2}

有一个不稳定极点.
\end{block}
\end{block}
\end{column}
\end{columns}
\end{frame}

\end{document}
