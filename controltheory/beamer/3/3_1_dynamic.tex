% Created 2015-09-16 Wed 17:36
\documentclass{beamer}
\usepackage{fixltx2e}
\usepackage{graphicx}
\usepackage{longtable}
\usepackage{float}
\usepackage{wrapfig}
\usepackage{soul}
\usepackage{textcomp}
\usepackage{marvosym}
\usepackage{wasysym}
\usepackage{latexsym}
\usepackage{amssymb}
\usepackage{hyperref}
\tolerance=1000
\usepackage{etex}
\usepackage{amsmath}
\usepackage{amssymb}
\usepackage{pstricks}
\usepackage{pgfplots}
\usepackage{tikz}
\usepackage[europeanresistors,americaninductors]{circuitikz}
\usepackage{colortbl}
\usepackage{yfonts}
\usetikzlibrary{shapes,arrows}
\usetikzlibrary{positioning}
\usetikzlibrary{arrows,shapes}
\usetikzlibrary{intersections}
\usetikzlibrary{calc,patterns,decorations.pathmorphing,decorations.markings}
\usepackage[BoldFont,SlantFont,CJKchecksingle]{xeCJK}
\setCJKmainfont[BoldFont=Evermore Hei]{Evermore Kai}
\setCJKmonofont{Evermore Kai}
\usepackage{pst-node}
\usepackage{pst-plot}
\psset{unit=5mm}
\mode<beamer>{\usetheme{Frankfurt}}
\mode<beamer>{\usecolortheme{dove}}
\mode<article>{\hypersetup{colorlinks=true,pdfborder={0 0 0}}}
\mode<beamer>{\AtBeginSection[]{\begin{frame}<beamer>\frametitle{Topic}\tableofcontents[currentsection]\end{frame}}}
\setbeamercovered{transparent}
\subtitle{线性系统动态性能分析}
\providecommand{\alert}[1]{\textbf{#1}}

\title{线性系统时域分析法}
\author{}
\date{}
\hypersetup{
  pdfkeywords={},
  pdfsubject={},
  pdfcreator={Emacs Org-mode version 7.9.3f}}

\begin{document}

\maketitle

\begin{frame}
\frametitle{Outline}
\setcounter{tocdepth}{3}
\tableofcontents
\end{frame}














\mode<article>{分析 $\sigma\%,t_s$ 等指标, $r(t)=1,R(s)=\frac{1}{s}$ }
\section{一阶系统动态性能}
\label{sec-1}
\begin{frame}
\frametitle{一阶系统单位阶跃响应}
\label{sec-1-1}

\begin{psmatrix}[rowsep=0.4,colsep=0.5]
%              
%              .------.
% R-->o----- ->| 1/Ts |--+--> C
%   _ ^        '------'  |
%     |                  |  
%     '------------------'
%
%
% 1                        2                        3             4              5    6
$R(s)$ &  \pscirclebox[framesep=-0.2em]{$\times$} &   &  \psframebox{$\frac{1}{Ts}$}   & {\hskip 1em}   & $C(s)$ \\
%link
\ncline{->}{1,1}{1,2}
\ncline{->}{1,2}{1,4}
\ncline{->}{1,4}{1,6}
%\ncangle[angleA=0,angleB=0,armA=0.5em,armB=0.5em]{1,4}{2,4}
\ncangles[angleA=0,angleB=-90,armA=1em,armB=2em]{->}{1,4}{1,2}
\naput[npos=3.6]{$-$}
\end{psmatrix}

\begin{eqnarray*}
G(s) & = & \frac{1}{Ts}\\
\Phi(s) &=& \frac{1}{Ts+1} \\
R(s) &= & \frac{1}{s} \\
C(s) &=& \Phi(s)R(s) \\
     &=& \frac{-T}{Ts+1}+\frac{1}{s} \\
c(t) &=& 1-e^{-t/T}
\end{eqnarray*}
\end{frame}
\begin{frame}
\frametitle{一阶系统单位脉冲响应}
\label{sec-1-2}


\begin{eqnarray*}
R(s) &= & 1 \\
C(s) &=& \Phi(s)R(s) \\
     &=& \Phi(s) \\
     &=& \frac{1}{Ts+1} \\
c(t) &=& \frac{1}{T}e^{-t/T}
\end{eqnarray*}
\end{frame}
\begin{frame}
\frametitle{一阶系统单位斜坡响应}
\label{sec-1-3}

\begin{eqnarray*}
R(s) &= & \frac{1}{s^{2}} \\
C(s) &=& \Phi(s)R(s) \\
     &=& \frac{1}{(Ts+1)s^{2}} \\
     &=& \frac{1}{s^{2}}-\frac{T}{s}+\frac{T^2}{Ts+1} \\
c(t) &=& (t-T)+Te^{-t/T}
\end{eqnarray*}
\end{frame}
\begin{frame}
\frametitle{一阶系统单位加速度响应}
\label{sec-1-4}

\begin{eqnarray*}
R(s) &= & \frac{1}{s^{3}} \\
C(s) &=& \Phi(s)R(s) \\
     &=& \frac{1}{(Ts+1)s^{3}} \\
     &=& \frac{1}{s^3}-\frac{T}{s^2}+\frac{T^2}{s}-\frac{T^3}{sT+1}\\
c(t) &=& \frac{1}{2}t^2-Tt+T^2(1-e^{-t/T})
\end{eqnarray*}
\end{frame}
\section{二阶系统时域分析}
\label{sec-2}
\begin{frame}
\frametitle{传递函数}
\label{sec-2-1}

\begin{psmatrix}[rowsep=0.4,colsep=0.5]
%              
%              .----------------------.
% R-->o----- ->| w_n^2/s^2+2\xi\w_n s |--+--> C
%   _ ^        '----------------------'  |
%     |                                  |  
%     '----------------------------------'
%
% 1                        2                        3             4              5    6
$R(s)$ &  \pscirclebox[framesep=-0.2em]{$\times$} &   &  \psframebox{$\frac{\omega_n^2}{s^2+2\xi\omega_n s}$}   & {\hskip 1em}   & $C(s)$ \\
%link
\ncline{->}{1,1}{1,2}
\ncline{->}{1,2}{1,4}
\ncline{->}{1,4}{1,6}
%\ncangle[angleA=0,angleB=0,armA=0.5em,armB=0.5em]{1,4}{2,4}
\ncangles[angleA=0,angleB=-90,armA=1em,armB=2em]{->}{1,4}{1,2}
\naput[npos=3.6]{$-$}
\end{psmatrix}

\begin{itemize}
\item $\xi$: 阻尼比
\item $\omega_n$:自然频率,无阻尼振荡频率
\end{itemize}
\begin{eqnarray*}
r(t) &=& 1 \\
R(s) &=& \frac{1}{s}\\
G(s) & =& \frac{\omega_n^2}{s^2+2\xi\omega_n s} \\
\Phi(s) &=& \frac{\omega_n^2}{s^2+2\xi\omega_n s+\omega_n^2}\\
p_{1,2} &=& -\xi\omega_n\pm\omega_n\sqrt{\xi^2-1}
\end{eqnarray*}
\end{frame}
\begin{frame}
\frametitle{$\xi\leq 0$}
\label{sec-2-2}

\begin{itemize}
\item <2-> $\xi< 0$ 时有正实根,不稳定
\item <3-> $\xi=0$ 时有两个纯虚根,无阻尼,临界稳定,等幅振荡,频率为$\omega_n$,
        \begin{eqnarray*}
        C(s) & = & \frac{\omega_n^2}{s^2+\omega_n^2}\cdot \frac{1}{s}  \\
             & =& \frac{-s}{s^2+\omega_n^2}+\frac{1}{s} \\
        c(t) &=& 1-\cos\omega_n t
        \end{eqnarray*}
\end{itemize}
\end{frame}
\begin{frame}
\frametitle{$\xi>1$}
\label{sec-2-3}

   系统闭环极点为两个不同的实根.过阻尼,相当于两个一阶系统并联, $\sigma\%=0$
      \begin{eqnarray*}
      \Phi(s) & = & \frac{\omega_n^2}{(s-p_1)(s-p_2)} \\
              & = & \frac{K_1}{s-p_1}+\frac{K_2}{s-p_2}\\
      c(t)    &=& 1-\frac{e^{p_1 t}}{1-\frac{p_1}{p_2}}-\frac{e^{p_2 t}}{1-\frac{p_2}{p_1}}
      \end{eqnarray*}
\end{frame}
\begin{frame}
\frametitle{$\xi=1$}
\label{sec-2-4}

\begin{itemize}
\item <1-> 闭环极点有两个相同的负实根$p_{1,2}=-\xi\omega_n=-\omega_n$
      \begin{eqnarray*}
      C(s) & = &\frac{\omega_n^2}{(s+\omega_n)^2}\cdot\frac{1}{s} \\
      c(t) &=& 1-e^{-\omega_n t}(1+\omega_n t)
      \end{eqnarray*}
\item <2-> 且有:
      \begin{eqnarray*}
      \frac{dc(t)}{dt} &=& \omega_ne^{-\omega_n t}(1+\omega_n t)-\omega_n e^{-\omega_n t}
       =  \omega_n^2 te^{-\omega_n t} 
       >  0 \\
      c(0) &=&0 \\
      c(\infty)&=&1\\
      \sigma \% &=& 0\\
      t_s &=& 4.75T \\
      T &=&\frac{1}{\omega_n}
      \end{eqnarray*}
\end{itemize}
\end{frame}
\begin{frame}
\frametitle{$0<\xi<1$}
\label{sec-2-5}

系统有一对实部小于零的共轭复根, $p_{1,2}  =  -\xi\omega_n\pm j\omega_n\sqrt{1-\xi^2}$
\begin{eqnarray*}
C(s) &=& \frac{\omega_n^2}{s^2+2\xi\omega_n s+\omega_n^2}\cdot\frac{1}{s} \\
     &=& \frac{1}{s}+\frac{p_2}{(p_1-p_2)(s-p_1)}+\frac{p_1}{(p_2-p_1)(s-p_2)} 
\end{eqnarray*}
\end{frame}
\begin{frame}
\frametitle{$0<\xi<1$}
\label{sec-2-6}

\begin{eqnarray*}
c(t) &=& 1+\frac{p_2}{p_1-p_2}e^{p_1 t}+\frac{p_1}{p_2-p_1}e^{p_2 t}\\
     &=& 1+2\Re\left[ \frac{p_2}{p_1-p_2}e^{p_1 t} \right]\\
     &=& 1+2\Re\left[ \frac{-\omega_n e^{j\beta}}{2j\omega_d}e^{-\xi\omega_n t}e^{j\omega_d t} \right]\\
     &=& 1-e^{-\xi\omega_n t}\Re\left[ \frac{\omega_n }{j\omega_d}e^{j(\omega_d t+\beta)} \right]\\
     &=& 1-\frac{\omega_n }{\omega_d}e^{-\xi\omega_n t}\sin(\omega_d t+\beta)\\
\beta & = & \tan^{-1}\frac{\sqrt{1-\xi^2}}{\xi} \qquad   \omega_d = \sqrt{1-\xi^2}\omega_n
\end{eqnarray*}
\end{frame}
\section{二阶系统阶跃响应指标计算}
\label{sec-3}
\begin{frame}
\frametitle{二阶欠阻尼系统阶跃响应指标}
\label{sec-3-1}


\begin{eqnarray*}
   c(t)  &=& 1-\frac{1}{\sqrt{1-\xi^2}}e^{-\xi\omega_n t}\sin(\omega_d t+\beta)\\
\end{eqnarray*}

\begin{itemize}
\item 欠阻尼. $\omega_d$ 称为有阻尼振荡频率.最佳阻尼比 $\xi=0.707$
\item 指标: $\sigma\% , t_s , t_p , t_r$ 等
\end{itemize}
\end{frame}
\begin{frame}
\frametitle{上升时间 $t_r$}
\label{sec-3-2}

\begin{itemize}
\item $100\%$ 的 $t_r$ : $c(t)$ 首次达到 $c(\infty)$ 的时间
\item $90\%$ 的 $t_r$ : $c(t)$ 首次达到 $90\%c(\infty)$ 的时间
\item $70\%$ 的 $t_r$ : $c(t)$ 首次达到 $70\%c(\infty)$ 的时间
\end{itemize}

\begin{eqnarray*}
c(t) & = & c(\infty) \\
1-\frac{1}{\sqrt{1-\xi^2}}e^{-\xi\omega_n t}\sin(\omega_d t+\beta) &=& 1 \\
sin(\omega_d t+\beta) &=& 0 \\
\omega_d t+\beta &=& k\pi \\
t_r &=& \frac{\pi-\beta}{\omega_d}
\end{eqnarray*}
\end{frame}
\begin{frame}
\frametitle{峰值时间 $t_p$}
\label{sec-3-3}


\mode<article>{$c(t)$ 达到最大值的时间}

\begin{eqnarray*}
\frac{dc(t)}{dt} &=& 0 \\
-\xi\omega_n e^{-\xi\omega_n t}\sin(\omega_d t+\beta)+e^{-\xi\omega_n t}\omega_d\cos(\omega_d t+\beta) & = & 0 \\
\omega_d\cos(\omega_d t+\beta) &=& \xi\omega_n \sin(\omega_d t+\beta) \\
\tan(\omega_d t+\beta) &=& \frac{\sqrt{1-\xi^2}}{\xi} \\
\tan(\omega_d t+\beta) &=& \tan\beta \\
\omega_d t &=& k\pi\\
t_p &=& \frac{\pi}{\omega_d}
\end{eqnarray*}
\end{frame}
\begin{frame}
\frametitle{超调量 $\sigma \%$}
\label{sec-3-4}

\begin{eqnarray*}
\sigma \% & = & \frac{c_{max}-c(\infty)}{c(\infty)}\times 100\% 
         = (c(t_p)-1) \\
         &=& -\frac{1}{\sqrt{1-\xi^2}}e^{-\xi\omega_n t_p}\sin(\omega_d t_p+\beta) \\
         &=& -\frac{1}{\sqrt{1-\xi^2}}e^{-\frac{\xi\omega_n\pi}{\omega_d}}\sin(\pi+\beta) \\
         &=& \frac{1}{\sqrt{1-\xi^2}}e^{-\frac{\xi\pi}{\sqrt{1-\xi^2}}}\sin(\beta) \\
         &=& e^{-\frac{\xi\pi}{\sqrt{1-\xi^2}}}\times 100\% \\
\end{eqnarray*}

\mode<article>{分析:}
\begin{itemize}
\item <2->$\sigma\%$ 只与 $\xi$ 有关,两者成反比关系
\item <3->工程上一般取 $\xi\in[0.4,0.8]$
\item <4->最佳阻尼比 $\xi=0.707,\sigma\%=4.3\%$
\end{itemize}
\end{frame}
\begin{frame}
\frametitle{调节时间 $t_s$}
\label{sec-3-5}


\mode<article>{近似估算:}

\begin{eqnarray*}
c(t) & = & 1-\frac{1}{\sqrt{1-\xi^2}}e^{-\xi\omega_n t}\sin(\omega_d t+\beta)\\
     &\approx & 1-\frac{1}{\sqrt{1-\xi^2}}e^{-\xi\omega_n t} \\
%     &\approx & 1-e^{-\xi\omega_n t} \\
e(t) &=& c(\infty)-c(t) \\
    &\approx& \frac{1}{\sqrt{1-\xi^2}}e^{-\xi\omega_n t}\\ 
%     &\approx& e^{-\xi\omega_n t}
\end{eqnarray*}

\begin{itemize}
\item <2-> $t_s$ 与 $\omega_n,\xi$ 有关:通常取 $\xi\omega_n t_s = 3.5,t_s=\frac{3.5}{\xi\omega}$
\end{itemize}
       
\end{frame}
\begin{frame}
\frametitle{二阶过阻尼系统阶跃响应指标}
\label{sec-3-6}

\begin{itemize}
\item <2->$\sigma\%=0$
\item <3->$\xi=1$ 时, 
       \[t_s=\frac{4.75}{\omega_n}\]
\item <4->$\xi>1,|p_1|\ll |p_2|$ 时,系统降阶,去掉极点 $p_2$ , 
      \[t_s=\frac{3}{|p_1|}\]
\end{itemize}
\end{frame}
\section{二阶系统单位斜坡响应}
\label{sec-4}
\begin{frame}
\frametitle{欠阻尼单位斜坡响应}
\label{sec-4-1}

\begin{eqnarray*}
C(s) & =& \frac{\omega_n^2}{s^2(s^2+2\xi\omega_n s+\omega_n^2)}\\
&=&\frac{1}{s^2}-\frac{2\xi}{\omega_n s}+\frac{2\xi(s+\xi\omega_n)+\omega_n(2\xi^2-1)}{\omega_n(s^2+2\xi\omega_n s+\omega_n^2)}\\
c(t)&=&t-\frac{2\xi}{\omega_n}+\frac{1}{\omega_n\sqrt{1-\xi^2}}e^{-\xi\omega_n t}\sin(\omega_d t+2\beta)\\
e(t)&=&\frac{2\xi}{\omega_n}\left[1-\frac{1}{2\xi\sqrt{1-\xi^2}}e^{-\xi\omega_n t}\sin(\omega_d t+2\beta)\right]
\end{eqnarray*}
\end{frame}
\begin{frame}
\frametitle{临界阻尼单位斜坡响应}
\label{sec-4-2}

\begin{eqnarray*}
c(t) & =& t-\frac{2}{\omega_n}+\frac{2}{\omega_n}(1+\frac{1}{2}\omega_n t)e^{-\omega_n t} \\
e(t) &=& \frac{2}{\omega_n}\left[1-(1+\frac{1}{2}\omega_n t)e^{-\omega_n t}\right] 
\end{eqnarray*}
\end{frame}
\begin{frame}
\frametitle{过阻尼单位斜坡响应}
\label{sec-4-3}

\begin{eqnarray*}
C(s) &= &\frac{1}{s^2}-\frac{2\xi}{\omega_n s}+\frac{2\xi(s+\xi\omega_n)+\omega_n(2\xi^2-1)}{\omega_n(s-p_1)(s-p_2)} \\
p_1 &=& -\omega_n\xi+\omega_n\sqrt{\xi^2-1} \\
p_2 &=& -\omega_n\xi-\omega_n\sqrt{\xi^2-1} \\
c(t) &=& t-\frac{2\xi}{\omega_n}+\frac{2\xi^2-1+2\xi\sqrt{\xi^2-1}}{2\omega_n\sqrt{\xi^2-1}}e^{p_1 t} \\
     & & -\frac{2\xi^2-1-2\xi\sqrt{\xi^2-1}}{2\omega_n\sqrt{\xi^2-1}}e^{p_2 t} 
\end{eqnarray*}
\end{frame}
\section{高阶系统时域分析(3阶及以上系统)}
\label{sec-5}
\begin{frame}
\frametitle{三阶系统}
\label{sec-5-1}

\begin{itemize}
\item <2-> 根的几种情况
\begin{itemize}
\item <3-> 3个负实根 $p_1,p_2,p_3$
\item <4-> 1个负实根,一对共轭复根 
      \[-s_0,-\xi\omega_n\pm j\omega_n\sqrt{1-\xi^2},(0<\xi<1)\]
\end{itemize}
\item <5-> 重点考虑有复根的情况.
\end{itemize}
\end{frame}
\begin{frame}
\frametitle{三阶系统($\Phi(s)$) 单位阶跃响应($C(s)$)}
\label{sec-5-2}

\begin{eqnarray*}
 \Phi(s) & = & \frac{s_0\omega_n^2}{(s+s_0)(s^2+2\xi\omega_n s+\omega_n^2)} \\
 C(s) &=& \frac{s_0\omega_n^2}{s(s+s_0)(s^2+2\xi\omega_n s+\omega_n^2)} \\
 c(t) &=& 1-\frac{e^{-s_0 t}}{b\xi^2(b-2)+1}-\frac{e^{-\xi\omega_n t}}{b\xi^2(b-2)+1} \\
     & & \left(b\xi^2(b-2)\cos\omega_d t + \frac{b\xi(\xi^2(b-2)+1)}{\sqrt{1-\xi^2}}\sin\omega_d t\right) \\
 \omega_d &=& \omega_n\sqrt{1-\xi^2} \\
 b &=& \frac{s_0}{\xi\omega_n}
\end{eqnarray*}
\end{frame}
\begin{frame}
\frametitle{$b$ 对 $c(t)$ 的影响}
\label{sec-5-3}


\begin{itemize}
\item <2->复根比实根离虚轴近得多
     \begin{eqnarray*}
     b & \gg & 1\\
     c(t) &\approx & 1-e^{-\xi\omega_n t}\left(\cos\omega_d t + \frac{\xi}{\sqrt{1-\xi^2}}\sin\omega_d t\right) 
     \end{eqnarray*}
     近似看作2阶欠阻尼系统.
\item <3->实根比复根离虚轴近得多
     \begin{eqnarray*}
     b & \approx & 0\\
     c(t) &\approx & 1-e^{-s_0 t}
     \end{eqnarray*}
     近似看作1阶系统
\item <4->实根与复根与虚轴距离同
     \begin{eqnarray*}
     b & \approx & 1\\
     c(t) &\approx & 1-\frac{e^{-\xi\omega_n t}}{1-\xi^2}\left(1+\xi\sin(\omega_d t-\beta)\right) 
     \end{eqnarray*}
\end{itemize}
\end{frame}
\begin{frame}
\frametitle{主导极点法}
\label{sec-5-4}

\begin{itemize}
\item <2->目的:分析高阶系统的性能
\item <3->内容:系统有多个极点,其中某些极点决定了整个系统的性能,对系统起主导作用,称这些极点为主导极点.
\item <4->确定方法:主导极点离虚轴距离为 $a$ ,其它极点离虚轴距离 $\geq 5a$
\end{itemize}
\end{frame}

\end{document}
