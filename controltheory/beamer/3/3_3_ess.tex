% Created 2014-10-11 星期六 12:06
\documentclass{beamer}
\usepackage{fixltx2e}
\usepackage{graphicx}
\usepackage{longtable}
\usepackage{float}
\usepackage{wrapfig}
\usepackage{soul}
\usepackage{textcomp}
\usepackage{marvosym}
\usepackage{wasysym}
\usepackage{latexsym}
\usepackage{amssymb}
\usepackage{hyperref}
\tolerance=1000
\usepackage{etex}
\usepackage{amsmath}
\usepackage{pstricks}
\usepackage{pgfplots}
\usepackage{tikz}
\usepackage[europeanresistors,americaninductors]{circuitikz}
\usepackage{colortbl}
\usepackage{yfonts}
\usetikzlibrary{shapes,arrows}
\usetikzlibrary{positioning}
\usetikzlibrary{arrows,shapes}
\usetikzlibrary{intersections}
\usetikzlibrary{calc,patterns,decorations.pathmorphing,decorations.markings}
\usepackage[BoldFont,SlantFont,CJKchecksingle]{xeCJK}
\setCJKmainfont[BoldFont=Evermore Hei]{Evermore Kai}
\setCJKmonofont{Evermore Kai}
\usepackage{pst-node}
\usepackage{pst-plot}
\psset{unit=5mm}
\newcommand*\diff{\mathop{}\!\mathrm{d}}
\allowdisplaybreaks
\usepackage{polynom}
\mode<beamer>{\usetheme{Frankfurt}}
\mode<beamer>{\usecolortheme{dove}}
\mode<article>{\hypersetup{colorlinks=true,pdfborder={0 0 0}}}
\mode<beamer>{\AtBeginSection[]{\begin{frame}<beamer>\frametitle{Topic}\tableofcontents[currentsection]\end{frame}}}
\setbeamercovered{transparent}
\subtitle{系统的稳态误差计算}
\providecommand{\alert}[1]{\textbf{#1}}

\title{线性系统时域分析法}
\author{}
\date{}
\hypersetup{
  pdfkeywords={},
  pdfsubject={},
  pdfcreator={Emacs Org-mode version 7.9.3f}}

\begin{document}

\maketitle

\begin{frame}
\frametitle{Outline}
\setcounter{tocdepth}{3}
\tableofcontents
\end{frame}














\section{误差传递函数}
\label{sec-1}
\begin{frame}
\frametitle{系统误差}
\label{sec-1-1}
\begin{columns}
\begin{column}{0.5\textwidth}
%% ignore
\label{sec-1-1-1}

\begin{psmatrix}[rowsep=0.4,colsep=0.5]
%              
%         E(s) .------.
% R-->o----- ->| G(s) |--+--> C
%   _ ^        '------'  |
%     |                  |  
%     '--------[ H(s) ]--'
%
%
% 1                        2                        3             4              5    6
$R(s)$ &  \pscirclebox[framesep=-0.2em]{$\times$} &$\cdots $   &  \psframebox{$G(s)$}   &   & $C(s)$ \\
       &                                          &       &  \psframebox{$H(s)$}&  &        \\
%link
\ncline{->}{1,1}{1,2}
\ncline{->}{1,2}{1,4}
\naput{$E(s)$}
\ncline{->}{1,4}{1,6}
\ncangle[angleA=0,angleB=0,armA=0.5em,armB=0.5em]{1,4}{2,4}
\ncangle[angleA=180,angleB=-90,armA=0.5em,armB=1em]{->}{2,4}{1,2}
\naput[npos=2.3]{$-$}
\end{psmatrix}

\mode<article>{系统误差有两种:}

\begin{itemize}
\item <2->输入端定义:$E_{2}(s)=E(s)$
\item <3->输出端定义:$E_{1}(s)=C_{expect}-C_{real}$
\item <4->不加特别说明,系统误差指的是输入端定义.
\end{itemize}
\end{column}
\begin{column}{0.5\textwidth}
\begin{block}<5->{$E(s)$ 与 $E_1(s)$}
\label{sec-1-1-2}


\begin{eqnarray*}
C_{expect} & = & \frac{R(s)}{H(s)}\\
E_{1}(s)   & = & \frac{R(s)}{H(s)}-C(s) \\
           & = & \frac{R(s)-C(s)H(s)}{H(s)}\\
           & =& \frac{E(s)}{H(s)}
\end{eqnarray*}
\end{block}
\end{column}
\end{columns}
\end{frame}
\begin{frame}
\frametitle{误差传递函数:}
\label{sec-1-2}

\begin{eqnarray*}
\Phi_{e}(s) & = & \frac{E(s)}{R(s)}\\
            & = & \frac{1}{1+G(s)H(s)} \\
            & = & \frac{R(s)-H(s)C(s)}{R(s)} \\
            & = & 1-H(s)\Phi(s)
\end{eqnarray*}

\begin{itemize}
\item <2-> 系统误差:$E(s)=\Phi_{e}(s)R(s)$
\end{itemize}
\end{frame}
\begin{frame}
\frametitle{稳态误差:}
\label{sec-1-3}

\begin{eqnarray*}
e_{ss} &=& \lim_{t\rightarrow \infty}e(t) \\
       &=& \lim_{s\rightarrow 0}sE(s)  \\
       &= & \lim_{s\rightarrow 0}s\Phi_{e}(s)R(s)
\end{eqnarray*}

\begin{itemize}
\item <2->稳态误差与输入信号有关
\item <3->求稳态误差前要判断系统稳定性
\end{itemize}
\end{frame}
\begin{frame}
\frametitle{扰动作用下的稳态误差}
\label{sec-1-4}

\begin{psmatrix}[rowsep=0.4,colsep=0.5]
%          1    2  3     4    5    6    7 
%                            | N(s)
%               E(s)         v +  
%         R---->o--> G_1(s)--o- G_2(s)--+--> C
%             _ ^                       |
%               |                       |  
%               '-----------H(s)--------+
%           
%
% 1                         2                           3                  4                     5
       &                                       &               &                        &         $N(s)$                                  &   \\  
$R(s)$  &\pscirclebox[framesep=-0.2em]{$\times$}& {\hskip 1em} &  \psframebox{$G_1(s)$} &  \pscirclebox[framesep=-0.2em]{$\times$} &  \psframebox{$G_2(s)$}  & \    & $C(s)$ \\
       &                                         &              &     &   \psframebox{$ H(s) $} 
%link
\ncline{->}{1,5}{2,5}
%\naput{$N(s)$}
\ncline{->}{2,1}{2,2}
\ncline{->}{2,2}{2,4}
\naput{$E(s)$}
\ncline{->}{2,4}{2,5}
\ncline{->}{2,5}{2,6}
\ncline{->}{2,6}{2,8}
\ncangles[angleA=180,angleB=0,armA=0em,armB=0em]{->}{2,7}{3,5}
\ncangle[angleA=180,angleB=-90,armA=0em,armB=0em]{->}{3,5}{2,2}
\naput[npos=1.6]{$-$}
\end{psmatrix}
\begin{columns}
\begin{column}{0.5\textwidth}
\begin{block}<2->{输入端定义:}
\label{sec-1-4-1}


\begin{eqnarray*}
E(s) & = &E_R(s)+E_N(s) \\
E_R(s)&=& \Phi_e(s)R(s) \\
E_N(s)&=& \Phi_{en}(s)N(s) \\
\end{eqnarray*}
\end{block}
\end{column}
\begin{column}{0.5\textwidth}
\begin{block}<3->{输出端定义:}
\label{sec-1-4-2}

令 $R(s)=0$ ,计算 $N(s)$ 单独引起的 $e_{ss}$ ,此时 $C_{expect}(s)=0$ 

\begin{eqnarray*}
E(s) & = & 0-C(s) \\
     & = & -\Phi_N(s)N(s)\\
\Phi_N(s) &=& \frac{G_2}{1+G_1G_2}\\
e_{ss}&=&\lim_{s\rightarrow 0}s(-\Phi_N(s)N(s)) 
\end{eqnarray*}
\end{block}
\end{column}
\end{columns}
\end{frame}
\section{系统类型与静态误差系数}
\label{sec-2}
\begin{frame}
\frametitle{阶跃输入:}
\label{sec-2-1}

\begin{eqnarray*}
r(t) & = & A \\
R(s) & = & \frac{A}{s} \\
e_{ss}&=& \lim_{s\rightarrow 0}s \cdot\frac{1}{1+G_{open}(s)}\cdot\frac{A}{s} \\
      &=& \lim_{s\rightarrow 0}\frac{A}{1+G_{open}(s)}
\end{eqnarray*}
\end{frame}
\begin{frame}
\frametitle{速度输入}
\label{sec-2-2}

\begin{eqnarray*}
r(t) & = & vt \\
R(s) & = & \frac{v}{s^{2}} \\
e_{ss}&=& \lim_{s\rightarrow 0}s \cdot\frac{1}{1+G_{open}(s)}\cdot\frac{v}{s^{2}} \\
      &=& \lim_{s\rightarrow 0}\frac{A}{s+sG_{open}(s)}\\
      &=& \lim_{s\rightarrow 0}\frac{A}{sG_{open}(s)}
\end{eqnarray*}
\end{frame}
\begin{frame}
\frametitle{加速度输入}
\label{sec-2-3}

\begin{eqnarray*}
r(t) & = & \frac{1}{2}at^{2} \\
R(s) & = & \frac{a}{s^{2}} \\
e_{ss}&=& \lim_{s\rightarrow 0}s \cdot\frac{1}{1+G_{open}(s)}\cdot\frac{a}{s^{3}} \\
      &=& \lim_{s\rightarrow 0}\frac{A}{s^{2}+s^{2}G_{open}(s)}\\
      &=& \lim_{s\rightarrow 0}\frac{A}{s^{2}G_{open}(s)}
\end{eqnarray*}
\end{frame}
\begin{frame}
\frametitle{系统类型}
\label{sec-2-4}

\begin{itemize}
\item <2->由开环传递函数定义
     \begin{eqnarray*}
      G_{open} & = & G(s)H(s) \\
               & = & \frac{K\prod_{i=1}^{m}(\tau_{i}s+1)}{s^{\nu}\prod_{j=1}^{n-\nu}(T_{j}s+1)}
     \end{eqnarray*}
\item <2->其中 $K$ 为开环增益.
\item <3->定义:
\begin{itemize}
\item $\nu=0$ 称为0型系统
\item $\nu=1$ 称为I型系统
\item $\nu=2$ 称为II型系统
\end{itemize}
\end{itemize}
\end{frame}
\begin{frame}
\frametitle{静态误差系数}
\label{sec-2-5}

\begin{itemize}
\item <2->静态位置误差系数
       \begin{eqnarray*}
       r(t) &=& A\\
       e_{ss}&=&\frac{A}{1+K_{p}}, \qquad
       K_{p}=\lim_{s\rightarrow 0} G_{open}(s)
       \end{eqnarray*}
\item <3->静态速度误差系数 
       \begin{eqnarray*}
       r(t)&=&vt\\
       e_{ss}&=&\frac{v}{K_{v}}, \qquad
       K_{v}=\lim_{s\rightarrow 0} sG_{open}(s)
       \end{eqnarray*}
\item <4->静态加速度误差系数 
       \begin{eqnarray*}
       r(t)&=&\frac{1}{2}at^{2}\\
       e_{ss}&=&\frac{a}{K_{a}}, \qquad
       K_{a}=\lim_{s\rightarrow 0} s^{2}G_{open}(s)
       \end{eqnarray*}
\end{itemize}
\end{frame}
\begin{frame}
\frametitle{零型系统($\nu=0$)}
\label{sec-2-6}

\begin{itemize}
\item <2-> $r(t)=A$ 时:
     \begin{eqnarray*}
     K_p &=& \lim_{s\rightarrow 0}G_o(s) 
         = \lim_{s\rightarrow 0}\frac{K\prod_{i=0}^m(\tau_i s+1)}{\prod_{j=1}^n (\tau_j s+1)} 
         = K \\
     e_{ss1} &=& \frac{A}{1+K_p}
     \end{eqnarray*}
     \mode<article>{称为有差系统.}
\item <3-> $r(t)=vt$ 时:
     \begin{eqnarray*}
     K_v &=& \lim_{s\rightarrow 0}sG_o(s) 
         = 0 \\
     e_{ss2} &=& \infty 
     \end{eqnarray*}
\item <4> $r(t)=\frac{1}{2}at^2$ 时:
     \begin{eqnarray*}
     K_a &=& \lim_{s\rightarrow 0}s^2 G_o(s) 
         = 0 \\
     e_{ss3} &=& \infty
     \end{eqnarray*}
\end{itemize}
\end{frame}
\begin{frame}
\frametitle{I型系统($\nu=1$)}
\label{sec-2-7}

\begin{itemize}
\item <2->$r(t)=A$ 时:
     \begin{eqnarray*}
     K_p &=& \lim_{s\rightarrow 0}G_o(s) 
         = \lim_{s\rightarrow 0}\frac{K\prod_{i=0}^m(\tau_i s+1)}{s\prod_{j=1}^{n-1}(\tau_j s+1)} 
         = \infty \\
     e_{ss1} &=& \frac{1}{1+K_p}
           = 0
     \end{eqnarray*}
     \mode<article>{无差系统.}
\item <3->$r(t)=vt$ 时:
     \begin{eqnarray*}
     K_v &=& \lim_{s\rightarrow 0}sG_o(s) 
         = K \\
     e_{ss2} &=& \frac{v}{K_v} 
             =\frac{v}{K}
     \end{eqnarray*}
\item <4->$r(t)=\frac{1}{2}at^2$ 时:
     \begin{eqnarray*}
     K_a &=& \lim_{s\rightarrow 0}s^2 G_o(s) 
         = 0 \\
     e_{ss3} &=& \infty
     \end{eqnarray*}
\end{itemize}
\end{frame}
\begin{frame}
\frametitle{II型系统($\nu=2$)}
\label{sec-2-8}

\begin{eqnarray*}
K_p & = & \infty\\
e_{ss1} &=& 0 \\
K_v & = & \infty \\
e_{ss2} &=& 0 \\
K_a &=& K \\
e_{ss3} &=& \frac{a}{K}
\end{eqnarray*}
\end{frame}
\begin{frame}
\frametitle{小结:}
\label{sec-2-9}

\begin{itemize}
\item <2->零型:
     \[e_{ss1}=\frac{A}{1+K},e_{ss2}=e_{ss3}=\infty\]
\item <3->I型:
     \[e_{ss1}=0,e_{ss2}=\frac{v}{K},e_{ss3}=\infty\]
\item <4->II型:
     \[e_{ss1}=e_{ss2}=0,e_{ss3}=\frac{a}{K}\]
\end{itemize}
\end{frame}
\begin{frame}
\frametitle{例:}
\label{sec-2-10}


若 $G(s)H(s) =\frac{10K_h}{s+1},K_h\in\{0.1,1\}$ ,求单位阶跃下的 $e_{ss}$ .

\mode<article>{解:}
\begin{columns}
\begin{column}{0.45\textwidth}
\begin{block}<2->{解法1}
\label{sec-2-10-1}


零型系统, $r(t)=1,e_{ss}=\frac{1}{1+K_p}$

\begin{eqnarray*}
K_p &=  &\lim_{s\rightarrow 0}G(s)H(s) \\
    &=& 10K_h \\
    &=&
\begin{cases}
1  & K_h =0.1 \\
10 & K_h = 1
\end{cases}\\
e_{ss} &=&
\begin{cases}
0.5 & K_h=0.1 \\
\frac{1}{11} & K_h=1
\end{cases}
\end{eqnarray*}
\end{block}
\end{column}
\begin{column}{0.45\textwidth}
\begin{block}<3->{解法2:}
\label{sec-2-10-2}


\begin{eqnarray*}
e_{ss} &=& \lim_{s\rightarrow 0}s\Phi_e(s)R(s)\\
    &=&\lim_{s\rightarrow 0}s\frac{1}{1+G(s)H(s)}R(s)\\
    &=&\lim_{s\rightarrow 0}s\frac{s+1}{s+1+10K_h}\frac{1}{s}\\
    &=& \frac{1}{1+10K_h} \\
    &=&
\begin{cases}
0.5 & K_h=0.1 \\
\frac{1}{11} & K_h=1
\end{cases}
\end{eqnarray*}
\end{block}
\end{column}
\end{columns}
\end{frame}
\begin{frame}
\frametitle{例:  求 $r(t)=2+3t$ 时的 $e_{ss}$}
\label{sec-2-11}

\begin{psmatrix}[rowsep=0.4,colsep=0.5]
% 1    2   3  4   5   6    7       8   9
%           R*    E(s)    .------.
% R-->2/s+1-->o------>o-->| G(s) |--+--> C
%           _ ^     _ ^   '------'  |
%             |       |             |  
%             |       '----0.8s-----+
%             |                     |
%             '---------------------'
%        
%
% 1                        2                 3        
$R(s)$ &  \psframebox{$\frac{2}{s+1}$} & ${\hskip 1em}  $ & %
\pscirclebox[framesep=-0.2em]{$\times$} &$ $   & \pscirclebox[framesep=-0.2em]{$\times$} & %
\psframebox{$\frac{5}{s(5s+1)}$}   & \   & $C(s)$ \\
  &   &     &  & & & \psframebox{$ 0.8s $} &  \ &  \\
\\
%link
\ncline{->}{1,1}{1,2}
\ncline{->}{1,2}{1,4}
\naput{$R^{*}(s)$}
\ncline{->}{1,4}{1,6}
\naput{$E(s)$}
\ncline{->}{1,6}{1,7}
\ncline{->}{1,7}{1,9}
\ncline{2,8}{2,7}
\ncangle[angleA=180,angleB=-90,armA=0.5em,armB=1em]{->}{2,7}{1,6}
\naput[npos=2.3]{$-$}
\ncangles[angleA=180,angleB=-90,armA=0em,armB=4.5em]{->}{1,8}{1,4}
\naput[npos=3.6]{$-$}
\end{psmatrix}

解:

\begin{eqnarray*}
G(s) & = \frac{C(s)}{E(s)} \\
     &=& \frac{\frac{5}{s(5s+1)}}{1+\frac{4s}{s(5s+1)}} \\
    & = & \frac{5}{5s^2+5s} \\
    & = & \frac{1}{s(s+1)} 
\end{eqnarray*}
\end{frame}
\begin{frame}
\frametitle{例:计算稳态误差}
\label{sec-2-12}

\mode<article>{判断稳定性:}

\begin{eqnarray*}
\Phi(s) &=& \frac{C(s)}{R^{*}(s)}
        = \frac{1}{s(s+1)+1} \\
\Phi_e(s) &=& \frac{s(s+1)}{s(s+1)+1}
\end{eqnarray*}
系统稳定.

\begin{eqnarray*}
R(s) &=& \frac{2s+3}{s^2}\\
e_{ss} & = &\lim_{s\rightarrow 0}s\Phi_e(s)R^{*}(s) \\
       &=& \lim_{s\rightarrow 0}s\cdot\frac{s(s+1)}{s(s+1)+1}\cdot\frac{2}{s+1}\cdot\frac{2s+3}{s^2} \\
       &=& 6
\end{eqnarray*}
\end{frame}
\section{动态误差系数}
\label{sec-3}
\begin{frame}
\frametitle{动态误差系数}
\label{sec-3-1}

动态误差系数可描述系统稳态误差随时间变化的规律,静态误差可看作动态误差的一个特例.

\begin{eqnarray*}
E(s) & = & \Phi_e(s)R(s)\\
\Phi_e(s) &=& \frac{E(s)}{R(s)}\\
         &=&\frac{1}{1+G(s)H(s)} \\
         &=& \frac{M(s)}{N(s)} 
\end{eqnarray*}
\end{frame}
\begin{frame}
\frametitle{在 $s=0$ 处展开,得:}
\label{sec-3-2}


\begin{eqnarray*}
\phi_e(s)  &=& \Phi_e(0)+\dot{\Phi}_e(0)s+\cdots+\frac{\Phi_e^{(n)}(0)s^n}{n!}+\cdots \\
E(s) & = & \Phi_e(0)R(s)+\dot{\Phi}_e(0)sR(s)+\cdots+\frac{\Phi_e^{(n)}(0)s^nR(s)}{n!}+\cdots \\
e_{ss}(t) & = & \Phi_e(0)r(t)+\dot{\Phi}_e(0)\dot{r}(t)+\cdots+\frac{\Phi_e^{(n)}(0)r^{(n)}(t)}{n!}+\cdots \\
          &= & \sum_{i=1}^{\infty}C_ir^{(i)}(t) ,\qquad
C_i = \frac{\Phi_e^{(i)}(0)}{i!}
\end{eqnarray*}

\begin{itemize}
\item 其中 $C_i$ 称为动态误差系数.
\begin{itemize}
\item $C_0$ 动态位置误差系数
\item $C_1$ 动态速度误差系数
\item $C_2$ 动态加速度误差系数
\end{itemize}
\end{itemize}
\end{frame}
\begin{frame}
\frametitle{动态误差系数示例:}
\label{sec-3-3}

\begin{itemize}
\item <2-> 零型系统 $r(t)=1$ 则 \[e_{ss}(t)=C_0 ,C_0=\frac{1}{1+K_p}\]
\item <3-> I型系统 $r(t)=t$ 则 \[e_{ss}(t)=C_0 t+C_1,C_0=0,C_1=\frac{1}{K_v}\]
\item <4-> II型系统 $r(t)=t$ 则 \[e_{ss}(t)=C_0 \frac{1}{2}at^2+C_1at+C_2 a,C_0=C_1=0,C_2=\frac{1}{K_a}\]
\end{itemize}
\end{frame}
\begin{frame}
\frametitle{讨论: $C_i$ 的计算}
\label{sec-3-4}

\begin{eqnarray*}
\Phi_e(s) &=& \frac{M(s)}{N(s)} \\
 & = & C_0+C_1s+C_2s^2+\cdots
\end{eqnarray*}
\end{frame}
\begin{frame}
\frametitle{例: $G(s)H(s)=\frac{1}{s(s+1)}$}
\label{sec-3-5}


综合除法:

\[
\begin{matrix}
\text{divident}      &      &   \text{divisor}  &    & \text{quotient} &   & \text{remainder} \\  
s^2+s             &\div  &    s^2+s+1     & \rightarrow &  s       &  & s^2+s-s(1+s+s^2) \\
-s^3             &\div  &   s^2+s+1       & \rightarrow  & -s^3     &   & -s^3-(-s^3)(1+s+s^2)\\
 s^4+s^5         &\div  &  s^ 2+s+1       & \rightarrow & s^4  &   & \cdots  \\
\cdots           &\div   &  s^2+s+1       &\rightarrow  & \cdots   &    &\cdots   
\end{matrix}
\]

得:
\begin{eqnarray*}
\Phi_e(s)  &=& s-s^3+s^4+\cdots
\end{eqnarray*}
\end{frame}
\begin{frame}
\frametitle{例: $G(s)H(s)=\frac{1}{s(s+1)}$ 另一种写法:}
\label{sec-3-6}


\begin{eqnarray*}
\frac{s^2+s}{s^2+s+1} & = & s + \frac{-s^2+s-s(1+s+s^2)}{s^2+s+1}  \\
\frac{-s^3}{s^2+s+1}  & = & -s^3+\frac{-s^3-(-s^3)(1+s+s^2)}{s^2+s+1}\\
\frac{s^4+s^5}{s^2+s+1} &=& s^4 +\cdots \\
\cdots                  &=& \cdots \\
\Phi_e(s)               &=& s-s^3+s^4+\cdots
\end{eqnarray*}
\end{frame}
\begin{frame}
\frametitle{例: $G(s)H(s)=\frac{1}{s(s+1)}$ 长除法}
\label{sec-3-7}

\[ 
\begin{array}{ccccccccc} 
 & & s &  & -s^3 & +s^4 &  &\cdots\\
\cline{2-8}
\multicolumn{1}{r|}{1+s+s^2} & & s & +s^2 \\
                             & & s  & +s^2 & +s^3 \\
\cline{3-5}
                             & &   &     & -s^3 \\
                             & &   &     & -s^3  & -s^4 & -s^5\\
\cline{5-7}
                             & &   &     &       &  s^4 &  +s^5\\
                             & &   &     &       &  s^4 &  +s^5  & +s^6\\
\cline{6-8}
                             & &   &     &       &      &       & -s^6\\
                             & &   &     &       &      &       & \cdots
\end{array} 
\]
\end{frame}
\begin{frame}
\frametitle{例:}
\label{sec-3-8}

单位负反馈系统开环传递函数: $G_o(s)=\frac{100}{s(0.s1+1)}$ 求输入信号为 $\sin(5 t)$ 时的稳态误差.

解:系统稳定,
\begin{eqnarray*}
r(t) &=& sin(\omega t),\omega=5 \\
E(s) &=& \Phi_e(s)R(s) \\
e_{ss}&=& \sum_{i=0}^{\infty}C_i r^{(i)} \\
\Phi_e(s)&=& \frac{1}{1+G_o(s)} \\
         &=& \frac{0.1s^2+s}{0.1s^2+s+100} 
\end{eqnarray*}
\end{frame}
\begin{frame}
\frametitle{解法1}
\label{sec-3-9}





\begin{itemize}
\item $\frac{0.1s^2+s}{0.1s^2+s+100}  =0.01s+\frac{0.1s^2+s-0.01s(0.1s^2+s+100)}{0.1s^2+s+100}$
\item $\frac{-10^{-3}s^3+0.09s^2}{0.1s^2+s+100} = 9\times 10^{-4}s^2+\frac{-10^{-3}s^3+0.09s^2-9\times 10^{-4}s^2(0.1s^2+s+100)}{0.1s^2+s+100}$
\item $\frac{-9\times 10^{-5}s^4-1.9\times 10^{-3}s^3}{0.1s^2+s+100}  = -1.9\times 10^{-5}s^3 + \cdots$
\end{itemize}
所以
\begin{itemize}
\item $\Phi_e(s) = 0+0.01s+9\times 10^{-4}s^2-1.9\times 10^{-5}s^3+\cdots$
\item $e_{ss}(t) = (C_0-C_2\omega^2+C_4\omega^4+\cdots)\sin(\omega t)+(C_1-C_3\omega^3+C_5\omega^5+\cdots)\cos(\omega t)$
\item $e_{ss}(t) = -0.055\cos(5t-249^{\circ})$
\end{itemize}
\end{frame}
\begin{frame}
\frametitle{解法2:}
\label{sec-3-10}

\begin{eqnarray*}
E(s) & = & \Phi_e(s)R(s) \\
     &=& \frac{s^2+10S}{s^2+10S+1000}\cdot\frac{5}{s^2+25}\\
     &=&\frac{-0.0498s-0.1115}{s^2+25}+\frac{as+b}{s^2+10s+1000}\\
e_{ss}(t)&=& -0.055\cos(5t-249^{\circ})+\Delta 
\end{eqnarray*}
其中: $\lim_{t\rightarrow\infty}\Delta = 0$
\end{frame}
\section{减小稳态误差的措施}
\label{sec-4}
\begin{frame}
\frametitle{减小 $e_{ss}$ 的措施}
\label{sec-4-1}

\begin{itemize}
\item <2->增大开环增益
\item <3->提高系统类型
\item <4->串级控制抑制扰动
\item <5->复合控制
\end{itemize}
\end{frame}
\begin{frame}
\frametitle{增大开环增益}
\label{sec-4-2}

\begin{align*}
G(s)    & = \frac{K\prod_{i=1}^{m}(\tau_{i}s+1)}{s^{\nu}\prod_{j=1}^{n-\nu}(T_{j}s+1)} \\
e_{ss}  &=
\begin{cases}
\frac{1}{1+K} & \nu=0,R(s)=\frac{1}{s} \\
\frac{1}{K} & \nu=1,R(s)=\frac{1}{s^2} \\
\frac{1}{K} & \nu=2,R(s)=\frac{1}{s^3} 
\end{cases}
\end{align*}
\end{frame}
\begin{frame}
\frametitle{提高系统类型}
\label{sec-4-3}

\begin{align*}
G(s)    & = \frac{1}{s}\frac{K\prod_{i=1}^{m}(\tau_{i}s+1)}{s^{\nu}\prod_{j=1}^{n-\nu}(T_{j}s+1)} \\
e_{ss}  &=
\begin{cases}
\frac{1}{K} & \nu=0,R(s)=\frac{1}{s^2} \\
\frac{1}{K} & \nu=1,R(s)=\frac{1}{s^3} 
\end{cases}
\end{align*}
\end{frame}
\begin{frame}
\frametitle{串级控制}
\label{sec-4-4}

\begin{psmatrix}[rowsep=0.4,colsep=0.5]
%          1    2  3     4    5    6    7 
%                            | N(s)
%               E(s)         v +  
%         R---->o-o-> G_1(s)--o- G_2(s)---G_3-+--> C
%             _ ^  \hspace{9ex}H(s)________/     |
%               |                             |  
%               '-----------------------------+
%           
%
       &         &            &         &                        & $N(s)$    \\  
$R(s)$ & $\circ$ & \hskip 1em & $\circ$ &  \psframebox{$G_1(s)$} & $\circ$   &  \psframebox{$G_2(s)$} & \  & \psframebox{$G_3(s)$} & \  & $C(s)$ \\
       &         &            &         &                        &   \psframebox{$ H(s) $} 
%link
\ncline{->}{1,6}{2,6}
%naput{$N(s)$}
\ncline{->}{2,1}{2,2}
\ncline{->}{2,2}{2,4}
\naput{$E(s)$}
\ncline{->}{2,4}{2,5}
\ncline{->}{2,5}{2,6}
\ncline{->}{2,6}{2,7}
\ncline{->}{2,7}{2,9}
\ncline{->}{2,9}{2,11}
\ncangle[angleA=-90,angleB=0,armA=0em,armB=0em]{->}{2,8}{3,6}
\ncangle[angleA=180,angleB=-90,armA=0em,armB=0em]{->}{3,6}{2,4}
\naput[npos=1.6]{$-$}
\ncangles[angleA=-90,angleB=-90,armA=4em,armB=0em]{->}{2,10}{2,2}
\naput[npos=2.6]{$-$}
\end{psmatrix}


\begin{align*}
C(s) &=(G_1(s)E'(s)+N(s))G_2(s)G_3(s) \\
E'(s) &=E(s)-\frac{H(s)}{G_3(s)}C(s)\\
C(s) &=\frac{(G_1(s)E(s)+N(s))G_2(s)G_3(s)}{1+G_1(s)G_2(s)H(s)}\\
C(s) &\approx\frac{G_3(s)E(s)}{H(s)} \qquad (G_1(s)>>1) \\
\end{align*}
\end{frame}
\begin{frame}
\frametitle{复合控制}
\label{sec-4-5}

\begin{psmatrix}[rowsep=0.4,colsep=0.5]
%   .-e/(1-e)-.         
%   |  E(s)   |   .------.
% R-->o-------o ->| G(s) |--+--> C
%   _ ^           '------'  |
%     |                     |  
%     '---------------------'
%
%
% 1          2         3          4                  5          6        7
       &          & \psframebox{$\frac{\epsilon}{1-\epsilon}$} \\
$R(s)$ &  $\circ$ & \hskip 1em & $\circ$  &  \psframebox{$G(s)$} & \  & $C(s)$ 
%link
\ncline{->}{2,1}{2,2}
\ncline{->}{2,2}{2,4}
\naput[labelsep=0pt]{$E(s)$}
\ncline{->}{2,4}{2,5}
\ncline{->}{2,5}{2,7}
\ncangles[angleA=0,angleB=180,armA=0.2em,armB=0.5em]{->}{2,1}{1,3}
\ncangle[angleA=0,angleB=90,armA=0em,armB=0.5em]{->}{1,3}{2,4}
\ncangles[angleA=-90,angleB=-90,armA=2em,armB=0em]{->}{2,6}{2,2}
\naput[npos=2.3]{$-$}
\end{psmatrix}


\begin{eqnarray*}
\epsilon &=& \frac{r(\infty)-c(\infty)}{r(\infty)}\\
c(\infty)&=& (1-\epsilon)r(\infty)\\
r(t)&=&\frac{r'(t)}{1-\epsilon}\\
c(\infty) &=& r'(\infty) \\
e'_{ss}&=&r'(\infty)-c(\infty)\\
 &=& 0
\end{eqnarray*}
\end{frame}

\end{document}
