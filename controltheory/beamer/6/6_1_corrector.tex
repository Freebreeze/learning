% Created 2014-11-25 星期二 10:42
\documentclass[table]{beamer}
\usepackage{fixltx2e}
\usepackage{graphicx}
\usepackage{longtable}
\usepackage{float}
\usepackage{wrapfig}
\usepackage{soul}
\usepackage{textcomp}
\usepackage{marvosym}
\usepackage{wasysym}
\usepackage{latexsym}
\usepackage{amssymb}
\usepackage{hyperref}
\tolerance=1000
\usepackage{etex}
\usepackage{amsmath}
\usepackage{pstricks}
\usepackage{pgfplots}
\pgfplotsset{compat=1.8}
\usepackage{tikz}
\usepackage[europeanresistors,americaninductors]{circuitikz}
\usepackage{colortbl}
\usepackage{yfonts}
\usetikzlibrary{shapes,arrows}
\usetikzlibrary{positioning}
\usetikzlibrary{arrows,shapes}
\usetikzlibrary{intersections}
\usetikzlibrary{calc,patterns,decorations.pathmorphing,decorations.markings}
\usepackage[BoldFont,SlantFont,CJKchecksingle]{xeCJK}
\setCJKmainfont[BoldFont=Evermore Hei]{Evermore Kai}
\setCJKmonofont{Evermore Kai}
\usepackage{pst-node}
\usepackage{pst-plot}
\psset{unit=5mm}
\mode<beamer>{\usetheme{Frankfurt}}
\mode<beamer>{\usecolortheme{dove}}
\mode<article>{\hypersetup{colorlinks=true,pdfborder={0 0 0}}}
\mode<beamer>{\AtBeginSection[]{\begin{frame}<beamer>\frametitle{Topic}\tableofcontents[currentsection]\end{frame}}}
\setbeamercovered{transparent}
\subtitle{系统设计与校正}
\providecommand{\alert}[1]{\textbf{#1}}

\title{线性系统校正方法}
\author{}
\date{}
\hypersetup{
  pdfkeywords={},
  pdfsubject={},
  pdfcreator={Emacs Org-mode version 7.9.3f}}

\begin{document}

\maketitle

\begin{frame}
\frametitle{Outline}
\setcounter{tocdepth}{3}
\tableofcontents
\end{frame}












\section{校正装置的目的}
\label{sec-1}
\begin{frame}
\frametitle{校正装置的目的}
\label{sec-1-1}

\begin{tikzpicture}[node distance=2em,auto,>=latex', thick]
% \path[use as bounding box] (-1,0) rectangle (10,-2); 
\path[->] node[] (r) {$R(s)$}; 
\path[->] node[ circle,inner sep=2pt,minimum size=1pt,draw,label=below left:$   $ ,right =of r] (p1) {}; 
\path[->](r) edge node {} (p1) ; 
\path[blue,->] node[draw, right =of p1] (gc) {$G_{c}(s)$}; 
\path[blue,->] node[draw, above =of gc] (gr) {$G_{r}(s)$}; 
\path[->] (p1) edge node {} (gc) ; 
\path[ draw] (r.east)+(1em,0)  |-  (gr); 
\path[->] node[ circle,inner sep=2pt,minimum size=1pt,draw,label=below left:$   $ ,right =of gc] (p2) {}; 
\path[->] (gc) edge node {} (p2) ; 
\path[->,draw] (gr) -| (p2) ; 
\path[red,->] node[draw, inner sep=5pt,right =of p2] (g) {$G(s)$}; 
\path[->] (p2) edge node {} (g); 
\path[->] node[ right =of g] (o) {$C(s)$}; 
\path[->] (g) edge node {} (o); 
\path[blue,->] node[draw, inner sep=5pt,below =of g] (gf) {$G_f(s)$}; 
\path[ draw] (g.east)+(1em,0)  |-  (gf); 
\path[->, draw] (gf.west) -| node[very near end] {$-$} (p2); 
\path[->, draw] (g.east)+(1em,0) -- +(1em,-7em) -| node[very near end] {$-$} (p1); 
\end{tikzpicture} 

\begin{itemize}
\item <2->改善系统稳定性
\item <2->改善系统的稳态性能
\item <2->改善系统的动态品质
\end{itemize}
\end{frame}
\section{设计指标的转换}
\label{sec-2}
\begin{frame}
\frametitle{设计指标的转换}
\label{sec-2-1}

\begin{itemize}
\item <2->二阶系统的时域与频域指标有明确的转换公式
\item <3->高阶系统的时域与频域指标有近似转换公式
\item <4->校正后系统的要求
\begin{itemize}
\item <4->低频段:积分环节, $K$ 尽量大,以减小稳态误差
\item <5->中频段:以斜率 $-20dB/dec$ 穿越 $0dB$ 线,使 $\omega_c$ 足够大,提高动态性能
\item <6->高频段:抗干扰要求,增益下降要快
\end{itemize}
\end{itemize}
\end{frame}

\end{document}
