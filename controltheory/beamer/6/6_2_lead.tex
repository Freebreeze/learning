% Created 2014-11-09 星期日 22:23
\documentclass[table]{beamer}
\usepackage{fixltx2e}
\usepackage{graphicx}
\usepackage{longtable}
\usepackage{float}
\usepackage{wrapfig}
\usepackage{soul}
\usepackage{textcomp}
\usepackage{marvosym}
\usepackage{wasysym}
\usepackage{latexsym}
\usepackage{amssymb}
\usepackage{hyperref}
\tolerance=1000
\usepackage{etex}
\usepackage{amsmath}
\usepackage{pstricks}
\usepackage{pgfplots}
\pgfplotsset{compat=1.8}
\usepackage{tikz}
\usepackage[europeanresistors,americaninductors]{circuitikz}
\usepackage{colortbl}
\usepackage{yfonts}
\usetikzlibrary{shapes,arrows}
\usetikzlibrary{positioning}
\usetikzlibrary{arrows,shapes}
\usetikzlibrary{intersections}
\usetikzlibrary{calc,patterns,decorations.pathmorphing,decorations.markings}
\usepackage[BoldFont,SlantFont,CJKchecksingle]{xeCJK}
\setCJKmainfont[BoldFont=Evermore Hei]{Evermore Kai}
\setCJKmonofont{Evermore Kai}
\usepackage{pst-node}
\usepackage{pst-plot}
\psset{unit=5mm}
\mode<beamer>{\usetheme{Frankfurt}}
\mode<beamer>{\usecolortheme{dove}}
\mode<article>{\hypersetup{colorlinks=true,pdfborder={0 0 0}}}
\mode<beamer>{\AtBeginSection[]{\begin{frame}<beamer>\frametitle{Topic}\tableofcontents[currentsection]\end{frame}}}
\setbeamercovered{transparent}
\subtitle{串联超前校正}
\providecommand{\alert}[1]{\textbf{#1}}

\title{线性系统校正方法}
\author{}
\date{}
\hypersetup{
  pdfkeywords={},
  pdfsubject={},
  pdfcreator={Emacs Org-mode version 7.9.3f}}

\begin{document}

\maketitle

\begin{frame}
\frametitle{Outline}
\setcounter{tocdepth}{3}
\tableofcontents
\end{frame}












\section{串联超前校正原理与方法}
\label{sec-1}
\begin{frame}
\frametitle{串联超前校正原理}
\label{sec-1-1}

利用超前校正网络的相角超前特性来提升系统的相角裕度 $\gamma$ 
\begin{eqnarray*}
G_c(s) &=& \frac{1+aTs}{1+Ts} ,a>1 \\
\phi_c(\omega) & = & \arctan(aT\omega)-\arctan(T\omega) 
\end{eqnarray*}
\begin{columns}
\begin{column}{0.5\textwidth}
\begin{block}<2->{校正网络Bode图}
\label{sec-1-1-1}

\begin{tikzpicture}[ yscale=0.55,yscale=0.7]
\draw[->] (-1,0) -- (3.5,0);
\draw[->] (0,-0.5) -- (0,2.1);
\draw (0,1.5) node[above left] {$L_c(\omega)$};
\draw [red,thick] plot coordinates {(0,0) (1,0) (2,1)  (3,1)};
\draw (1,0) node[below] {$\frac{1}{aT}$};
\draw (2,0) node[below] {$\frac{1}{T}$};
\draw[pink] (1.5,0) -- +(0,0.5);

\begin{scope}[shift={(0,-3)}]
\draw[->] (-1,0) -- (3.5,0);
\draw[->] (0,-0.5) -- (0,2.1);
\draw (0,1.5) node[above left] {$\phi_c(\omega)$};
\draw [red,thick] plot [smooth] coordinates {(0,0) (0.5,0.1) (1,0.45) (1.5,1) (2,0.45) (2.5,0.1) (3,0)};
\draw (1,0) node[below] {$\frac{1}{aT}$};
\draw (2,0) node[below] {$\frac{1}{T}$};
\draw[pink] (1.5,0) -- +(0,1);
\end{scope}
\end{tikzpicture}
\end{block}
\end{column}
\begin{column}{0.5\textwidth}
\begin{block}<3->{超前校正示意图:}
\label{sec-1-1-2}

\begin{tikzpicture}[scale=0.7]
\coordinate (o) at (0,0);
\coordinate (ox) at (4.5,0);
\draw[->] (o) -- (ox);
\draw[->] (0,-1.1) -- (0,1.5);
\draw (0,0.5) node[above left] {$L(\omega)$};
\draw (o) node[below left] {$o$};
%\draw [red,thick] plot coordinates {(0,0) (1,0) (2,-1)  (3,-1)};
\coordinate (s) at (1,1);
\coordinate (s1) at ($(s) + (160:1)$);
\coordinate (s2) at ($(s)+(-40:4.5)$);
\coordinate (w1) at (intersection of s--s2 and o--ox);
\coordinate (w2) at ($(w1)+(0.5,0)$);
\coordinate (w2down) at ($(w2)+(0,-5)$);
\coordinate (w2v) at (intersection of w2--w2down and s--s2);
\coordinate (w2e) at ($(w2)+(160:5)$);
\coordinate (w2left) at (intersection of w2--w2e and s--s2);
\coordinate (w2right) at ($2*(w2)-(w2left)$);
\draw[red] (s1)--(s)--(s2);
\draw[blue] (w2left)--(w2right)--+(-40:1);
\draw[purple] (w2)--(w2v);
\draw (w1) node[below left] {$\omega_c'$};
\draw (w2) node[above right] {$\omega_c''$};

\begin{scope}[shift={(0,-2.3)}]
\draw[->] (o) -- (ox);
\draw[->] (0,-1.1) -- (0,0.5);
\draw (0,0.2) node[above left] {$\phi(\omega)$};
\draw [red,thick] plot [smooth] coordinates {(0,0) (0.3,-0.1) (0.6,-0.25) (1.1,-0.5) (1.5,-0.65) (2,-0.7) (3.7,-0.9)};
\draw [blue,thick] plot [smooth] coordinates {(1.1,-0.5) (1.5,-0.6) (2.7,-0.6) (3.7,-0.9)};
\draw[dashed,red] (0,-1) -- (4.5,-1);
\draw (0,-1) node[left] {$-180^\circ$};
\end{scope}
\end{tikzpicture}
\end{block}
\end{column}
\end{columns}
\end{frame}
\begin{frame}
\frametitle{超前校正网络分析}
\label{sec-1-2}
\begin{columns}
\begin{column}{0.5\textwidth}
\begin{block}<2->{最大超前角:}
\label{sec-1-2-1}

\begin{eqnarray*}
\frac{d\phi_c(\omega)}{d\omega} & = & 0 \\
\omega_m &=& \frac{1}{T\sqrt{a}}\\
\phi_c(\omega_m) &=& 2\arctan\sqrt{a}-\frac{\pi}{2}\\
                 &=& \arcsin\frac{a-1}{a+1}
\end{eqnarray*}
工程中一般 $\phi_m\leq 60^{\circ},a<20$ 
\end{block}
\end{column}
\begin{column}{0.5\textwidth}
\begin{block}<3->{幅值}
\label{sec-1-2-2}

\begin{eqnarray*}
\lg\omega_m &=& \frac{1}{2}(\lg\frac{1}{T}+\lg\frac{1}{aT}) \\
L_c(\omega) &=& 20\lg\omega_m-20\lg\frac{1}{aT} \\
            &=& 10\lg a
\end{eqnarray*}
\begin{itemize}
\item <4-> $\omega_m$ 位于 $\frac{1}{aT}$ 与 $\frac{1}{T}$ 的几何中心
\item <5-> 超前网絡设计使得设计后的 $\omega_c\approx\omega_m$ ,以获得最大相角提升
\end{itemize}
\end{block}
\end{column}
\end{columns}
\end{frame}
\begin{frame}
\frametitle{适用范围}
\label{sec-1-3}

\begin{itemize}
\item 超前校正适用范围:
\begin{itemize}
\item <2->原系统稳定, $\gamma$ 不够
\item <3->希望的截止频率比原系统大,主要用于提高系统的瞬态性能(动态品质)
\end{itemize}
\end{itemize}
\end{frame}
\begin{frame}
\frametitle{设计步聚(设计参数 $K,a,T$ )}
\label{sec-1-4}

\begin{enumerate}
\item 根据 $e_{ss}$ 的指标要求,确定开环增益 $K$
\item 计算未校正系统的 $\omega_c',\gamma'$
\item 转换时域指标到频域指标,得到希望的 $\omega_c'',\gamma''$
\item <2->设计超前校正网络的参数 $a,T$
\begin{itemize}
\item <2->确定 $a$  :  
         \[L(\omega_c'')=-10\lg a\]
\item <2->确定 $T$  :  
        \[\omega_c'' = \omega_m = \frac{1}{T\sqrt{a}}\]
\end{itemize}
\item <3->检验校正后系统的 $\omega_c,\gamma$ .
\end{enumerate}
\end{frame}
\begin{frame}
\frametitle{设计步聚(设计参数 $K,a,T$ )}
\label{sec-1-5}

\begin{enumerate}
\item 根据 $e_{ss}$ 的指标要求,确定开环增益 $K$
\item 计算未校正系统的 $\omega_c',\gamma'$
\item 转换时域指标到频域指标,得到希望的 $\gamma''$
\item <2->设计超前校正网络的参数 $a,T$
\begin{itemize}
\item <2->确定 $a$  : 
       \begin{align*}
        \phi_m &=\gamma''-\gamma'+\epsilon \\
        a &=\frac{1 +\sin\phi_m}{1-\sin\phi_m}\\
       \end{align*}
\item <2->确定 $T$  :  
       \begin{align*}
        L(\omega_c'') &=-10\lg a \\
        \omega_c'' &= \omega_m = \frac{1}{T\sqrt{a}}
       \end{align*}
\end{itemize}
\item <3->检验校正后系统的 $\gamma$ .
\end{enumerate}
\end{frame}
\section{超前校正示例}
\label{sec-2}
\begin{frame}
\frametitle{超前校正示例1}
\label{sec-2-1}

单位负反馈开环传递函数  $G(s)=\frac{200}{s(0.1s+1)}$ ,设计校正网絡,使已校正系统相角裕度  $\gamma''\geq 45^{\circ}$ , 截止频率  $\omega_c''\geq 50 rad/s$  .

\begin{itemize}
\item <2->解:
      \begin{eqnarray*}
       L(\omega) & = &\begin{cases} 20\lg\frac{200}{\omega} & \omega < 10 \\
      20\lg\frac{200}{0.1\omega^2} & \omega\geq 10
      \end{cases} \\
      \omega_c' &=& 44.7 \\
        &<& \omega_c'' \\
      \gamma' &=& 12.6^{\circ} \\
       &<& \gamma''
      \end{eqnarray*}
      选择超前校正网絡.
\end{itemize}
\end{frame}
\begin{frame}
\frametitle{超前校正示例1(续)求 $a$}
\label{sec-2-2}

\begin{itemize}
\item 先根据 $\phi_m$ 尝试确定参数  $a$ 
      \begin{eqnarray*}
      \phi_m & = &\gamma''-\gamma'+\epsilon \\
      \epsilon &=& 10^{\circ} \\
      \phi_{m} &=& 45^{\circ}-12.6^{\circ}+10^{\circ} \\
       &=& 42.4^{\circ} \\
      a &=& \frac{1+\sin\phi_m}{1-\sin\phi_m} \\
       &=& 5.1
      \end{eqnarray*}
      其中  $\epsilon$  是因为估算有误差而留的余量.
\end{itemize}
\end{frame}
\begin{frame}
\frametitle{超前校正示例1(续)求 $T$}
\label{sec-2-3}

\begin{itemize}
\item 求解  $T$
     \begin{eqnarray*}
     L(\omega_c'') +10\lg a &= & 0 \\
     20\lg\frac{2000}{(\omega_c'')^2}+20\lg\sqrt{a} &=& 0 \\
     \frac{2000\sqrt{5}}{(\omega_c'')^2} &=& 1 \\
     \omega_c'' &=& 66.9 \\
     \omega_m &=& \frac{1}{T\sqrt{a}} \\
     \omega_c'' &=& \omega_m \\
     T &=& 0.0066
     \end{eqnarray*}
\item <2->计算此时的相角裕度: 
     \begin{eqnarray*}
     \gamma'' &=& 180^{\circ}+42.4^{\circ}-90^{\circ}-\arctan(0.1\omega_c'') \\
      &=& 50.9^{\circ}
     \end{eqnarray*}
\item <2->满足要求的超前校正网絡为  $G(s)=\frac{1+0.034s}{1+0.0066s}$
\end{itemize}
\end{frame}
\begin{frame}
\frametitle{超前校正示例2}
\label{sec-2-4}

 $G(s)=\frac{K}{s(s+1)}$ ,设计校正网絡使  $\gamma''\geq 45^{\circ}$  ,单位斜坡作用下  $e_{ss} \leq \frac{1}{15}$ 

\begin{itemize}
\item <2->解: 根据稳态误差要求,得:
     \begin{eqnarray*}
     e_{ss} & = & \frac{1}{K} 
            \leq  \frac{1}{15} \\
     K & \geq & 15
     \end{eqnarray*}
     取  $K=15$  .
     \begin{eqnarray*}
     L(\omega) & = & \begin{cases} 20\lg\frac{15}{\omega}  & \omega< 1 \\
                                  20\lg\frac{15}{\omega^2} & \omega\geq 1   \end{cases}\\
     \omega_c' &=& 3.9 \\
     \gamma'  &=& 14.5^{\circ} 
      < 45^{\circ}
     \end{eqnarray*}
     选择超前校正网絡.
\end{itemize}
\end{frame}
\begin{frame}
\frametitle{超前校正示例2(续)  求 $a$}
\label{sec-2-5}

\begin{itemize}
\item 先根据  $\phi_m$  确定  $a$  
     \begin{eqnarray*}
     \phi_m & = &\gamma''-\gamma'+\epsilon \\
     \epsilon &=& 10^{\circ} \\
     \phi_m &=& 40.5^{\circ} \\
     a &=& \frac{1+\sin\phi_{m}}{1-\sin\phi_m} \\
      &=& 4.7 
     \end{eqnarray*}
     其中  $\epsilon$  是因为估算有误差而留的余量.
\end{itemize}
\end{frame}
\begin{frame}
\frametitle{超前校正示例2(续)  求 $T$}
\label{sec-2-6}

\begin{itemize}
\item 然后求解  $T$ 
     \begin{eqnarray*}
     L(\omega_c'') +10\lg a &= & 0 \\
     20\lg\frac{15}{(\omega_c'')^2}+20\lg\sqrt{a} &=& 0 \\
     \frac{15\sqrt{4.7}}{(\omega_c'')^2} &=& 1 \\
     \omega_c'' &=& 5.7 \\
     \omega_m &=& \frac{1}{T\sqrt{a}} \\
     \omega_c'' &=& \omega_m \\
     T &=& 0.08
     \end{eqnarray*}
\item <2-> 计算此时的相角裕度: 
     \begin{eqnarray*}
     \gamma'' &=& 180^{\circ}+40.5^{\circ}-90^{\circ}-\arctan(\omega_c'') \\
      &=& 50.5^{\circ}
     \end{eqnarray*}
\item <2-> 满足要求的超前校正网絡为  $G(s)=\frac{1+0.38s}{1+0.08s}$
\end{itemize}
\end{frame}
\begin{frame}
\frametitle{超前校正示例3}
\label{sec-2-7}

 单位负反馈  $G(s)=\frac{K}{s(0.05s+1)(0.2s+1)}$ ,设计超前校正网絡使  $K_v\geq 5,\sigma\%\leq 25\%, t_s\leq 1s$ 

\begin{itemize}
\item <2-> 解: 由性能指标知:
      \begin{eqnarray*}
      K & = &5 \\
      \sigma\% &=& 0.16+0.4(M_r-1) \\
      0.25 &=& 0.16+0.4(M_r-1) \\
      M_r &=& 1.225 \\
      t_s &=&\frac{K_0\pi}{\omega_c''} \\
      K_0 &=& 2+1.5(M_r-1)+2.5(M_r-1)^2 
          = 2.5 \\
      \omega_c'' &=& 7.7 \\
      \gamma'' &=&\arcsin\frac{1}{M_r}
         = 55^{\circ}
      \end{eqnarray*}
\end{itemize}
\end{frame}
\begin{frame}
\frametitle{超前校正示例3(续)频率特性分析}
\label{sec-2-8}


\begin{eqnarray*}
L(\omega) & = & \begin{cases} 20\lg\frac{5}{\omega}  & \omega< 5 \\
                             20\lg\frac{5}{0.2\omega^2} & 5\leq\omega< 20  \\
                             20\lg\frac{5}{0.01\omega^3} & \omega \geq 20 \end{cases}\\
\omega_c' &=& 5 \\
\gamma'  &=& 180^{\circ}-90^{\circ}-\arctan0.2\omega_c' -\arctan0.05\omega_c'\\
  &=& 31.0^{\circ}
\end{eqnarray*}

选择超前校正网絡.
\end{frame}
\begin{frame}
\frametitle{超前校正示例3(续):计算  $a$}
\label{sec-2-9}


\begin{eqnarray*}
\omega_c'' &=& 7.7 \\
L(\omega_c'') +10\lg a &= & 0 \\
20\lg\frac{5}{0.2(\omega_c'')^2}+20\lg\sqrt{a} &=& 0 \\
\frac{5\sqrt{a}}{0.2(\omega_c'')^2} &=& 1 \\
a &=& 5.6 \\
\end{eqnarray*}
\end{frame}
\begin{frame}
\frametitle{超前校正示例3(续):根据截止频率计算  $T$}
\label{sec-2-10}

\begin{itemize}
\item 计算 $T$
      \begin{eqnarray*}
      \omega_m &=& \frac{1}{T\sqrt{a}} \\
      \omega_c'' &=& \omega_m \\
      T &=& 0.055 \\
      \phi_m &=& \arcsin\frac{a-1}{a+1} \\
      &=& 44^{\circ} 
      \end{eqnarray*}
\item <2->计算此时的相角裕度: 
      \begin{eqnarray*}
      \gamma'' &=& 180^{\circ}+44^{\circ}-90^{\circ}-\arctan(0.05\omega_c'')-\arctan(0.2\omega_c'') \\
       &=& 56^{\circ}
      \end{eqnarray*}
\item <2->满足要求的超前校正网絡为  $G(s)=\frac{1+0.3s}{1+0.055s}$
\end{itemize}
\end{frame}

\end{document}
