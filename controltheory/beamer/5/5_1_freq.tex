% Created 2013-10-25 Fri 12:42
\documentclass[table]{beamer}
\usepackage[T1]{fontenc}
\usepackage{fixltx2e}
\usepackage{graphicx}
\usepackage{longtable}
\usepackage{float}
\usepackage{wrapfig}
\usepackage{soul}
\usepackage{textcomp}
\usepackage{marvosym}
\usepackage{wasysym}
\usepackage{latexsym}
\usepackage{amssymb}
\usepackage{hyperref}
\tolerance=1000
\usepackage{amsmath}
\usepackage[usenames]{color}
\usepackage{pstricks}
\usepackage{pgfplots}
\pgfplotsset{compat=1.8}
\usepackage{tikz}
\usepackage[europeanresistors,americaninductors]{circuitikz}
\usepackage{colortbl}
\usepackage{yfonts}
\usetikzlibrary{shapes,arrows}
\usetikzlibrary{positioning}
\usetikzlibrary{arrows,shapes}
\usetikzlibrary{intersections}
\usetikzlibrary{calc,patterns,decorations.pathmorphing,decorations.markings}
\usepackage[BoldFont,SlantFont,CJKchecksingle]{xeCJK}
\setCJKmainfont[BoldFont=Evermore Hei]{Evermore Kai}
\setCJKmonofont{Evermore Kai}
\xeCJKsetup{CJKglue=\hspace{0pt plus .08 \baselineskip }}
\usepackage{pst-node}
\usepackage{pst-plot}
\psset{unit=5mm}
\usepackage{beamerarticle}
\mode<beamer>{\usetheme{Frankfurt}}
\mode<beamer>{\usecolortheme{dove}}
\mode<article>{\hypersetup{colorlinks=true,pdfborder={0 0 0}}}
\mode<beamer>{\AtBeginSection[]{\begin{frame}<beamer>\frametitle{Topic}\tableofcontents[currentsection]\end{frame}}}
\setbeamercovered{transparent}
\subtitle{频率分析介绍}
\providecommand{\alert}[1]{\textbf{#1}}

\title{线性系统的频域分析法}
\author{}
\date{}
\hypersetup{
  pdfkeywords={},
  pdfsubject={},
  pdfcreator={Emacs Org-mode version 7.9.3f}}

\begin{document}

\maketitle

\begin{frame}
\frametitle{Outline}
\setcounter{tocdepth}{3}
\tableofcontents
\end{frame}













\section{频率法基本概念}
\label{sec-1}
\begin{frame}
\frametitle{频域法特点:}
\label{sec-1-1}

\begin{enumerate}
\item <2->工程使用广泛,有自己一套指标体系
\item <3->时域与频域指标可由经验公式相互转化
\item <4->根据Nyquist判据可由系统的开环频率特性判断闭环系统的稳定性,频率特性可由实验测定
\item <5->频率特性还可适用于典型非线性环节系统
\item <6->可以方便地设计出各种滤波器
\end{enumerate}
\end{frame}
\begin{frame}
\frametitle{频率特性基本概念}
\label{sec-1-2}
\begin{columns}
\begin{column}{0.5\textwidth}
\begin{block}<2->{RC网絡:}
\label{sec-1-2-1}


\begin{circuitikz}[american voltages,scale=0.7]
%       o---R --+-------o
%               |
%      U_r   C ===      U_c
%               |
%       o-------+-------o
\draw
  % rotor circuit
  (0,0) to  [short, o-o] (5,0)
  to [open, v^>=$U_c$ ,-o](5,3)
  to [short] (3,3)
  to [C, l_=$C$] (3,0)

  (0,0) to [open, v>=$U_r$ ,-o] (0,3)
  to [R,l=$R$] (3,3);
\end{circuitikz}

\begin{eqnarray*}
U_r &=& U_c + RC\dot{U}_c \\
U_r(s) &=& U_c(s) + RsU_c(s) 
\end{eqnarray*}
\end{block}
\end{column}
\begin{column}{0.5\textwidth}
\begin{block}<3->{传递函数:}
\label{sec-1-2-2}

\begin{eqnarray*}
%U_r &=& U_c + RC\dot{U}_c \\
%U_r(s) &=& U_c(s) + RsU_c(s) \\
G(s) &=& \frac{U_c(s)}{U_r(s)} \\
   &=&\frac{1}{1+RCs} \\
  &=& \frac{1}{1+Ts} 
\end{eqnarray*}
其中, $T=RC$ ,
\end{block}
\end{column}
\end{columns}
\end{frame}
\begin{frame}
\frametitle{频率特性基本概念(续)}
\label{sec-1-3}

\begin{itemize}
\item <2->当 $U_r=A\sin\omega t$ 时,
        \begin{eqnarray*}
        U_c(s) & =& G(s)U_r(s)\\
        U_c(t) &=& \frac{A\omega t}{1+\omega^2 T^2}e^{-\frac{t}{T}}+\frac{A}{\sqrt{1+\omega^2 T^2}}\sin(\omega t-\beta)
        \end{eqnarray*}
\item <3->稳态分量为 
      \[\frac{A}{\sqrt{1+\omega^2 T^2}}\sin(\omega t-\beta)\]
       其中, $\tan\beta=\omega T$ .
\item <4->结论:
\begin{itemize}
\item <4->对线性系统而言,输入为正弦信号,输出也为相同频率的正弦信号,但幅值与相位发生变化.
\item <5->幅值变化:输出是输入的  $\frac{1}{\sqrt{1+\omega^2 T^2}}$  倍
\item <6->相角变化:输出比输入滞后  $\arctan \omega T$  .
\end{itemize}
\end{itemize}
\end{frame}
\begin{frame}
\frametitle{频率特性定义}
\label{sec-1-4}

\begin{itemize}
\item <2->幅频特性:系统稳态正弦输出量与输入量的幅值比  $A(\omega)$
\item <3->相频特性:系统稳态正弦输出量与输入量的相角差  $\phi(\omega)$
\item <4->令  $s=j\omega$  ,则
      \begin{eqnarray*}
      A(j\omega)&=&|G(j\omega)|  \\
      \phi(j\omega) &=& \angle G(j\omega)
      \end{eqnarray*}
\end{itemize}
\end{frame}
\section{频率特性的图示表示法}
\label{sec-2}
\begin{frame}
\frametitle{Bode图}
\label{sec-2-1}

\begin{itemize}
\item <2->横坐标:  $\log_{10}\omega$
\item <2->纵坐标:$L(\omega)=20\log_{10}A(\omega),\phi(\omega)$
\end{itemize}
\begin{block}<3->{例:}
\label{sec-2-1-1}


\[G(s)=\frac{10(0.1s+1)}{(s+1)(0.01s+1)}\]

\begin{tikzpicture}[scale=0.5]
\begin{semilogxaxis}[
grid=both,ylabel=$L(\omega)$ ,xlabel=$\omega$ ,ymin=-20,ymax=20,xmin=0.1,xmax=1000,domain=0.1:1000,every axis plot post/.append style={mark=none}]
\addplot gnuplot { 20*log(10*abs(1/(1*x*{0,1}+1)*(0.1*x*{0,1}+1)/(0.01*x*{0,1}+1)))/log(10)};
%\addplot gnuplot { 20* log(10/(x>1?x:1) *(0.1*x>1?0.1*x:1) /(0.01*x>1?0.01*x:1))/log(10)} ;
\end{semilogxaxis}
\end{tikzpicture}
\begin{tikzpicture}[scale=0.5]
\begin{semilogxaxis}[
grid=both,
ylabel=$\phi(\omega)$ ,xlabel=$\omega$ ,ymin=-90,ymax=0,xmin=0.1,xmax=1000,domain=0.1:1000,every axis plot post/.append style={mark=none}]
\addplot gnuplot { 180/3.1415*arg(1/(1*x*{0,1}+1)*(0.1*x*{0,1}+1)/(0.01*x*{0,1}+1))};
%\addplot gnuplot { 20* log(10/(x>1?x:1) *(0.1*x>1?0.1*x:1) /(0.01*x>1?0.01*x:1))/log(10)} ;
\end{semilogxaxis}
\end{tikzpicture}
\end{block}
\end{frame}
\begin{frame}
\frametitle{Nyquist图}
\label{sec-2-2}


\begin{itemize}
\item <2->横坐标:$\Re [G(j\omega)]$
\item <2->纵坐标:$\Im [G(j\omega)]$
\end{itemize}
\begin{block}<3->{例:}
\label{sec-2-2-1}

\[G(s)=\frac{10(0.1s+1)}{(s+1)(0.01s+1)}\]

\begin{tikzpicture}[scale=0.5]
\begin{axis}[
grid=both,
ylabel=Im ,xlabel=Re,ymin=-5,ymax=0,xmin=0,xmax=10,every axis plot post/.append style={mark=none}]
\addplot
shell {
octave -q --eval "s=tf('s');g=10*(0.1*s+1)/(s+1)/(0.01*s+1);[re,im]=nyquist(g);disp([re,im]);"
};
\end{axis}

%\begin{axis}[ylabel=$\phi(\omega)$ ,xlabel=$\omega$ ,ymin=-10,ymax=0,xmin=0,xmax=10,domain=0:10000,samples=1000,every axis plot post/.append style={mark=none}]
%\addplot[/pgfplots/parametric=true] gnuplot { real(10/(1*t*{0,1}+1)*(0.1*t*{0,1}+1)/(0.01*t*{0,1}+1)), imag(10/(1*t*{0,1}+1)*(0.1*t*{0,1}+1)/(0.01*t*{0,1}+1))};
%\addplot gnuplot { 20* log(10/(x>1?x:1) *(0.1*x>1?0.1*x:1) /(0.01*x>1?0.01*x:1))/log(10)} ;
%\end{axis}
\end{tikzpicture}
\end{block}
\end{frame}
\begin{frame}
\frametitle{Nichols图}
\label{sec-2-3}

\begin{itemize}
\item <2->横坐标:$\phi(j\omega)$
\item <2->纵坐标:$20\log_{10} A(\omega)$
\end{itemize}
\begin{block}<3->{例:}
\label{sec-2-3-1}

\[G(s)=\frac{10(0.1s+1)}{(s+1)(0.01s+1)}\]

\begin{tikzpicture}[scale=0.5]
\begin{axis}[ylabel=$L(\omega)$ ,xlabel=$\phi(\omega)$ ,ymin=-45,ymax=20,xmin=-95,xmax=0,every axis plot post/.append style={mark=none}]
\addplot[blue,->]
shell {octave -q --eval "s=tf('s');g=10*(0.1*s+1)/(s+1)/(0.01*s+1);[re,im]=nichols(g);disp([im,20*log(re)/log(10)]);"};
%\draw[red,->] (axis cs:-80,-40)--(axis cs:0,0);
\end{axis}
%\begin{axis}[ylabel=$\phi(\omega)$ ,xlabel=$\omega$ ,ymin=-10,ymax=0,xmin=0,xmax=10,domain=0:10000,samples=1000,every axis plot post/.append style={mark=none}]
%\addplot[/pgfplots/parametric=true] gnuplot { real(10/(1*t*{0,1}+1)*(0.1*t*{0,1}+1)/(0.01*t*{0,1}+1)), imag(10/(1*t*{0,1}+1)*(0.1*t*{0,1}+1)/(0.01*t*{0,1}+1))};
%\addplot gnuplot { 20* log(10/(x>1?x:1) *(0.1*x>1?0.1*x:1) /(0.01*x>1?0.01*x:1))/log(10)} ;
%\end{axis}
\end{tikzpicture}
\end{block}
\end{frame}

\end{document}
