% Created 2013-10-25 Fri 13:31
\documentclass[table]{beamer}
\usepackage[T1]{fontenc}
\usepackage{fixltx2e}
\usepackage{graphicx}
\usepackage{longtable}
\usepackage{float}
\usepackage{wrapfig}
\usepackage{soul}
\usepackage{textcomp}
\usepackage{marvosym}
\usepackage{wasysym}
\usepackage{latexsym}
\usepackage{amssymb}
\usepackage{hyperref}
\tolerance=1000
\usepackage{amsmath}
\usepackage[usenames]{color}
\usepackage{pstricks}
\usepackage{pgfplots}
\pgfplotsset{compat=1.8}
\usepackage{tikz}
\usepackage[europeanresistors,americaninductors]{circuitikz}
\usepackage{colortbl}
\usepackage{yfonts}
\usetikzlibrary{shapes,arrows}
\usetikzlibrary{positioning}
\usetikzlibrary{arrows,shapes}
\usetikzlibrary{intersections}
\usetikzlibrary{calc,patterns,decorations.pathmorphing,decorations.markings}
\usepackage[BoldFont,SlantFont,CJKchecksingle]{xeCJK}
\setCJKmainfont[BoldFont=Evermore Hei]{Evermore Kai}
\setCJKmonofont{Evermore Kai}
\xeCJKsetup{CJKglue=\hspace{0pt plus .08 \baselineskip }}
\usepackage{pst-node}
\usepackage{pst-plot}
\psset{unit=5mm}
\usepackage{beamerarticle}
\mode<beamer>{\usetheme{Frankfurt}}
\mode<beamer>{\usecolortheme{dove}}
\mode<article>{\hypersetup{colorlinks=true,pdfborder={0 0 0}}}
\mode<beamer>{\AtBeginSection[]{\begin{frame}<beamer>\frametitle{Topic}\tableofcontents[currentsection]\end{frame}}}
\setbeamercovered{transparent}
\subtitle{频域稳定性判据}
\providecommand{\alert}[1]{\textbf{#1}}

\title{线性系统的频域分析法}
\author{}
\date{}
\hypersetup{
  pdfkeywords={},
  pdfsubject={},
  pdfcreator={Emacs Org-mode version 7.9.3f}}

\begin{document}

\maketitle

\begin{frame}
\frametitle{Outline}
\setcounter{tocdepth}{3}
\tableofcontents
\end{frame}













\section{Nyquit稳定性判据}
\label{sec-1}
\begin{frame}
\frametitle{辐角原理}
\label{sec-1-1}

\begin{itemize}
\item 设  $s$  为复变量,  $F(s)$  为  $s$  的有理分式函数.对于  $s$  平面上任意一点  $s$  , 通过复变函数  $F(s)$  的映射关系,可以确定  $s$  的象.
\item 在  $s$  平面上任选一条闭合曲线  $\Gamma$  ,且不通过  $F(s)$  任一零点和极点,  $s$  沿闭合曲线  $\Gamma$  运动一周,则相应地  $F(s)$  形成一条闭合曲线  $\Gamma_F$ .
\end{itemize}
\end{frame}
\begin{frame}
\frametitle{辐角原理(续):}
\label{sec-1-2}

设  $s$  平面闭合曲线  $\Gamma$  包围  $F(s)$  的  $Z$  个零点和  $P$  个极点,则  $s$  沿  $\Gamma$  顺时针运动一周时,在  $F(s)$  平面上,  $F(s)$  沿闭合曲线  $\Gamma_F$  逆时针包围原点的圈数为  $R=P-Z$ .
          
          \begin{tikzpicture}
          \draw[->] (-1,0) -- (4.5,0);
          \draw[->] (0,-2) -- (0,2);
          \draw (0,2) node[above left] {$j$};
          \draw (2,2) node[above right] {$\Gamma$};
          %\draw[dashed] (-4,-5) -- (-4,0);
          \draw [red] plot [smooth] coordinates {(1,0.5) (2,2)  (3,1.5) (3.5,0) (1.1,0) (1,0.5)};
          \draw (1,0.5) node {$\cdot$};
          \draw (1,0.5) node[left] {$s$};
          \draw[blue,->,thick] (1,0.5)-- ++(0.3,0.6);
          \draw (2,1) node {$\times$};
          \draw (2,-1) node {$\times$};
          \draw (2.3,0) node {$\times$};
          \draw (2.7,0.3) node {$\circ$};
          \draw (2.7,-0.3) node {$\circ$};
          
          \begin{scope}[shift={(7,0)}]
          \draw[->] (-2,0) -- (2,0);
          \draw[->] (0,-2) -- (0,2);
          \draw (0,2) node[above left] {$j$};
          \draw (1,1) node[above right] {$\Gamma_F$};
          \draw[red] (0,0) ++(0:1) arc (0:360:1);
          \draw[thick] (120:1) node {$\cdot$};
          \draw (120:1) node[above left] {$F(s)$};
          \draw (0,0) node[below left] {$o$};
          \draw[blue,->,thick] (120:1)-- ++(-0.3,-0.2);
          \end{scope}
          \end{tikzpicture}
\end{frame}
\begin{frame}
\frametitle{辐角原理的应用}
\label{sec-1-3}

\begin{eqnarray*}
\Phi(s) &= &\frac{G(s)}{1+G(s)H(s)} \\
       &=&\frac{G(s)}{1+G_o(s)} \\
       &=&\frac{G(s)}{F(s)} \\
 F(s)&=&1+G_o(s)
\end{eqnarray*}
\begin{itemize}
\item $F(s)$  的极点是系统开环极点,
\item $F(s)$  的零点是系统的闭环极点.
\end{itemize}
\end{frame}
\begin{frame}
\frametitle{辐角原理的应用(续)}
\label{sec-1-4}
\begin{columns}
\begin{column}{0.2\textwidth}
%% 示意图
\label{sec-1-4-1}

\begin{tikzpicture}
\draw[->] (-0.1,0) -- (2,0);
\draw[->] (0,-2) -- (0,2);
\draw (0,2) node[above left] {$j$};
\draw (0,0) node[below left] {$o$};
\draw[red,thick,->] (0,-1.7) -- (0,1.7);
\draw[violet,dashed,->] (0,2) arc (90:-90:2);
\draw[green,thick,->] (0,0) -- (60:2);
\draw (60:2) node[above] {$R$};
\draw (0,1) node[left] {$\Gamma_1$};
\draw (45:2) node[right] {$\Gamma_2$};
\end{tikzpicture}
\end{column}
\begin{column}{0.8\textwidth}
\begin{block}<2->{将 $\Gamma$ 分为两段:}
\label{sec-1-4-2}

\begin{itemize}
\item $\Gamma_1$ : $s=j\omega,\omega\in[-\infty,\infty]$
\item $\Gamma_2$ : $s=\lim_{R\rightarrow\infty}Re^{j\theta}$ , $\theta$ 从 $\frac{\pi}{2}$ 到 $-\frac{\pi}{2}$
\item 可得对应的 $G_o(s)$ 曲线.
\begin{itemize}
\item $s$ 在 $\Gamma_1$ 上时,与Nyquist图对应.($\omega\in[0,\infty]$)
\item $s$ 在 $\Gamma_1$ 上时, $F(s)=1+G_o(s)=1+\lim_{R\rightarrow\infty}Re^{j\theta}G_o(s)=1$
\end{itemize}
\item <3-> Nyquist判据
\begin{itemize}
\item 对于开环稳定系统($P=0$),若Nyquist曲线不包含 $(-1,0)$ 点,则系统稳定.
\item 对于开环稳定系统($P>0$),若Nyquist曲线逆时针包围 $(-1,0)$ 点的次数为 $\frac{P}{2}$ ,则系统稳定.
\end{itemize}
\end{itemize}
\end{block}
\end{column}
\end{columns}
\end{frame}
\begin{frame}
\frametitle{Nyquist判据,例1:}
\label{sec-1-5}

某负反馈开环传递函数为 $G_o(s)=\frac{10}{s-1}$ ,用Nyquist判据判断系统稳定性.
\begin{columns}
\begin{column}{0.5\textwidth}
\begin{block}{Nyquist图}
\label{sec-1-5-1}

\begin{tikzpicture}[scale=0.5]
%g=10/(s-1)
\begin{axis}[
%axis x line=middle,axis y line= middle, 
ylabel=$j$ ,xlabel=$   $ ,
ymin=-5.7,ymax=1,xmin=-11,xmax=1,every axis plot post/.append style={mark=none},
grid=both]
\addplot[blue,thick,->]
shell {
octave -q --eval "s=tf('s');g=10/(s-1);[re,im]=nyquist(g);disp([re,im]);"
};
\end{axis}
\end{tikzpicture}
\end{block}
\end{column}
\begin{column}{0.5\textwidth}
\begin{block}<2->{稳定性判断}
\label{sec-1-5-2}

\begin{eqnarray*}
P & = & 1\\
N &=& \frac{1}{2} \\
P-Z &=& 2N \\
Z &=& P-2N \\
  &=&0 
\end{eqnarray*}
系统稳定.
\end{block}
\end{column}
\end{columns}
\end{frame}
\begin{frame}
\frametitle{虚轴上有极点时}
\label{sec-1-6}

\begin{itemize}
\item 零型系统 $F(s)$ 沿 $\Gamma$ 解析且不为0.
\item I型及以上系统 $F(s)$ 在 $s=0$ 处不解析,不满足辐角原理条件.
\end{itemize}
\begin{columns}
\begin{column}{0.5\textwidth}
%% 示意图
\label{sec-1-6-1}

\begin{tikzpicture}
\draw[->] (-1,0) -- (2,0);
\draw[->] (0,-2) -- (0,2);
\draw (0,2) node[above left] {$j$};
\draw (0,0) node[below left] {$o$};
\draw[red,thick,->] (0,0.5) -- (0,1.7);
\draw[red,thick,->] (0,-1.7) -- (0,-0.5);
\draw[blue,thick,->] (0,0)++(-90:0.5) arc (-90:90:0.5);
\draw[violet,dashed,->] (0,2) arc (90:-90:2);
\draw[green,thick,->] (0,0) -- (60:0.5);
\draw (60:0.5) node[above] {$\epsilon$};
\draw (0,1) node[left] {$\Gamma_2$};
\draw (45:0.5) node[right] {$\Gamma_0$};
\draw (0,-1) node[left] {$\Gamma_1$};
\draw (45:2) node[right] {$\Gamma_3$};
\end{tikzpicture}
\end{column}
\begin{column}{0.5\textwidth}
\begin{block}<2->{将 $\Gamma$ 分为四段:}
\label{sec-1-6-2}

\begin{itemize}
\item $\Gamma_1$ : $s=j\omega,\omega\in[-\infty,0^-]$
\item $\Gamma_2$ : $s=j\omega,\omega\in[0^+,\infty]$
\item $\Gamma_3$ : $s=\lim_{R\rightarrow\infty}Re^{j\theta},\theta\in[-\frac{\pi}{2},\frac{\pi}{2}]$
\item $\Gamma_0$ : $s=\lim_{\epsilon\rightarrow 0}\epsilon^{j\theta}$ , $\theta$ 从 $\frac{\pi}{2}$ 到 $-\frac{\pi}{2}$
\item <3->对增补后的Nyquist图可使用Nyquist判据.
\end{itemize}
\end{block}
\end{column}
\end{columns}
\end{frame}
\begin{frame}
\frametitle{穿越次数}
\label{sec-1-7}
\begin{columns}
\begin{column}{0.5\textwidth}
\begin{block}{Nyquist图}
\label{sec-1-7-1}

\begin{tikzpicture}[scale=0.5]
\begin{axis}[
axis x line=middle,axis y line= middle, 
ylabel=$j$ ,xlabel=$   $ ,
ymin=-2.5,ymax=1,xmin=-3.5,xmax=3.5,every axis plot post/.append style={mark=none}]
grid=both,
\addplot[smooth,blue,thick,->]
shell {
octave -q --eval "
a=[0.01 3   0;
   0.1  0   -2; 
   1   -3   0; 
   2   -2.3  0.5; 3 -1.7 0; 4 -1 -0.5; 3 -0.5 0 ;5 -0.25 0.25; 7 0 0];
disp(a(:,2:3));"};
\end{axis}
\end{tikzpicture}
\end{block}
\end{column}
\begin{column}{0.5\textwidth}
\begin{block}{穿越次数}
\label{sec-1-7-2}

\begin{itemize}
\item 根据增补后的Nyquist曲线穿越 $(-1,0)$ 点左侧的次数可得 $\Gamma_F$ 包围原点的圈数
     \begin{eqnarray*}
     R &=  &2N \\
       &=& 2(N_+ - N_-)
     \end{eqnarray*}
   其中,
\begin{itemize}
\item $N_+$ 为正穿越(自上向下)次数
\item $N_-$ 为负穿越(自下向上)次数
\end{itemize}
\end{itemize}
\end{block}
\end{column}
\end{columns}
\end{frame}
\begin{frame}
\frametitle{例: $G_o(s)=\frac{10}{s(s+1)}$}
\label{sec-1-8}


\begin{tikzpicture}
%g=10/s/(s+1)
\begin{axis}[
axis x line=middle,axis y line= middle, 
ylabel=$j$ ,xlabel=$   $ ,
ymin=-9,ymax=1,xmin=-6,xmax=10,every axis plot post/.append style={mark=none}]
grid=both,
\addplot[blue,thick,->]
shell {
octave -q --eval "s=tf('s');g=10/s/(s+1);[re,im]=nyquist(g);disp([re,im]);"
};
\addplot[red,dashed,->]shell {octave -q --eval "t=[-0.1:-0.1:-pi*1.5/2]';disp(8*[cos(t),sin(t)]);"};
\end{axis}
\end{tikzpicture}
\end{frame}
\section{Bode稳定性判据}
\label{sec-2}
\begin{frame}
\frametitle{Bode稳定性判据}
\label{sec-2-1}
\begin{columns}
\begin{column}{0.5\textwidth}
%% Bode图
\label{sec-2-1-1}

\begin{tikzpicture}[scale=0.42]
\begin{semilogxaxis}[
%axis x line=middle,axis y line= left, 
ylabel=$L(\omega)/L_a(\omega)$ ,xlabel=$\omega$ ,
every axis plot post/.append style={mark=none},
grid=both,
ymin=-10,ymax=20,xmin=0.01,xmax=10]
\addplot[smooth,blue,thick]
 shell {
octave -q --eval "
a=[0.01 3   0;
   0.1  0   -2; 
   1   -3   0; 
   2   -2.3  0.5; 
   3 -1.7 0;
   5 -1 -0.5;
   7 -0.5 0 ;
   10 -0.25 0.25];
w=a(:,1);
a=a(:,2)+i*a(:,3);
m=abs(a);
disp([w,20*log(m)/log(10)]);"};
\end{semilogxaxis}
\end{tikzpicture}
\begin{tikzpicture}[scale=0.42]
\begin{semilogxaxis}[
%axis x line=middle,axis y line= left, 
ylabel=$\phi(\omega)$ ,xlabel=$\omega$ ,
every axis plot post/.append style={mark=none},
grid=both,
ymin=-200,ymax=10,xmin=0.01,xmax=11]
%\draw[blue,thick] (axis cs:0.1,90)--(axis cs:10,90);
\addplot[smooth,blue,thick]
shell {
octave -q --eval "
a=[0.01 3   0;
   0.1  0   -2; 
   1   -3   0; 
   2   -2.3  0.5; 
   3 -1.7 0;
   5 -1 -0.5;
   7 -0.5 0 ;
   10 -0.25 0.25];
w=a(:,1);
a=a(:,2)+i*a(:,3);
p=angle(a)*180/pi;
p(p>0)=p(p>0)-360;
disp([w,p]);"};
\draw[red,dashed] (axis cs:0.01,-180) --(axis cs:10,-180);
\draw[red,dashed] (axis cs:5,-190) --(axis cs:5,-170);
\end{semilogxaxis}
\end{tikzpicture}
\begin{tikzpicture}[scale=0.42]
\begin{axis}[
%axis x line=middle,axis y line= middle, 
ylabel=$j$ ,xlabel=$   $ ,
ymin=-2.5,ymax=1,xmin=-3.5,xmax=3.5,every axis plot post/.append style={mark=none},
grid=both]
\addplot[smooth,blue,thick,->]
shell {
octave -q --eval "
a=[0.01 3   0;
   0.1  0   -2; 
   1   -3   0; 
   2   -2.3  0.5; 3 -1.7 0; 4 -1 -0.5; 3 -0.5 0 ;5 -0.25 0.25; 7 0 0];
disp(a(:,2:3));"};
\end{axis}
\end{tikzpicture}
\end{column}
\begin{column}{0.5\textwidth}
\begin{block}<2->{稳定性判断}
\label{sec-2-1-2}

\begin{itemize}
\item 截止频率 $\omega_c$ : $A(\omega_c)=0$
\item 穿越频率 $\omega_x$ : $\phi(\omega_x)=(2k+1)\pi$
\item <3->Bode判据:
\begin{itemize}
\item 最小相位系统,若在 $\omega<\omega_c$ 前 $N_+-N_-=0$ ,则系统稳定
\item 非最小相位系统,若在 $\omega<\omega_c$ 前 $N_+-N_-=\frac{P}{2}$ ,则系统稳定
\end{itemize}
\end{itemize}
\end{block}
\end{column}
\end{columns}
\end{frame}

\end{document}
